\begin{landscape}

\setlist{nolistsep}
\setitemize{noitemsep,topsep=0pt,parsep=0pt,partopsep=0pt,nolistsep,leftmargin=*,label={}}
\chapter{Vier zentrale politisch-administrative Traditionen}	
	
\renewcommand{\arraystretch}{1.4} 
\begin{table}[!hbt]\tiny
\begin{tabular}{|L{4cm}|L{5cm}|L{5cm}|L{5cm}|L{5cm}|}\hline
&\textbf {\footnotesize Deutsch}&	\textbf{\footnotesize Französisch}&\textbf {\footnotesize Angelsächsisch}&	\textbf{\footnotesize Skandinavisch}\\\hline
Verhältnis Staat-Gesellschaft&	organisch&	antagonistisch&	pluralistisch&	organisch\\\hline
Politische Organisation&föderalistisch&zentralistisch	&begrenzt föderalistisch&dezentralisiert, unitaristisch\\\hline
Politikstil&	legalistisch	&korporatistisch, legalistisch&	inkrementell&	konsenuell, technokratisch\\\hline
Dezentrale Elemente&	kooperativer Föderalismus&	regionalisierter Einheitsstaat&	State Power (US), Local Government (UK)&	starke lokale Autonomie\\\hline
Vorherrschende Sichtwiese auf öffentliche Verwaltung&	Öffentliches Recht	&Öffentliches Recht&	Politische Wissenschaft/ Soziologie	&Öffentliches Recht (SWE), Organisationstheorie (NO)\\\hline
Historische Dimension&	Preussische Tradition	&Napoleonische Tradition&	Civic culture Tradition&	Wohlfahrtsstaatsmodell\\\hline
Legale Basis der öffentlichen Verwaltung
&
\begin{itemize}
\item gesonderte Gesetze zu öfentlichem Dienst
\item Verfassungsstatus des öffentlichen Dienstes
\end{itemize}
&
\begin{itemize}
\item gesonderte Gesetze zu öfentlichem Dienst            
\item negative Definition öffentlicher Verwaltung 
\end{itemize}
&
 \vspace{-2mm}
\begin{itemize}
\item gesonderte Gesetze zu öfentlichem Dienst
\item keine Verankerung des öffentlichen Dienstes in der Verfassung
\item Rolle der öffentlichen Verwaltung eher in dienenderer Tradition als in Kontinentaleuropa
 \vspace{-2mm}
 \end{itemize}

&
\begin{itemize}
\item gesonderte Gesetze zu öfentlichem Dienst            
\item Mischung aus Deutschem und Angelsächsischem Modell
\end{itemize}\\\hline
Grad der Zentralisierung&
\begin{itemize}
\item vertikale und horizontale Fragmentierung
\item administrative Dezentralisierung
\item hierarchische Strukturen
\end{itemize}
&
\begin{itemize}
\item unitaristische und stark zentralistische Regierung und öffentliche Verwaltung
\item hierarchische Strukturen
\end{itemize}
 &
\begin{itemize}
\item unitaristische und zentralistische politisch-administrative Strukturen
\item wenig hierarchische Strukturen 	
\end{itemize}
&
Mischung aus Deutschem und Angelsächsischem Modell\\\hline
Koordination innerhalb der öffentlichen Verwaltung	&inter-minsterielle Koordination	&begrenzte inter-ministerielle Koordination&	inter-minsterielle Koordination&	Mischung aus Deutscher und Angelsächsischer inter-ministerieller Koordination\\\hline
Administrativer Rahmen	&
 \vspace{-2mm}
\begin{itemize}
\item einheitlicher administrativer Rahmen auf allen Ebenen
\item föderaler Rahmen mit regional und kommunalverwaltung
\item unterschiedliche Sub-Verwaltungen mit eigenen Kompetenzen
\item vertikale Verteilung von Kompetenzen zwischen den verschiedenen föderalen Ebenen
 \vspace{-2mm}\end{itemize}

&
\begin{itemize}
\item einheitlicher administrativer Rahmen
\item administrative Untereinheiten direkten Weisungen der Zentralregierung unterstellt
\item strikt zentralistische Orientierung
\end{itemize}
&
\begin{itemize}
\item weitgehend autnomome Exekutiv Organe
\item untergeordnete administrative Einheiten mit eingeschränkter finanzieller Autonomie           
\end{itemize}
&
\begin{itemize}
\item zentralistischer Aufbau
\item Mischung aus Deutschem und Angelsächsischem Modell
\end{itemize}\\\hline
Verhältnis von Politik und öffentlicher Verwaltung&	Trennung von öffentlicher Verwaltung und Politik&
\begin{itemize}
\item Trennung von öffentlicher Verwaltung und Politik
\item enge Beziehungen zwischen Politikern und Verwaltern
\end{itemize}
&
 \vspace{-2mm}
\begin{itemize}
\item civic culture und individualistische Tradition
\item Trennung von öffentlicher Verwaltung und Politik
\item Werte des politischen Systems bestimmen auch die öffentliche Verwaltung
 \vspace{-2mm}
 \end{itemize}

&	Mischung aus Deutschem und Angelsächsischem Modell\\\hline
Personalpolitik und Rekrutierung&
\begin{itemize}
\item Primat von Universitätsausbildung im höheren Dienst
\item Hauptsächlich Juristen
\item Beamte sind der personifizierte Staat
\item Ernennung aufgrund von Qualifikation und Leistung, mit begrenzten politischen Ernennungen (höhere Positionen)
\item Lebenszeit und Staatsbediensteten auf Vertragsbasis	
\end{itemize}&
 \vspace{-2mm}
\begin{itemize}
\item vorwiegend administrative Elite
\item hauptsächlich Juristen, aber auch Generalisten
\item homogene mentale und kognitive Übereinstimmung der Bediensteten in der Verwaltung                             \item Rekrutierung vor allem aus spezialisierten Verwaltungsschulen
\item Ernennung aufgrund von Qualifikation und Leistung, mit begrenzten politischen Ernennungen (höhere Positionen)
\item Karriereorientierung
 \vspace{-2mm}
\end{itemize}
&
\begin{itemize}
\item kein Einfluss von Politikern auf Beförderung
\item Universitätsausbildung für höhere Positionen
\item  vorwiegend Generalisten
\item Bestimmte Universitäten bei der Rekrutierung bevorzugt
\item Karrieresystem
\end{itemize}&	Mischung aus Deutschem und Angelsächsischem Modell\\\hline
Länder&	Deutschland, Österreich, Niederlande, Spanien (nach 1978), Belgien (nach 1988)&	Frankreich, Italien, Spanien (bis 1978), Portugal, Griechenland, Belgien (bis 1988) 	&UK, US, Irland&	Schweden, Norwegen, Dänemark\\\hline
\end{tabular}
\end{table}	\footnote{nach Loughlin (1994) aus: Lippert/Umbach (2005), S. 65ff.}
\end{landscape}	
