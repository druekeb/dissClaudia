
\newgeometry{left=15mm,right=20mm, top=20mm, bottom=25mm}
\begin{landscape}
\renewcommand*\chapterheadstartvskip{\vspace*{0cm}}
\chapter{Vier zentrale politisch-administrative Traditionen}	
\label{tab:poltisch-administrative Traditionen}
\begin{scriptsize}
\setlist{nolistsep}
\setitemize{noitemsep,topsep=0pt,parsep=0pt,partopsep=0pt,nolistsep,leftmargin=*,label={}}

\renewcommand{\arraystretch}{1.5} 
\begin{longtable}[H]{|L{4cm}|L{5cm}|L{5cm}|L{5cm}|L{4cm}|}\hline
&\textbf {\footnotesize Deutsch}&	\textbf{\footnotesize Französisch}&\textbf {\footnotesize Angelsächsisch}&	\textbf{\footnotesize Skandinavisch}\\\hline
\endfirsthead
\caption{(Fortsetzung)}\\\hline
&\textbf {\footnotesize Deutsch}&	\textbf{\footnotesize Französisch}&\textbf {\footnotesize Angelsächsisch}&	\textbf{\footnotesize Skandinavisch}\\\hline
\endhead 
\endfoot
\multicolumn{5}{l}{}\\
\multicolumn{5}{c}{\normalsize nach Loughlin (1994) aus: \cite{lipumb05}: 65ff.}
\endlastfoot
Verhältnis Staat-Gesellschaft&	organisch&	antagonistisch&	pluralistisch&	organisch\\\hline
Politische Organisation&föderalistisch&zentralistisch	&begrenzt föderalistisch&dezentralisiert, unitaristisch\\\hline
Politikstil&	legalistisch	&korporatistisch, legalistisch&	inkrementell&	konsensuell, technokratisch\\\hline
Dezentrale Elemente&	kooperativer Föderalismus&	regionalisierter Einheitsstaat&	State Power (US), Local Government (UK)&	starke lokale Autonomie\\\hline
Vorherrschende Sichtweise auf öffentliche Verwaltung&	Öffentliches Recht	&Öffentliches Recht&	Politische Wissenschaft/ Soziologie	&Öffentliches Recht (SWE), Organisationstheorie (NO)\\\hline
Historische Dimension&	Preußische Tradition	&Napoleonische Tradition&	Civic culture Tradition&	Wohlfahrtsstaatsmodell\\\hline
\parbox[t]{4cm}{Legale Basis der öffentlichen Verwaltung}
&
\parbox[t]{5cm}{gesonderte Gesetze zu öffentlichem Dienst\\
Verfassungsstatus des öffentlichen Dienstes}
&
\parbox[t]{5cm}{gesonderte Gesetze zu öffentlichem Dienst\\
negative Definition öffentlicher Verwaltung}
&
\parbox[t]{5cm}{gesonderte Gesetze zu öffentlichem Dienst\\
keine Verankerung des öffentlichen Dienstes in der Verfassung\\
Rolle der öffentlichen Verwaltung eher in dienenderer Tradition als in Kontinentaleuropa}
&
\parbox[t]{4cm}{gesonderte Gesetze zu öffentlichem Dienst\\
Mischung aus Deutschem und Angelsächsischem Modell}
\\\hline
\parbox[t]{4cm}{Grad der Zentralisierung}
&
\parbox[t]{5cm}{vertikale und horizontale Fragmentierung\\
administrative Dezentralisierung\\
hierarchische Strukturen\\}
&
\parbox[t]{5cm}{unitaristische und stark zentralistische Regierung und öffentliche Verwaltung\\
hierarchische Strukturen}
&
\parbox[t]{5cm}{unitaristische und zentralistische politisch-administrative Strukturen\\
wenig hierarchische Strukturen}
&
\parbox[t]{4cm}{Mischung aus Deutschem und Angelsächsischem Modell}\\\hline
\parbox[t]{4cm}{Koordination innerhalb der öffentlichen Verwaltung}
&
\parbox[t]{5cm}{inter-ministerielle Koordination}
&
\parbox[t]{5cm}{begrenzte inter-ministerielle Koordination}
&
\parbox[t]{5cm}{inter-ministerielle Koordination}
&
\parbox[t]{4cm}{Mischung aus Deutscher und Angelsächsischer inter-ministerieller Koordination\\}\\\hline
\parbox[t]{4cm}{Administrativer Rahmen}
&
\parbox[t]{5cm}{einheitlicher administrativer Rahmen auf allen Ebenen\\
föderaler Rahmen mit Regional- und Kommunalverwaltung\\
unterschiedliche Sub-Verwaltungen mit eigenen Kompetenzen\\
vertikale Verteilung von Kompetenzen zwischen den verschiedenen föderalen Ebenen\\}
&
\parbox[t]{5cm}{einheitlicher administrativer Rahmen\\
administrative Untereinheiten direkten Weisungen der Zentralregierung unterstellt\\
strikt zentralistische Orientierung}
&
\parbox[t]{5cm}{weitgehend autonome Exekutiv Organe\\
untergeordnete administrative Einheiten mit eingeschränkter finanzieller Autonomie}           
&
\parbox[t]{4cm}{zentralistischer Aufbau\\
Mischung aus Deutschem und Angelsächsischem Modell}
\\\hline
\parbox[t]{4cm}{Verhältnis von Politik und öffentlicher Verwaltung}
&
\parbox[t]{5cm}{Trennung von öffentlicher Verwaltung und Politik}
&
\parbox[t]{5cm}{Trennung von öffentlicher Verwaltung und Politik\\
enge Beziehungen zwischen Politikern und Verwaltern}
&
\parbox[t]{5cm}{civic culture und individualistische Tradition\\
Trennung von öffentlicher Verwaltung und Politik\\
Werte des politischen Systems bestimmen auch die öffentliche Verwaltung}
&
\parbox[t]{4cm}{Mischung aus Deutschem und Angelsächsischem Modell}\\\hline
\parbox[t]{4cm}{Personalpolitik und Rekrutierung}
&
\parbox[t]{5cm}{Primat von Universitätsausbildung im höheren Dienst\\
Hauptsächlich Juristen\\
Beamte sind der personifizierte Staat\\
Ernennung aufgrund von Qualifikation und Leistung, mit begrenzten politischen Ernennungen (höhere Positionen)\\
Lebenszeit und Staatsbediensteten auf Vertragsbasis}
&
\parbox[t]{5cm}{vorwiegend administrative Elite\\
hauptsächlich Juristen, aber auch Generalisten\\
homogene mentale und kognitive Übereinstimmung der Bediensteten in der Verwaltung\\
Rekrutierung vor allem aus spezialisierten Verwaltungsschulen\\
Ernennung aufgrund von Qualifikation und Leistung, mit begrenzten politischen Ernennungen (höhere Positionen)\\
Karriereorientierung}
&
\parbox[t]{5cm}{kein Einfluss von Politikern auf Beförderung\\
Universitätsausbildung für höhere Positionen\\
vorwiegend Generalisten\\
Bestimmte Universitäten bei der Rekrutierung bevorzugt\\
Karrieresystem}
&
\parbox[t]{4cm}{Mischung aus Deutschem und Angelsächsischem Modell}\\\hline
\parbox[t]{4cm}{Länder}
&
\parbox[t]{5cm}{Deutschland, Österreich, Niederlande, Spanien (nach 1978), Belgien (nach 1988)}
&
\parbox[t]{5cm}{Frankreich, Italien, Spanien (bis 1978), Portugal, Griechenland, Belgien (bis 1988)}
&
\parbox[t]{5cm}{UK, US, Irland}
&
\parbox[t]{4cm}{Schweden, Norwegen, Dänemark}\\\hline

\end{longtable}	
\end{scriptsize}
\end{landscape}	
