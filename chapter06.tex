\chapter{Abschließende Betrachtung}\label{chap:abschlBetrachtung}
Im nun folgenden Abschlusskapitel werden die Ergebnisse der einzelnen Kapitel der Arbeit zum Einfluss der EU-Perspektive auf die Verwaltungsmodernisierung in den Ländern des Westlichen Balkans im Hinblick auf die Untersuchungsfragen zusammengeführt. Noch offene Forschungsfragen werden benannt und mögliche Strategien formuliert. Hier noch einmal die eingangs formulierten Fragen der Untersuchung:
\begin{itemize} \itemsep1pt \parskip0pt \parsep0pt
\item Sind Erfahrungen hinsichtlich der Entwicklung der öffentlichen Verwaltung in den Ländern der letzten Aufnahmewelle übertragbar auf den Westlichen Balkan?
\item Liefert die historische Betrachtung der Verwaltungsentwicklung unter Einschluss früherer Regime der kommunistischen oder sozialistischen Zeit, aber auch der zeitlich davor gelagerten Einflüsse der Imperien verwertbare Erkenntnisse für den aktuellen Modernisierungsprozess?
\item Wie fördert die EU die Verwaltungsmodernisierung in den Beitrittsländern?
\item Wie schätzen Experten der EU und Akteure im Westbalkan die Verwaltungsentwicklung im Kontext der EU-Erweiterung ein?
\item Welche Optionen bestehen für die Verwaltungsentwicklung in den Westbalkanstaaten?
\end{itemize}
Das Thema Verwaltungsmodernisierung im Kontext der EU-Erweiterung fand bislang in der wissenschaftlichen Erforschung überraschenderweise wenig Beachtung. Für die vorliegende Untersuchung sichtete die Autorin Literatur und wertete Berichte der EU und anderer Institutionen aus. Auf der Basis der Ergebnisse dieser Sichtung wurde eine Expertenbefragung durchgeführt.\par
Um verallgemeinerbare Aussagen zu erhalten, wurden drei Länder des Westlichen Balkan für die Untersuchung herangezogen. Die noch labilen staatlichen Gebilde des Balkan (Kosovo und Bosnien-Herzegowina) wurden von der Untersuchung ausgeschlossen, ebenso Kroatien, da es kurz vor der Aufnahme in die EU stand. Als Untersuchungsländer eigneten sich drei benachbarte kleinere Länder des Westlichen Balkan: Montenegro, Mazedonien und Albanien.\par
Hauptbezugspunkt der Untersuchung ist die Verwaltungsentwicklung in den Untersuchungsländern und der Einfluss der EU-Beitrittsperspektive auf die Verwaltungsmodernisierung. Vor diesem Hintergrund wurden die aktuellen Entwicklungen der Europäisierungsforschung dargestellt, die im Rahmen der Politikwissenschaft aus der Transitionsforschung hergeleitet wurden. Für den hier interessierenden EU-Einfluss auf die Verwaltungsentwicklung in den Beitrittsländern ist die Konditionalitätsforschung ein besonders relevanter Strang der Europäisierungsforschung.\par
Neben diesem theoretischen Gerüst wurden weitere Rahmen in die Untersuchung eingezogen: Die Erfahrungen aus den bisherigen Erweiterungsrunden in Bezug auf die Verwaltungsmodernisierung arbeitete die Autorin mittels einer Literaturanalyse heraus. Die Verwaltungsentwicklung der Untersuchungsländer seit ihrer Demokratisierung wurde dargestellt. Auch die zeitlich davor bestehenden Systeme und ihre Bedeutung für die Verwaltungsentwicklung fanden Eingang in die Untersuchung. Ebenso wurden die EU-Pro"-gramme in ihrer Bedeutung für die Verwaltungsentwicklung in den Beitrittskandidaten beleuchtet.\par
Zur Überprüfung der Befunde und Klärung noch offener Fragen wurde eine Expertenbefragung mittels halbstandardisierter Interviews konzipiert und durchgeführt. Die Befragung erstreckte sich auf mit dem Thema Verwaltungsmodernisierung praktisch und thematisch vertraute Experten aus EU-Institutionen und den drei Untersuchungsländern.\par
Bei der Literaturauswertung zur EU-Osterweiterung fiel auf, dass in den Fortschrittsberichten der EU die Verwaltungsmodernisierung als wichtiges Element betrachtet wurde, nach der vollzogenen Erweiterung aber kein Monitoring zur Verwaltungsentwicklung mehr stattfand.
Aktuelle Untersuchungen zum Status quo der Verwaltungsentwicklung nach der Ost-Erweiterung stellen für viele der Länder der letzten Erweiterungswelle eine Stagnation oder ein Zurückgehen hinter schon Erreichtes fest. Einige dieser Länder haben Reformen angestoßen und unter anderem Gesetze im Bereich Verwaltungsmodernisierung erlassen, jedoch oft nicht umgesetzt, und „some even backtracked once they joined EU“ (\cite{pickering}: 23).\par
Die EU erwartet von potenziellen Mitgliedern die Übernahme des Acquis communautaire. Dabei handelt es sich um das Regelwerk der EU, das Kandidatenländer in nationales Recht übernehmen müssen. Dieses ist in 35 Kapitel gegliedert und bildet die Grundlage für die Beitrittsverhandlungen. Öffentliche Verwaltung ist keines der Kapitel des Acquis, sondern wird unter politischen Kriterien in den jährlichen Fortschrittsberichten der EU begutachtet. Während sich die Erweiterungspolitik der EU evolutionär entwickelt, hat sich in Bezug auf die Verwaltungsmodernisierung seit der Osterweiterung keine wesentliche Weiterentwicklung ergeben. Nach wie vor ist die Verwaltungsmodernisierung als politisches Kriterium verankert und entfaltet daher im Gegensatz zu den Themen der Kapitel des Acquis nur begrenzten Zugzwang für die Beitrittskandidaten. Die von der EU propagierte Konditionalität gegenüber den Beitrittskandidaten greift beim Thema Verwaltungsentwicklung daher kaum. Dieser Befund wurde schon in der Analyse der EU-Osterweiterung deutlich. Überraschenderweise wird dennoch keine wesentliche Weiterentwicklung der Herangehensweise der EU zum Thema öffentliche Verwaltung und EU-Erweiterung nach Südosteuropa erkennbar. Die Auswertung der Experteninterviews bestätigt diesen Befund.\par
Auf der praktischen Ebene wird dies in folgender Aussage deutlich:
\begin{itemize}[label={}]
\item \textit{The EU always states in their progress reports that there should be a stable, professional public administration and career system established and not having political appointees in the institutions. But on the other hand, this message is not clear for the Albanian politicians and nothing happens to them if they change their staff. I think the heads of the institutions need to start feeling responsible for the outcome of their institution”} (NGO representative Albania, Frage 6, Seite \pageref{sec:par view}).
\end{itemize}
Hier wird einerseits die propagierte politische Konditionalität der EU gegenüber den Beitrittsländern bestätigt. Gleichzeitig wird deutlich, dass die Nichterfüllung in Bezug auf die öffentliche Verwaltung, hier den öffentlichen Dienst, keine Auswirkungen auf den Erweiterungsprozess hat. Eine Änderung der von der EU kritisierten Praxis der öffentlichen Verwaltung in den Beitrittsländern findet nicht statt.\par
Eine Studie zur EU-Konditionalität im Bereich Demokratie und Rechtsstaatlichkeit kommt zu dem Ergebnis, dass Konditionalität in diesen Bereichen, unter die auch Verwaltungsentwicklung fällt, versagt hat. Gründe werden in den fehlenden Vergleichskriterien und Standards der EU gesehen. „The Commission acted as a prolific myth-maker, asking the candidate countries to embrace non-existent ‚European standards’” (\cite{kochenov}: 300).\par
Dimitrova stellt dazu fest, dass administrative Konditionalität zwar eine “partial conditionality” ist, diese aber ein Land nicht am Beitritt hindern kann, “a candidate country may view itself as to be able to skirt full compliance in that particular policy area without retribution” (\cite{dimit05}: 79). \par
Diese Befunde werden von der vorliegenden Untersuchung gestützt, die auch darauf hinweist, dass Konditionalität im Bereich Verwaltungsentwicklung nicht annähernd gut greift wie in den im Acquis communautaire ausdrücklich geforderten Bereichen. Die Frage ist allerdings, ob man von einem Versagen sprechen kann, wie Kochenov dies vorschlägt (\cite{kochenov}: 300), oder ob es sich dabei nicht vielmehr um eine logische Folge handelt, angesichts der Tatsache, dass kein EU-Acquis zur Verwaltungsmodernisierung definiert ist.\par
In der Analyse der Fortschrittsberichte für die drei Untersuchungsländer von 2006 bis 2012 fällt auf, dass in der Zusammenfassung vor allem die Politisierung des öffentlichen Dienstes als Problem benannt wird. Beim Vergleich der Länder über die Jahre hinweg ergibt sich ein erstaunlich einheitliches Bild in der Einschätzung durch die EU. Eine Entwicklung ist kaum festzustellen und die Politisierung des öffentlichen Dienstes in allen drei Untersuchungsländern wird Jahr um Jahr angeprangert. Auffällig und überraschend in der Auswertung des Interviewmaterials ist die starke und fast ausschließliche Konzentration aller Interviewpartner auf das Thema civil service. Im Rahmen der verwaltungswissenschaftlichen Betrachtung würde man das Thema civil service allenfalls als ein Thema unter vielen anderen Themen (z.B. E-Government, Dezentralisierung, regionale Verwaltungszusammenarbeit etc.) der Debatte um Verwaltungsmodernisierung erwarten.\par
Möglicherweise ist diese (einseitige) Konzentration im Zusammenhang mit der Berichterstattung der EU im Rahmen der Fortschrittsberichte zu sehen: Wenn, wie im Fall der Zusammenfassungen zur Verwaltungsentwicklung in den EU-Fortschrittsberichten, (fast) ausschließlich der civil service besprochen wird, überrascht der Befund der vorliegenden Arbeit nicht, dass kaum andere Themen der Verwaltungsmodernisierung von den Gesprächspartnern benannt werden. Mit dieser Schwerpunktsetzung wird der politische Diskurs zwischen der EU und den beitrittswilligen Staaten unter Umständen thematisch vorbestimmt. Diskurse bilden, so Foucault, Wirklichkeit nicht nur ab, sondern stellen sie auch her (vgl. Foucault 1973: 42). Im Sinne Foucaults könnte man sagen, dass die EU mit ihren Fortschrittsberichten die Wahrnehmung aller Akteure beeinflusst und zur Erschaffung der Realität, im Sinne der Konstruktion einer komplexen gesellschaftlichen Situation, beiträgt. Andere Themen der Verwaltungsmodernisierung wie die Erbringung öffentlicher Aufgaben, E-Government oder Kundenorientierung werden, wie die Interviewauswertung zeigt, weder von den EU officials noch von den Experten in den Untersuchungsländern benannt. Um diesen Befund weiter auszuleuchten, wäre weitergehende verwaltungswissenschaftliche Forschung wünschenswert. \par

Die Annahme, dass die vor der Demokratisierung bestehenden Systeme immer noch einen Einfluss auf die aktuelle Situation der öffentlichen Verwaltung entfalten (Legacy-Ansatz), wird von der Auswertung der Interviews gestützt. Die befragten Experten verweisen in einer Reihe von Fällen auf langfristige kulturelle Prägungen bei aktuellen Problemen der Verwaltungsentwicklung. Kulturelle Prägungen sind insbesondere in einem komplexen System wie der öffentlichen Verwaltung nur langsam veränderbar und entfalten sehr langfristigen Einfluss.\par
Diese sind zum großen Teil in der historischen Verwaltungsentwicklung zu suchen. Für Mazedonien und Montenegro ist dieser historische Bezugspunkt Jugoslawien mit seinem spezifischen „Selbstverwaltungssozialismus“, für Albanien dagegen ist es das kommunistische System in seiner isolationistischen Ausprägung. Aber auch die diesen Systemen zeitlich vorgelagerten Einflüsse der Imperien setzen sich in der Verwaltungskultur der Untersuchungsländer immanent fort. Während der osmanischen Zeit entfaltete das Reich vor allem in den Bergregionen kaum administrativen Einfluss und das Gemeinwesen wurde durch die traditionellen Clanbeziehungen geregelt. Diese Besonderheit fand sich in den Bergregionen Albaniens und Montenegros, während in Mazedonien unterschiedliche Nachbarländer immer wieder auf das Land zugriffen.\par
Besonders die in den EU-Fortschrittsberichten oft kritisierte fehlende Implementierung von Gesetzen und Reformen kann zurückverfolgt werden in die Zeit Jugoslawiens. In der spezifischen staatlichen Verfassung in Jugoslawien entwickelte sich die Praxis, Projekte und Initiativen der Föderation zwar aufzunehmen, diese aber in den jeweiligen Republiken nicht umzusetzen und stattdessen regionale Interessen zu verfolgen. Albanien hat aktuell vor allem mit einer gegenseitigen Blockadepolitik zweier etwa gleich starker Parteien zu kämpfen. Hier kommen Veränderungen nicht voran, da die jeweils andere Gruppierung Gesetze und Reformen blockt. Dies kommt besonders in der Beziehung vom Zentrum zu der lokalen Ebene zum Tragen, wenn diese eine unterschiedliche politische Orientierung haben. Im kommunistischen System war die öffentliche Verwaltung straff auf das Zentrum ausgerichtet, ohne Spielraum für die kommunale Ebene. Historisch gesehen gab es in Albanien zu keiner Zeit eine Tradition der Kooperation von lokaler und zentraler Ebene.\footnote{Hier ergibt sich auch ein wesentlicher Unterschied zu den Ländern der Ost-Erweiterung der EU. In diesen Ländern hatte vor ihrer Zugehörigkeit zur Sowjetunion mit entsprechend zentralistischer öffentlicher Verwaltung eine kontinentaleuropäische Verwaltungstradition bestanden, an die angeknüpft werden konnte.
}\par
Zu den weiteren historischen Einflüssen, die im heutigen System der öffentlichen Verwaltung fortwirken, gehört die in der sozialistischen Zeit in Jugoslawien abgeschaffte Unterscheidung in Beamte und andere öffentlich Angestellte. Alle öffentlich Bediensteten fielen in Jugoslawien unter das allgemeine Arbeitsrecht. Aktuell ist zu beobachten, dass es vor allem in Mazedonien mit dieser Tradition schwierig ist, eine Trennung in Beamte und Angestellte wieder einzuführen. In Albanien, wo es historisch kein Modell der civil servants gegeben hat, steht die Einführung des Konzeptes der civil servants ebenfalls vor großen Schwierigkeiten. Lediglich in Montenegro, das in der Zeit vor seiner Zugehörigkeit zu Jugoslawien über ein funktionierendes Berufsbeamtentum verfügte, sind historisch gesehen Anknüpfungspunkte vorhanden. \par
Eine andere Tradition der kontinentaleuropäischen öffentlichen Verwaltung, die Verwaltungsgerichtsbarkeit und Überprüfbarkeit des Verwaltungshandelns, überdauerte die jugoslawische Zeit fast unbeschadet. Die Verwaltungsgerichtsbarkeit in Montenegro erhält regelmäßig gute Noten in den Fortschrittsberichten der EU. Ganz anders stellt sich die Situation in Albanien dar, wo es historisch zu keiner Zeit eine Verwaltungskontrolle gab. Eine Verwaltungsgerichtsbarkeit ist in Albanien erst im Entstehen. \par
Bei der historischen Betrachtung fiel auf, dass Elemente moderner öffentlicher Verwaltung vor allem in Montenegro identifiziert wurden. Montenegro war in der historischen Verwaltungsentwicklung von Einflüssen des französischen und des k.u.k.-Systems der Verwaltung geprägt, bevor es Teil des sozialistischen Jugoslawien wurde. Aber auch die Verwaltung in Jugoslawien mit ihrer besonderen Form des Selbstverwaltungssozialismus führte Elemente moderner Staatlichkeit fort, wie die gerichtliche Überprüfbarkeit von Verwaltungsentscheidungen. Auch war die dezentrale Organisation des Staates im sozialistischen Jugoslawien zumindest in begrenztem Umfang hilfreich nach der Demokratisierung bei der Etablierung von Strukturen unterhalb des Zentralstaates. Hinderlich ist in diesem Zusammenhang allerdings bis heute die während der sozialistischen Zeit etablierte Durchdringung aller Ebenen der öffentlichen Verwaltung mit Parteigängern der Regierungspartei. Diese Tradition setzt sich trotz aller Bemühungen um Reformen in der Gesetzgebung zum civil service fast ungehindert fort. Es stellt sich die Frage, ob dies nur der schwierigen ökonomischen Lage geschuldet ist, oder ob es sich um tiefer liegende Strukturen handelt. \par
Deutlich wurde, dass die aktuellen Entwicklungen und Probleme der Modernisierung der Verwaltungen in den Untersuchungsländern eine starke historische Prägung haben. Für die weitere Modernisierung der öffentlichen Verwaltungen in den Beitrittsländern ist es wichtig, diese Prägungen bei der Konzipierung von Unterstützung einer weiteren Annäherung an die EU zu berücksichtigen.

\section{Konsequenzen für die Förderinstrumente der EU}
%In den Untersuchungsländern besteht eine Reihe von hemmenden Faktoren für die Verwaltungsmodernisierung. Zunächst ist die Tradition der zentralistischen Organisation der öffentlichen Verwaltung in den vor-demokratischen Zeiten zu nennen. Die politischen Umwälzungen nach 1989 stellten die Untersuchungsländer vor die Aufgabe eine öffentliche Verwaltung auszubauen, die der geografischen Verortung der Länder in Europa entsprach. Während der politische Umbau hin zu einer parlamentarischen Demokratie in allen Untersuchungsländern schnell stattfand mit der Einführung von demokratischen Wahlen und einem Mehrparteiensystem, ist der Umbau, bzw. Aufbau einer rechtsstaatlichen Verwaltung eine langfristige Aufgabe. Die vorliegende Arbeit macht deutlich, dass dieser Umbau noch nicht vollständig stattgefunden hat. Dabei stehen vor allem Traditionen im Weg, die in der vordemokratischen Zeit geprägt wurden. Fehlender Bezug des Einzelnen zum Gemeinwesen mit stärkerer Prägung auf die Familie und den Clan, der wie gezeigt wurde geschichtlich erklärt werden kann, ist zentral. Ebenfalls geschichtlich herleitbar ist die mangelnde Umsetzung von Gesetzen in den Untersuchungsländern, die eine effektive Verwaltungsmodernisierung behindert. So wurde deutlich, dass einerseits in allen Untersuchungsländern Gesetze zum civil service erlassen wurden und Institutionen eingerichtet wurden zu ihrer Umsetzung. Andererseits werden jedoch Wege gefunden, mit denen die Anforderungen umgangen werden, wie z.B. temporäre Verträge für civil servants.\par
Im Rahmen der Heranführungsstrategie für die Länder, die der EU beitreten wollen, stellt die EU Mittel für die Beitrittskandidaten bereit. Für die Staaten des Westbalkans stand EU-Finanzhilfe zunächst vor allem für Wiederaufbau und Infrastrukturprojekte zur Verfügung. Im Zuge der konkreteren EU-Perspektive der Länder seit dem Thessaloniki-Gipfel 2003 ist auch Institutionenaufbau Teil der Unterstützung. In dem aktuellen IPA-Programm der EU für die beitrittswilligen Länder wird auch die administrative Kapazität gefördert.\par
Auf der Ebene der praktischen Unterstützung der Beitrittskandidaten durch die EU findet sich vor allem die Entsendung von Experten und Praktikern aus den EU-Ländern in die Beitrittskandidaten im Rahmen der Institutionenhilfe. Diese Entsendungen erfolgen meist im Rahmen des Twinning-Ansatzes der EU, der aus der Verwaltungshilfe für Entwicklungsländer entstanden ist. Dieser Ansatz wird von den Akteuren in Brüssel und in den Untersuchungsländern generell als angemessen eingeschätzt. Allerdings gibt es Fragezeichen der befragten Experten hinsichtlich der Eignung dieses Ansatzes, insbesondere in Abwesenheit einer verbindlichen Richtschnur der umzusetzenden Standards zur Verwaltungsentwicklung. Als problematisch wird die Tendenz erlebt, Konzepte der jeweiligen Entsendeländer in Bezug auf öffentliche Verwaltung in den Beitrittsländern umzusetzen. Die Nachhaltigkeit und Angemessenheit eines solchen Ansatzes wird seitens der Interviewpartner durchaus in Frage gestellt. Auch zu diesem Zusammenhang böte sich weitere verwaltungswissenschaftliche Forschung an. \par
%Die Auswertung der Experteninterviews weist auf Verbesserungsmöglichkeiten bei den EU-Pro"-grammen zur Unterstützung der administrativen Kapazität in den Beitrittsländen hin. So gibt es außer den Hinweisen aus den EU-Fortschrittsberichten keine konkrete Orientierung für Projekte zur Verbesserung der öffentlichen Verwaltung. Dies führte und führt oft zur Übertragung von Modellen aus der Verwaltungspraxis der Länder, die die Experten stellen. Auch in diesem Zusammenhang wurde von den Interviewpartnern fehlende Nachhaltigkeit der Hilfe bemängelt. Oft sind die Projekte nicht ausreichend vor Ort abgestimmt und werden in Brüssel konzipiert, womit Akzeptanz vor Ort und institutionelles Lernen erschwert wird.\par
Eine Steuerung der Hilfe, die durch die EU zur Verfügung gestellt wird, ist durch ein komplexes Zuständigkeitsgeflecht in der EU zu bewältigen. Für den Bereich Verwaltungsentwicklung kommen weitere Stellen innerhalb der EU dazu, da es sich um eine horizontale Aufgabe handelt, die nicht in einem Acquis-Kapitel definiert ist. Die Koordinierung wird von den Interviewpartnern sowohl der EU als auch aus den Empfängerländern als schwierig erlebt. \par
Das aktuelle Instrument IPA ist in seinem finanziellen Rahmen auf große Projekte mit einem bestimmten Finanzvolumen ausgelegt. In der Erfahrung der Interviewpartner stellt dies vor allem kleinere Länder vor Herausforderungen. Auch werden die Antragsmodalitäten in ihrer Komplexität als problematisch benannt. Die Interviewauswertung weist darauf hin, dass sich aus diesen strukturellen Schwierigkeiten Hemmnisse für die Anwendung dieses Instrumentariums für die Verwaltungsmodernisierung in den Untersuchungsländern ergeben. \par
In der Zusammenschau der fehlenden Tradition einer rechtsstaatlich orientierten öffentlichen Verwaltung sowie der mangelnden Umsetzung von Gesetzen und Reformen in den Untersuchungsländern mit den Schwierigkeiten bei der Förderung von Verwaltungsmodernisierung durch EU-Pro"-gramme besteht die Gefahr von Reform-Attrappen. Für die EU-Erweiterung stellt dies ein ernstes Problem dar, weil, wie gezeigt wurde, die Konditionalität in Bezug auf die öffentliche Verwaltung keine Kraft entfalten kann und zudem nach dem Beitritt kein Monitoring zur Verwaltungsentwicklung mehr stattfindet. Im Ergebnis würden weitere Länder in die EU aufgenommen, deren öffentliche Verwaltungen den Mindestanforderungen an Professionalität, Transparenz und Effektivität nicht entsprechen. 

\section{Mögliche Strategien }
In der Auswertung der Interviews wird deutlich, dass vor allem in den Untersuchungsländern selbst eine stärkere Orientierung und Anleitung zur Verwaltungsmodernisierung durch die EU gewünscht wird. Es wäre für die weitere Modernisierung der Beitrittsländer zielführend, eine Art von Katalog an die Hand zu bekommen. Auch für die Konzipierung der EU-Projekte zur Unterstützung der Verwaltungsmodernisierung wäre dies sinnvoll. In Abwesenheit verbindlicherer Leitlinien der EU zu Verwaltungsstruktur und Verwaltungsmodernisierung werden in den EU-Projekten meist Modelle aus den jeweiligen Ländern der entsandten Experten übertragen. Eine Praxis, die in Anbetracht der grundsätzlichen Probleme mit der Verwaltungsentwicklung in den Untersuchungsländern aus Sicht von Nachhaltigkeit und „local ownership“ zu hinterfragen ist.\par
Auf der praktischen Ebene wird eine stärkere Outcome- und Impact-Orientierung der EU-Hilfe im Bereich Verwaltungsmodernisierung als förderlich angesehen. Dies würde ein langfristiges Monitoring der Ergebnisse nach Projektabschluss voraussetzen, eine Praxis, die bislang nicht entwickelt ist.\par
Weiterhin sollten gesellschaftliche Gruppen an der Entwicklung eines Fahrplanes zur Verwaltungsmodernisierung beteiligt werden. Dabei handelt es sich um eine kulturelle Umorientierung, die nur langfristig zu erreichen ist. Möglicherweise ist die Ausgangssituation in Montenegro und Mazedonien etwas günstiger einzuschätzen, da dort in der jugoslawischen Zeit zumindest nominell eine Beteiligung gesellschaftlicher Gruppen an Planungsprozessen propagiert worden war. In Albanien dagegen hat historisch gesehen eine Beteiligung von gesellschaftlichen Gruppen keine Tradition. \par
Zentral wäre eine kontinuierliche Begutachtung der Verwaltungsentwicklung auch nach einem Beitritt. Die Erfahrung aus der EU-Osterweiterung zeigt, dass es nach der Aufnahme in die EU meist zu einer deutlichen Verlangsamung der Reformfortschritte der Länder oder gar zu Rückschritten gekommen ist.\par
Für die Untersuchungsländer wird auch deutlich, dass die Instrumente der EU, vor allem das IPA-Programm, das mit großen Finanzvolumen arbeitet, an die Bedürfnisse von kleineren Ländern angepasst werden sollte. \par
Eine aktuelle Untersuchung zu civil service-Reformen und -Professionalisierung im Westbalkan kommt zu dem Schluss, dass die Bedingungen für Reformen auf nationaler Ebene in den Beitrittsländern derzeit ungünstig sind. Dies ist zum Teil der aktuellen Finanzlage und damit zusammenhängend Verschärfungen der sozialen Situation geschuldet, aber auch der Abschwächung der europäischen Perspektive der Staaten des Westbalkans. Es wird konstatiert, dass die fehlende Definition administrativer Konzepte seitens der EU zunehmend ein Problem darstellt für die Reformschritte in den Beitrittsländern. So werden dem europäischen Standard entsprechende Konzepte für den civil service zunehmend aufgeweicht mit größerem Einfluss und Entscheidungsfreiheit der Vorgesetzten über die Einstellungspraxis. ”The contestation of the European principles as the most desired concept to guide civil service reform and the lack of specific guidelines for institutional reform have increasingly undermined the capacity of the international community to direct and monitor reform efforts in the Western Balkans. Further reform slippage is likely unless the European principles are reviewed, clarified and re-fashioned in the area of civil service reform” (\cite{meyersah12}: 8). In diesem Zusammenhang schlägt der Autor der Studie vor, die europäischen Prinzipien zum civil service neu zu definieren und in ein umfassenderes Konzept von „better governance“ einzubetten. „Indeed it is worth considering a re-launch of the European principles as a wider initiative for better governance in Europe“ (\cite{meyersah12}: 9). \par
Im Verlauf der vorliegenden Arbeit zeigte sich, dass die EU zwar von administrativer Konditionalität spricht, es aber kein Konzept gibt, an dem sich die Beitrittskandidaten orientieren könnten. Versuche einer PAR checklist sind im Sande verlaufen, nicht zuletzt aufgrund von Widerstand aus den Mitgliedstaaten, die keine Einmischung in die Ausgestaltung ihrer historisch gewachsenen öffentlichen Verwaltung wünschen. Auch das Konzept „European Administrative Space“ hat bislang keine nachhaltige Wirkung entfaltet, nicht zuletzt aufgrund der fehlenden Verbindlichkeit. Somit besteht in der Realität keine wirksame administrative Konditionalität. Notwendig ist eine neue Debatte um eine EU-Vision von Better Governance, die umfassend ist und auch über das Thema civil service hinaus eine moderne und den Herausforderungen angemessene öffentliche Verwaltung in der erweiterten EU entwirft. Dieses Konzept wäre dann nicht nur eine Richtschnur für die Beitrittsländer des Westbalkans, sondern auch anwendbar für die Modernisierung der Verwaltungen in den Ländern, die schon zur EU gehören. Ein Konzept von Better Governance für die EU zu entwickeln, ginge auch über die bekannten Elemente von New Public Management hinaus, da es zusätzlich zu Effektivitätskriterien weitere Elemente einbeziehen müsste und wahrscheinlich auch zu einer gewissen Vereinheitlichung administrativer Strukturen führen würde. Dass eine solche Reform insgesamt vonnöten ist, zeigen nicht zuletzt die aktuellen Probleme in der EU, die in Ländern mit intransparenter und nicht effektiver öffentlicher Verwaltung verschärft zum Tragen kommen.

