\documentclass[parskip=full]{scrreprt}
% PDF-Kompression
\pdfminorversion=5
\pdfobjcompresslevel=1
% Allgemeines
\usepackage[automark]{scrpage2} % Kopf- und Fußzeilen
\usepackage{amsmath,marvosym} % Mathesachen
\usepackage[T1]{fontenc} % Ligaturen, richtige Umlaute im PDF
\usepackage[utf8]{inputenc}% UTF8-Kodierung für Umlaute usw
\usepackage[official]{eurosym}
\usepackage{url}%für online-Zitate
\usepackage{anysize} %Seitenränder verändern


\usepackage{geometry}
\geometry{a4paper,left=30mm,right=25mm, top=25mm, bottom=25mm}
% Schriften
\usepackage{mathpazo} % Palatino für Mathemodus
%\usepackage{mathpazo,tgpagella} % auch sehr schöne Schriften
\usepackage{setspace} % Zeilenabstand
\usepackage{caption} %Zeilenumbruch in Bildunterschriften
\onehalfspacing % 1,5 Zeilen
% Schriften-Größen
\setkomafont{chapter}{\Large\rmfamily} % Überschrift der Ebene
\setkomafont{section}{\large\rmfamily}
\setkomafont{subsection}{\rmfamily}
\setkomafont{subsubsection}{\rmfamily}


\setkomafont{chapterentry}{\large\rmfamily} % Überschrift der Ebene in Inhaltsverzeichnis
\setkomafont{descriptionlabel}{\bfseries\rmfamily} % für description Umgebungen
\setkomafont{captionlabel}{\small\bfseries}
\setkomafont{caption}{\small}
% Nummerierung der Sections und subsections und subsubsections
\setcounter{secnumdepth}{2}
% Sprache: Deutsch
\usepackage{ngerman}
\usepackage[ngerman, english]{babel} % Silbentrennung
% PDF
\usepackage[ngerman]{hyperref}
\usepackage[final]{microtype} % mikrotypographische Optimierungen
\usepackage{pdflscape} % einzelne Seiten drehen können
% Tabellen
\usepackage{multirow} % Tabellen-Zellen über mehrere Zeilen
\usepackage{multicol} % mehre Spalten auf eine Seite
\usepackage{tabularx} % Für Tabellen mit vorgegeben Größen
\usepackage{longtable} % Tabellen über mehrere Seiten
\usepackage{array}

\usepackage[table]{xcolor}
%  Bibliographie
%\usepackage{bibgerm} % Umlaute in BibTeX
% Tabellen
\usepackage{float}
\usepackage{wrapfig}
% Bilder
\usepackage{graphicx} % Bilder
\usepackage{color} % Farben
\usepackage{colortbl}%Hintergrundfarben für Tabellen	
% Define user colors using the RGB model
\definecolor{dunkelgrau}{rgb}{0.8,0.8,0.8}
\definecolor{hellgrau}{rgb}{0.95,0.95,0.95}
\definecolor{red}{rgb}{1.0,0.5,0.5}
\graphicspath{{images/}}
\DeclareGraphicsExtensions{.pdf,.png,.jpg} % bevorzuge pdf-Dateien
\usepackage{subfigure} % mehrere Abbildungen nebeneinander/übereinander
\newcommand{\subfigureautorefname}{\figurename} % um \autoref auch für subfigures benutzen
\setcapindent{0em} % kein Einrücken der Caption von Figures und Tabellen
%
%\setcapwidth[c]{0.9\textwidth}
\setlength{\abovecaptionskip}{0.2cm} % Abstand der zwischen Bild- und Bildunterschrift

%Abkürzungsverzeichnis

\usepackage[]{acronym}


% 
% 
% % Quellcode
\usepackage{listings} % für Formatierung in Quelltexten
\definecolor{grau}{gray}{0.25}
\definecolor{lightgray}{rgb}{.9,.9,.9}
\definecolor{darkgray}{rgb}{.4,.4,.4}
\definecolor{purple}{rgb}{0.65, 0.12, 0.82}

\lstdefinelanguage{JavaScript}{
  keywords={typeof, new, true, false, catch, function, return, null, catch, switch, var, if, in, while, do, else, case, break},
  keywordstyle=\color{blue}\bfseries,
  ndkeywords={class, export, boolean, throw, implements, import, this},
  ndkeywordstyle=\color{darkgray}\bfseries,
  identifierstyle=\color{black},
  sensitive=false,
  comment=[l]{/* */},
  morecomment=[s]{/*}{*/},
  commentstyle=\color{purple}\ttfamily,
  stringstyle=\color{red}\ttfamily,
  morestring=[b]',
  morestring=[b]"
}
\lstset{
   language=JavaScript,
   backgroundcolor=\color{lightgray},
   extendedchars=true,
   basicstyle=\footnotesize\ttfamily,
   showstringspaces=false,
   showspaces=false,
   %numbers=right,
   numberstyle=\scriptsize\ttfamily,
   numbersep=9pt,
   tabsize=2,
   breaklines=true,
   showtabs=false,
   captionpos=b
}

% linksbündige Fußboten
\deffootnote{1.5em}{1em}{\makebox[1.5em][l]{\thefootnotemark}}

%typearea{14} % typearea am Schluss berechnen lassen, damit die Einstellungen oben berücksichtigt werden
% für autoref von Gleichungen in itemize-Umgebungen
\makeatletter
\newcommand{\saved@equation}{}
\let\saved@equation\equation
\def\equation{\@hyper@itemfalse\saved@equation}
\makeatother 
\renewcommand{\labelenumi}{\arabic{enumi}.}
\renewcommand{\labelenumii}{\arabic{enumi}.\arabic{enumii}}

 % Importiere die Einstellungen aus der Präambel
\newcolumntype{P}[1]{>{\raggedright\let\newline\\\arraybackslash\hspace{0pt}}b{#1}}
\newcolumntype{Q}[1]{>{\raggedright\let\newline\\\arraybackslash\hspace{0pt}}p{#1}}
\newcolumntype{L}[1]{>{\raggedright\let\newline\\\arraybackslash\hspace{-2pt}}m{#1}}
\newcolumntype{M}[1]{>{\raggedright\let\newline\\\arraybackslash\hspace{0pt}}m{#1}}
\newcolumntype{C}[1]{>{\centering\let\newline\\\arraybackslash\hspace{0pt}}m{#1}}
\newcolumntype{R}[1]{>{\raggedleft\let\newline\\\arraybackslash\hspace{-2pt}}m{#1}}

\renewcommand{\arraystretch}{1.2}

\newcommand*\oldurlbreaks{}
\let\oldurlbreaks=\UrlBreaks
\renewcommand{\UrlBreaks}{\oldurlbreaks\do\- \do\t}

% Eigene Trennungsregeln* 
\hyphenation{Demo-kra-ti-sie-rung}
\hyphenation{Er-wei-te-rung}
\hyphenation{Schwie-rig-kei-ten}
\hyphenation{Unter-su-chungs-län-dern}
\hyphenation{Bei-tritts-kan-di-daten}
\hyphenation{EU-Erwei-te-rungs-welle}
\hyphenation{EU-Mit-glieds-län-der}
\hyphenation{Ver-wal-tungs-ge-richts-bar-keit}

\usepackage{enumitem}
\usepackage{soul}
\sethlcolor{lightgray}
\setlength\belowcaptionskip{0.5\baselineskip} 
\usepackage{helvet}
 \usepackage[babel,german=guillemets]{csquotes}
\usepackage[backend=biber,
bibencoding=utf8,
style=authoryear, 
autocite=inline,
labeldate=true,  
uniquename=false,
uniquelist=minyear, 
maxcitenames=2]{biblatex}
\addbibresource{bibliography.bib}% Syntax for version >= 1.2
\DeclareNameAlias{author}{last-first}
\renewcommand{\labelnamepunct}{\addcolon\space} % Doppelpunkt nach Label 
\renewcommand*{\multinamedelim}{\addslash}
\renewcommand*{\finalnamedelim}{\multinamedelim}
% \renewcommand*{\multinamedelim}{\addsemicolon\space}% mehrere Namen durch Semikolon plus Leerzeichen 
 % \renewcommand*{\finalnamedelim}{\addsemicolon\space}% mehrere Namen durch Semikolon plus Leerzeichen 
\DeclareFieldFormat{citetitle}{#1\isdot}
\DeclareFieldFormat[book]{citetitle}{#1}
\DeclareFieldFormat[article]{citetitle}{#1}
\DeclareFieldFormat[incollection]{booktitle}{#1}
\DeclareFieldFormat[inbook]{booktitle}{#1}
\DeclareFieldFormat[book]{title}{#1}
\DeclareFieldFormat[article]{title}{#1}
\DeclareFieldFormat[manual]{title}{#1}
\DeclareNameAlias{sortname}{last-first}
\DefineBibliographyStrings{ngerman}{andothers={et \addabbrvspace al\adddot}} 

\makeatletter
\def\uplabel#1{{{\textsf{#1}}\hfill}}
\renewenvironment{acronym}[1][1]{%
   \providecommand*{\acro}{\AC@acro}%
   \providecommand*{\acroplural}{\AC@acroplural}%
   \long\def\acroextra##1{##1}%
   \def\@tempa{1}\def\@tempb{#1}%
   \ifx\@tempa\@tempb%
      \global\expandafter\let\csname ac@des@mark\endcsname\AC@used%
      \ifAC@nolist%
      \else%
         \begin{list}{}%
                {\settowidth{\labelwidth}{\normalfont{\textsf{#1}}\hspace*{6em}}% change according to your needs
                \setlength{\itemsep}{-\parsep} %kein Abstand, kompakte Darstellung 
                \setlength{\leftmargin}{\labelwidth}%
                \addtolength{\leftmargin}{\labelsep}%
                \renewcommand{\makelabel}{\uplabel}}
      \fi%
   \else%
      \begin{AC@deflist}{#1}%
   \fi%
  }%
  {%
   \ifx\AC@populated\AC@used\else%
      \ifAC@nolist%
      \else%
          \item[]\relax%
      \fi%
   \fi%
   \expandafter\ifx\csname ac@des@mark\endcsname\AC@used%
      \ifAC@nolist%
      \else%
        \end{list}%
      \fi%
   \else%
      \end{AC@deflist}%
   \fi}%
\renewenvironment{AC@deflist}[1]%
        {\ifAC@nolist%
         \else%
            \raggedright\begin{list}{}%
                {\settowidth{\labelwidth}{\normalfont{\textsf{#1}}\hspace*{3em}}% change according to your needs
                \setlength{\leftmargin}{\labelwidth}%
                \addtolength{\leftmargin}{\labelsep}%
                \renewcommand{\makelabel}{\uplabel}}%
          \fi}%
        {\ifAC@nolist%
         \else%
            \end{list}%
         \fi}%
 \makeatother


% Dokumentenanfang
\begin{document}

% Seitennummerierung für Titel, Widmung, Danksagung, Zusammenfassung,
% Inhaltsverzeichnis werden in römischen Zahlen gesetzt

\pagestyle{empty} 

% Titelseite
%\clearscrheadings\clearscrplain

\newpage
\setlength{\voffset}{20mm}
%\begin{center}
\openup 1em
{\LARGE \bfseries Verwaltungsmodernisierung im Westlichen\\ Balkan im Kontext der EU-Erweiterung\\am Beispiel von Albanien, Mazedonien und Montenegro}\par
%\end{center}
\openup -1em
\vspace{70mm}
\begin{large}
\begin{tabular}{p{180mm}}
Dissertation zur \\
Erlangung des akademischen Grades einer \\
Doktorin der Wirtschafts- und Sozialwissenschaften (Dr. rer. pol.) \\
im Fachbereich Wirtschaftswissenschaften\\
der Universität Kassel\\
\end{tabular}

\vspace{1cm}

\begin{tabular}{p{24cm}}
Vorgelegt von Claudia Vollmer\\
Kassel, im November 2013\\
\end{tabular}
\newpage
\vspace{1cm}

\begin{tabular}{ll}
{\bf Erstgutachterin:} &  Prof. Dr. Silke Laskowski\\
{\bf Zweitgutachter:}&Prof. Dr. Jürgen Reese\\
\end{tabular}
\end{large}
\clearpage
\setlength{\voffset}{0mm}
% Widmung*
%\include{dedication}

% Danksagung*
%\include{acknowledgement}

% Zusammenfassung/Abstract
%\include{abstract}
  
% Inhalts-, Abbildungs*-, Tabellen*-Verzeichnis
\pagestyle{useheadings} 
\pagenumbering{roman}
\setkomafont{chapterentry}{\normalfont} % Überschrift der Ebene in
\tableofcontents
\newpage
\listoffigures
\newpage
\listoftables
\newpage
\pagestyle{plain}
\chapter*{Verzeichnis der Abkürzungen}

\begin{acronym}
%\setlength{\itemsep}{-\parsep} %kein Abstand, kompakte Darstellung 
%\settowidth{\labelwidth}{\hspace*{6em}}
%\addtolength{\leftmargin}{\labelsep}%
 \acro{BdKJ }{		Bund der Kommunisten Jugoslawiens}
 \acro{BENF}{ 		Beneficaries}
 \acro{BiH }{		Bosnia and Herzegovina} 
 \acro{CAF	}{	Common Assessment Framework}
 \acro{CARDS}{ 	European Financing Programme for assisting the countries of the Western Balkans}
 \acro{CCLSGR}{ 	Committee for Coordination of Local Self-Government Reform}
 \acro{CEE 	}{	Central and Eastern European}
 \acro{CEEC 	}{	Central and Eastern European Countries }
 \acro{CoE 	}{	Council of Europe}
 \acro{COMECON}{	Council for Mutual Economic Assistance}
 \acro{CS }{		Civil Service }
 \acro{CSA}{ 		Civil Service Agency (Mazedonien)}
 \acro{DEZA}{	Direktion für Entwicklung und Zusammenarbeit}
 \acro{DG 	}{	Directorate-General}
 \acro{DG ADMIN}{	Directorate-General for Administration}
 \acro{DG ELARG}{	Directorate-General for Enlargement}
 \acro{DG HR	}{Directorate-General for Human Resources}
 \acro{DoPA}{		Department of Public Administration (Albania) }
 \acro{EAR	}{	European Agency for Reconstruction}
 \acro{EAS	}{	European Administrative Space }
 \acro{EBRD}{		European Bank for Reconstruction and Development}
 \acro{EC	}{	European Commission}
 \acro{EGPA}{		European Group of Public Administration }
 \acro{EIPA}{		European Institute of Public Administration }
 \acro{EU	}{	European Union}
 \acro{EuGH}{		Europäischer Gerichtshof}
 \acro{EUPAN}{	European Public Administration Network }
 \acro{FYROM}{	Former Yugoslav Republic of Macedonia }
 \acro{GTZ	}{	Deutsche Gesellschaft für Technische Zusammenarbeit	}
 \acro{HHStA}{	Österreichisches Haus-, Hof- und Staatsarchiv Wien }
 \acro{HRMA}{	Human Resource Management Authority (Montenegro)}
 \acro{IMF	}{	International Monetary Fund}
 \acro{IPA 	}{	Instrument for Pre-accession Assistance}
 \acro{ISPA}{		Instrument for Structural Policies for Pre-Accession}
 \acro{MIFF}{		Multi-Annual Indicative Financial Framework}
 \acro{MIPD}{		Multi-Annual Indicative Planning Document}
 \acro{MPA}{		Master of Public Administration }
 \acro{NGO}{		Non-governmental Organisation }
 \acro{NRO}{		Nichtregierungsorganisation}
 \acro{NISPA}{		Network of Institutions and Schools of Public Administration in Central and Eastern Europe}
 \acro{NPM}{		New Public Management }
 \acro{NUTS}{	Nomenclature des unités territoriales statistiques }
 \acro{OECD}{		Organisation for Economic Cooperation and Development}
 \acro{OSCE}{		Organisation for Security and Cooperation in Europe}
 \acro{OSZE}{		Organisation für Sicherheit und Zusammenarbeit in Europa}
 \acro{PA	}{	Public Administration}
 \acro{PAR	}{	Public Administration Reform}
 \acro{Phare}{		Poland and Hungary Assistance for the Restructuring of the Economy}
 \acro{PIFC}{		Public Internal Financial Control}
 \acro{RESPA}{	Regional School for Public Administration }
 \acro{RGW}{		Rat für gegenseitige Wirtschaftshilfe}
 \acro{SAA	}{	Stabilisation and Association Agreement }
 \acro{SAI	}{	Supreme Audit Institution}
 \acro{SaM}{		Serbia and Montenegro}
 \acro{SAP	}{	Stabilisierungs- und Assoziierungsprozess }
 \acro{SAPARD}{	Special Accession Programme for Agriculture \& Rural Development}
 \acro{SFRJ	}{	Sozialistische Föderative Republik Jugoslawien}
 \acro{SFRY	}{	Socialist Federal Republic of Yugoslavia}
 \acro{SIGMA}{	(OECD) Support for Improvement in Governance and Management}
 \acro{TA	}{	Technical Assistance}
 \end{acronym}

\newpage
% Seitennummerierung im Hauptteil#

\pagestyle{useheadings} 
\pagenumbering{arabic}
\setcounter{page}{1}
% Kapitel
\chapter{EU-Beitritt als Option}
Seit dem Ende der kriegerischen Auseinandersetzungen auf dem Balkan in den 1990er Jahren in Folge des Auseinanderbrechens Jugoslawiens besteht für die Länder des Westlichen Balkans \footnote{Kroatien, Serbien, Montenegro und Mazedonien, Albanien, Bosnien-Herzegowina, Kosovo.}  die Option für einen Beitritt zur EU. Allen Staaten des Westlichen Balkans wurde im Juni 2003 vom Europäischen Rat in Thessaloniki eine konkrete EU-Beitrittsperspektive eröffnet. Mazedonien\footnote{Aufgrund eines Namensstreites mit Griechenland ist der offizielle Name „Former Yugoslav Republic of Macedonia“ (FYROM), unter dem das Land 1993 in die Vereinten Nationen aufgenommen wurde. Für einen besseren Lesefluss wird in der vorliegenden Arbeit die Bezeichnung Mazedonien verwendet.}, Montenegro und Serbien haben inzwischen die Stufe von Beitrittskandidaten erreicht, während Albanien, Bosnien-Herzegowina und Kosovo potenzielle Beitrittskandidaten sind. Slowenien, auch ein Nachfolgestaat Jugoslawiens, war schon in der letzten Aufnahmewelle 2004 zusammen mit Ländern Osteuropas in die EU aufgenommen worden, Kroatien ist am 1. Juli 2013 in die EU aufgenommen worden.\par

Der Westbalkan ist von EU-Mitgliedern umgeben (Ungarn, Griechenland, Slowenien, Bulgarien, Italien, Slowakei). So gesehen handelt es sich bei den noch nicht beigetretenen Ländern des Westbalkans quasi um einen „weißen Fleck“ mitten in Europa. Vor dem Hintergrund der fragilen Situation auf dem Balkan nach der Auflösung Jugoslawiens und der potenziell instabilen staatlichen Gebilde (Bosnien-Herzegowina, Kosovo) erscheint eine konkrete EU-Perspektive nachvollziehbar.\par

Für die Europäische Union ist das Bestehen dieses „weißen Fleckes“ ein Problem, für das Lösungen gesucht wurden. Eine anscheinend sinnvolle Lösung ist die erneute Erweiterung der EU mittels der Integration der Länder des Westbalkans. Das Interesse der EU ist dabei auch ein strategisches, nicht zuletzt zur Vermeidung eines Konfliktherdes „vor der Haustüre“. Dieses strategische Interesse schließt gleichzeitig die Hoffnung auf einen Reformmotor mit ein, wie es auch in der folgenden Äußerung Javier Solanas\footnote{Generalsekretär des Rates der Europäischen Union von 1999 bis 2009 und in dieser Zeit Hoher Vertreter für die Gemeinsame Außen- und Sicherheitspolitik (GASP).} zum Ausdruck kommt:\par

„It is in the European interest that countries on our borders are well governed. Neighbours who are engaged in violent conflict, weak states where organised crime flourishes, dysfunctional societies or exploding population growth on its borders all pose problems for Europe. […] The credibility of our foreign policy also depends on the successes achieved in the region. The European perspective is both a strategic goal and an incentive for reform” \cite{solana}.\par

In den Staaten des Westbalkans selbst ist die Aufnahme in die Europäische Union ein vorrangiges Ziel. Damit verbunden sind Hoffnungen auf ein vor allem ökonomisch schnelleres Heranrücken an europäische (Lebens-)Standards.\par

Im Prozess der Ausrichtung der neu entstandenen Westbalkanstaaten auf demokratische Werte und Marktwirtschaft spielten internationale Organisationen seit Anfang der 1990er Jahre eine wesentliche Rolle als Unterstützer sowohl finanzieller Art als auch mit Beratungsleistungen. Im Westbalkan waren dies vor allem das United Nations Development Programm (UNDP), die US-amerikanische Entwicklungshilfeorganisation USAID, die Organisation für Sicherheit und Zusammenarbeit in Europa (OSZE) und die Weltbank sowie die Einrichtung eines Stabilitätspaktes für den Westbalkan (1999). Die EU mit ihrer Agency for Reconstruction (EAR), die in Kosovo, Serbien, Montenegro und Mazedonien die EU-Hilfe koordinierte, kam Ende 2002 hinzu. Ursprünglich war die Hilfe für die neu entstandenen Staaten des Westbalkans auf strukturelle Hilfen und Entwicklungshilfe ausgelegt. Mit zunehmendem Engagement der EU in der Region und nach dem EU-Gipfel von Thessaloniki von 2003 wurde den Ländern des Westlichen Balkans die konkrete EU-Perspektive eröffnet. In der Folge legte die EU für die Region ein Instrumentarium auf, das gezielte EU-Unterstützung im Vorfeld einer möglichen EU-Mitgliedschaft beinhaltete. Die Staaten des Westbalkans werden von der EU als Beitrittskandidaten oder potenzielle Beitrittskandidaten \footnote{Der Statuswechsel von potenziellem Beitrittskandidaten zum Beitrittskandidaten wird vom Rat der EU auf Grundlage einer Stellungnahme der Europäischen Kommission einstimmig entschieden.} durch Programme der pre-accessioning assistance in erheblichem Maße finanziell unterstützt. Diese Instrumente sollen ihnen die Heranführung an Standards ermöglichen, die Voraussetzung für die Aufnahme in die EU sind. Die Unterstützung bezieht sich im Rahmen des Institutionenaufbaus explizit auch auf die Reform der öffentlichen Verwaltung.
\section{Untersuchungsgegenstand}
Für eine Aufnahme in die EU müssen alle zukünftigen Mitgliedstaaten bestimmte Bedingungen erfüllen. Dabei geht es im Wesentlichen um die Annäherung an europäische Standards im Bereich Wirtschaft, Institutionenaufbau und Rechtsstaatlichkeit. Für die erfolgreiche Umsetzung einer Annäherung an die EU muss der gesamte gemeinsame Besitzstand (Acquis communautaire) übernommen werden. Das heißt, die Länder müssen ca. 100.000 Seiten gemeinschaftliche gesetzliche Normen in ihr nationales Recht übernehmen.
Für einen so umfangreichen Prozess, die Übernahme der Europäischen Gesetzgebung in nationales Recht, ist eine effektive öffentliche Verwaltung notwendig.\par
Dennoch ist die Ausgestaltung der öffentlichen Verwaltung im Rahmen des Annäherungsmechanismus kein vorrangiges Thema im Erweiterungsprozess. In den jährlichen Fortschrittsberichten\footnote{Die Fortschrittsberichte der EU werden von der EU-Kommission jährlich im Spätherbst für jedes der Länder erstellt, die einen Beitrittsantrag zur EU gestellt haben. In diesen Berichten wird aus Sicht der EU der Fortschritt des jeweiligen Landes zu den Themen der Kapitel des Acquis communautaire beschrieben.} der EU wird die öffentliche Verwaltung zwar regelmäßig thematisiert, allerdings nur unter politischen Gesichtspunkten. Als wesentlicher Grund hierfür kann gelten, dass der öffentlichen Verwaltung kein eigenes Kapitel im Acquis communautaire zukommt.\par
Im Rahmen der von der EU konzipierten Heranführungshilfe wird (finanzielle) Förderung bereitgestellt, um die Kandidatenländer für die EU „fit“ zu machen. Im Rahmen dieser Heranführungshilfe ist auch die Förderung einer effektiven öffentlichen Verwaltung der Länder im Blick. Der Umfang der finanziellen Förderung für die öffentliche Verwaltung in den Beitrittskandidaten ist allerdings nicht zu vergleichen mit der Förderung von solchen Themen, die im Acquis klar geregelt sind.\par
Es handelt sich gewissermaßen um eine paradoxe Situation. Einerseits ist die öffentliche Verwaltung essenziell wichtig für die erfolgreiche Heranführung von Beitrittskandidaten, andererseits scheint das Thema nicht im Zentrum des EU-Interesses zu stehen.\par
In der vorliegenden Arbeit wird dieser Zusammenhang in den Blick genommen. Die Entwicklung der öffentlichen Verwaltung in drei ausgewählten Ländern des Westlichen Balkans wird von verschiedenen Perspektiven aus betrachtet. Hauptbezugspunkt ist dabei die EU-Perspektive der Westbalkanstaaten in ihrer Bedeutung für die öffentliche Verwaltung.\par
Die Länder des Westlichen Balkans befinden sich mit der EU quasi in einem Verhandlungsprozess. Den Westbalkanländern wird die Mitgliedschaft in der EU in Aussicht gestellt, sie müssen dafür aber bestimmte Bedingungen erfüllen (Konditionalität). Dies ist ein intensiver, wechselseitiger Prozess der Annäherung seitens der Länder, welche die Mitgliedschaft anstreben und seitens der EU, die deren Bemühungen beobachtet, unterstützt und bewertet. \par
Die vorliegende Arbeit untersucht das Verhältnis zwischen den Ländern des Westlichen Balkans und der EU unter dem Gesichtspunkt der öffentlichen Verwaltung. Wichtig ist in diesem Zusammenhang die Darstellung des Status quo der öffentlichen Verwaltung in den drei Untersuchungsländern. Neben der Entwicklung seit der Demokratisierung wird auch die historische Entwicklung der öffentlichen Verwaltung in diesen drei Ländern näher beleuchtet und gefragt, wieweit frühere Zusammenhänge und Entwicklungen in die Gegenwart hinein wirken.\par
Die Untersuchung erfolgt im Wesentlichen auf zwei Ebenen.
\begin{itemize}
\item Die Theorie und Praxis der EU-Erweiterung mit der öffentlichen Verwaltung als Bezugspunkt.
\item Die vergleichende Perspektive am Beispiel der Verwaltungsentwicklung in drei benachbarten Westbalkanländern.
\end{itemize}
Die Kriterien für demokratische und verantwortungsvolle Regierungsführung haben sich international in den beiden letzten Dekaden verändert. Nicht mehr nur formale Parameter wie freie und faire Wahlen, Gewaltenteilung oder Rechtsstaatlichkeit stehen im Fokus der Betrachtung. Zunehmend wird nach „substantive democracy“ gefragt und damit eher outcome-orientierte Kriterien wie Institutionenentwicklung und die Wirkung von „Good Governance“ herangezogen (vgl. \cite{pridham99}). Die Perspektive der europäischen Integration ist dabei ein bestimmendes Moment, denn stabile Regierungspraxis und funktionsfähige Verwaltungsstrukturen sind Voraussetzungen für den Beitritt zur EU und die Übernahme des gemeinschaftlichen Besitzstandes (vgl. \cite{kemmeu}: 49).
\section{Untersuchungsziel}
Ziel der Untersuchung ist eine verwaltungswissenschaftliche Perspektive auf ein bisher eher politikwissenschaftlich bearbeitetes Thema: Die Erweiterung der EU nach Südosteuropa. Eine dezidiert verwaltungswissenschaftliche Herangehensweise erscheint sinnvoll, angesichts des großen Stellenwertes, den die nationalen öffentlichen Verwaltungen der Beitrittsländer im Zuge des Erweiterungsprozesses haben. Die Umsetzung der EU-Aufnahmebedingungen ist in den Beitrittsländern vor allem von der öffentlichen Verwaltung zu leisten. Dennoch ist dieser Zusammenhang von der Forschung bislang wenig beachtet worden. Die vorliegende Arbeit geht erste Schritte zur Schließung dieser bemerkenswert großen Forschungslücke. Erkenntnisse aus der vorliegenden Arbeit könnten bei der Entwicklung von Grundlagen für weitergehende Forschung hilfreich sein.\par
Die Erfahrungen mit der letzten Welle von Ländern, die in die EU aufgenommen wurden, zeigen, dass Verwaltungsreform in ehemals kommunistischen Ländern nicht auf einer tabula rasa stattfindet; vielmehr wurde in vielen Ländern der letzten Erweiterungswelle an Verwaltungstraditionen angeknüpft, die bereits vor der kommunistischen Zeit bestanden. „Revitalization of common administrative roots was helpful for a smooth integration and normalisation of the post-communist state administrations“. (\cite{lipumb05}: 161) \par
Forschungen zur Transformation der vormals zentralistisch organisierten öffentlichen Verwaltungen der letzten EU-Aufnahmenwelle zeigen, dass hier besondere Problemlagen bei der Modernisierung und damit der „Europäisierung“ auftraten.\par
Unter Rückbezug auf die historische Verwaltungsentwicklung in ausgewählten Ländern des Westbalkans lautet die zentrale Frage der vorliegenden Untersuchung:\par
{\bf Welchen Einfluss hat die EU-Perspektive auf die Verwaltungsmodernisierung in den Ländern des Westlichen Balkans?}

Eine Reihe von weiterführenden Fragen schließt sich an:
\begin{itemize}
\item Sind Erfahrungen hinsichtlich der Entwicklung der öffentlichen Verwaltung in den Ländern der letzten Aufnahmewelle übertragbar auf den Westlichen Balkan?
\item  Liefert die historische Betrachtung der Verwaltungsentwicklung unter Einschluss früherer Regime der kommunistischen oder sozialistischen Zeit, aber auch der zeitlich davor gelagerten Einflüsse der Imperien verwertbare Erkenntnisse für den aktuellen Modernisierungsprozess?
\item Wie fördert die EU die Verwaltungsmodernisierung in den Beitrittsländern?
\item Wie schätzen Experten der EU und Akteure im Westbalkan die Verwaltungsentwicklung im Kontext der EU-Erweiterung ein?
\item Welche Optionen bestehen für die Verwaltungsentwicklung in den Westbalkanstaaten?
\end{itemize}
\section{Begründung der Länderauswahl}
In der vorliegenden Arbeit erfolgt eine Länderanalyse von drei benachbarten Staaten des Westbalkans – Mazedonien und Montenegro (Beitrittskandidaten) und Albanien (potenzieller Beitrittskandidat).\footnote{Bosnien-Herzegowina und Kosovo, beide immer noch mit starken internationalen Komponenten in der Administration, wurden von vorneherein ausgeschlossen.} Montenegro und Mazedonien sind Nachfolgestaaten des ehemaligen Jugoslawien. Albanien war von einem isolationistischen kommunistischen Regime geprägt.
\par
Gemeinsames Kennzeichen der Länder des Westbalkans ist die Existenz von großen nicht wettbewerbsfähigen Bürokratien, unterentwickelten Marktwirtschaften, unzureichenden Ressourcen und unzureichender demokratischer Staatlichkeit (Governance). Ineffektive Verwaltungen, in vielen Bereichen Nicht-Erfüllung öffentlicher Aufgaben und Korruption sind weiter bestehende Kennzeichen der öffentlichen Verwaltung in den Westbalkanstaaten. Vor diesem Hintergrund stellt die Reform der öffentlichen Verwaltung im Hinblick auf einen EU-Beitritt eine immense Herausforderung dar.
Eine Länderanalyse des Reformprozesses in drei benachbarten Auswahlländern (Albanien, Mazedonien und Montenegro) ermöglicht einen Vergleich und ggf. Abgrenzung der Problematiken in den untersuchten Ländern. Bei der länderspezifischen Analyse wird neben dem kommunistischen/sozialistischen Vermächtnis auch auf die zeitlich davor bestehenden Verwaltungstraditionen innerhalb des Osmanischen Reiches bzw. Österreich-Ungarns eingegangen. Diese historische Betrachtung ermöglicht Rückschlüsse auf das historische Vermächtnis der unterschiedlichen Verwaltungstraditionen auf den Status quo.\par
Die historisch-kulturelle Gemengelage in den Ländern des Westlichen Balkans wird im Folgenden prägnant charakterisiert: „The countries are both similar and different. They are similar in that all but Albania have a common heritage of modern history -- through their participation in Yugoslavia from 1918 and then their participation in the Socialist Federal Republic after 1945. This heritage has left many traces in law, institutions and exposure to administrative and economic concepts, as well as the involvement in the tragic wars that marked the end of the SFRY. But they are different because of their pre-1918 history. The boundary between the Ottoman and Austro-Hungarian empires ran through the region. The empires deeply marked the legal and administrative systems. The differences include their cultures and traditions, and they are divided amongst Muslim, Catholic and Orthodox peoples“ (\cite{oecd04}: 5).\par
Die Länder der letzten Beitrittswelle im Rahmen der Osterweiterung 2004 \footnote{Bulgarien und Rumänien wurden 2007 aufgenommen.} waren zum ersten Mal Staaten, die einen Systemwechsel von einem kommunistischen zu einem demokratischen Regime vollzogen hatten. Auch die Länder des Westbalkans erlebten einen Systemwechsel, wobei es sich bei dem Sozialismus Jugoslawiens um ein von der staatlichen Verfasstheit der Sowjetunion zu unterscheidendes System handelt. In Jugoslawien wurde eine spezifische Form des Selbstverwaltungssozialismus, in Abgrenzung zur sowjetischen Staatsorganisation umgesetzt. Für die öffentliche Verwaltung bedeutete diese staatliche Verfasstheit einige Ähnlichkeiten mit der kommunistischen Verwaltung, es sind aber auch entscheidende Unterschiede festzustellen. In Albanien hat sich ebenfalls ein von der sowjetischen Staatsorganisation abgegrenztes kommunistisches Regime herausgebildet, wiederum mit spezifischen Ausprägungen in der öffentlichen Verwaltung. Die drei untersuchten Länder unterscheiden sich insofern von den Ländern der letzten EU-Erweiterungswelle, welche die Verwaltungstradition der Sowjetunion übernommen hatten oder Teil der Sowjetunion waren.\par
Bei der detaillierten Betrachtung der historischen Verwaltungsentwicklung in den drei Untersuchungsländern wird deutlich, dass nicht pauschal von einer Verwaltungstradition des Westbalkans ausgegangen werden kann. Es bestehen Unterschiede in der Verwaltungsentwicklung, die zum Teil auf die unterschiedlichen historischen Einflüsse auf die betreffenden Länder zurückgeführt werden können. \par
Die umfassende historische Betrachtung der drei Untersuchungsländer liefert wertvolle Anhaltspunkte für die Beurteilung der zukünftigen Perspektive der EU-Erweiterung im Westbalkan.

\section{Stand der Forschung}
In der Literatur wurden zunächst die aktuellen theoretischen Ansätze gesichtet, die sich mit der Erweiterung der EU und dem Einfluss der EU auf die beigetretenen oder im Prozess des Beitritts befindlichen Staaten beschäftigen. Es handelt sich dabei im Wesentlichen um Untersuchungen, die als „Europäisierungsforschung“ bezeichnet werden können. Diese Ansätze entwickelten sich vor allem in der Politischen Wissenschaft innerhalb der Transitionsforschung. Die Europäisierungsforschung beschäftigt sich mit den Auswirkungen der Mitgliedschaft in der Europäischen Union auf nationale und internationale Prozesse. Im Zuge der Erweiterung der EU nahm die Europäisierungsforschung auch Prozesse im Umfeld der EU-Erweiterung in den Blick.\par
Studien zur Demokratisierung haben eine lange Tradition, während Studien zur Demokratisierung durch Regimewechsel, dem Gegenstand der Transitionsforschung, neueren Datums sind. In diesem Forschungsfeld nehmen Untersuchungen zu den Nachfolgestaaten der Sowjetunion eine große Rolle ein. Besonderes Merkmal dieser Transformation ist die gleichzeitige Neuausrichtung der Länder von einer zentral geplanten Ökonomie hin zu einer marktorientierten Wirtschaftsordnung (\cite{diaman}: 25). Für diejenigen Nachfolgestaaten der Sowjetunion, die inzwischen in die EU aufgenommen wurden, liegen eine Reihe von Studien der Transitionsforschung vor (\cite{dimit02,linden,grab05,kneuer07}). Für die Länder des Westbalkans, die ebenfalls einen Systemwechsel vollzogen haben, ist die Literaturlage dagegen sehr überschaubar. 
\par
Der Transitionsforschung liegen in der Regel zwei Analysekriterien zugrunde: einerseits kulturelle und andererseits politische. Im ersten Fall wird meist auf langfristige Entwicklungen und historische Bedingtheiten abgestellt, die Menschen einschränken. Im Gegensatz dazu wird im letzteren Ansatz das Element der Wahlmöglichkeit stärker betont und von Handlungsoptionen im Zusammenhang mit individuellen und kollektiven Akteuren ausgegangen. Hensell kritisiert die Transformationsforschung zu Osteuropa in ihrer generellen Reduzierung auf den politischen Strukturwandel. Er sieht die analytische Vernachlässigung des bürokratischen Staates in der Forschung als „blinden Fleck“. Untersuchungen zur Verwaltungspraxis wären wünschenswert, auch angesichts der Tatsache, dass der Staatsapparat, zumindest in Teilen, wechselnde Regime überdauert, seien sie dynastischer, kommunistischer oder demokratischer Art (vgl. \cite{hens09}: 27). Aber auch angesichts der Bedeutung der Verwaltungsstrukturen für die Funktionsfähigkeit demokratischer Regierungen ist erstaunlich, wie wenig dieser Aspekt bisher zum Thema politikwissenschaftlicher oder verwaltungswissenschaftlicher Untersuchungen gemacht wurde.
\par
Bereits Mitte der 1990er Jahre haben Linz und Stepan gefordert, einen systematischen Ansatz zur Staatsentwicklung in die Theorien der Transformation zu integrieren (vgl. \cite{linz} 366ff.). Jedoch nur wenige Arbeiten haben ausdrücklich die bürokratisch-staatliche Dimension der Transformation betrachtet. Ergebnis dieser Studien ist, dass sich die staatliche Verwaltung, im Gegensatz zu den politischen Institutionen in vielen Ländern nach dem Ende des Sozialismus, nur langsam wandelt. „Both communist and pre-communist structures and practices are still very much part of today’s bureaucratic order. Instead the overwhelming tendency has been one of structural conservatism. The ‘Big Bang’ of economic reforms has not extended to the administrative sphere, where change has been incremental, ad hoc, and, in the main, un-strategic” (\cite{dimgoe}: 225). 
\par
Ein funktionierender Justizapparat, Polizeiwesen, regelhafte Besteuerung, Fachbeamtentum, formal-rationale Verfahren, nach Max Weber Kernelemente moderner Staatlichkeit, werden in der Transformationsforschung meist vorausgesetzt (vgl. \cite{hens09}: 28).
\par
Es wird deutlich, dass die Erforschung des Verwaltungsumbaus im Zusammenhang mit dem Regimewechsel ein Schattendasein führt. Ebenso wurde das wichtige Thema der Verwaltungsentwicklung im Kontext der EU-Erweiterung bisher von der Forschung nur am Rande behandelt.
\par
Bei Sichtung der Literatur wurde weiterhin gefragt, inwieweit Untersuchungen zur Verwaltungsmodernisierung in den zuletzt in die EU eingetretenen Ländern Osteuropas, vorliegen. Dies geschah im Hinblick auf möglicherweise übertragbare Erkenntnisse auf die EU-Erweiterung nach Südosteuropa. Zu den Ländern Osteuropas, die 2004 bzw. 2007 der EU beitraten, liegen einige Untersuchungen vor, die sich auch mit der Verwaltungsentwicklung beschäftigen. Zu diesem Thema sind mehrere Studien erschienen, die sich vor allem mit der Ausgestaltung des civil service \footnote{In der vorliegenden Arbeit wird der Begriff „civil service“ verwendet. Diese Vorgehensweise erlaubt einen umfassenderen Bezugsrahmen als die deutschen Begriffe Beamtentum, Staatsdienst und öffentlicher Dienst, die jeweils Teilbereiche des civil service bezeichnen.} beschäftigen.
\par
Da der Vergleich dreier Länder des Westlichen Balkans ein zentrales Element der vorliegenden Arbeit ist, wurde die aktuelle Situation der Verwaltungsentwicklung in diesen drei Untersuchungsländern beschrieben. Die Literaturlage ist im Bereich der westlichen Sprachen überschaubar. Vor allem die SIGMA-Initiative der OECD und die Europäische Union haben Studien verfasst, die für die vorliegende Arbeit ausgewertet wurden. Auch einzelne Think Tanks in den jeweiligen Ländern befassen sich mit dem Thema Verwaltungsmodernisierung. Diese Studien wurden ebenfalls berücksichtigt.
\par
Um die möglichen Nachwirkungen der historischen Verwaltung auf die aktuelle Situation in jedem der drei Länder genauer betrachten zu können, wurde die Literatur zur Geschichte der betreffenden Länder gesichtet. Überraschendes Ergebnis dieser Sichtung ist, dass die Ausgestaltung der öffentlichen Verwaltung kaum Thema historischer Darstellungen ist. Allenfalls am Rande kommen die Ausprägungen und die Auswirkungen der Verwaltung in historischen Darstellungen vor. Für die Zeit der sozialistischen bzw. kommunistischen Verfasstheit waren in der Literatur einige Studien zu finden, die sich mit der Verwaltung beschäftigen. Die jeweils zweijährige österreichisch-ungarische Besatzungszeit im ersten Weltkrieg in Montenegro und Albanien ist über Archivmaterial im Österreichischen Staatsarchiv in Wien zugänglich und wurde von der Autorin im Hinblick auf die Beschreibung der historischen öffentlichen Verwaltung vor Ort ausgewertet. Weiterhin knüpft die vorliegende Arbeit an die Masterarbeit der Autorin an, die im Studiengang Master of Public Administration der Universität Kassel verfasst wurde (vgl. Vollmer 2007). Ergänzend zu dieser Masterarbeit wurde 2009 ein thematisch daran anknüpfender Beitrag in der Schriftenreihe „Moderne Verwaltungsentwicklung“ veröffentlicht (vgl. \cite{vollmer09}).
\par
Insgesamt wurde deutlich, dass die Betrachtung der historischen Verwaltung in den Ländern des Westlichen Balkans ein weitgehend unerforschtes Feld ist. Generell findet sich in der Literatur zur Geschichte der Untersuchungsländer in den meisten Fällen ein starker Fokus auf die politischen Entwicklungen, oft unter Einbezug des Einflusses externer Akteure. Die Verwaltungsentwicklung ist bislang kaum im Fokus der Forschung. Die Darstellung und Analyse des Verwaltungssystems im historischen Kontext in länderspezifischer, aber auch regionaler Perspektive könnte sich als lohnend erweisen für zukünftige Forschung. Die vorliegende Arbeit mit Berücksichtigung der historischen Perspektive in der Verwaltungsentwicklung liefert Ansätze in diese Richtung. Aufgrund der weitgehenden Abwesenheit von Literatur zur historischen Verwaltungsentwicklung in den Untersuchungsländern gleicht die Darstellung dieses Themenbereiches allerdings dem Zusammentragen einzelner „Puzzleteilchen“, die einen Eindruck vermitteln, jedoch kein umfassendes Bild abgeben.
\par
Verwaltungsentwicklung und Verwaltungsmodernisierung in den drei Untersuchungsländern werden aus unterschiedlichen Blickwinkeln betrachtet, um auf diese Weise die Zusammenhänge zwischen historischer Entwicklung, Entwicklung seit der Demokratisierung und der EU"=Perspektive soweit wie möglich auszuleuchten.

\subsection{Verwaltungsreform/Verwaltungsmodernisierung }
Zentral für die vorliegende Untersuchung ist das Konzept von „Governance“, bzw. „Good Governance“, auch „Gute Regierungsführung“. Good Governance kann als Weiterentwicklung der unterschiedlichen nationalen Verwaltungstraditionen im Zuge der zunehmenden Internationalisierung und Globalisierung verstanden werden.
\par
Die spezifische Ausrichtung der Modernisierung der öffentlichen Verwaltung wird im Wesentlichen bestimmt durch den jeweiligen historischen und kulturellen Kontext der Verwaltungsentwicklung. In Europa werden grundsätzlich vier verschiedene Verwaltungstraditionen unterschieden. Unterschieden wird in kontinentaleuropäisch-deutsche Tradition, kontinentaleuropäisch-französische (napoleonische) Tradition, angelsächsische Tradition und skandinavische Tradition als Mischung der angelsächsischen und deutschen Tradition (vgl. \cite{lipumb05}: 61). Hill fragt, ob nicht auch von einer spezifischen südeuropäischen Verwaltungstradition ausgegangen werden muss, und vor allem von einer mittel- und osteuropäischen Verwaltungstradition (vgl. \cite{hill06}: 15). Es wird also deutlich, dass es nicht „die” öffentliche Verwaltung gibt, die als Modell dienen kann \footnote{In den Institutionen der Europäischen Union finden sich Elemente aller dieser Traditionen.}.
\par
Die europäischen Verwaltungstraditionen sind im folgenden Schema im Überblick dargestellt:\footnote{Für eine detailliertere Beschreibung der Kennzeichen der vier Modelle von Verwaltungsstilen siehe Anhang }.
\begin{figure}[H]
  \caption{Europäische Verwaltungstraditionen im Überblick}
  \centering
  \includegraphics[width=5in]{Material/VerwaltungsModelle}\\
\scriptsize{Quelle: \cite{lipumb05}: 68}
\end{figure}

Aus dem Überblick wird erkennbar, dass es in der EU kein gemeinsames Modell gibt, nach dem eine öffentliche Verwaltung organisiert ist. Vielmehr gibt es nach dieser Darstellung vier Haupttraditionen, die eng mit der historischen Entwicklung in den einzelnen Ländern verbunden sind. Dass die Orientierung in eine bestimmte Richtung nicht statisch ist, zeigen die Beispiele Belgien und England, deren Administrationen zunächst dem französischen Modell angelehnt waren und nun eher dem deutschen Modell folgen, wie im obigen Schema ersichtlich. Weiterhin unterliegen die öffentlichen Verwaltungen Einflüssen, die aus einer generellen Internationalisierung resultieren. So hat das Konzept des „New Public Management“ in allen EU-Mitgliedstaaten zu einem Umdenken und vielfach auch Umbau der Verwaltungen geführt (vgl.\cite{dunhoo}).
\par
Die Modernisierung der öffentlichen Verwaltung ist weltweit in fast allen Ländern zum Thema geworden. In vielen Ländern auf allen Kontinenten sind Aktivitäten zu verzeichnen, deren Ziel eine rationale und effektive Verwaltungsführung ist. Damit verbunden ist meist die Hoffnung auf Kosteneinsparungen, die konsequente Trennung von Verwaltung und Politik, größere Bürgernähe der Verwaltung und/oder verbesserte Aufgabenerfüllung der öffentlichen Hand. Dabei sind mehrere Beweggründe für Verwaltungsmodernisierung zu beobachten:
\par
\begin{itemize}
\item In Ländern mit entwickelten öffentlichen Verwaltungen die Notwendigkeit, Ausgaben im öffentlichen Sektor einzusparen.
\item In Transformationsländern mit nicht existenten Verwaltungsstrukturen oder vormals Kommandowirtschaft die Notwendigkeit, Verwaltungsstrukturen nach modernen Kriterien aufzubauen.
\item Internationale Ausrichtung, z.B. EU-Mitgliedschaft.
\item Globalisierung: Wettbewerb der Länder untereinander bei der Schaffung optimaler Bedingungen für global operierende Unternehmen und Investoren.
\end{itemize}
Man kann zwei Phasen unterscheiden bei dem Austausch von Erfahrungen zwischen Ländern im Feld öffentlicher Verwaltung. Die erste Phase bestand ungefähr von den 1970er Jahren bis Ende des 20. Jahrhunderts. Diese Phase war geprägt von Informalität, Spontaneität und Freiwilligkeit. Kennzeichen war die gegenseitige Beeinflussung der administrativen Systeme, basierend auf Verwaltungsrecht und Gewohnheitsrecht, sowohl in Europa als auch in der Welt. Die zweite Phase des Austausches von Erfahrung und Wissen im Bereich öffentlicher Verwaltung ist seit Beginn des 21. Jahrhunderts auszumachen. Die Überzeugung, dass man die nationalstaatlichen öffentlichen Verwaltungen nicht nur sich selbst überlassen sollte, wurde in der Milleniumserklärung der Vereinten Nationen 2000 in den Blick genommen, die die enge Verzahnung des Kampfes gegen Armut und das Recht auf Entwicklung und „Good Governance“ definierte. 
\subsection{Das Konzept „Good Governance”}
In der vorliegenden Arbeit wird der Begriff „Reform der öffentlichen Verwaltung“ (Englisch: Public Administration Reform, PAR) in einem umfassenden Sinne verstanden; er umfasst den Beamtenapparat, seine rechtliche Verankerung, seine Funktionen, Kompetenzen und Verfahren, unter Einschluss der Verwaltung des Justizsystems. Über den klassischen Begriff der öffentlichen Verwaltung hinaus wird ein umfassenderes Konzept von ‘governance’ zugrunde gelegt. Dieses schließt die Kultur des Regierungshandelns im Sinne der nationalen Entscheidungen hinsichtlich der Ausgestaltung von Staatlichkeit (Legitimität, Effektivität, Transparenz, Pluralität und Verantwortlichkeit) ein, ebenso wie die Beziehungen zwischen Regierung und Parlament.\par
Der Begriff Governance, der zentral ist in der Betrachtung der Verwaltungsmodernisierung in den Ländern, die einen EU-Beitritt anstreben, wird von verschiedenen Institutionen unterschiedlich beschrieben.\par
Die Vereinten Nationen gehen von einer umfassenden Definition von „Governance“ aus, die die Definition einer modernen öffentlichen Verwaltung einschließt. „Governance“ im Gegensatz zum traditionellen Verständnis von „Öffentlicher Verwaltung“ legt einen Schwerpunkt auf Mitgestaltung und Partnerschaft. „Public administration needs to be transformed into a responsive instrument to meet the needs of all citizens“ (\cite{unpan}).\par
Das gewachsene Interesse an einer Governance-Orientierung der Vereinten Nationen wird auf folgende Entwicklungen zurückgeführt:
\begin{itemize}
\item den Erfolg der Marktwirtschaft und das Scheitern der Planwirtschaft;
\item die Tendenz, demokratische Regierungsformen mit wirtschaftlichem Erfolg gleichzusetzen;
\item die weltweite Krise der öffentlichen Finanzen, die in vielen Staaten die Frage nach der Rolle und der Effizienz des Staates neu gestellt hat; 
\item die gestiegene Wahrnehmung und Verärgerung über Korruption in Regierung und Verwaltung;
den Zusammenbruch der ehemaligen Sowjetunion und die ethnischen Konflikte auf dem Balkan und in Afrika, die die neuen Staaten vor große Umbauaufgaben auch ihres politischen Systems stellen (vgl. \cite{undp}: 18).
\end{itemize}
Die Weltbank beschreibt das Konzept als “the traditions and institutions by which authority in a country is exercised for the common good” (\cite{weltbank}: 1). Die sechs Governance-Indikatoren der Weltbank gelten mittlerweile als Standardkriterien zur Bewertung von Good Governance: 
\begin{itemize}
\item Politische Mitspracherechte (Voice and Accountability),
\item Politische Stabilität und Gewaltkontrolle (Political Stability and No Violence);
\item Effektivität des Regierens (Government Effectiveness),
\item Qualität regulativer Politik (Regulatory Quality),
\item Rechtsstaatlichkeit (Rule of Law),
\item Korruptionskontrolle (Control of Corruption) (vgl. \cite{kaufmann}).
\end{itemize}
Im Laufe der 1990er Jahre übernahmen auch die in der OECD zusammengeschlossenen Geber sowie die EU das Konzept der „guten Regierungsführung“.\par
Die SIGMA-Initiative der OECD versucht in ihrer Veröffentlichung zu ‘European Principles of Public Administration’ im Jahr 1999, ebenfalls an den Weltbankindikatoren angelehnt, Prinzipien für die öffentlichen Verwaltungen in ihren Mitgliedstaaten aufzustellen. Folgende Indikatoren werden genannt: 
\begin{itemize}
\item reliability and predictability (legal certainty or judicial security); 
\item openness and transparency; 
\item accountability; 
\item efficiency and effectiveness (vgl. \cite{oecd99}: 8ff).
\end{itemize}
An dem Good-Governance-Verständnis von Weltbank und IWF sowie OECD orientiert sich auch die Europäische Union. Die Europäische Kommission hat im Jahr 2001 ihr Weißbuch „Europäisches Regieren“ veröffentlicht, das ebenfalls Kriterien guter Regierungsführung enthält (vgl. \cite{czada2010}). Dort werden folgende Prinzipien als Merkmale von Good Governance definiert:
\begin{itemize}
\item Transparenz: Institutionen sollten in ihrem Handeln transparent sein und erklären, wie ihre Entscheidungen zustande kommen.
\item Partizipation: Die Qualität und Effektivität von Politik hängt wesentlich von umfassender Partizipation ab.
\item Übernahme von Verantwortung: Institutionen müssen erklären, warum sie etwas tun, und auch dafür Verantwortung übernehmen.
\item Effektivität: Policies müssen effektiv und zeitnah umgesetzt werden mit dem Ziel, das Benötigte auf der Basis von definierten Zielen zu liefern.
\item Kohärenz: Policies und Handlungen müssen kohärent und nachvollziehbar sein.
(vgl. \cite{euko01}: 13).
\end{itemize}
Gute Regierungsführung ist demnach im Wesentlichen gleichbedeutend mit einer leistungsfähigen, berechenbaren und transparenten staatlichen Verwaltung. Sie setzt „ein funktionierendes öffentliches Buchführungs- und Rechnungswesen ebenso voraus wie einen verbindlichen rechtlichen Rahmen, der privatwirtschaftlichen Wettbewerb ermöglicht“ (\cite{schmitz09}: 132).
\par
In der Debatte um die Notwendigkeit stabiler Institutionen und einer effektiven öffentlichen Verwaltung schwingt explizit oder implizit mit, dass die Modernisierung der öffentlichen Verwaltung die Demokratisierung vorantreibt. „Die Demokratie ist heute eigentlich keine Volksregierung, sondern eine Volksverwaltung – die Administration ist die eigentliche Aufgabe der Demokratie“ (Masaryk, zit. nach \cite{czerwick}: 14). Czerwick merkt dazu an, dass in der wissenschaftlichen Literatur davon ausgegangen wird, dass demokratische Systeme nur überleben können, wenn sichergestellt ist, dass die öffentlichen Verwaltungen ein Mindestmaß an struktureller Übereinstimmung mit demokratischen Normen, Institutionen und Prinzipien aufweisen.
\par
Vor diesem Hintergrund ist im Zusammenhang mit der (weiteren) Demokratisierung der Staaten des Westlichen Balkans und ihrem Wunsch einer Aufnahme in die Europäische Gemeinschaft die Frage nach dem Status quo der Verwaltung in diesen Ländern unumgänglich. Es wird deutlich, dass die öffentliche Verwaltung ein zentrales Element ist für die weitere Entwicklung, Demokratisierung und ultimativ den EU-Beitritt der Balkanstaaten. Dennoch ist dieser Zusammenhang von der Forschung bislang wenig beachtet worden. Die vorliegende Arbeit versucht erste Schritte, um diese bemerkenswert große Forschungslücke zu schließen. In Anbetracht der kaum vorhandenen Literatur wird das Thema Verwaltungsmodernisierung im Westbalkan im Kontext der EU-Erweiterung in der vorliegenden Arbeit von mehreren Seiten betrachtet. Die historische Perspektive fließt mit ein, in der Hoffnung auch aus dieser Betrachtung Hinweise zur Beantwortung der Forschungsfragen zu erhalten.
\section{Methodisches Konzept}
In der vorliegenden Arbeit wird eine Annäherung an ein bisher weitgehend nicht untersuchtes Thema, die Bedeutung der Reform der öffentlichen Verwaltung im Westlichen Balkan im Kontext der EU-Erweiterung, vorgenommen. Da mit dieser Untersuchung quasi Neuland betreten wird, wurde eine Methode gewählt, die es ermöglicht, das Thema von unterschiedlichen Seiten aus zu betrachten. Gewählt wurde ein kaleidoskopisches Verfahren, mittels dessen versucht wird, Antworten auf die zentrale Frage und die weiterführenden Fragen der Untersuchung finden. Die Vorgehensweise in der vorliegenden Arbeit ist im Rahmen des Forschungsansatzes der Triangulation verortet. Triangulation findet vor allem in der empirischen Sozialforschung Anwendung. Mit unterschiedlichen Methoden oder Sichtweisen, bzw. Daten wird versucht ein Phänomen zu erklären. Ursprünglich kommt der Begriff Triangulation aus der Landvermessung, wo er folgendermaßen verwendet wird:\par
„Triangulation is the method of location of a point from two others of known distance apart, given the angles of the triangle formed by three points. By repeated application of the principle, if a series of points form the apices of a chain or network of connected triangles of which the angles are measured, the lengths of all the unknown sides and the relative positions of the points may be computed when the length of one of the sides is known” (\cite{clark}: 145).\par
In der sozialwissenschaftlichen Forschung wurde die Triangulation als Methode entwickelt, um von verschiedenen Referenzpunkten aus die Position des (Forschungs-)Objektes zu lokalisieren (vgl. Smith 1975, zit. nach \cite{jick}: 136). Auf dieser Grundlage definiert Flick: „Triangulation beinhaltet die Einnahme verschiedener Perspektiven auf einen untersuchten Gegenstand oder allgemeiner: bei der Beantwortung von Forschungsfragen. Diese Perspektiven können sich in unterschiedlichen Methoden, die angewendet werden, und/oder unterschiedlichen gewählten theoretischen Zugängen konkretisieren, wobei beides wiederum miteinander in Zusammenhang steht bzw. verknüpft werden sollte. Weiterhin bezieht sie sich auf die Kombination unterschiedlicher Datensorten jeweils vor dem Hintergrund der auf die Daten jeweils eingenommenen theoretischen Perspektiven“ (\cite{flick08}:12).\par

In ähnlicher Weise resümiert Schirmer: „Triangulation meint – allgemein gesprochen – die Betrachtung eines ‚Punktes’ aus mehreren Perspektiven mit dem Ziel, diesen Punkt umfassender oder vollständiger zu verstehen, sozusagen ein kompletteres Bild zu entwerfen; damit sollen gleichzeitig Verzerrungen oder Fehlblicke vermieden oder relativiert werden, die Resultat einer bestimmten Perspektive sind.“ (\cite{schirmer}: 100). \par
Der Kern der Triangulation als Methode in der sozialwissenschaftlichen Forschung ist die Kontrastierung. In der vorliegenden Arbeit werden die Erkenntnisse aus der Literaturanalyse anhand von Interviews mit Experten überprüft und so eine Kontrastierung vollzogen, die möglicherweise zu neuen Erkenntnissen und weiterführenden Fragestellungen führt. Diese Vorgehensweise erscheint als eine gute Grundlage für weitergehende Forschung zu dem bisher wenig beleuchteten Thema dieser Arbeit.\par

In der praktischen Anwendung kann man zwei unterschiedliche Lesarten zu Triangulation feststellen. Im ersten Fall wird die Triangulation als Validierung von Forschungsergebnissen durch die Verwendung unterschiedlicher Methoden gesehen. Eine andere Lesart ist die Triangulation mit dem Ziel, ein umfassenderes Bild des Gegenstandsbereichs zu erzielen und den Untersuchungsgegenstand von unterschiedlichen Perspektiven her zu betrachten (vgl. \cite{kelle}: 50). Für die vorliegende Untersuchung wird die Triangulation im letzteren Sinne verwandt.
\par
Es wird versucht, über eine Betrachtung der historischen Verwaltungstradition und Verwaltungsentwicklung in drei benachbarten Ländern des Westlichen Balkans (Albanien, Mazedonien und Montenegro) Gemeinsamkeiten und Unterschiede aufzuspüren, die für den Status quo der Verwaltungsentwicklung in den Ländern relevant sein könnten.\par
Der aktuelle Bezugspunkt der Untersuchung und allen drei Untersuchungsländern gemein ist die Aufnahmeperspektive in die EU. Neben der Entwicklung der Beziehungen der Untersuchungsländer zur EU wird daher auch die Unterstützung der EU im Rahmen der Heranführungshilfe für die Aufnahmekandidaten beschrieben.\par
In einem weiteren Teil der Arbeit werden von der Verfasserin durchgeführte Experteninterviews zu den Forschungsfragen der Arbeit ausgewertet. Die Interviews betreffen thematisch den Status quo der Verwaltungsmodernisierung in den Untersuchungsländern und die Unterstützung durch die EU. Sechs Interviews wurden hierzu mit hochrangigen Beamten der EU und OECD / SIGMAs durchgeführt. Um die Perspektive in den Untersuchungsländern zu erfassen, wurden außerdem in jedem der drei Länder jeweils ein Vertreter der Regierung und ein Vertreter einer Nichtregierungsorganisation (NRO/NGO) interviewt. Alle Interviewpartner waren intensiv mit Verwaltungsmodernisierung befasst und sind in diesem Sinne Experten zum Thema. Insgesamt wurden also zwölf Interviews durchgeführt, die für die vorliegende Arbeit ausgewertet wurden.
\par
In Anbetracht der noch sehr überschaubaren Literatur zum Prozess der Verwaltungsmodernisierung in den Westbalkanländern unter dem Einfluss des EU-Erweiterungsprozesses stellt die Auswertung der Experteninterviews einen unverzichtbaren Beitrag dar. Dabei geht es nicht vorrangig um die Validierung von Ergebnissen der Literaturdurchsicht, sondern um eine Erweiterung der Forschungsperspektive unter Einschluss der persönlichen Erfahrungen und Meinungen von Experten, die sich mit dem Thema in der täglichen Berufspraxis beschäftigten.\par
In die vorliegende Untersuchung fließen ferner Eindrücke aus der eigenen beruflichen Praxis ein. Die Verfasserin hat mehrere Jahre für die OSZE in Südosteuropa im Bereich Demokratisierung vor allem in der Durchführung und Beobachtung von Wahlen gearbeitet (Kosovo noch unter internationaler Administration, Bosnien-Herzegowina, Albanien und Montenegro). Ein wiederkehrender Befund durchzieht diese jahrelange Beschäftigung. Einerseits handelt es sich bei dieser Region geografisch um Europa. Im gemeinsamen Land Jugoslawien bestanden über viele Jahre politisch und ökonomisch enge Kontakte mit den Ländern der EU und der EU als Institution. Andererseits erscheint die Region heute weit von Europa entfernt. So findet Berichterstattung in den westeuropäischen Medien kaum statt. Dieses mangelnde Interesse kann nur zum Teil mit den Kriegen Anfang der 90er Jahre erklärt werden, die zu ökonomischen und politischen Rückschritten führten.\par
Der Widerspruch zwischen einerseits geografischer Zugehörigkeit zu Europa und andererseits wahrnehmbar großer Entfernung zu Westeuropa spiegelt sich auch in der Herangehensweise der EU gegenüber der Region. Einerseits hat die EU den Staaten des Westbalkans seit 2003 einen Beitritt zur EU in Aussicht gestellt. Andererseits sind Auflagen seitens der EU formuliert worden, die in wesentlichen Punkten über die Anforderungen der früheren Aufnahmewellen hinausgehen. Auch wird die bisher gängige Praxis einer gemeinsamen EU-Aufnahme mehrerer Staaten für die Länder des Westbalkans ausgeschlossen.
\section{Aufbau der Arbeit }
Übergeordneter Bezugsrahmen für die vorliegende Arbeit ist zunächst die Konditionalität der EU, die derzeit den theoretischen Hauptstrang der Europäisierungsforschung darstellt. Das Konzept der Konditionalität wird im zweiten Kapitel aus der Demokratisierungs- und Transformationsforschung hergeleitet und bildet den ersten Rahmen der Untersuchung.
Ebenfalls im zweiten Kapitel wird der Prozess der EU-Erweiterung in der Praxis dargestellt. Dabei wird auch ausführlich auf den Stellenwert der Verwaltungsmodernisierung in diesem Prozess eingegangen. Hierbei ist zentral, dass Verwaltungsentwicklung im Erweiterungsprogramm der EU nicht explizit vorkommt. Eine implizite Bezugnahme ist dennoch erkennbar mit dem Konzept des Europäischen Verwaltungsraums, das ebenfalls dargestellt wird. Für dieses Kapitel werden Veröffentlichungen der EU und anderer internationaler Akteure (OECD/SIGMA, UNDP, Weltbank usw.) hinsichtlich des Stellenwertes von Verwaltungsentwicklung innerhalb des Erweiterungsprozesses ausgewertet. Ein wichtiger Teilaspekt der Betrachtung in diesem Kapitel ist die Auswertung der letzten Aufnahmewelle der EU hinsichtlich der Erfahrungen mit und der Nachhaltigkeit von Verwaltungsmodernisierung. Abschließend werden im zweiten Kapitel die Hilfen der EU für die Verwaltungsentwicklung in Kandidatenländern vorgestellt. \par
Die Ausgangslage auf dem Westbalkan ist Gegenstand des dritten Kapitels und bildet einen weiteren Rahmen für die Arbeit. Es wird auf die sozialistische (Jugoslawien) und kommunistische (Albanien) Verwaltungsgeschichte ebenso eingegangen wie auf die zeitlich davor gelagerten imperialen Einflüsse auf die Verwaltung (Österreich-Ungarn, Osmanisches Reich). Dieser Teil der Arbeit wird vorwiegend mittels einer Literatur-Auswertung durchgeführt. Auch die Sichtung von Akten der Österreichischen Militärverwaltung in Montenegro und Albanien während des Ersten Weltkrieges durch die Autorin im Staatsarchiv in Wien fließt in diesen Teil der Untersuchung ein. Als Bezugsrahmen dieses Teiles der Arbeit dient der Legacy-Ansatz, der auch für die vergleichende Verwaltungsforschung anwendbar ist. An die historische Betrachtung schließt sich die kursorische Darstellung der Verwaltungsentwicklung in den drei Untersuchungsländern in der Demokratie an.
\par
Der empirische Teil der Arbeit folgt im vierten Kapitel, bestehend aus zwölf Experteninterviews. Im ersten Teil dieses Kapitels wird die angewandte Methode für die Durchführung der Experteninterviews dargestellt, einschließlich des Vorgehens bei der Auswertung. Im Hauptteil des vierten Kapitels werden die Ergebnisse der Experteninterviews analog der behandelten Themenstränge dargestellt und analysiert.\par
Das Gesamtergebnis der Untersuchung wird im fünften Kapitel zusammengefasst dargestellt.
\chapter{Grundlagen der EU-Erweiterung }
Ob und in welchem Umfang die EU erweitert werden soll, ist eine politische Entscheidung. Auch die zeitlichen Abläufe für den Beitritt von Staaten zur EU werden durch politische Entscheidungen dominiert, wenngleich für diesen Aspekt binnenorganisatorische Abläufe informell ebenfalls von Bedeutung sein könnten. Die Gestaltung eines Beitrittsprozesses wirft viele Fragen auf, die teils fallspezifisch, teils allgemeiner Art sind. In Betracht kommen politische, ökonomische, soziale und administrative Probleme des gewünschten oder notwendigen Wandels. \par
Prozesse des Wandels sind sowohl Gegenstand verschiedener theoretischer Überlegungen als auch eine Gelegenheit, entsprechende Praxiserfahrungen zu sammeln. Da die EU seit ihrer Gründung bereits mehrfach erweitert wurde, liegt schon ein umfangreiches Praxiswissen zu Beitrittsprozessen zur EU vor; von diesem ist allerdings nicht genau bekannt, in welchem Umfang es fallspezifisch oder wieweit es übertragbar ist.
\section{Forschungsansätze zum politischen Wandel in Europa}
Prozesse des politischen Wandels sind weltweit mit unterschiedlicher Intensität und aufgrund verschiedenartiger Impulse unter dem Einfluss unterschiedlicher Rahmenbedingungen zu beobachten. Schwerpunkte sind die Umwandlung traditionell regierter Länder zu modernen Staaten, wie insbesondere im Rahmen der westlichen Entwicklungshilfe, und aktuell die Umwälzungen in Nordafrika, sowie die Transformation vormals sozialistisch beherrschter Länder in demokratisch regierte Staaten in Osteuropa. Die Aufnahme von Staaten in die Europäische Union stellt für diese ebenfalls einen grundlegenden Wandel dar, zu dessen Beschreibung und Gestaltung allgemeine Forschungsergebnisse aus den verschiedenen Prozessen des politischen Wandels herangezogen werden können. Die Forschungsrichtung der Europäisierungsforschung und hier insbesondere das Konzept der Konditionalität ist der in der Europäisierungsforschung vorherrschende Ansatz zur Erklärung verschiedenster Prozesse, welche die EU als supranationale Organisation entfaltet. Die Europäisierungsforschung wird als zentraler Ansatz ausführlich dargestellt, insbesondere die in ihrem Umfeld entwickelte Konditionalitätsforschung. Diese Forschungsrichtung ist von besonderer Bedeutung für die vorliegende Arbeit, da die Verwaltungsmodernisierung ein wesentliches Element der EU-Bedingungen für die Aufnahme der Westbalkanstaaten ist. Im Folgenden werden diese Forschungsrichtungen aus der Transitionsforschung hergeleitet mit den konkreten Fragestellungen der Europäisierungsforschung und der Konditionalitätsforschung. 
\subsection{Transitionsforschung}
Die Transitionsforschung, die sich traditionell mit Entwicklungsländern beschäftigte, erfuhr durch die politische Wende in Osteuropa eine neue Ausrichtung. Nach dem Systemwechsel in den Ländern Mittel- und Osteuropas war die wissenschaftliche Aufarbeitung der Geschehnisse zunächst von wirtschaftswissenschaftlichen Konzeptionen beherrscht, die sich vor allem mit dem Wechsel der Wirtschaftsweise von zentraler Planwirtschaft zu Marktwirtschaft befassten. Die auch stattfindenden politischen Transformationsprozesse wurden in der Folge ebenfalls nach und nach mit Erklärungsmodellen begleitet. Es entstanden Staatenanalysen und Analysen spezifischer Systembereiche mit ökonomischem, demokratietheoretischem oder soziologischem Schwerpunkt (vgl. \cite{huszak} : 54ff).\par
König definiert den Prozess der Transition in den neu entstandenen Ländern Osteuropas eher als einen der Transformation. „It is evident that the transition from command to market economy and from totalitarian state to a pluralist state, multiparty democracy is not only a transition in itself but rather a long process of transformation and it requires essential reforms in the basic functions and institutions of the state” (vgl. \cite{koenig}). \par
Der Einfluss der EU als einer der wesentlichen Geber kam zunehmend in den Blick als externer Akteur der Transformation. Von Beyme konstatiert in diesem Zusammenhang, dass der „internationale Einfluss der etablierten Demokratien auf die neuen Systeme (…) eine neue Dimension in der Weltgeschichte“ darstellt (\cite{beyme} : 158). Whitehead geht von drei Formen der Demokratisierung aus, erstens der auferlegten Demokratisierung, zweitens Demokratisierung durch Dekolonisierung und drittens Demokratisierung durch Konvergenz (vgl. \cite{whitehead}). Pridham entwickelt ein Konzept der interaktiven Prozesse zwischen externen (vor allem internationalen Organisationen) und innerstaatlichen Akteuren (vgl. \cite{pridham91,pridham95,pridham08}), während das Konzept der Diffusion bzw. einer Art „Schneeballeffekt“ bei der Demokratisierung von Huntington stammt (vgl. \cite{hunting}). Eine Aufarbeitung des Systemwandels in Osteuropa unter Betrachtung externer Faktoren findet statt; diese werden allerdings noch nicht in einem ausreichenden Maße in theoretische Erklärungszusammenhänge eingebunden: „…even though the influence of international factors has been widely acknowledged, these still have not been fully integrated into theoretical frameworks aiming to explain the dynamics or failure of post communist transitions“ (\cite{dimpri} : 93).\par
Im Rahmen der Transitionsforschung wird der Erkenntnis, dass der Beitritt zur EU spezifische Transformationsergebnisse zeitigt, zunehmend Raum gegeben. Forschung zum Institutionenwandel zentralstaatlicher Administration in nachkommunistischen Ländern im Rahmen der Transformations- und Integrationsforschung ist dagegen noch eher unterentwickelt. Nur selten wird auf die Entwicklung der Ministerialbürokratien und Regierungen in einem engeren Sinne Bezug genommen. Insbesondere Probleme mit der administrativen Kapazität der neuen Mitgliedsländer werden, so Lippert und Umbach, lediglich auf allgemeine Weise abgehandelt. “Therefore, the cross-country research on the administrative developments under the pressure of Europeanisation is particularly relevant” (\cite{lipumb05} : 17). Auch Luchterhand konstatiert schon 2001 im Vorwort seiner Analyse zu Verwaltung und Verwaltungsrecht im Erneuerungsprozess Osteuropas: „Dass die tatsächliche Erfüllung der Beitrittsvoraussetzungen – unterhalb einer demokratischen und menschenrechtskonformen Verfassung – nicht nur von einem EU-kompatiblen Wirtschaftssystem abhängt, sondern kaum weniger von einer leistungsstarken und rechtsstaatlich fundierten öffentlichen Verwaltung, hat man daneben weithin kaum zur Kenntnis genommen“ (\cite{lucht} : 6).\par
Die Transitionsforschung mit ihrer Untersuchung des politischen Wandels ist für die vorliegende Arbeit besonders fruchtbar. Im Zuge der EU-Erweiterung kam auch die EU als externer Akteur mit Einfluss auf Beitrittsländer in den Blick der Transitionsforschung. Die Betrachtung von Institutionenwandel als erklärtem Ziel dieser Forschungsrichtung bietet sich somit auch für die Betrachtung der Verwaltungsentwicklung an. 
\subsection{Neo-Institutionalismus als Ansatz zur Erklärung des Wandels}
Der Neo-Institutionalismus, der von der Bedeutung der Institutionen für soziales Handeln ausgeht, kann als Gegenbewegung zu dem in den USA seit den 1960er Jahren dominanten „Behaviouralismus“ betrachtet werden. Dieser versuchte politische Phänomene vor allem über individuelle Einstellungen und individuelles Verhalten zu erklären. Ausgangspunkt der Kritik an einem rein verhaltenswissenschaftlichen Erklärungskonzept war die fehlende Erfassung der wachsenden Bedeutung von Institutionen. Der Zusammenbruch der Staaten in Ost- und Mitteleuropa und die damit einsetzende Erforschung der Transformationsprozesse brachte die Institutionen erneut in den Blick. Die zunächst vorherrschende Einschätzung, dass es sich in diesen Ländern im Wesentlichen um „nachholende Modernisierung“ handelt, wurde angesichts wirtschaftlicher Probleme und ethnischer Auseinandersetzungen zunehmend fragwürdig. Die Sichtweise verschob sich zunehmend hin zur Annahme, dass die Wandlungsprozesse nicht in logischer Folge ablaufen, sondern dass man von Prozessen ausgehen muss, die geprägt sind von Verteilungskämpfen und traditionellen institutionellen Einflüssen. So „finden sich in den Gesellschaften Ost- und Mitteleuropas zahlreiche so genannte institutionelle Hinterlassenschaften (institutional legacies), d.h. Routinen, Regeln und soziale Bindungen1, die den Verlauf der Transformation maßgeblich beeinflussen“ (vgl. \cite{schulze} : 5). Es wird also nach der Veränderung des institutionellen Gefüges durch Anpassungsprozesse unterschiedlichster Art und Geschwindigkeit gefragt. \par
Im „neuen“ Institutionalismus in der Politikwissenschaft, der seit den 1970er Jahren verstärkt zum Einsatz kommt, unterscheidet man im Wesentlichen drei Varianten.\par
Erstens gibt es eine stark vom Rational-Choice-Ansatz bestimmte Richtung, die sich mit der Wirkung politischer Institutionen in den verfassungsmäßigen Entscheidungsgremien befasst.
In Rational-Choice-Ansätzen ist das politisch-soziale System Untersuchungsgegenstand. Grundlage für die Analyse ist das Konzept des methodologischen Individualismus, wonach Entscheidungen immer nur von weitgehend rational handelnden Individuen getroffen werden können und somit Handlungen von Kollektiven (z.B. Behörden) eine Anhäufung von Einzelfallentscheidungen seien. Dabei werden strukturelle Faktoren weitgehend ausgeblendet bei vorwiegender Berücksichtigung der angenommenen Interessen der beteiligten Akteure. Putnam entwickelt für den Blick auf die europäische Integration ein Zwei-Ebenen-Modell. Er geht davon aus, dass bei Verhandlungen über internationale Kooperationen gesellschaftliche Akteure auf nationaler Ebene Druck auf die Regierung ausüben, um ihre Ziele zu realisieren. Gleichzeitig werden von den nationalen Regierungen die Verhandlungen auf internationaler Ebene genutzt, um den Erwartungen der einheimischen Akteure nachzukommen bzw. zu entkommen (vgl. \cite{putnam}). Kritik an den akteursorientierten Ansätzen stellt vor allem auf die mangelnde Berücksichtigung struktureller und institutioneller Rahmenbedingungen ab.\par
Zweitens gibt es eine kulturalistisch-konstruktivistische Variante, die sich vom Rational-Choice-Modell abgrenzt. Hierfür steht der Ansatz von March und Olsen. Diese definieren Institutionen als ein Gefüge aus Regeln und Verhaltensroutinen, die durch soziale Werte und Normen bedingt sind und so die Akteursreaktionen beeinflussen. Es kann daher kaum zu identisch ausgeprägten Institutionen kommen bei unterschiedlichen kulturellen und sozialen Rahmenbedingungen (vgl. \cite{marols} : 17).\par
Drittens gibt es die vermittelnde Variante des sogenannten historischen Institutionalismus. (\cite{haltay, steinmo}). Die Wirkung von Institutionen wird historisch sowie national und sektoral vergleichend untersucht. Der historisch-soziologische Ansatz entspringt der vergleichenden Regierungslehre und stellt den Staat als zentralen Akteur mit seinen Machtpotenzialen in den Mittelpunkt.\par
Im Zentrum der Untersuchungen im Rahmen des Neo-Institutionalismus stehen die Beweggründe für institutionelle Änderungen und die Frage, wie die neuen Spielregeln nach Überwindung der alten Regeln und Handlungsmuster verfestigt und angenommen werden. Vor allem die Mechanismen des Wandels sind im Blick sowie der Einfluss des veränderten Umfelds auf die Politikgestaltung (vgl. \cite{huszak}).\par
Kennzeichnend für die aktuellen Forschungen nach den neo-institutionalistischen Konzepten ist die Konzentration auf die Bedeutung der Institutionen bei der Betrachtung gesellschaftlichen Wandels. Dies insbesondere in Abgrenzung zu verhaltenswissenschaftlichen Erklärungsansätzen, die in den Sozialwissenschaften in der USA seit den 1960er Jahren vorherrschend waren. Die Europäisierungsforschung kann als eine Weiterentwicklung der neo-institutionellen Theorien gesehen werden. \par
Auf Basis der bisher dargestellten Ansätze zum politischen Wandel wird zunächst die Europäisierungsforschung näher beleuchtet; diese stellt eine weitere Konkretisierung der Transitionsforschung im Zusammenhang mit der EU-Erweiterung dar. In einem weiteren Schritt wird auf eine Unterkategorie der Europäisierungsforschung, die Konditionalitätsforschung, eingegangen. Forschungen zur Konditionalität haben insbesondere in Bezug auf die politischen Kriterien im Erweiterungsprozess Relevanz. Die Verwaltungsentwicklung in den Beitrittsländern wird von der EU unter politischen Kriterien betrachtet. 
\subsection{Europäisierungsforschung}
Die Rolle der EG/EU im Zusammenhang mit Demokratisierung war im Prinzip vor 1989 nicht im Blick und auch anlässlich der Süderweiterung (überraschenderweise) nicht beleuchtet worden (vgl. Kneuer 2007). Bezogen auf Mittel- und Osteuropa beschäftigte sich die Europaforschung vor allem mit den technischen Aspekten der Assoziierung (Europaabkommen) im Rahmen des Heranführungs- und Beitrittsprozesses (\cite{lipbec, lipsch}). Weiterhin wurden die sich entwickelnden Beziehungen zwischen der EU und den Beitrittsländern thematisiert (\cite{mayhew, torre}). Die klassische Integrationsforschung beleuchtete vor allem, ob und wie die Mitgliedstaaten auf die Entwicklung supranationaler Institutionen und Politiken einwirkten. Ein Perspektivwechsel seit Mitte der 1990er Jahre führte zur zunehmenden Beschäftigung mit der Frage nach dem Einfluss der EU und den Effekten auf nationale Systeme (vgl. \cite{kneuer09} : 21). Damit war der Paradigmenwechsel vollzogen und die nun Europäisierungsforschung genannte Betrachtungsweise basierte auf der These, dass die EU unterschiedliche Effekte in den Mitgliedstaaten hervorrufen kann. Die einsetzende Theoriebildung versuchte den Einfluss und die Wirkung der EU auf die Mitgliedstaaten und die dort ablaufenden Prozesse, Politikinhalte, Einstellungen und Normen zu beschreiben (\cite{boerzel,boeris00, radaelli00, kohler, fearad03}). Es wurde der Frage nachgegangen, ob die EU zu policy-Veränderungen führt, zur Transformation von Institutionen, oder sogar zu Identitätsveränderungen (\cite{meny, knilen,fearad03,boeris07}). \par


Im Allgemeinen wird unter Europäisierung das Zusammenwirken der folgenden drei Zusammenhänge verstanden: 
\begin{itemize}
\item Die Herausbildung und Entwicklung spezifischer Strukturen von governance auf europäischer Ebene (vgl. \cite{risseetal} : 3), (vgl. \cite{radpas} : 36).
\item Europäisierung als „top-down“ Prozess, der durch Institutionen und Entscheidungen auf der Ebene der EU die nationalen policies und Institutionen formt (vgl. \cite{herit}).
\item Ein Prozess mit folgenden Schritten: a) Konstruktion, b) Diffusion und c) Institutionalisierung von Normen, Glaubenssätzen und informellen Regeln, Abläufen, policy Paradigmen, Stilen und „der Art wie Dinge getan werden“. Diese sind zunächst durch den EU-policy-Prozess definiert und werden dann auf die nationale Ebene übertragen und in die öffentlichen Debatten, politischen Vorgaben und Institutionen übernommen. Diese letzte Beschreibung basiert auf der Annahme der Europäisierung als Institutionalisierung und interaktivem Prozess, der über einen ein-direktionalen Mechanismus als Reaktion auf Europa und auch über das Konzept des „impact“ oder Einflusses der EU auf nationale Systeme hinausgeht. Damit ist keine vertikale Anpassung gemeint, sondern ein Sozialisierungsprozess im umfassenden Sinne (vgl. \cite{fearad03,Featherstone/Radaelli 2003, olsen}).
\end{itemize}
Der letztgenannte Ansatz betrachtet unter verschiedenen Blickwinkeln und in einer diskursiven Herangehensweise, wie nationale Veränderungen geschehen. Zwar kann man mit definierten Kriterien den Grad der Europäisierung messen oder doch beschreiben, allerdings ist ein besonderes Problem immer die Abgrenzung von Anpassung und Transformation (vgl. \cite{radpas} : 40).\par
Die klassischen Probleme von Forschung zur Europäisierung sind a) Voreingenommenheit bei der Beurteilung des Einflusses der EU auf die nationalen Policies und Politik und b) die Annahme, dass es sich bei nationalen Veränderungen, die den Brüsseler Vorschlägen ähnlich sind, um Europäisierung handelt (vgl. \cite{radpas} : 40). Oder wie Goetz warnt: „Europeanization can very easily become a cause in the search of an effect (at the domestic level)” (\cite{goetz01a} : 211). Noch kritischer wird von Mair angemerkt: „Europeanization and globalization are becoming catch-all, default explananda for almost everything that cannot otherwise be explained at the domestic level” (\cite{mair} : 339).\par
Die Konzepte der Europäisierung wurden zunächst fast ausschließlich auf Mitglieder der EU angewandt. Erst seit der letzten Erweiterungswelle gibt es Studien, die sich im Rahmen der Europäisierungsforschung auch mit Regionen außerhalb der Grenzen der EU beschäftigen (\cite{lipumwes, grab03, papadi, lavenex, schsed05b,schsed05c}). Die aktuellen Ansätze der Europäisierungsforschung sind für empirische Studien vor allem in drei Richtungen nutzbar gemacht worden: Europäisierung als Policy-Veränderung, Europäisierung als Institutionenveränderung sowie Europäisierung und EU-Erweiterung. Diese drei Stränge der Europäisierungsforschung werden im Folgenden kurz dargestellt.
\subsubsection{Europäisierung als Policy-Veränderung}
Eine Reihe von empirischen Studien wurde zur Europäisierung der politischen Institutionen und Entscheidungsprozesse einzelner Staaten oder als Ländervergleiche durchgeführt. Vor allem Frankreich, Deutschland und Großbritannien dienten dabei als Untersuchungsländer. Weiterhin gibt es Untersuchungen zu einzelnen policy-Feldern. Hier wird oft nach der nationalen Umsetzung der EU-Vorgaben in den Mitgliedstaaten gefragt. Studien in dieser Kategorie beschäftigen sich bislang vor allem mit Umweltpolitik, Sozial- oder Regionalpolitik, seltener mit Landwirtschafts-, Gesundheits- Wettbewerbs- oder Kulturpolitik. Die Themen Außen- und Sicherheitspolitik sowie Justiz- und Innenpolitik waren bislang selten im Forschungsinteresse, mit Ausnahmen zur Immigrations- und Asylpolitik (vgl. \cite{bulmer07} : 57). Die vorgelegten Studien zeigen, dass der Einfluss der EU im Bereich der Umweltpolitik und der Sozialpolitik zu höheren Standards in den Mitgliedsländern geführt hat, wobei die südlichen Mitgliedstaaten stärker von Veränderung betroffen waren. Auch wurden neue Instrumente der Politikgestaltung übernommen, die z.B. auf einer stärkeren Einbeziehung von verschiedenen sozialen Gruppen basieren oder auf politikfeldübergreifender Kooperation. In Bereichen, in denen die EU großen Eingriffseinfluss hat, ist es dennoch nicht zu einem einheitlichen Politikstil gekommen (vgl. \cite{boeris07} : 486).
\subsubsection{Europäisierung als Institutionenveränderung }
Studien in diesem Bereich haben sich mit Fragen beschäftigt, inwieweit europäische Prozesse sich auf die Beziehungen zwischen Regierungen, nationalen Bürokratien und administrativen Prozessen, Regulierungsstrategien, Justizstrukturen oder die Beziehungen zwischen Legislative und Exekutive auswirken. Diese Studien kommen zu keinem eindeutigen Ergebnis. Manche Studien fanden heraus, dass nationale Institutionen dem europäischen Einfluss im Wesentlichen standgehalten haben, während andere Studien davon ausgehen, dass die EU die nationalen Systeme föderalisiert oder pluralisiert habe. Börzel und Risse sehen in diesen Ergebnissen die Kontroverse gespiegelt, ob die EU-Integration den Staat stärkt, schwächt oder transformiert. Für nationale Verwaltungen konstatieren sie, dass diese die Anforderungen der EU erfüllt haben, aber die konkrete Umsetzung unterschiedlich ausfällt und maßgeblich von den schon existierenden Institutionen abhängt. „National administrations have responded to the ‚demands of EU membership’ but institutional adaptation differs significantly and is mediated by pre-existing institutions“ (\cite{boeris07} : 487).
\subsubsection{Europäisierung und EU-Erweiterung }
Die EU-Erweiterung nach Osteuropa bot eine gute Möglichkeit, die Hypothesen der Europäisierungsforschung zu testen. Die Länder Ost- und Mitteleuropas, insbesondere die post-kommunistischen Länder, hatten eine andere historische Einbindung als westeuropäische Demokratien und sie hatten wenig Möglichkeit, selbst Einfluss auf die EU-Politik auszuüben. Diese Länder hatten den Acquis zu übernehmen und standen damit unter einem erheblichen Anpassungsdruck. Damit verbunden war die Vermutung, dass sie die EU-Modelle stärker internalisieren aufgrund der Schnelligkeit, mit der sie EU-Vorgaben übernehmen mussten angesichts des großen Umfangs der zu übernehmenden EU-Agenda und dank der größeren Offenheit für EU-Modelle im Rahmen des post-kommunistischen Transformationsprozesses (vgl. \cite{grab03}). Doch die Studien zeichnen kein eindeutiges Bild. Die meisten Forscher stimmen darin überein, dass die EU-Erweiterung den Hauptstimulus darstellte und die Übernahme des Acquis communautaire ja auch die Aufnahmebedingung war. Dies bedeutete auch, dass Europäisierung in diesem Zusammenhang eher ein top-down Prozess und eine „Einbahnstraße“ war. Zwar zeigte sich, dass die wesentlichen Verwaltungseinheiten gestärkt wurden, die Entwicklung eines nicht-politisierten civil service begünstigt wurde und ein gewisser Grad an Dezentralisierung erreicht wurde, zumindest im Gegensatz zur kommunistischen Zeit. Dennoch variieren die Auswirkungen auf Institutionen und Politik erheblich. Schimmelfenning zeigt, dass die politische Konditionalität der EU nur in instabilen Demokratien zur Festigung von liberalen Politiken führte. In Ländern mit starken Demokratien oder autokratisch regierten Ländern wirkte die politische Konditionalität nicht. Darüber hinaus könnte die starke top-down Orientierung zu einer oberflächlichen Europäisierung führen oder gar zu Revisionen, da die Veränderungen sehr schnell vollzogen wurden und wenig Spielraum zum sozialen Lernen boten (vgl. \cite{boeris07} : 490).\par
Zusammenfassend kann gesagt werden, dass die Europäisierungsforschung im Rahmen der der Transformationsforschung entstanden ist. Zunächst wurde vor allem der Einfluss der EU auf nationale Politik und Institutionen in den Mitgliedsländern untersucht. In der praktischen Anwendung auf unterschiedliche Politikfelder stellte sich heraus, dass vor allem im Bereich der Umwelt- und Sozialpolitik durch Anforderungen der EU insgesamt höhere Standards zur Durchsetzung kamen.
\par
In der vergleichenden Betrachtung zur institutionellen Veränderung durch EU-Politik zeichnen entsprechende Studien kein eindeutiges Bild. In einigen Fällen wird ein Standhalten der Institutionen gegenüber EU-Einflüssen konstatiert, während andere Untersuchungen von einer Pluralisierung der nationalen Systeme ausgehen.\par
In Bezug auf die EU-Erweiterung kommt die Europäisierungsforschung ebenfalls zu unterschiedlichen Einschätzungen. Allerdings geht die Mehrheit der Untersuchungen von einem “top down“ Prozess aus, der mit der Übernahme des Acquis communautaire als Beitrittsbedingung verbunden ist. Nach diesen Studien ist die Übernahme europäischer Standards nur vordergründig erfolgt, um den Anforderungen für eine EU-Mitgliedschaft zu genügen. Eine wesentliche Institutionenveränderung hätte nach dieser Sichtweise nicht stattgefunden. Dies vor allem, weil unter Zeitdruck ein umfassendes soziales Lernen als Voraussetzung zu umfassender Veränderungen der nationalen Institutionen nicht stattgefunden hat.
\subsection{Konditionalität als Konzept}
Das Konzept der politischen Konditionalität kommt aus der Entwicklungszusammenarbeit als ein Instrument bei der Durchsetzung von Reformen, die explizit oder implizit auf Demokratisierung abzielen. Dabei werden generell positive und negative Konditionalität unterschieden. Positive Konditionalität macht die Mittelvergabe von der Implementierung von Reformmaßnahmen abhängig, während negative Konditionalität die Kürzung oder Einstellung der Unterstützungsleistungen bedeutet, wenn die Empfängerseite vereinbarte Auflagen nicht eingehalten hat (vgl. \cite{schmitz09} : 127). Bis in die 1990er Jahre waren Zuwendungen der internationalen Finanzinstitutionen meist mit Strukturanpassungsmaßnahmen verbunden, die von den Empfängerländern durchzuführen waren. Untersuchungen zur Wirksamkeit solcher Programme kamen generell zu dem Ergebnis, dass die Wirksamkeit der ökonomischen Konditionalität der Strukturanpassungsprogramme oft nicht nachweisbar ist oder bestehende Probleme noch verschärft (\cite{killick, morrissey}). 
\par
Eine andere Richtung schlagen die Konzepte „policy transfer“ und „lessons learning“ vor. Diese entstammen dem Forschungsfeld der Vergleichenden Politikwissenschaft, das vor allem in den 1990er Jahren neue Impulse entwickelte. Gefragt wird hier, wie nationale Politik durch das Lernen von erfolgreichen Beispielen anderer Länder verbessert werden kann. Der Bertelsmann-Index und der Governance-Index der Weltbank stehen in dieser Tradition. Die neue Denkrichtung geht von einer „demokratisierten“ Konditionalität aus, die als wechselseitiger Prozess verstanden wird, in dessen Verlauf sich Geber und Empfänger auf gemeinsame Ziele verständigen unter Einbezug von Dialog und Monitoring.
\subsubsection{Konditionalitätsforschung im Rahmen der Europäisierungsforschung }
Der zunehmende Gebrauch der Konditionalität seitens der EU in den späten 1990er und frühen 2000er Jahren ging einher mit einer Expansion der Forschung zum Einfluss der Konditionalität auf unterschiedliche Länder, Politikfelder und institutionelle Gegebenheiten (\cite{grab99,grab01,grab03, schsed04, schsed05b,schsed05c,vachudova01, vachudova05}). Es sind einige vergleichende Studien entstanden zu den Demokratisierungseffekten der EU. Diese Studien kommen zu einer Reihe von übereinstimmenden Erkenntnissen hinsichtlich der Effektivität der EU als Demokratie-Förderer. Es wird davon ausgegangen, dass die Anwendung von Konditionalität wesentliche Erfolgsvoraussetzung ist. Dabei ist zunächst politische Konditionalität zu nennen (\cite{kelley, kubicek, pridham05,schetal, vachudova05, youngs}). Die als wahrscheinlich angenommene Aufnahme in die EU bei erfolgreichen demokratischen Reformen wird als das effektivste Element der EU-Strategien eingeschätzt. Weiterhin stimmen die Studien darin überein, dass außerhalb von Europa, d.h. ohne Mitgliedsperspektive, die politische Konditionalität mit ihrer Demokratieförderung weniger erfolgreich ist. Grundsätzlich kommen die Studien zu dem Ergebnis, dass sogar in einer Situation, wo die Mitgliedsperspektive sehr glaubhaft ist, weitere Faktoren hinzukommen müssen. Förderliche politische Umstände in den Zielländern sind dabei wesentlich, um einen positiven Demokratisierungseffekt zu erreichen (vgl. \cite{schsch07} : 273).\par
Inzwischen liegen auch einige empirische Untersuchungen vor zu den Auswirkungen des EU-Beitritts in den mittel- und osteuropäischen Staaten (\cite{dimit02, grab05, kneuer07, linden,schsed05a}). Diese Studien gehen von einer generell erfolgreichen Wirkung der EU-Konditionalität aus, da die Reformen in den entsprechenden Ländern umgesetzt wurden oder Regierungen, die von der EU kritisiert wurden, abgewählt wurden (vgl. \cite{brusis09} : 196).\par

Einen Überblick zu den Ansätzen zur Erforschung des Europäischen Integrationsprozesses auf die institutionellen Strukturen der Mitgliedstaaten liefert folgendes Schema:
\begin{figure}[H]
\caption{ Überblick über die Forschung zu Europäisierung und Konditionalisierung}
  \centering
  \includegraphics[width=5in]{Material/ForschungZuEuropUndKondi_ohneRand}\\
  
\scriptsize{(Quelle: in Anlehnung an \cite{huszak} : 75)}
\end{figure}
Erkennbar ist aus diesem Schema die Einbettung der Europäisierungs- und der Konditionalitätsforschung in die übergeordneten theoretischen Konzepte Integrationsforschung, Forschung zu institutionellem Wandel und Transitionsforschung.\par
Moravcsik und Vachudova gehen von einer asymmetrischen Interdependenz aus zwischen Beitrittskandidat und der EU. Bei positiv verlaufender Konditionalität schätzen die Kandidatenländer die politischen Kosten der Anpassung ihrer nationalen Politiken niedriger ein als einen möglichen Ausschluss aus der EU und die damit verbundenen Nachteile (vgl. \cite{morvac} : 44).
\par
Schimmelfennig und Sedelmeier schlagen ein „external incentives“ Modell vor, das den Erfolg der EU-Konditionalität anhand von vier Faktoren beschreibt. Diese Faktoren führen dazu, dass nationale Regierungen EU-Regeln übernehmen, wenn die Vorteile größer sind als die Kosten der Anpassung. Die vier Faktoren sind “the determinancy of conditions, the size and speed of rewards, the credibility of threats and promises and size of adaption costs” (\cite{schsed05b} : 12). Angewandt auf die neuen EU-Mitgliedsländer kommen die Autoren zu dem Schluss, dass das „external incentives“ Modell von dem Typ der Konditionalität abhängt, wobei die Acquis-Konditionalität besser abschnitt als die politische Konditionalität (vgl. \cite{schsed05c} : 212). Die empirische Überprüfung führt zu dem Schluss, dass die Glaubwürdigkeit der Belohnung und die Höhe der politischen Anpassungskosten ausschlaggebend waren bei der Entscheidung der Anpassung an EU-Konzepte. Hinsichtlich der Glaubwürdigkeit erhöhte die Eröffnung von Verhandlungen die Wahrscheinlichkeit von nationalen Anpassungen, da sich damit in den Augen der Kandidatenländer der Wille der EU zeigte, die Verhandlungen auch zu einem Abschluss zu bringen (vgl. \cite{schsed05c} : 215). Weiterhin nimmt die Gefahr des Ausschlusses von der EU-Mitgliedschaft ab, je weiter der Assoziierungsprozess fortschreitet (vgl. \cite{dimit05}). Allerdings zeigte sich auch, dass hohe Anpassungskosten, die die Sicherheit oder Integrität des Staates oder das Überleben der Regierung gefährdeten, eine starke Behinderung darstellten, sogar bei glaubwürdigen Anreizen der EU. Nur im allerletzten Stadium der Verhandlungen („endgame“) haben die Staaten die Anpassungsleistungen vollzogen, sogar bei kurzfristigen hohen politischen Anpassungskosten im eigenen Land (vgl. \cite{schetal} : 921).\par

Huszka merkt in Bezug auf die Anwendung dieses “external incentives” Modells auf den Balkan an: „However, while this ‘external incentive model’ according to which external rewards help elites to overcome domestic costs worked effectively in Central and Eastern Europe, its application to the Western Balkans is more problematic” (\cite{huszka} : 10). Hinzu kommt, dass die Mitgliedschaft für die in der vorliegenden Arbeit betrachteten Balkanstaaten noch stark in der Zukunft liegt. Daher sind die Belohnungen, die aktuell möglich sind, eher beschränkt.\par

Brusis konstatiert, dass demokratische Reformen verschiedene Ursachen haben und er geht davon aus, dass die Konditionalität der EU einen wesentlichen Einfluss hat, gibt aber auch zu bedenken: „Von der EU oder anderen externen Demokratisierungsakteuren gestellte Anforderungen sind aber weder a priori notwendige, noch hinreichende Bedingungen für innerstaatlichen Wandel“ (\cite{brusis05} : 298).

\subsubsection{Öffentliche Verwaltung und politische Konditionalität}

Die Notwendigkeit einer stabilen, effektiven und transparenten Verwaltung ist im Hinblick auf die Fähigkeit zur Übernahme des Acquis communautaire wichtig und wird in den Handreichungen der EU zur Übernahme des Acquis folgendermaßen formuliert: “A candidate country preparing for accession to the EU must bring its institutions, management capacity and administrative and judicial systems up to Union standards with a view to implementing the acquis effectively… At the general level, this requires a well-functioning and stable public administration built on an efficient and impartial civil service, and an independent and efficient judicial system” (\cite{eurcom05} : 7).\par
Die Existenz einer gut funktionierenden und stabilen öffentlichen Verwaltung ist eines der wesentlichen Kriterien innerhalb der EU-Konditionalität. Allerdings ist die Existenz einer funktionierenden öffentlichen Verwaltung kein Kapitel des Acquis und unterliegt damit nicht der direkten Überprüfung anhand eines Kriterienkataloges. Die EU hat gemeinsame grundrechts- und allgemein rechtsstaatsbezogene Normen vereinbart. Doch gibt es keine konkreten Vorgaben, wie demokratische Institutionen (Parlament, Regierung, Gerichte, Verwaltungsaufbau) organisiert sein sollen. Und in dieser Hinsicht existieren keine konkreten benchmarks, an denen sich die Beitrittsländer orientieren und deren Erfüllung man untersuchen könnte (vgl. \cite{brusis09} : 196). Von der EU wird das Thema Verwaltungsreform unter „politische Bedingungen“ behandelt und diese „politischen Kriterien“ nehmen einen festen Raum ein in den jährlichen Fortschrittsberichten der EU zu den Beitrittskandidaten.\par
Die Konditionalitätsforschung geht also von einem starken Zugzwang aus, in den die Kandidatenländer geraten und der dazu führt, dass sie die Anforderungen der EU zum Umbau ihrer nationalen Strukturen erfüllen. Dies ist deutlich im Rahmen der geforderten Übernahme des Acquis mit konkreten Kapiteln, die im nationalen Rahmen umzusetzen sind. Im Zusammenhang mit der Verwaltungsmodernisierung ist dies nicht so eindeutig nachvollziehbar, da es sich nicht um ein Kapitel des Erweiterungsacquis handelt.\par
Kennzeichnend für die Konditionalitätsforschung ist also die Konzentration auf die Frage, was die Veränderungen insbesondere in den Beitrittskandidaten befördert. Zentraler Gesichtspunkt sind dabei die Bedingungen der EU, die einem Beitritt vorausgehen, d.h. die Konditionalität. Im Kontext der vorliegenden Arbeit geht es hierbei insbesondere um die politische Konditionalität, unter die das Thema Verwaltungsmodernisierung fällt. Verwaltungsmodernisierung ist kein Kapitel des Acquis und entfaltet daher vergleichsweise geringere Konditionalität. Dennoch ist die Struktur der Verwaltung und ihre Modernisierung Thema unter politischen Kriterien, wie z.B. in den jährlichen Fortschrittsberichten deutlich wird.\par
Insofern ist die Konditionalitätsforschung auch auf das Thema Verwaltungsmodernisierung in den Beitrittsländern anwendbar und kann wertvolle Hinweise liefern. \par
Im nächsten Abschnitt der Untersuchung werden deshalb die praktischen Aspekte der EU-Erweiterung, jeweils mit Rückbindung an das Thema Verwaltungsentwicklung \/ Verwaltungsmodernisierung, überblicksartig dargestellt.Es wird im weiteren Verlauf der Arbeit zu prüfen sein, welchen Stellenwert Verwaltungsmodernisierung für die EU im Zuge der Erweiterungsstrategie hat und wie Verwaltungsmodernisierung in der Erweiterungspolitik vorkommt.
\section{Die Erweiterung der EU in der Praxis }
Die Europäische Union in den 1950er Jahren, zunächst mit der Gründung der Europäischen Gemeinschaften, umfasste sechs Staaten (Belgien, Frankreich, Deutschland, Luxemburg, Italien und die Niederlande). Ziel war es, nach dem Zweiten Weltkrieg einen wirtschaftlichen Staatenverbund zu schaffen, der die Gefahr gewaltsamer Auseinandersetzungen vermindern und durch einen gemeinsamen Markt die Wirtschaft ankurbeln sollte. Der EU, mit den Römischen Verträgen von 1957 gegründet, gehören inzwischen 27 Länder an, die in sogenannten Erweiterungsrunden aufgenommen wurden. Länder, die geografisch zu Europa gehören und demokratisch verfasst sind, können in die EU aufgenommen werden. Die umfassendste Erweiterung wurde 2004 umgesetzt mit der Aufnahme von Zypern, der Tschechischen Republik, Estland, Ungarn, Lettland, Malta, Polen, der Slowakei und Sloweniens. In derselben Erweiterungsrunde, aber mit einer Verzögerung, traten Bulgarien und Rumänien 2007 der EU bei. Die Aufnahme Kroatiens ist für 2013 vorgesehen.\par
\begin{table}[!hbt]\vspace{1ex}\centering
\caption{Geschichte der Verträge zur Europäischen Gemeinschaft}
\small
%\texttt{
\begin{tabular}{|p{12mm}|p{12mm}|p{4cm}|p{7cm}|}\hline
Unter\-zeichnet &In Kraft & Name&Inhalt\\\hline
1951&1952&Vertrag von Paris&Europäische Gemeinschaft für Kohle und Stahl (EGKS)\\\hline
1957&1958&Verträge von Rom&EWG-Vertrag, Europäische Wirtschaftsgemeinschaft (EWG) und der EURATOM-Vertrag\\\hline
1965&1967&Fusionsvertrag&Einsetzung eines gemeinsamen Rates und einer gemeinsamen Kommission der Europäischen Gemeinschaften\\\hline
1986&1987&Einheitliche Europäische Akte&Binnenmarkt eingeführt\\\hline
1992&1993&Vertrag von Maastricht&Europäische Union\\\hline
1997&1999&Vertrag von Amsterdam&Änderungen des Maastrichter Vertrages\\\hline
2001&2003&Vertrag von Nizza&Änderungen der Verträge von Rom und Amsterdam\\\hline
2004& &Verfassungsvertrag&Verfassung für Europa (abgelehnt)\\\hline
2007&2009&Vertrag von Lissabon&Änderungen des Vertrages über die Europäische Union (EUV) und des Vertrages über die Arbeitsweise der Europäischen Union (AEUV)\\\hline
\end{tabular}\\
%}
\scriptsize{Quelle: nach \cite{moller} : 4 (eigene Ergänzung zu Vertrag von Lissabon)}
\end{table}

\subsection{Das Verfahren zur Aufnahme eines Staates }
Die Aufnahme neuer Mitglieder war von Anfang an in der Gründungsidee der EU enthalten. In Art. 6 Absatz 1 des Vertrages über die Europäische Union (EUV) ist festgelegt: „Die Union beruht auf den Grundsätzen der Freiheit, Demokratie, der Achtung der Menschenrechte und Grundfreiheiten sowie der Rechtsstaatlichkeit; diese Grundsätze sind allen Mitgliedstaaten gemeinsam“. Artikel 49 des Vertrages legt fest: „Jeder europäische Staat, der die in Art. 6 Abs. 1 genannten Grundsätze achtet, kann beantragen, Mitglied der Union zu werden“. Über diese allgemeine Verfügung hinaus muss die EU in der Lage sein, neue Mitglieder aufzunehmen, was im Einzelfall entschieden wird. Eine Aufnahme geschieht durch Konsensbeschluss der EU-Mitgliedstaaten mittels ihrer Vertreter im Ministerrat oder Europäischen Rat. Nach dem Antrag auf Aufnahme wird aufgrund einer Stellungnahme der Europäischen Kommission entschieden, ob das Land als Beitrittskandidat anerkannt wird. Innerhalb der Kommission ist die Generaldirektion Erweiterung zuständig für Koordination, regelmäßige Berichterstattung sowie enge Zusammenarbeit mit den Line DGs und den Arbeitsgruppen des Europäischen Rates (vgl. \cite{summa} : 13). Der Delegation der EU in den Kandidatenländern kommt ebenfalls eine wichtige Rolle zu in der Koordination zwischen der Europäischen Kommission in Brüssel und den Kandidatenländern.\par
Vor der Aufnahme in die EU findet ein Prozess der Verhandlungen statt zu unterschiedlichen Politikbereichen, um die Übernahme des vollständigen gemeinschaftlichen Besitzstandes zu gewährleisten. Dies ist eine Aufnahmebedingung. Vor einer Aufnahme muss dann der entsprechende Vertrag in den Mitgliedstaaten nach dem dafür vorgesehenen Verfahren ratifiziert werden. Schließlich muss noch das Europäische Parlament seine Zustimmung geben (\cite{euko07} : 6f).
\par
In den Kandidatenländern wird die Arbeit im Zusammenhang mit dem Beitrittsprozess meist von einem Minister in einem bestehenden Ministerium oder aus einem eigens geschaffenen Ministerium für den Erweiterungsprozess koordiniert (\cite{summa} : 14).\par
Die Stadien im Erweiterungsprozess sind im folgenden Schema überblicksartig dargestellt:
\begin{figure}[H]
  \centering
   \caption{Phasenmodell des EU-Beitritts }
  \includegraphics[width=5in]{Material/Phasenmodell_ohneRand}\\

\scriptsize{Quelle: in Anlehnung an \cite{wessels} : 449}
\end{figure}
In der Darstellung ist der gesamte Prozess der Aufnahme in den einzelnen erforderlichen Schritten schematisch dargestellt. Dabei werden drei Hauptphasen unterschieden: Die Initiativ-, die Verhandlungs- und die Ratifizierungsphase. Der angestrebte EU-Beitritt der drei Untersuchungsländer befindet sich in der Verhandlungsphase, aktuell geprägt von der Umsetzung und Abarbeitung der Heranführungsstrategie und der jährlichen Berichterstattung (im Schema Fortschrittsberichte genannt).\par

Nachdem ein Land offiziell den Aufnahmeantrag gestellt hat, bekommt es von der EU-Kommission einen Fragebogen zu allen Bereichen des Acquis communautaire übersandt. Diese Fragebögen zu den bestehenden Institutionen, policies und Infrastruktur müssen von dem Antragste
 ausgefüllt und der EU-Kommission übersandt werden. Auf der Grundlage dieser Antworten erarbeitet die Kommission eine vorläufige Stellungnahme zu dem Beitrittsgesuch des Landes, mit einer Empfehlung, ob und ggf. wann das Land die Beitrittsverhandlungen beginnen kann. Um offiziell Beitrittskandidat zu werden, muss der Europäische Rat/Council of Ministers formal beschließen, mit dem Land Beitrittsverhandlungen aufzunehmen. Die Kommission tritt dann in einen Prozess des “screening” ein, in dem die Gesetzgebung und policies des Landes mit der EU verglichen werden, “in order to make longterm plans to bring applicant countries up to EU standards“ (Grabbe et al. 2010, 3).
Die praktische Durchführung der Erweiterung ist ein komplexer Prozess, der sich evolutionär entwickelt hat. Keine Erweiterungsrunde war bisher mit der vorherigen identisch und in jedem Fall wurden neue Erweiterungsinstrumente eingeführt oder bestehende weiterentwickelt (vgl. Kochenov, 2008: 16). Die bisherigen Aufnahmen mit den Schritten vom Beitrittsantrag bis zum Beitritt sind im folgenden Schema mit der in der Literatur vorherrschenden Bezeichnung der jeweiligen Erweiterung dargestellt.\\
\begin{table}[H]\vspace{1ex}\centering
\caption{Bisherige EU-Erweiterungen}
\begin{tabular}{|p{2cm}|p{2cm}|p{2cm}|p{2cm}|p{2cm}|p{2cm}|}\hline
&Beitritts\-antrag&Stellung\-nahme Kommis\-sion&Beginn Beitritts\-verhand\-lungen&Unter\-zeichnung Beitritts\-vertrag&Beitritts\-datum\\\hline
\multicolumn{6}{|p{12cm}|}{1. Norderweiterung} \\\hline 
Vereinigtes Königreich&
10.05.1967 (09.08.61)*&
19.09.1967&
30.06.1970&
22.01.1972&
01.01.1973\\\hline
Dänemark&
11.05.1967 (10.08.61)*&
19.09.1967&
30.06.1970&
22.01.1972&
01.01.1973\\\hline
Irland&
11.05.1967 (10.08.61)*&
19.09.1967&
30.06.1970&
22.01.1972&
01.01.1973\\\hline
\multicolumn{6}{|p{12cm}|}{2. Süderweiterung}\\\hline
Griechenland &
12.06.1975&
29.01.1976&
27.07.1976&
28.05.1979&
01.01.1981\\\hline
Portugal&
28.03.1977&
19.05.1978&
17.10.1978&
12.06.1985&
01.01.1986\\\hline
Spanien&
28.07.1977&
29.11.1978&
05.02.1979&
12.06.1985&
01.01.1986\\\hline
\multicolumn{6}{|p{12cm}|}{3. EFTA-Erweiterung}\\\hline
Österreich&
17.07.1989&
01.08.1991&
01.02.1993&
24.04.1994&
01.01.1995\\\hline
Schweden&
01.07.1991&
31.07.1992&
01.02.1993&
24.04.1994&
01.01.1995\\\hline
Finnland&
18.03.1992&
04.11.1992&
01.02.1993&
24.04.1994&
01.01.1995\\\hline
\multicolumn{6}{|p{12cm}|}{4. Erweiterung Mittel- und Osteuropa}\\\hline
Ungarn&
31.03.1994&
16.07.1997&
31.03.1998&
16.04.2003&
01.05.2004\\\hline
Polen&
05.04.1994&
16.07.1997&
31.03.1998&
16.04.2003&
01.05.2004\\\hline
Slowakei&
27.06.1995&
16.07.1997&
15.02.2000&
16.04.2003&
01.05.2004\\\hline
Lettland&
13.10.1995&
16.07.1997&
15.02.2000&
16.04.2003&
01.05.2004\\\hline
Estland&
24.11.1995&
16.07.1997&
31.03.1998&
16.04.2003&
01.05.2004\\\hline
Litauen&
08.12.1995&
16.07.1997&
15.02.2000&
16.04.2003&
01.05.2004\\\hline
Tschechien&
17.01.1996&
16.07.1997&
31.03.1998&
16.04.2003&
01.05.2004\\\hline
Slowenien&
10.06.1996&
16.07.1997&
31.03.1998&
16.04.2003&
01.05.2004\\\hline
Rumänien&
22.06.1995&
16.07.1997&
15.02.2000&
25.04.2005&
01.01.2007\\\hline
Bulgarien&
14.12.1995&
16.07.1997&
15.02.2000&
25.04.2005&
01.01.2007\\\hline
\multicolumn{6}{|p{12cm}|}{* In Klammern der Zeitpunkt des jeweils ersten Beitrittsantrages: In Folge des Scheiterns der Verhandlungen mit dem Vereinigten Königreich kam es ebenfalls zum Abbruch der Verhandlungen mit den übrigen Bewerbern.}\\\hline
\end{tabular}\\
\scriptsize{Quelle: \cite{wessels} : 451.}
\end{table}
Aus dieser Darstellung wird ersichtlich, dass die bisherigen EU-Erweiterungen in Wellen stattgefunden haben, mit der Aufnahme von Ländern oft in Gruppen. Darauf sind auch die umgangssprachlichen Benennungen wie „EU-Süderweiterung“ oder „EU-Osterweiterung“ zurückzuführen. 

\subsection{EU-Erweiterungen in der Wahrnehmung der Öffentlichkeit}
Die Wahrnehmung der Öffentlichkeit in EU-Ländern hinsichtlich einer erneuten EU"=Erweiterung lässt deutliche Unterschiede erkennen. Dabei wird in den „neuen“ EU-Ländern eine erneute Erweiterung am positivsten gesehen und in den „alten“ EU-15 Ländern am negativsten. Weiterhin sinkt im Durchschnitt die Befürwortung einer Erweiterung um ca. 3 Prozentpunkte jährlich. In Polen wird die Erweiterung am positivsten bewertet, mit 74\% im Jahre 2008. Die Befürwortung einer erneuten Erweiterung lag im EU-Durchschnitt im selben Jahr nur bei 47\% (EU-27). In Ländern mit geringer Begeisterung der Bevölkerung für eine erneute Erweiterung, wie Österreich, Frankreich und Deutschland, stehen die Regierungen einer Südosterweiterung positiv gegenüber, ungeachtet der Einstellung der Mehrheit der Bevölkerung (vgl. \cite{mus} : 20).\par

In den Beitrittsländern, die Gegenstand der vorliegenden Arbeit sind, hat sich die Unterstützung der Bevölkerung für den EU-Beitritt unterschiedlich entwickelt. So ist in Montenegro die Zustimmung zur EU-Orientierung des Landes im Zeitraum 2009-2010 von 67\% auf 73\% gestiegen. In Mazedonien dagegen fiel die Zustimmung zu einem EU-Beitritt des Landes im selben Zeitraum von 62\% auf 60\%. Im gesamten Westlichen Balkan ist die EU-Orientierung der Bevölkerung in Albanien im Jahr 2010 mit 81\% am höchsten, hat aber dennoch 9 Prozentpunkte Zustimmung gegenüber dem Vorjahr verloren (vgl.\cite{gallup10}).\par
In einer repräsentativen Umfrage in den Mitgliedsländern der EU wird deutlich, dass die Bevölkerung dort im Zusammenhang mit einer erneuten Erweiterung vor allem Freiheit und demokratische Werte, noch vor ökonomischen Überlegungen, in Bezug auf Europa als Ganzes wichtig findet. In Bezug auf das eigene Land war den Befragten allerdings der ökonomische Aspekt der Erweiterung wichtiger, wie aus folgender Tabelle ersichtlich:
\begin{figure}[H]
  \caption{Umfrage zur EU-Erweiterung in den Mitgliedstaaten der EU }
  \centering
  \includegraphics[width=5in]{Material/Umfrage}\\
\scriptsize{Quelle: Eurobarometer 2009: 20}\footnote{Das Eurobarometer ist eine in regelmäßigen Abständen von der Europäischen Kommission in Auftrag gegebene öffentliche Meinungsumfrage in den Ländern der EU. Dabei werden jeweils immer die gleichen Standardfragen als auch wechselnde Fragen zu unterschiedlichen Themen gestellt.}
\end{figure}

In den Mitgliedstaaten der EU wurde folgende Frage gestellt: Falls Die EU in der Zukunft über die Aufnahme neuer Mitglieder entscheiden würde, welche zwei Themen (von der vorgegebenen Liste) sollten dabei berücksichtig werden?

\subsection{Politische Konditionalität im Aufnahmeverfahren}
Um Mitgliedstaat zu werden, muss ein Land den kompletten gemeinschaftlichen Besitzstand der Union (Acquis communautaire) akzeptieren, d.h. in nationales Recht übernehmen. Das gesamte Recht der Europäischen Gemeinschaften und der Europäischen Union wird unter dem Begriff „gemeinschaftlicher Besitzstand“ zusammengefasst. Es handelt sich um rund 15.000 Rechtsakte auf nahezu 100.000 Druckseiten (Stand 2008). Der Acquis communautaire, der seit 1973 in 31 thematische Kapitel eingeteilt war, wurde nach Abschluss der letzten Erweiterungswelle in 35 Kapiteln reorganisiert (vgl. \cite{summa} : 14). \par

\begin{table}[H]
\caption{Die Kapitel des Acquis communautaire:}
\begin{tabular}{|c p{7cm}|c p{7cm}|}\hline
1:&Freier Warenverkehr&19:&Beschäftigung und Soziales\\
2:&Freizügigkeit für Arbeitnehmer&20:&Unternehmen und Industrie\\
3:&Niederlassungsrecht und freier Dienstleistungsverkehr&21:&Transeuropäische Netze\\
4:&Freier Kapitalverkehr&22:&Regionalpolitik und Koordinierung der strukturellen Instrumente\\
5:&Öffentliches Auftragswesen&23:&Judikative und Grundrechte\\
6:&Gesellschaftsrecht&24:&Justiz, Freiheit und Sicherheit\\
7:&Rechte am geistigen Eigentum&25:&Wissenschaft und Forschung\\
8:&Wettbewerb&26:&Bildung und Kultur\\
9:&Finanzdienstleistungen&27:&Umwelt\\
10:&Informationsgesellschaft und Medien&28:&Verbraucher- und Gesundheitsschutz\\
11:&Landwirtschaft und ländliche Entwicklung&29:&Zollunion\\
12:&Lebensmittelsicherheit, Tier- und Pflanzenschutzpolitik&30:&Außenbeziehungen\\
13:& Fischerei&31:&Außen-, Sicherheits- und Verteidigungspolitik\\
14:&Verkehr&32:&Finanzkontrolle\\
15:&Energie&33:&Finanz- und Haushaltsvorschriften\\
16:&Steuern&34:&Institutionen\\
17:&Wirtschaft und Währung&35:&Sonstiges \\
18:&Statistik&&\\\hline
\end{tabular}
{\footnotesize \url{http://www.europarl.europa.eu/brussels/website/media/modul_01/Zusatzthemen/Pdf/Acquis.pdf} (Aufgerufen: 10.9.2012).}\\
\end{table}
Obwohl es während der Süderweiterung keine auf bestimmte Länder abgestimmten Bedingungskataloge und kein regelmäßiges Monitoring gab, war neben der Übernahme des Acquis communautaire die Demokratie als Staatsmodell ein Aufnahmekriterium (vgl. \cite{pridham07} : 451). Die Assoziierungsvereinbarung mit Griechenland wurde nach dem Staatsstreich der Generäle 1967 eingefroren und der ursprüngliche Aufnahmeantrag Spaniens blieb zunächst unbeantwortet. Dies zeigt, dass die demokratische Verfasstheit schon in dieser Zeit der EU-Erweiterung ein, wenn auch implizites Kriterium war. In der Folge ist die EU von der reinen Annahme des Acquis communautaire als Beitrittsbedingung zu weiter gefassten Reform- und Transformationszielen mit zusätzlichen Voraussetzungen übergegangen (vgl. \cite{dimit04} : 8-9).\par
Eine Art politischer Konditionalität wurde erstmals Mitte der 1990er Jahre eingeführt mit den Europe Agreements, die ausgesetzt werden konnten. Gefordert wurde Rechtsstaatlichkeit, Achtung der Menschenrechte, ein Mehrparteiensystem und freie Wahlen (Grabbe zit.in \cite{pridham07}).\par
Seit der Eröffnung einer Beitrittsperspektive für die ehemals kommunistischen Staaten Mittel- und Osteuropas wurden die Kriterien für einen potenziellen Beitritt mit den sogenannten Kopenhagen-Kriterien von 1993 konkreter benannt. Neben ökonomischen Voraussetzungen wird institutionelle Stabilität als eine notwendige Bedingung und Grundlage für demokratische und rechtsstaatliche Ordnung gefordert (vgl. \cite{kreile} : 654). So legte der Rat fest:\par

„Als Voraussetzung für die Mitgliedschaft muss der Beitrittskandidat eine institutionelle Stabilität als Garantie für demokratische und rechtsstaatliche Ordnung, für die Wahrung der Menschenrechte, sowie die Achtung und den Schutz von Minderheiten verwirklicht haben; sie erfordert ferner eine funktionsfähige Marktwirtschaft sowie die Fähigkeit, dem Wettbewerbsdruck und den Marktkräften innerhalb der Union standzuhalten.“ \footnote{Europäischer Rat Kopenhagen, Schlussfolgerungen des Vorsitzes, 21. bis 23. Juni 1993,\url{http://www.consilium.europa.eu/ueDocs/cms_Data/docs/pressData/de/ec/72924.pdf}, 13, Aufgerufen: 21.9.2012).}\par

Um die Kriterien von Kopenhagen in konkrete Maßnahmen zu überführen, wurde 1994 das Instrument der „pre-accession strategy“ eingeführt (\cite{lipsch}). \par

Damit waren die Erweiterungsbedingungen für die Zukunft fixiert. Die EU-Institutionen überprüfen regelmäßig, ob die (potenziellen) Beitrittskandidaten dieses sogenannte politische Kriterium erfüllen. In jährlichen Berichten, die Fortschrittsberichte genannt werden und 1998 erstmals von der Kommission erstellt wurden, überprüft diese, ob Legislative, Exekutive und Verwaltung sowie das Gerichtswesen in den beitrittswilligen Ländern verfassungskonform arbeiten, ob Minderheitenrechte und individuelle Grundrechte geschützt werden und Korruption bekämpft wird (vgl. \cite{brusis09} : 195).
\par
Seitdem kommt der mit den Kopenhagen-Kriterien verbundenen politischen Konditionalität im Rahmen der EU-Erweiterung besondere Bedeutung zu. Ein Wertekatalog gekoppelt an finanzielle Unterstützung im Beitrittsprozess ist das Kennzeichen dieser Entwicklung. Kneuer bezeichnet die Konditionalität des Erweiterungsprozesses als Anreiz-Druck-System, das den attraktiven Anreiz der Mitgliedschaft bereithält, aber ebenso Druck zu allgemeiner Demokratisierung aufrechterhält und bei demokratischen Defiziten oder Fehlentwicklungen den Druck mit Negativmaßnahmen verstärken kann (vgl.\cite{kneuer07} : 108).\par
Weitere Kriterien bezogen auf die Beitrittsverhandlungen wurden in Madrid anlässlich der Tagung des Europäischen Rates am 14./15. Dezember 1995 formuliert, unter anderem die Forderung nach Verwaltungsreformen in den Kandidatenländern (vgl. \cite{dimit02}). Bezogen auf die Heranführungsstrategie für die Länder Mittel- und Osteuropas forderte der Europäische Rat:
„diese Strategie muß intensiviert werden, um die Voraussetzungen für eine schrittweise und harmonische Integration dieser Länder zu schaffen, und zwar insbesondere durch die Entwicklung der Marktwirtschaft, die Anpassung der Verwaltungsstrukturen dieser Länder und die Schaffung stabiler wirtschaftlicher und monetärer Rahmenbedingungen“ (Europäischer Rat: 1995).\par
Damit haben die Mitgliedsländer Elemente von „Good Governance“ eingeführt, mit expliziter Erwähnung der Anpassung von Verwaltungsstrukturen (vgl. \cite{sabzei} : 307).\par
Rechtliche und administrative Kapazitäten mussten nun vorhanden sein und wurden als Voraussetzung der Umsetzung des Acquis communautaire gesehen. Damit war die administrative Konditionalität als neues Instrument des EU-Erweiterungsprozesses in der letzten Dekade des 20. Jh.s eingeführt worden (vgl. \cite{tomtul} : 380).\par
Die zusätzlichen Bedingungen beziehen sich zum Teil auf Bereiche, in denen die EU selbst nicht über Normenkompetenz oder einheitliche Regelungen verfügt. Zwar gibt es keine direkte Sanktionsmöglichkeit bei der Nichterfüllung der entsprechenden Kriterien, z.B. durch finanzielle Sanktionen. Dennoch ist die „politische Konditionalität“ ein wesentliches, neues Element im EU-Erweiterungsprozess. Die Zunahme der Bedeutung von institution building bzw. staatlicher Kapazitäten innerhalb der Erweiterungspolitik der EU wird von Sabel und Zeitlin sogar als Paradigmenwechsel bezeichnet, der sich graduell innerhalb der 1990er Jahre vollzog (vgl.\cite{sabzei} : 307).\par
Der Stellenwert und der Umgang mit der Reform der öffentlichen Verwaltung seitens der EU im Erweiterungsprozess wird im folgenden Kapitel dargestellt.

\subsection{Kompetenz der EU bezüglich der Verwaltung}
\label{subsec:Kompetenz der EU bezüglich der Verwaltung}
Im Bereich öffentliche Verwaltung hat die EU keine Regelungskompetenz in den Mitgliedstaaten und keinen direkten Einfluss auf das administrative System ihrer Mitglieder. Alle Mitgliedsländer können die institutionellen Arrangements ihrer öffentlichen Verwaltungen frei wählen, einschließlich des Systems der Ausführung von Staatsaufgaben. Die EU ist bei der Umsetzung des Gemeinschaftsrechts auf die nationalen Verwaltungen angewiesen, ohne Organisationsstrukturen oder Personalstrukturen direkt beeinflussen zu können. Weiterhin gibt es innerhalb der EU keine „Blaupause“, die nationale öffentliche Verwaltungen übernehmen können. Es gibt nicht einmal eine Vorstellung eindeutiger „Best Practice“-Beispiele in Bezug auf Strukturen und Verfahrensweisen, obwohl das „White Paper on European Governance“ von 2001 Ausführungsstandards zu beschreiben versucht: „The lack of a clear overarching public administration model and the relatively weak European powers for the imposition of specific changes in domestic administrations might also be considered as a factor for slowing down European Integration“ (\cite{sverdrup} : 44).\par

Das Dilemma der EU hinsichtlich der öffentlichen Verwaltung in den Beitrittskandidaten wird in einem Evaluierungsbericht zur Institutionenentwicklung für die osteuropäischen Beitrittskandidaten deutlich: “Although there are, in many sectors, highly detailed acquis requirements as to what the public administration should deliver, and to what standards, there is of course no acquis on public administration per se” (\cite{omas} : 3). Die Evaluatoren konstatieren, dass die öffentliche Verwaltung nicht Teil des Acquis ist und seitens der EU auch keine Vorgaben für Reformprojekte zur öffentlichen Verwaltung gemacht wurden: “There is no acquis on public administration and there is no evidence that any coordinated attempt has been made by the Commission Services to orientate the PHARE national Public Administration Reform Programmes in any defined direction.” (\cite{omas} : 1). \par

Die Europäische Union hat anlässlich der Erweiterungen 1973, 1980, 1986 und 1995 keine Begutachtung der administrativen Kompetenzen der Aufnahmeländer durchgeführt (vgl. \cite{ziller} : 138). Dennoch sind seit 1997 administrative Fragen im Zusammenhang mit der EU-Erweiterung zunehmend in den Vordergrund gerückt. „Candidate countries have been put under pressure to modernize their administrations, that is, to develop a professional civil service and build institutional capacity to implement and enforce legal norms” (\cite{olsen}). So hat die Präsidentschaft des Europäischen Rates nach ihrem Treffen in Kaeken am 14. und 15. Dezember 2001 angemahnt, dass die Kandidatenländer ihre Anstrengungen fortsetzen müssen, insbesondere um ihre administrativen Kapazitäten auf das geforderte Niveau zu bringen (vgl. \cite{olsen}). Hintergrund dieser Forderung ist vor allem die Notwendigkeit, nationale Verwaltungsstrukturen zu entwickeln, die für die Interaktion mit der EU verantwortlich sind für die Verhandlungen mit Brüssel und die Umsetzung des Acquis.

\subsection{Das Konzept Europäischer Verwaltungsraum}
Nachdem nun deutlich wurde, dass es keinen konkreten Acquis zur Ausgestaltung der öffentlichen Verwaltung gibt, stellt sich die Frage, ob es möglicherweise durch die Zunahme supranational zustande gekommener Entscheidungsprozesse zu einer Europäisierung im Verwaltungshandeln kommt. Einerseits hat die EU keinen direkten Einfluss auf die verwaltungsmäßige Umsetzung von Gemeinschaftsrecht, andererseits erscheint in Anbetracht der möglichen Aufnahme von Ländern, die als neue Staatsgebilde (Nachfolgestaaten von Jugoslawien) und/oder mit schwacher demokratischer Staatstradition (z.B. Bosnien-Herzegowina, Albanien) eine Art Blaupause für Strukturen einer europäisch orientierten Verwaltung sinnvoll. Vor allem die Organisation für Ökonomische Zusammenarbeit und Entwicklung (OECD) mit ihrem SIGMA-Programm (Support for Improvement in Governance and Management in Central and Eastern European Countries) war seit 1998 maßgeblich an der Entwicklung eines Konzeptes zu einem europäischen Verwaltungsraum (European Administrative Space) beteiligt. SIGMA kam im Rahmen einer Studie zu der Auffassung, dass die strikte Auslegung von Artikel 39 Absatz 4 EG-Vertrag und die strikte Auslegung des Begriffs der „Öffentlichen Verwaltung“ durch den EuGH zur Schaffung eines Verwaltungsraumes in den Mitgliedstaaten führen wird. Die European Group of Public Administration (EGPA) des Internationalen Instituts für Verwaltungswissenschaften richtete 2002 eine Tagung aus zum Thema: „ The European Administrative Space: Governance in Diversity“ (\cite{mangenot} : 13).\par
In Anbetracht der unterschiedlichen Traditionen der öffentlichen Verwaltung in Europa gibt es Stimmen, die vor der Ausrufung eines gemeinsamen europäischen Rahmens für den öffentlichen Sektor warnen (vgl. \cite{olsen}). Gründe für diese angemahnte Vorsicht sind unter anderem nach wie vor nationale Weichenstellungen. Die EU hat gerade erst angefangen, die Reform der öffentlichen Verwaltung als eigenständiges Politikfeld zu sehen (vgl. \cite{schmar}). Weiterhin hat die EU kein starkes Rollenmodell im Bereich Public Management für die Mitgliedstaaten geschaffen und die zentrale Verwaltung der EU erscheint als ein Patchwork einzelner unterschiedlicher nationaler Traditionen. Dennoch kann von einem Trend hin zu einem spezifischen europäischen Ansatz im Bereich öffentlicher Verwaltung ausgegangen werden. Dies kann auch damit erklärt werden, dass es in den EU-Mitgliedsländern eine gemeinsame Idee von Staat, Souveränität und Demokratie gibt, zumindest im Vergleich mit anderen Kontinenten. Insbesondere der ‘acquis communautaire’, der zentrale Teil der EU-Gesetzgebung, stellt eine Basis dar für gemeinsame administrative Standards und Regeln, die Eingang finden in das System der nationalen öffentlichen Verwaltung (vgl. \cite{raarut} : 31). Weiterhin entspricht das Konzept eines Europäischen Verwaltungsraumes der Forderung von Wirtschaftsakteuren nach einheitlichen Wettbewerbsbedingungen und nach administrativer Kooperation über Ländergrenzen hinweg (vgl. \cite{bogjan} : 245). \par
Gemeinsame Grundwerte und -prinzipien der europäischen öffentlichen Verwaltung haben zunehmend zu Konvergenz zwischen nationalen Verwaltungen geführt. „The European Administrative Space (EAS) represents an evolving process of increasing convergence between national administrative legal orders and administrative practices of member states. The EAS concerns basic institutional arrangements, processes, common administrative standards, civil service values and administrative culture. It is difficult to speak of a European model of Public Administration, but the EAS, albeit a metaphor, signifies a convergence and states the basic values of public administration as a practice and profession in Europe” (\cite{oecd99} : 15). \par
Mit der Stärkung und Erweiterung der EU sind im Hinblick auf die administrative Konvergenz viele Hoffnungen verbunden. So wird von einer Homogenisierung der administrativen Kapazitäten ausgegangen, unter Einbezug nationaler und kultureller Besonderheiten. Dabei geht es nicht um eine „Gleichschaltung“ der administrativen Systeme, sondern um eine Angleichung der Serviceerbringung hinsichtlich Qualität und Effizienz im Sinne eines Public Service Standards, so eine Sichtweise (vgl. \cite{dorta} : 8f.). Allerdings gibt es auch Stimmen, die auf die Bedeutung der nationalen Verwaltungstraditionen hinweisen. Selbst eindeutig auf europäischen Regelungen basierende Veränderungsprozesse der nationalen Verwaltungen sind nach dieser Sichtweise entscheidend durch den nationalen Kontext geprägt (vgl. \cite{herit}). Daneben gibt es auch Hoffnung, dass mit dem Konzept des European Administrative Space politischer Nationalismus relativiert werden kann. „A development towards an EAS stands in contrast to national administrative systems as ‘solid bedrock for nationalism’, that is, idiosyncratic arrangements where the structure of public administration reflects the identity, history and traditions of a specific state and society (\cite{olsen} : 1).\par
Auf einer praktischen Ebene kann man bei dem ‚European Public Administration Network’ (EUPAN) \footnote{\url{http://www.eupan.eu/en/content/show/&tid=188}}, in dem Minister und Generaldirektoren des öffentlichen Dienstes vertreten sind, von einer Europäisierung durch Verwaltungskooperation sprechen. In dem Netzwerk war von Anfang an auch die Europäische Kommission vertreten mit dem für die Verwaltungsreform zuständigen Kommissar der Generaldirektion Verwaltung und Personal (ADMIN); dennoch steht EUPAN außerhalb des förmlichen Rahmens der Gemeinschaft. EUPAN dient als Netzwerk und fördert durch das Zusammentreffen von nationalen Beamten den Austausch über Ländergrenzen hinweg (vgl. \cite{mangenot} : 49).\par
Eine andere (europäische) Initiative, das ‚Common Assessment Framework’ (CAF) mit dem Ziel, Exzellenz in der europäischen öffentlichen Verwaltung zu fördern, soll hier ebenfalls erwähnt werden. Das CAF wurde im Anschluss an das Ministertreffen vom November 1998 entwickelt, als die Schaffung eines „Europäischen Qualitätspreises“ auf Grundlage von Leistungsindikatoren vorgeschlagen wurde. Das im Mai 2000 in Lissabon auf der ersten europäischen Qualitätskonferenz vorgestellte CAF basiert auf Modellen der Europäischen Stiftung für Qualitätsmanagement (European Foundation for Quality Management) und ist ein Instrument der Selbstbewertung von öffentlichen Verwaltungen (vgl. \cite{mangenot} : 52).\par
Im Lissabonner Vertrag, der 2009 in Kraft trat, wurde erstmals Verwaltungszusammenarbeit direkt erwähnt. Mit dem Vertrag von Lissabon wurden die beiden „Gründungsverträge“ der EU, d.h. der Vertrag über die Europäische Union (EUV) und der Vertrag zur Gründung der Europäischen Gemeinschaft (EGV), der in „Vertrag über die Arbeitsweise der Europäischen Gemeinschaft“ (AEUV) umbenannt wurde, grundlegend und umfassend geändert. Der eigentliche Lissabon-Vertrag, enthält die jeweiligen Änderungen am EUV und am AEUV (ex-EGV). Dies betrifft auch Artikel 176 des AEUV, in dessen veränderter Version unter Artikel 176d erstmals ausdrücklich Verwaltungszusammenarbeit erwähnt wird: „Die Union kann die Mitgliedstaaten in ihren Bemühungen um eine Verbesserung der Fähigkeit ihrer Verwaltung zur Durchführung des Unionsrechts unterstützen. Dies kann insbesondere die Erleichterung des Austauschs von Informationen und von Beamten sowie die Unterstützung von Aus- und Weiterbildungsprogrammen beinhalten. Die Mitgliedstaaten müssen diese Unterstützung nicht in Anspruch nehmen. Das Europäische Parlament und der Rat erlassen die erforderlichen Maßnahmen unter Ausschluss jeglicher Harmonisierung der Rechtsvorschriften der Mitgliedstaaten durch Verordnungen gemäß dem ordentlichen Gesetzgebungsverfahren“ (\cite{verLis} : C306/90). \par
Ein weiterer Versuch, das Thema Verwaltungsmodernisierung im Erweiterungsprozess zu operationalisieren, stellt die sogenannte PAR checklist der Generaldirektion Verwaltung (DG ADMIN) dar. 

{\bf “PAR checklist”}
\begin{enumerate}
\item PAR framework 
\begin{itemize}
\item Political will
\item Authority in charge with the coordination of the PAR
\item Comprehensive reform programme: reform strategy/action plan established + implemented after consultation with different stakeholders
\item Acceptance of the reform at all central/local/regional levels
\item Legal background on PA organization and administrative procedures endorsed and implemented
\item Integration of principles of a sound PA derived from the Community law
\end{itemize}
\item Civil service quality
\begin{itemize}
\item Structure in charge with civil service management
\item Legal acts endorsed and implemented (civil service act + secondary legislation – rights and obligations, ethics and integrity, merit + equal chances + transparency based recruitment, fair appraisal and promotion systems, appeals procedures, basic salary systems formalised + transparent bonus allocation policy, training, pension systems …)
\item HR instruments (CAF, competency frameworks, personal benchmarks, career guidance schemes, fast track…etc.)
\end{itemize}
\item Anti-corruption policy
\begin{itemize}
\item Political will to fight against corruption
\item Establishment/existence of independent anti-corruption bodies
\item Instruments to prevent, detect and penalise corruption (effective legal framework, anti-corruption strategies or laws, watchdog agencies, codes of conduct, penal laws, regulation of conflict of interest and incompatibilities, rules to ensure transparency and accountability in financial management, disciplinary procedures…)
\item Facultative requirements: ethic counselors, exchange of best practices, awareness campaigns, whistleblowers procedures…
\end{itemize}
\item Transparency and citizen orientation 
\begin{itemize}
\item Body/ies representing public interest (ombudsman.…)
\item Legal acts (law on free access to public information, law on treating citizens’ complaints, provisions on regular consultation of citizens…) 
\item Transparency instruments (public events, citizens’ charters, e-government instruments, regular consultation of public opinion, one-stop shops, information centers…)”
(\cite{butiu} : 13)
\end{itemize}
\end{enumerate}

Diese checklist wurde im Entwurf diskutiert und die Idee von einzelnen Akteuren in Brüssel, Paris sowie in den Beitrittsländern selbst aufgenommen. Eine verbindliche Richtschnur ist damit allerdings nie verbunden worden. Dies wohl auch zum Teil aus Sorge der Mitgliedsländer, dass damit eine Einmischung in die Strukturen ihrer eigenen öffentlichen Verwaltungen gerechtfertigt werden könnten. \par
Die Ausgestaltung der öffentlichen Verwaltung liegt im Ermessen der EU-Mitgliedsländer und ist eng verbunden mit der historischen Entwicklung der Verwaltung. In Europa gibt es unterschiedliche Verwaltungstraditionen und eine Vereinheitlichung erscheint aus diesem Grund schwer möglich und nicht gewünscht. Dennoch ist die Bedingung der Übernahme des Acquis communautaire bei einer EU Mitgliedschaft eine immense Herausforderung insbesondere für die öffentlichen Verwaltungen der Beitrittsländer. Im Zusammenhang mit Zunahme des Einflusses der EU auf Binnenprozesse gewinnen supranationale Ansätze auch in Bezug auf die Ausgestaltung, bzw. Modernisierung der nationalen öffentlichen Verwaltungen Gewicht. Stichworte dazu sind European Administrative Space, Common Assessment Framework und EU PAR checklist zu Veraltungsmodernisierung. Auch der Lissabonner Vertrag von 2009 erwähnt zum ersten Mal die Verwaltungskooperation. Dennoch sind alle diese Ansätze nur punktuell und haben keinen verbindlichen Charakter.\par
Dass sich aus dieser fehlenden Verbindlichkeit zum Thema Verwaltungsentwicklung und Verwaltungsmodernisierung Probleme für die EU bei der Erweiterung ergeben, zeigt ein Blick auf die letzte Erweiterungswelle nach Osteuropa. Einen Überblick über die Erfahrungen zur Verwaltungsentwicklung in den Ländern der letzten Erweiterungsrunde bietet der folgende Abschnitt.

\subsection{Erfahrungen zur Verwaltungsentwicklung in den Staaten Osteuropas}
Die vorliegende Literatur zu den Erfahrungen der EU-Erweiterung in den Ländern Osteuropas, also der Erweiterungswelle 2004/7, wurde im Hinblick auf den Stellenwert der Reform der öffentlichen Verwaltung im Erweiterungsprozess gesichtet. Da es sich bei den Ländern der Osterweiterung ebenfalls um vormals zentralistisch organisierte Staaten handelte, können möglicherweise übertragbare Erkenntnisse für die anstehende Südosterweiterung gewonnen werden. \par

Beitrittsvereinbarungen, die so genannten Europe Agreements, wurden von der EU mit den osteuropäischen Beitrittskandidaten zwischen 1991 und 1996 unterzeichnet und die Länder stellten zwischen 1994 und 1996 die offiziellen Aufnahmeanträge. In den darauffolgenden Jahren, in denen die Heranführung der Länder an die EU seitens der EU unterstützt wurde, führte der Wunsch nach schneller Aufnahme oft zu vordergründigen Reformen „The overarching goal of wanting to join the EU as quickly as possible, however, dominated the governments’ behaviour and more strategic work on policy priorities seldom got significant early attention“ (\cite{summa} : 10).\par
In den früher zentralistisch regierten Ländern des Ostblocks waren die Dezentralisierung und die Etablierung demokratischer lokaler Regierungsstrukturen nach dem Systemwechsel vorrangige Ziele. Mit der Dezentralisierung eng verknüpft ist auch die effektive Erfüllung öffentlicher Aufgaben. Ein wesentliches Ziel in diesem Zusammenhang ist die Trennung der zentralen und der lokalen Verwaltung bei der Aufgabenerfüllung, im Gegensatz zu der vormals direkten Unterstellung der lokalen Verwaltung unter die Zentralgewalt. Dezentralisierung und Reform der öffentlichen Verwaltung gingen und gehen also in den betroffenen Ländern Hand in Hand. Der nächste Schritt ist in der Regel die Reform des civil service, weg von politischer Loyalität hin zu neutralen Mitarbeitern, die an Recht und Gesetz gebunden sind. Dabei sind veränderte Gesetze zentral, aber nur der erste Schritt. Entsprechende Implementierung mit klaren Karriereschritten und Training der öffentlichen Bediensteten muss folgen. Für diese umfassenden Reformschritte stellen internationale Institutionen wie die Weltbank, EBRD, UNDP, bilaterale Institutionen und auch die EU-Mittel zur Verfügung.\par
Die Erfahrungen der Länder der östlichen EU-Erweiterung zeigen, dass die Veränderungen auf allen oben erwähnten Ebenen gleichzeitig angegangen werden müssen. Dabei gab es in den Ländern durchaus Unterschiede in der Entwicklung. In Ungarn und Polen wurden politische und institutionelle Veränderungen anfänglich in großem Tempo vorgenommen, danach jedoch dauerte der Prozess der Umsetzung fast eine Dekade. In Bulgarien und Lettland führten die revolutionären Ereignisse zu Unabhängigkeit und neuer Verfassung, doch die Reform des öffentlichen Sektors wurde vernachlässigt. Nach mehreren Jahren der Stagnation wurden die Gebietsreform und die Modernisierung der lokalen Verwaltungen erst Ende der 90er Jahre begonnen. In einer dritten Gruppe von Ländern (vor allem Kroatien und Slowakei) begannen die wesentlichen Strukturreformen nicht nur spät, sondern in den ersten zehn Jahren der Transformation wurden keine wesentlichen Reformen durchgeführt (vgl. \cite{petzen} : 19).  \par

Studien zur EU-Osterweiterung zeigen, dass je stärker die nationalen administrativen Strukturen und Traditionen waren, desto mehr Resistenz gegenüber Anpassungsdruck im Zusammenhang mit der EU-Erweiterung entstand (vgl. \cite{knill01, goetz01b}). Im Wesentlichen hat in diesen Fällen Policy-Transfer stattgefunden, der innerhalb der bestehenden Strukturen umgesetzt werden konnte, ohne die Verwaltungsstrukturen entscheidend zu beeinflussen. Ein einheitliches Modell der öffentlichen Verwaltung ist nicht entstanden. Eine Studie zu Ungarn fasst zusammen: „Of course, the national administrative culture is not untouchable or completely intact towards external impacts. But it is also true that it has considerable ability to make processes slow or resist new behavioural patterns and untested ideas and practices” (\cite{szente} : 59). Zu einer insgesamt kritischen Einschätzung der externen Unterstützungsmaßnahmen und ihrer Auswirkungen im Rahmen der Osterweiterung der EU kommt Coombes, der von häufig unklaren oder gar widersprüchlichen Zielvorstellungen der Programme spricht. Oft würden diese nicht der Realität in den Empfängerländern gerecht. Dennoch käme es durch die Projekte zu sogenannten „trickle-down“ Effekten. In diesem Sinne ist der Wissenstransfer, auch durch EU-Förderprogramme, bei aller angebrachten Kritik positiv zu bewerten: „…there is usually some, more or less hidden, indirect benefit of knowledge transfer and learning – albeit in aspects not specifically projected by donors – which enhance the value of human capital in the recipient countries” (\cite{coombes} : 6). \par

In eine ähnliche Richtung geht Brusis’ Beobachtung, „dass sich im Bereich der Verwaltungsreform viele vage und doppeldeutige EU-Signale beobachten lassen, da sich die demokratische und administrative Konditionalität überlagerten“ (\cite{brusis09} : 99). Während Fälle beschrieben werden, in denen sich Regierungen auf EU-Erwartungen berufen haben, um eigene Reformprojekte zu realisieren, gibt es auch die so genannte „Potemkinsche Harmonisierung“, bei der formale Strukturen aufgebaut wurden, um den Forderungen der EU nachzukommen, diese aber wenig oder keinen Einfluss auf die nationalen ‚outcomes’ hatten (vgl. \cite{schsed05a} : 17). Obwohl einige sektorale Ansätze sich mit der Reform der öffentlichen Verwaltung beschäftigten, gab es im Zusammenhang mit der Ost-Erweiterung in dieser Hinsicht keine systematische Vorgehensweise seitens der EU. „There were no systematic attempts at benchmarking key public administration reform objectives, no formal mechanism for disseminating good or best practices, and donor coordination fell short of expectations” (\cite{summa} : 21). Eine andere Sichtweise geht davon aus, dass die EU die falschen Anreize gegeben habe mit ihrer Konzentration auf die Berichterstattung zum Fortschritt der Übernahme des Acquis. „The detailed requirements of the Commission in the field of governance and the related conditionality created a relationship where the EC became the sole principal (instead of domestic publics or their representatives) and the government its agent. Reforms were not driven by impact evaluations, but by the need to satisfy the pressing bureaucratic reporting needs for the EC regular monitoring reports. ‘One-off special efforts” to reach certain EU deadlines and ‘islands of excellence’ units …prevailed while sound, system-building administrative reform was neglected” (\cite{mungiu} : 17). \par

Verheijen identifiziert im Hinblick auf die Verwaltungsentwicklung der Länder der letzten Erweiterungswelle ein „mixed picture of overall setbacks“. Zentrale Elemente der öffentlichen Verwaltung werden weiterhin als problematisch eingestuft, allen voran der civil service. So konstatiert er eine große Bandbreite unterschiedlicher Reformen, die nicht auf einem allgemein anerkannten Modell beruhen. Auch die weiterhin zum Teil starke Politisierung und damit sich häufig ändernden Reformorientierungen in den Ländern der letzten Erweiterungswelle tragen zu dem Bild bei. „The lack of underlying consensus has been clearly visible in the merry-go-round of reforms in the civil service in both Poland and Hungary” (\cite{verheijen06} : 48). \par

In einer Studie zur Reform des civil service in den Ländern der östlichen EU"=Erweiterung werden als zentrale Elemente strukturelle Probleme in nach-kommunistischen Verwaltungen angeführt: 

Fast vollständige Politisierung der öffentlichen Verwaltung. Die Vorbereitung von policies und Gesetzen sowie die Umsetzung der policies waren mit großem politischem Einfluss der Zentralregierung und der Regierungspartei verbunden. 
Ethische Prinzipien wie „Neutralität“ und „Unbestechlichkeit“ wurden oft gebrochen aufgrund der starken Politisierung.
Fehlende Mobilität im civil service. Darüber hinaus waren Karrierechancen und Beförderung eng mit der Anpassung an politische Präferenzen verbunden.
Ein hoher Grad an Fragmentierung von Verantwortlichkeit in der Personalpolitik. 
In den meisten Fällen gab es keine Stelle, die für Rekrutierung zuständig war, sondern die entsprechenden Minister hatten diese Kompetenz.
Schlechtes Image des Beamtentums und schlechte Verdienstmöglichkeiten.
(vgl. \cite{bosdem} : 3) \par

Als problematisch benennt Verheijen die Umsetzung nicht sinnvoll zugeschnittener Reformmodelle, insbesondere wenn die Angst bestand, dass die EU nicht ausreichende administrative Kapazität als Grund für einen Aufschub der EU-Mitgliedschaft anführen könnte. Besonders in der Slowakei als „Spätreformierer“ dürfte dies zugetroffen haben. Wesentliche Gründe für Rückschritte in der administrativen Reformagenda waren Uneinigkeit über die Richtung der Reformen (Polen, Estland, evtl. Ungarn). Auch ein Mangel an Interesse unter Politikern an einer Reform der öffentlichen Verwaltung wirkte sich negativ aus, besonders in der Tschechischen Republik. Hier ging man davon aus, ökonomische und politische Erfolge würden verhindern, dass zum Thema Verwaltungsreformen seitens der EU Druck ausgeübt werden konnte (vgl. \cite{verheijen06} : 43). \par

In Lettland und Litauen waren nach anfänglichem Experimentieren mit Verwaltungsreformen in den 1990er Jahren ab 2000 drei wesentliche Voraussetzungen für erfolgreiche Reformen erfüllt. Es gab:
\begin{enumerate}
\item  eine weitgehende Übereinstimmung über die Richtung der Reformen,
\item eine angemessen große Zahl reformorientierter Entscheidungsträger mit entsprechender Motivation, 
\item eine relative kleine Gruppe von Verwaltungsreformern, die das Vertrauen und die Unterstützung wechselnder politischer Gruppen sichern konnten (vgl.  \cite{verheijen06} : 44).
\end{enumerate}

Die Erfahrung der EU-Erweiterung mit der Aufnahme Bulgariens und Rumäniens 2007 als letzte Länder der Osterweiterung macht deutlich, dass Konditionalität in diesen beiden Fällen strikter angewendet wurde als bei früheren Aufnahmen. Auch die administrative Kapazität, als wichtige Voraussetzung für die Übernahme des Acquis communautaire, gelangte stärker in den Blick. Es wurden Monitoring-Maßnahmen eingeführt und bisher nicht bekannte Verzögerungsklauseln aufgenommen. Diese Verzögerungsklauseln für Bulgarien und Rumänien beinhalteten die Möglichkeit einer Verschiebung der Aufnahme um ein Jahr, fanden aber keine Anwendung (vgl. \cite{phinnemore} : 2). Stattdessen entwickelte die EU-Kommission einen „Mechanism for Cooperation and Verification of Progress in the areas of judicial reform and the fight against corruption, money-laundering and organized crime” (CVM). In diesem Rahmen mussten die beiden Länder Bulgarien und Rumänien zu speziell festgelegten Eckpunkten regelmäßig an die EU-Kommission berichten (vgl. \cite{summa} : 24). 
\par
Einerseits zeigen Erfahrungen in den Ländern Osteuropas, dass externe Anreize und finanzielle Unterstützung durchaus zu einer Modernisierung des Verwaltungsapparates führen können. Andererseits wird auch deutlich, dass der Druck der EU nicht ausreichte, um erfolgreiche Verwaltungsreformen auf den Weg zu bringen, da:
\begin{itemize}
\item der Druck der EU hinsichtlich der administrativen Kapazität nicht effektiv ist, 
\item der Zeitrahmen bis zur Aufnahme nicht ausreichend war für eine effektive und umfassende Umsetzung einer Verwaltungsreform (vgl. \cite{verheijen06} : 44).
\end{itemize}
Angesichts des nur indirekten Mandates der EU, Verwaltungsreformen einzufordern, ist der überwiegend negative Befund für die letzte Welle der Beitrittsländer hinsichtlich ihrer administrativen Kapazitäten nicht überraschend.\par
Auch nach der Aufnahme in die EU bleibt die administrative Kapazität der Länder Osteuropas problematisch. Obwohl es einige positive Beispiele in bestimmten Bereichen gibt, die unter großem Ressourceneinsatz zustande kamen, ist das übergeordnete Bild ernüchternd. Ein spezifisches Problem ergibt sich daraus, dass die Länder oft nicht in der Lage waren, die EU-Strukturfonds angemessen zu managen. Besonders die geringe Absorptionsquote der EU-Gelder im Zeitraum 2004 bis 2006 gab Anlass zur Sorge. Am Beispiel von Ungarn und Polen, die ungeachtet ihres Status als „best pupils of enlargement“ nicht das gesamte ihnen zur Verfügung stehende PHARE-Budget nutzten, konstatierte die EU-Kommission, „the lack of administrative capacity and political will as well as the poor involvement of civil society in shaping the reforms.“ (\cite{tulmets06} : 6).\par
In einer Untersuchung zu den neuen EU-Mitgliedsländern stellt Verheijen eine hohe Korrelation fest zwischen Ländern, die ihre öffentliche Verwaltung substanziell verbesserten und dem allgemeinen Funktionieren innerhalb der EU. Solche Staaten, wie Lettland, Litauen und mit etwas Abstand die Slowakei haben in einer Studie mit performance-indicators für die öffentliche Verwaltung der Weltbank wesentlich besser abgeschnitten als Länder mit geringerer Reformorientierung, wie Polen und Ungarn. „While the former states (Lettland, Litauen, Slowakei C.V.) made progress both on the introduction of performance-based public management systems and, to a lesser degree, on civil service reform, virtually no progress has been made on these aspects of administrative governance in the latter (Polen, Ungarn C.V.)” (\cite{verheijen09} : 4).\par
In einem internen Diskussionspapier der EU\footnote{Das Papier wurde der Verfasserin dieser Arbeit vom Autor zur Verfügung gestellt.} werden weitere Gedanken zur relativen Erfolglosigkeit der Unterstützung von Verwaltungsreformen im Zuge der EU-Osterweiterung entwickelt. Als problematisch benannt werden Geberdominanz, fehlender politischer Wille der nationalen Akteure und geringe Absorption der bereitgestellten Mittel: „Mainstream evidence shows that in the past a considerable part of the support for horizontal public administration reform largely failed as it was supply-driven without sufficient consideration of the demand-side issues of political commitment, change management and absorption capacity” (\cite{apelb07} : 2). \par
Den begrenzten Effekt der EU-Erweiterung auf die öffentliche Verwaltung in den neuen Mitgliedsländern der EU erklären Lippert und Umbach mit dem Umstand, dass Verwaltungsmodernisierung nicht Teil des Acquis communautaire ist und somit keine starken Interventionen seitens der EU stattfinden können. „Die Europäische Union, die in ihrer Erweiterungspolitik frühzeitig der Normalisierungsthese im Hinblick auf die Entwicklung der Verwaltung in den Ländern Mittel- und Osteuropas anhing, hat sich zunehmend auf die Rolle eines Wächters und nachfrageorientierten Assistenten und Vermittlers von Lösungsangeboten aus den EU-15 beschränkt, statt die Rolle einer aktiven Entscheidungsinstanz oder eines Exporteurs eines idealtypischen EU-kompatiblen Verwaltungsmodells zu beanspruchen“ (\cite{lipumb04} : 73).
\par
In die gleiche Richtung geht die Analyse eines Experten zur Reform der öffentlichen Verwaltung innerhalb der EU-Kommission: „Enlargement is rightly largely seen as a ‘success story’ but the benefits which citizens and enterprises could have enjoyed from their countries joining the EU would probably have been greater if the support to horizontal PAR had been more effective, more timely, more coordinated and more strategic” (\cite{apelb09} : 5).\par
Verbindet man die theoretischen Ansätze zur EU-Erweiterung mit den praktischen Erfahrungen, so zeigt sich, dass die Praxis der EU in Bezug auf die Verwaltungsentwicklung im Erweiterungsprozess vor einem Dilemma steht. Einerseits wird seitens der EU von politischer Konditionalität gesprochen. Nur mit entsprechender administrativer Kapazität ist die Übernahme des Acquis communautaire umzusetzen. Andererseits hat die EU kein Modell der öffentlichen Verwaltung, an dem sich die Beitrittsländer orientieren könnten. Alle Versuche in diese Richtung, mündeten bisher nicht in verbindlichen Modellen, was auch mit der Hoheit der einzelnen Länder in der Verwaltungsgestaltung erklärbar ist. Folgerichtig ist die Ausgestaltung der Verwaltung oder Verwaltungsmodernisierung kein Kapitel des Acquis. Um weiteres Licht in diese von der Forschung bisher weitgehend unausgeleuchteten Zusammenhänge zu bringen, soll ein Blick auf die Geschichte der Beziehungen der EU zu den Ländern des Westlichen Balkans bei der weiteren Spurensuche helfen.

\section{Institutionelle Beziehungen der EU zu Südosteuropa}
Anders als zu den Ländern der Osterweiterung, die in der Zeit des Kalten Krieges unter dem Einfluss der Sowjetunion standen und kaum Beziehungen zur EU hatten, kann die EU auf schon frühe Beziehungen zu Jugoslawien zurückblicken. Finanzielle Unterstützung für Jugoslawien in Form von direkten Hilfen durch die Europäische Wirtschaftsgemeinschaft (EWG) begannen ab Anfang der 1960er Jahre. Wirtschaftliche Beziehungen der EWG zu Südosteuropa gehen zurück in die 1970er Jahre, als Jugoslawien ein Handelsabkommen mit der EWG schloss für eine schrittweise Handelsliberalisierung und eingeschränkte Kreditmöglichkeiten. Weiterhin ermöglichten entsprechende Abkommen, dass ab den 1970er Jahren sogenannte Gastarbeiter in den Länder der EWG arbeiten konnten (vgl. \cite{weithmann} : 447). Dennoch führten die Entwicklungen der 1990er Jahre im ehemaligen Jugoslawien mit einem zerfallenden föderalen Staat und schneller bilateraler Anerkennung von Teilstaaten als eigenständigen Ländern (Kroatien und Slowenien) sowie Kriege in der Region zunächst zu einer Art Bewegungsunfähigkeit der EU in Bezug auf Südosteuropa (vgl. Inotai 2007:21)TODO:find.\par
Während die osteuropäischen Staaten in den 1990er Jahren mit verschiedenen Mechanismen auf eine Integration in die EU vorbereitet wurden, spitzten sich in dieser Zeit auf dem Westbalkan ethnische Konflikte zu, Kriege wurden ausgetragen und Staaten zerfielen. Das Ergebnis waren neue schwache Staaten, zum Teil unter internationaler Verwaltung. Dennoch blieb die EU-Orientierung ein Hauptziel praktisch aller Regierungen der Länder des Westbalkans. Und auch die internationale Staatengemeinschaft hatte ein Interesse daran, den Westbalkan in den europäischen Kontext einzubinden. Ein „Regionalkonzept“ diente als Rahmen für die Entwicklung der Beziehungen der EU zu den Ländern der Region. Die Ziele dieses Regionalkonzepts von 1996 waren die Unterstützung der Friedensabkommen von Dayton/Paris sowie die Schaffung einer Zone politischer Stabilität.\par
Der Stabilitätspakt für Südosteuropa (1999) zielte auf Ablösung der bisherigen Politik der Krisenintervention in der Region durch eine umfassende und langfristige Konfliktpräventionsstrategie. Der Stabilitätspakt bestand aus einer politischen Verpflichtungserklärung und einer Rahmenvereinbarung zur internationalen Kooperation in Südosteuropa zwischen mehr als 40 Staaten, Organisationen und regionalen Zusammenschlüssen. Am Stabilitätspakt beteiligt waren u.a. die Vereinten Nationen, die OSZE, die Europäische Kommission, alle EU-Mitgliedsländer, die Länder Südosteuropas aber auch die USA, Japan, Kanada und Russland. Ziel war es, die Länder in Südosteuropa in der Demokratisierung und ökonomischen Entwicklung zu unterstützen\footnote{\url{http://www.stabilitypact.org/about/default.asp} (Aufgerufen: 1.9.2012).}. Ein Sonderkoordinator führte den Vorsitz beim Regionaltisch, dem wichtigsten politischen Instrument des Stabilitätspaktes. Diesem untergeordnet waren drei Arbeitstische:
\begin{enumerate}[label=Tisch {\roman*}:,align=left,  leftmargin=*]
\item Demokratisierung und Menschenrechte
\item Wirtschaftlicher Wiederaufbau, Zusammenarbeit und Entwicklung
\item Sicherheitsfragen
\end{enumerate}
Innerhalb des Arbeitstisches I beschäftigte sich eine Arbeitsgruppe (Task Force) „Good Governance“ unter Vorsitz des Europarats vorrangig mit der Entwicklung der Kommunalverwaltungen, mit der Einsetzung von Ombudspersonen sowie der Reform der öffentlichen Verwaltung (vgl. \cite{calic01} : 9). Die Bedeutung des Stabilitätspaktes hat abgenommen in dem Maße, wie die EU ihre Südosteuropapolitik weiterentwickelte.

\subsection{Der Stabilisierungs- und Assoziierungsprozess}
Im Rahmen des Stabilitätspaktes legte die Europäsche Union 1999 einen Stabilisierungs- und Assoziierungsprozess (SAP) für Südosteuropa auf. Dieser hatte folgende Ziele für die Region:
\begin{itemize}
\item Konzipierung von Stabilisierungs- und Assoziierungsabkommen mit der Aussicht auf einen
Beitritt zur Europäischen Union, wenn die Kopenhagener Kriterien erfüllt sind;
\item Ausbau der wirtschaftlichen und handelspolitischen Beziehungen zu dieser Region sowie innerhalb der Region;
\item Erhöhung der wirtschaftlichen und finanziellen Hilfe;
\item Unterstützung der Demokratisierung, der Zivilgesellschaft und des Aufbaus von Institutionen;
\item Zusammenarbeit im Bereich Justiz und Inneres;
\item Intensivierung des politischen Dialogs (vgl. \cite{marwedel} : 9).
\end{itemize}

Ausgangspunkt einer erstmals konkreteren EU-Perspektive für die Staaten des Westbalkans war die Tagung des Europäischen Rates im Juni 2000 in Feira: „Ziel ist, die [westlichen Balkanstaaten] im Wege des Stabilisierungs- und Assoziierungsprozesses, des politischen Dialogs, der Liberalisierung des Handels und der Zusammenarbeit im Bereich Justiz und Inneres so weit wie möglich in den Strom der allgemeinen politischen und wirtschaftlichen Entwicklung Europas einzubeziehen. Alle betroffenen Länder sind potenzielle Bewerber für den Beitritt zur EU“ (\cite{euko05} : 3). Im November 2000 in Zagreb wurde der SAP-Prozess offiziell eingerichtet und drei wesentliche Instrumente wurden eingeführt: 1) bilaterale begünstigte Handelsabkommen, 2) das CARDS\footnote{„Community Assistance for Reconstruction, Democratization and Stabilization“ (siehe auch \ref{subsec:Kompetenz der EU bezüglich der Verwaltung})}-Programm zur finanziellen Unterstützung ausgewählter Projekte und 3) die Unterzeichnung von Stabilisierungs- und Assoziierungsabkommen bei Erfüllen bestimmter Kriterien. Die erfolgreiche Umsetzung des SAP eröffnete den Weg zu einem Stabilisierungs- und Assoziierungsabkommen (SAA) und darauf aufbauend zu der Beantragung der EU-Mitgliedschaft (vgl. Inotai 2007:25)TODO:find. \par
Bestätigt wurde die Beitrittsperspektive der Länder des Westlichen Balkans im Juni 2003 in Thessaloniki mit der „Agenda von Thessaloniki für die westlichen Balkanstaaten: Auf dem Weg zur Europäischen Integration“. Die nächste Stufe der institutionellen Kooperation zwischen der EU und den Ländern des Westbalkans waren Stabilisierungs- und Assoziierungsabkommen. Diese Abkommen haben eine ähnliche Funktion wie die Europäischen Partnerschaften mit den osteuropäischen Beitrittskandidaten vor deren Aufnahme in die EU (vgl. Inotai 2007: 30)TODO:find.\par
Die Stabilisierungs- und Assoziierungsabkommen (SAA) mit den Ländern des Westlichen Balkans sind mit Rechten und Pflichten verbunden und benennen konkrete Reformziele für die Kandidaten- und potenziellen Kandidatenländer. Finanzielle Unterstützung durch die EU ist ebenfalls wichtiger Bestandteil der Heranführungshilfe (vgl. \cite{euko07} : 9). Die SAAs, die mit den Ländern gesondert abgeschlossen wurden, bestehen aus verschiedenen Teilen allgemeinerer Art, z.B. zu politischem Dialog und regionaler Kooperation, und konkreteren Anweisungen zur Übernahme des Acquis und der rechtlichen Angleichung (vgl. \cite{marwedel} : 24).\par
Mit der Einführung des SAP wurde ein länderspezifisches Monitoring eingeführt, das die Einhaltung der Bedingungen des Stabilisierungs- und Assoziierungsprozesses durch die einzelnen Länder regelmäßig dokumentiert\footnote{\url{http://europa.eu/legislation_summaries/enlargement/western_balkans/r18003_de.htm} (Aufgerufen: 19.10.2012).}.\par
Im Gegensatz zu früheren Aufnahmen von Ländern in Gruppen verweist der Europäische Rat im Juni 2005 auf eine länderspezifische Betrachtung für die Länder des Westlichen Balkans und macht deutlich, dass „die Fortschritte der einzelnen Länder auf dem Weg zur europäischen Integration unter Berücksichtigung der Entwicklung des Besitzstands von ihren Bemühungen um die Einhaltung der Kopenhagener Kriterien und der im Stabilisierungs- und Assoziierungsprozess genannten Auflagen abhängen“ (\cite{euko07} : 9). In dem EU-Papier „Enlargement Strategy and Main Challenges 2006-2007“ wird weiter konkretisiert, dass es sich bei der Beitrittsperspektive für die Länder des Westlichen Balkans um eine mittel- bzw. langfristige Perspektive handelt. Der unterschiedliche Entwicklungsstand der Länder wird konstatiert; eine Aufnahme zum gleichen Zeitpunkt wird ausgeschlossen. Neben der politischen und ökonomischen Entwicklung wird die Kapazität der Verwaltung als wichtiger Indikator zur Beurteilung der Europa-Reife benannt (vgl. \cite{eurcom06b} : 18).\par
Seitens der EU wird die Erweiterungsstrategie auf drei wesentliche Prinzipien gegründet, die in der Strategie 2005 definiert wurden: „consolidating existing commitments towards countries engaged in the integration process, applying fair and rigorous conditions to be fulfilled by countries prior to their accession, and intensifying communication with the general public on the advantages of EU’s enlargement policy”\footnote{\url{	http://www.svez.gov.si/nc/en/splosno/cns/news/article/2028/1265/Enlargement Package 2006}, Aufgerufen: 15.7.2010.}.\par

In der folgenden Tabelle werden die wesentlichen Schritte im Prozess der bisherigen EU-Heranführung für die Untersuchungsländer im Überblick dargestellt:
\begin{table}[H]
\caption{ Status der EU-Annäherung in den Untersuchungsländern}
\begin{tabular}{|L{ 3,5cm}|L{3,5cm}|L{3,5cm}|L{3,5cm}|}\hline
&Mazedonien&
Montenegro&
Albanien\\\hline
EU**&
Handels- und Koopera\-tions\-abkommen seit 1998&
Bis 2006 zusammen mit Serbien&
Handels- und Kooperations\-abkommen seit 1992\\\hline
Europarat**&
1995&
2007&
1996\\\hline
SAAVerhandlungen begonnen*&
2000&
2005&
2003\\\hline
SAA verhandelt*&
2000&
2007&
2006\\\hline
SAA unterzeichnet*&
09.04.2001&
15.10.2007&
12.06.2006\\\hline
SAA in Kraft&
01.04.2004&
01.05.2010&
01.04.2009\\\hline
Aufnahmeantrag gestellt&
2004&
2008&
2009\\\hline
Visa-Liberalisierung&
2009&
2009&
2010\\\hline
Kandidatenstatus&
2005&
2010&
noch nicht\\\hline
\end{tabular}\\

\scriptsize{Quellen: EU website; * \cite{mus} : 12; ** \cite{brusgal} : 59. }
\end{table}


Aus dem Überblick wird ersichtlich, dass die Prozesse der Annäherung an die EU in unterschiedlichem Tempo stattfinden. Mazedonien erhielt schon 2005 Kandidatenstatus. Allerdings ist der Namensstreit mit Griechenland eine formale Hürde für weitere konkrete Schritte. Montenegro, erst seit 2006 als unabhängiges Land mit Beziehungen zur EU, stellte 2009 den Aufnahmeantrag und erhielt schon im Jahr darauf Kandidatenstatus. Albanien hat den Aufnahmeantrag 2009 gestellt. \par
Während also die Erfahrungen mit der Osterweiterung der EU deutlich machten, dass die administrative Kapazität der Beitrittsländer zentral ist, hat sich für die Vorbereitung der Westbalkanstaaten seitens der EU keine Änderung ergeben. Weiterhin ist die öffentliche Verwaltung, bzw. Verwaltungsmodernisierung kein Kapitel des Acquis. Die Annäherung der Westbalkanstaaten an die EU erfolgte auf der Basis von in den 1960er Jahren begonnenen wirtschaftlichen Beziehungen. Nach den Balkankriegen der 1990er Jahre schritt die EU ab 2000 zügig voran mit dem Angebot einer Beitrittsperspektive für die Balkanländer. Die sicherheitspolitische Konsolidierung der Region spielte dabei eine große Rolle. Die Heranführungsinstrumente der EU wurden aufbauend auf der Erfahrung mit der Osterweiterung weiterentwickelt. Im Folgenden wird ein Blick auf die einzelnen Förderinstrumente geworfen, insbesondere in ihrer Relevanz für die Verwaltungsmodernisierung.

\section{Förderprogramme der EU}
\label{sec:foerderprogramme}
Im Vorfeld des Beitrittes eines Landes zur EU stellt die EU Instrumentarien zur Unterstützung der Kandidatenländer in der Heranführungsphase bereit. Eine Reihe von speziell zugeschnittenen Programmen, aufbauend auf den Erfahrungen mit der letzten Erweiterungswelle, stehen somit für die Länder des Westbalkans zur Verfügung. Diese Unterstützungsprogramme der EU sind seit 1994 elementarer Teil einer „Heranführungsstrategie“, ebenso wie ein begleitendes Kontrollverfahren (Monitoring). In diesem Monitoring beurteilt die Kommission in jährlichen Fortschrittsberichten (Progress Reports), wie weit die einzelnen Länder bei der Erfüllung der Aufnahme-Kriterien vorangeschritten sind\footnote{Diese Fortschrittsberichte der EU werden im Internet auf der Website der DG Enlargement veröffentlicht.}. Der beschriebene Prozess bis zur tatsächlichen Aufnahme kann unter Umständen zeitlich extensiv sein, wie die Beispiele Rumänien und Bulgarien gezeigt haben, die 2007 als letzte Länder der vierten Erweiterungswelle aufgenommen wurden (vgl. \cite{scherman} : 63). 
\par
In diesem Abschnitt wird zunächst das als Trainings-Programm konzipierte Twinning-Programm als ein wesentliches Instrument zur Verwaltungsentwicklung vorgestellt. Zentral für die Aktivitäten der EU zur Heranführung der Kandidaten an den EU-Standard ist das Programm IPA, Instrument for Pre-Accession Assistance, das mit Blick auf die Verwaltungsmodernisierung dargestellt wird. Auch die Tätigkeit der OECD mit ihrem SIGMA-Programm, das die EU zu verwaltungsrelevanten Fragen der Erweiterung berät und im Auftrag der EU-Länderstudien zu Themen der öffentlichen Verwaltung erarbeitet, ist für die vorliegende Untersuchung relevant und wird vorgestellt. Inzwischen beendete EU-Programme, von denen die aktuellen Kandidatenländer profitiert haben, sind die Tätigkeit der European Agency for Reconstruction (EAR) sowie die Programme PHARE und CARDS, die jeweils bezüglich der Verwaltungsmodernisierung ebenfalls kurz dargestellt werden. \par
Die Entwicklung und Reform der Verwaltung auf dem Westlichen Balkan ist seit der Unabhängigkeit der Länder ab Anfang der 1990er Jahre außer von der EU auch von einer Reihe von internationalen Akteuren begleitet und gefördert worden. Wesentliche Institutionen sind dabei das Entwicklungsprogramm der UN (UNDP), die Weltbank, die OSZE, die EBRD und das Open Society Institute. Auch bilaterale Entwicklungshilfe wurde für den Institutionenaufbau und den Aufbau administrativer Kapazitäten eingesetzt. Diese allgemeinen Programme werden hier jedoch nicht weiter betrachtet.
\subsection{Twinning}
Im Rahmen der Heranführung von Beitrittskandidaten und potenziellen Beitrittskandidaten ist das als Twinning bezeichnete Programm eines der zentralen Elemente bei der Institutionenentwicklung. Aufbauend auf Erfahrungen in der Entwicklungshilfe, werden Experten aus Ministerien einzelner Mitgliedstaaten zu ihren jeweiligen counterparts in den Empfängerländern geschickt. Bei Twinning wird somit die Erfahrung von Experten aus den Mitgliedsländern im Bereich Verwaltung und im Justizwesen eingesetzt, um gute Beispiele an die Beitrittskandidaten weiterzugeben. Institutionelles Lernen soll erreicht werden durch Nachahmung, Imitation und Sozialisation (vgl. \cite{tulmets05}).
\par
Erfahrungen mit Twinning im Erweiterungsprozess wurden während der letzten EU-Erweiterungswelle gesammelt. „Programmes for the long-term secondment to applicant countries of experts from the administrations of the Member States must be drawn up for each applicant in the light of the needs identified, particularly in the opinions [avis]” (European Commission, 1997: 4). Finanziert wurden diese Initiativen durch das PHARE Programm, das seit 1997 ca. 30\% seines Budgets für “institution building” in den CEE-Ländern vorsah. Nach einer fünfmonatigen Vorbereitungszeit wurde Twinning offiziell im Mai 1998 eingeführt. Ziel sollte die Hilfe für Beitrittskandidaten sein bei der Etablierung einer “…modern, efficient administration that is capable of applying the acquis communautaire to the same standards as the current member states” (European Commission, 1998: zit nach: \cite{papadi} : 8). Die EU hat zwischen 1998 und 2002 ca. 700 Langzeitexperten (Twinning) für die zehn mittel- und osteuropäischen Beitrittskandidaten bereitgestellt (vgl. \cite{lipumb04} : 63). Im Westbalkan begannen Twinning-Aktivitäten im Sommer 2002 für bestimmte Programme des Institutionenaufbaus, vor allem im Bereich Justiz und Inneres (vgl. European Commission 2007: 10)TODO:find.\par

Im Gegensatz zur klassischen technischen Unterstützung (Technical Assistance – TA) sind bei Twinning die wesentlichen Elemente:
\begin{itemize}
\item die Zusammenarbeit zwischen Verwaltungen, 
\item die permanente Anwesenheit eines Beamten eines Mitgliedslandes und
\item ein Projektmanagement-Ansatz mit zu erzielenden Ergebnissen. 
\end{itemize}
Dies führte laut einer Untersuchung allerdings zu Widerstand, da die Einführung dieses Ansatzes nicht mit den Aufnahmestaaten abgesprochen war. Das Konzept Twinning, so empfand man es, war ohne ausreichende Konsultationen mit den Empfängerstaaten „aufgesetzt” worden. Einige der entsandten Beamten „had to fight hard to overcome the feeling that they were 'spies' appointed by the Commission” (\cite{coojoh} : 5). In der gleichen Untersuchung wird aber auch festgestellt, dass Twinning verschiedene durchaus sehr wichtige Begleiteffekte hatte, die nicht zu unterschätzen sind. Die Autoren der Studie nennen in diesem Zusammenhang Verhaltensänderungen, die quasi nebenbei entstanden sind, wie z.B. Verbesserungen des Managementstils, bessere Koordination und Kommunikation zwischen und innerhalb von Ministerien der Beitrittsländer, selbst wenn es sich um einfache Dinge handelte, wie kundenfreundlichere Abwicklung von Telefonanfragen. Im Zusammenhang mit dem Konzept von Lernen und Sozialisation führt Twinning im Idealfall zu individuellem, kollektivem und organisationalem Lernen in bestimmten Politikfeldern. Auch hat die Twinning-Beziehung im Rahmen des EU-Programmes zwischen einem Mitgliedsland und einem Beitrittskandidaten durchaus zu weiterer bilateraler Zusammenarbeit der beteiligten Länder geführt (vgl. \cite{coojoh} : 6.). Allerdings ist bei der Beurteilung des Programmes auch die Gefahr des „policy-shopping”, wie sie für die Länder der letzten Erweiterungswelle stattgefunden hat und dem Twinning-Programm inhärent ist, nicht zu vernachlässigen (vgl. \cite{meyersah08b} : 3).\par
Eine Studie im Auftrag der Europäischen Kommission kommt zu der Einschätzung, dass Twinning oftmals durch bürokratische Formalitäten seitens der Projektdurchführung, schwache administrative Kapazitäten und fehlende eigene Ressourcen in den Partnerländern an Grenzen kam (vgl. \cite{koenigova} : 54). Um dem Bedarf der Beitrittskandidaten nach größerer Flexibilität entgegenzukommen, wurde das sogenannte „Twinning Light“ entwickelt. Dabei ist die permanente Anwesenheit eines Beamten aus einem EU-Mitgliedsland nicht erforderlich und die finanzielle Ausstattung ist geringer (150.000 Euro im Gegensatz zu 1 Million Euro bei Twinning-Projekten).\par
In einer insgesamt kritischen Evaluierung wird davon ausgegangen, dass Erfolge von Twinning im Bereich der öffentlichen Verwaltung gering waren. Dies wird vor allem auf das Fehlen von Strategien zur Reform der öffentlichen Verwaltung zurückgeführt. „Twinning has not engaged – other than in a very few instances – with horizontal PAR. The argument is that PAR is not acquis-related. Despite this, the Twinning philosophy – of cooperation between administrations – would be ideally suited to assisting the ACs to develop better PAR strategies and practices” (\cite{coojoh} : 23).\par
Einen Überblick über die Gesamtanzahl der Twinning (TW) und klassische Technical Assistance (TA) Projekte der Jahre 2005–2008 im Westbalkan bietet die folgende Tabelle:

\begin{table}[H]
\caption{Anteil der Twinning (TW) Projekte an Technischer Hilfe (TA) insgesamt 2005–2008 nach Ländern}
\rowcolors{3}{lightgray}{lightgray} 
\scriptsize{
\begin{tabular}{|R{16mm}|R{5mm}|R{4mm}|R{7mm}|R{4mm}|R{4mm}|R{7mm}|R{5mm}|R{4mm}|R{7mm}|R{5mm}|R{4mm}|R{7mm}|R{5mm}|R{5mm}|R{7mm}|}\hline
%\multirow{2}{*}{{\bf Land}}&
%\multicolumn{3}{ |c| }{\multirow{2}{*}{{\bf 2005}}}&
%\multicolumn{3}{ |c| }{\multirow{2}{*}{{\bf 2006}}}&
%\multicolumn{3}{ |c| }{\multirow{2}{*}{{\bf 2007}}}&
%\multicolumn{3}{ |c| }{\multirow{2}{*}{{\bf 2008}}}&
%\multicolumn{3}{ |c| }{\multirow{2}{*}{{\bf Projekte insgesamt}}}\\\hline
{\bf Land}&\multicolumn{3}{ |c| }{\bf 2005}&
\multicolumn{3}{ |c| }{\bf 2006}&
\multicolumn{3}{ |c| }{\bf 2007}&
\multicolumn{3}{ |c| }{\bf 2008}&
\multicolumn{3}{ |c| }{\bf Projekte insgesamt}\\[3ex]
\hline
 &{\bf TA}&{\bf TW}&{\bf \%TW}&{\bf TA}&{\bf TW}&{\bf \%TW}&{\bf TA}&{\bf TW}&{\bf \%TW}&{\bf TA}&{\bf TW}&{\bf \%TW}&{\bf TA}&{\bf TW}&{\bf \%TW}\\
\hiderowcolors Albanien&9&2&\cellcolor[gray]{0.9}18\%&11&0&\cellcolor[gray]{0.9}0\%&4&1&\cellcolor[gray]{0.9}20\%&8&6&\cellcolor[gray]{0.9}43\%&32&9&\cellcolor[gray]{0.9}22\%\\
BiH&16&3&\cellcolor[gray]{0.9}16\%&22&4&\cellcolor[gray]{0.9}15\%&23&5&\cellcolor[gray]{0.9}18\%&n/a&4&\cellcolor[gray]{0.9}n/a&61&16&\cellcolor[gray]{0.9}21\% \\
Kroatien&19&19&\cellcolor[gray]{0.9}50\%&16&12&\cellcolor[gray]{0.9}43\%&9&20&\cellcolor[gray]{0.9}69\%&n/a&9&\cellcolor[gray]{0.9}n/a&44&60&\cellcolor[gray]{0.9}58\% \\
Mazedonien&15&4&\cellcolor[gray]{0.9}21\%&20&2&\cellcolor[gray]{0.9}9\%&16&2&\cellcolor[gray]{0.9}11\%&14&1&\cellcolor[gray]{0.9}7\%&65&9&\cellcolor[gray]{0.9}12\% \\
Kosovo&0&0&\cellcolor[gray]{0.9}0\%&0&0&\cellcolor[gray]{0.9}0\%&18&2&\cellcolor[gray]{0.9}10\%&23&8&\cellcolor[gray]{0.9}26\%&41&10&\cellcolor[gray]{0.9}20\% \\
Montenegro&0&0&\cellcolor[gray]{0.9}0\%&6&0&\cellcolor[gray]{0.9}0\%&8&3&\cellcolor[gray]{0.9}27\%&4&4&\cellcolor[gray]{0.9}50\%&18&7&\cellcolor[gray]{0.9}28\% \\
Serbien&26&7&\cellcolor[gray]{0.9}28\%&19&4&\cellcolor[gray]{0.9}17\%&25&6&\cellcolor[gray]{0.9}19\%&8&5&\cellcolor[gray]{0.9}38\%&78&25&\cellcolor[gray]{0.9}24\% \\
Türkei&25&7&\cellcolor[gray]{0.9}22\%&30&13&\cellcolor[gray]{0.9}30\%&25&13&\cellcolor[gray]{0.9}34\%&29&12&\cellcolor[gray]{0.9}29\%&109&45&\cellcolor[gray]{0.9}29\% \\
\showrowcolors Insgesamt&110&45&29\%&124&35&22\%&128&52&29\%&87&50&36\%&448&181&29\% \\\hline
\end{tabular}
}\\
\scriptsize{Quelle: \cite{epec11} : 5.}
\label{tab:AnteilTwinning}
\end{table}

Aus dieser Tabelle ist ersichtlich, dass in jedem der drei Untersuchungsländer Albanien, Montenegro und Mazedonien die Gesamtzahl der Twinning-Projekte für die Jahre 2005–2008 zusammen jeweils bei unter 10 Projekten pro Land lag.\par
Wenn man die Verteilung der Twinning-Projekte per Sektor in der folgenden Grafik betrachtet, wird ersichtlich, dass Twinning für den gesamten Sektor Justice and Home Affairs ein Viertel der durchgeführten Projekte ausmacht. PAR-Projekte sind in diesem Sektor eingeschlossen, aber nicht separat aufgeführt. Es wird deutlich, dass Twinning, welches als optimales Instrument besonders auch für die Verwaltungsentwicklung angesehen wird, zumindest bis 2008 für die Untersuchungsländer eher zurückhaltend eingesetzt wurde.
\begin{figure}[H]
  \centering
\caption{Verteilung der Twinning-Projekte (inkl. Twinning light) nach Sektor}
  \includegraphics[width=4.5in]{Material/TwinningProjects}\\
\scriptsize{Quelle: \cite{epec11} : 8}
\end{figure}
Aus dieser Grafik wird ersichtlich, dass der Anteil von Justice und Home Affairs 25\% an allen Twinning- und Twinning-Light-Projekten ausmachte. Überträgt man dies auf die in den Untersuchungsländern durchgeführten Projekte für die Jahre 2005-2008 (\ref{tab:AnteilTwinning}), kommt man auf maximal 3 Projekte pro Untersuchungsland, die dem Bereich Justice and Home Affairs zugeordnet sind. Es kann davon ausgegangen werden, dass es sich nicht in allen Fällen um Projekte zur Verwaltungsmodernisierung handelte, da auch die Modernisierung des Justizsystems unter diese Kategorie fällt. Für die Untersuchungsländer kann also festgehalten werden, dass Twinning, eines der Instrumente, das für Verwaltungsmodernisierung als besonders angemessen eingeschätzt wird, zumindest für die Zeit von 2005-2008 nur in geringem Umfang zum Einsatz kam.
\subsection{IPA – Instrument für Pre-Accession Assistance}
Das Programm IPA ersetzt die bis 2006 geltenden Instrumente PHARE, ISPA\footnote{ISPA: Instrument for Structural Policies for Pre-Accession.}, SAPARD\footnote{SAPARD: Special Accession Programme for Agriculture \& Rural Development.}, Heranführungshilfe für die Türkei und CARDS. Damit ist IPA seitdem das zentrale Instrument der Heranführungshilfe für die Beitrittsländer und die potenziellen Beitrittsländer in dem Prozess der Angleichung an die Standards der Europäischen Union zur Erlangung der Beitrittsreife.\par

Insbesondere soll das Instrument dazu beitragen:
\begin{itemize}
\item „die Demokratie und Rechtsstaatlichkeit zu stärken,
\item die öffentliche Verwaltung zu reformieren,
\item Wirtschaftsreformen durchzuführen,
\item die Menschen- und Minderheitenrechte zu achten,
\item die Gleichstellung der Geschlechter zu achten,
\item die Entwicklung der Zivilgesellschaft voranzubringen,
\item die regionale Zusammenarbeit sowie Versöhnung und Wiederaufbau zu fördern“ (Senatskanzlei Berlin, 2008). 
\end{itemize}
Grundlage für die Entwicklung von Projekten im Rahmen des IPA-Programmes sind folgende Dokumente:
\begin{itemize}
\item die Erweiterungsstrategie der Europäischen Kommission;
\item die jährlichen Fortschrittsberichte von der Europäischen Kommission. Für jedes (potenzielle) Beitrittsland von der Europäischen Kommission erstellt;
\item die Accession Partnerships (vgl. \cite{eurcom11a}).
\end{itemize}
Auf Basis dieser Dokumente werden mittelfristige Planungsdokumente, sog. Mehrjahresprogramme (über 3–4 Jahre), für jedes Land bzw. für mehrere Länder erstellt, die eine grobe Mittelaufteilung ausweisen, und je nach den Bedürfnissen jedes einzelnen Empfängerlandes für die fünf Komponenten des IPA-Instruments:
\begin{enumerate}[label=IPA {\roman*}:,align=left,  leftmargin=*]
\item Übergangshilfe und Institutionenaufbau
\item  grenzübergreifende Zusammenarbeit
\item regionale Entwicklung (nur für Beitrittsländer)
\item Entwicklung der Humanressourcen (nur für Beitrittsländer)
\item ländliche Entwicklung.(nur für Beitrittsländer)
\end{enumerate}
Die Komponenten I und II werden durch die Kommission selbst oder durch die Delegationen der Kommission in den Empfängerländern verwaltet. Die Komponenten III bis V stehen nur den Beitrittsländern zur Verfügung und werden nach den Prinzipien der Strukturfondsförderung in den Empfängerländern mit den entsprechenden Verwaltungsstrukturen abgewickelt. Verwaltungsaufbau und Verwaltungsmodernisierung fallen unter Institutionenaufbau, also IPA I, und können durch das IPA-Programm in den drei Untersuchungsländern gefördert werden.\par
Neben der Förderung von Projekten in den einzelnen Ländern können auch länderübergreifende Programme im Rahmen von IPA umgesetzt werden. Aus der folgenden Tabelle wird die Finanzplanung pro Sektor für diese länderübergreifenden Programme deutlich:
\begin{table}[H]
\caption{Mehrjähriger indikativer Finanzrahmenplan 2011–2013, länderübergreifend.}
\small{
\begin{tabular}{|L{56mm}|R{25mm}|R{25mm}|R{25mm}|}\hline
\multicolumn{4}{ |l| }{{\bf Indikative finanzielle Zuwendung pro Sektor (in Millionen EUR)}}\\\hline
&{\bf2007–2010}&{\bf 2010–2013}&\\\hline
Justiz und Innere Angelegenheiten, inkl. Grundrechte und vulnerable groups &285&24&4,61\%\\\hline
Öffentliche Verwaltung&60,92&38,5&7,39\%\\\hline
Zivilgesellschaft&40,5&30&5,76\%\\\hline
Entwicklung des privaten Sektors&123,9&70&13,44\%\\\hline
Transport und Energie Infrastruktur, inkl. Nukleare Sicherheit&124,45&108&20,73\%\\\hline
Umwelt und Klimawandel&16,3&17&3,26\%\\\hline
Soziale Entwicklung &71,98&96,5&18,52\%\\\hline
Andere Investitionen&173,86&96&18,43\%\\\hline
Reserve&0&40,97&7,86\%\\\hline
{\bf Gesamt}&{\bf 640,41}&{\bf 502,97}&{\bf 100\%}\\\hline
\end{tabular}
}\\
\scriptsize{Quelle: \cite{eurcom11b} : 15 (eigene Übersetzung aus dem Englischen)}
\end{table}

Aus dieser Tabelle wird ersichtlich, dass für regionale, also länderübergreifende Programme im Bereich der öffentlichen Verwaltung für den Zeitraum 2011–13 Mittel in Höhe von 24 Millionen Euro zur Verfügung stehen. Diese werden im Balkan vor allem für die Unterstützung der regionalen Trainingseinrichtung für die öffentliche Verwaltung verwendet, die Regional School for Public Administration (RESPA), die 2010 in Montenegro eröffnet wurde.\par

„Gemäß den in den Mehrjahresprogrammen festgelegten Prioritäten werden für jedes Land in jährlichen Programmen konkret die Ziele, die Interventionsbereiche, die erwarteten Ergebnisse, die Verwaltungsverfahren und der für die Finanzierung vorgesehene Gesamtbetrag, eine Kurzbeschreibung der Art der zu finanzierenden Vorhaben, Angaben über die vorgesehenen Beträge je Vorhaben und ein vorläufiger Durchführungszeitplan festgelegt. Sobald die Programme im nicht-öffentlichen IPA-Programmausschuss durch die Vertreter der EU-Mitgliedstaaten genehmigt sind, erfolgen die Ausschreibungen“\footnote{\url{http://ec.europa.eu/enlargement/pdf/countries/ipa_miff_081106_en.pdf} (Aufgerufen 11.12.2012)}.

Die Förderung durch IPA kann unterschiedliche Formen annehmen; dies sind vor allem:
\begin{itemize}
\item Zuschüsse zu öffentlichen Investitionen;
\item Unterstützung von Projekten der Zivilgesellschaft;
\item Twinning;
\item Direkte Budgethilfe (in Ausnahmefällen und unter Überwachung);
\item Technische Hilfe (\cite{epec11} : 2).
\end{itemize}
Für den Zeitraum 2007–2013 sind insgesamt Mittel in Höhe von 11,565 Milliarden. Euro vorgesehen (\cite{senatskanzlei}).\par


Für den Bereich Institutionenbildung, unter den die Verwaltungsmodernisierung fällt sind folgende Ausgaben in den Untersuchungsländern vorgesehen. 
\begin{table}[H]
\caption{ IPA 2007-13 Übergangshilfe und Institutionenaufbau (in Euro)}
\small{
\begin{tabular}{|L{12mm}|R{16mm}|R{16mm}|R{16mm}|R{16mm}|R{16mm}|R{16mm}|R{16mm}|}\hline
&2007&
2008&
2009&
2010&
2011&
2012&
2013\\\hline
Maze\-donien&
41.641.613&
41.122.001&
39.328.499&
36.317.068&
28.803.410&
28.207.479&
27.941.228\\\hline
Monte\-negro&
27.490.504&
28.112.552&
28.632.179&
29.238.823&
29.843.599&
30.446.471&
30.996.035\\\hline
Alba\-nien&
54.318.790&
62.117.756&
71.377.079&
82.711.421&
84.301.650&
85.987.683&
87.446.037\\\hline
\end{tabular}
}\\

\scriptsize{Quelle: \cite{eurcom09b} }
\end{table}
In einer Studie zur Implementierung von IPA Funds kommen die Autoren eines Think Tank in Mazedonien zu dem Ergebnis: „So far, it has been a general rule that the key roles in IPA program management for WB countries are fulfilled jointly by the central governments and the EU delegations to these countries. It has been observed that the participation of local authorities and Civil Society Organisations (CSOs) in the process of designing IPA priorities and drafting the national or local strategic documents has been limited. All WB candidate countries have encountered some common difficulties in dealing with IPA rules” (\cite{eurmov10} : 4). Die nicht ausreichende Einbeziehung der lokalen Akteure und vor allem der Zivilgesellschaft in den Empfängerländern wird als Problem identifiziert. Dies beeinträchtigt die Identifikation mit den Projekten, eine „local ownership“ wird dadurch erschwert. Problematisch wird auch das Vorherrschen der englischen Sprache im Rahmen der IPA-Planung empfunden, was wiederum ebenfalls eine Reihe von Akteuren in den Empfängerländern ausschließt.\par
Eine Studie zur Effektivität des IPA-Instrumentes, im Auftrag der EU durchgeführt, konstatiert die Erfolge des Instruments vor allem für Bereiche, die konkret im Acquis geregelt sind: „Effectiveness was judged to be strongest in those areas where actions are related to the alignment/adoption of the acquis, notably where the acquis is clearly defined in terms of a legal and administrative framework to be achieved”. Für Themen, die nicht konkret im Acquis definiert sind, so auch zum Thema Reform der öffentlichen Verwaltung, sehen die Evaluatoren das IPA-Programm kritischer: „Where the acquis is defined in a looser framework or there is not a formal acquis chapter (e.g. public administration), effectiveness is less evident. For this type of interventions the BENF needs to establish its own, appropriate strategic/implementation frameworks, often involving interagency cooperation” (\cite{htspe} : 37).\par
Eine andere Analyse des bisherigen IPA-Programmes gibt zu bedenken, dass schwache administrative Kapazitäten in den Kandidaten- und potenziellen Kandidatenländern dazu führen können Ressourcen zu binden, die anderswo nötiger und effektiver eingesetzt werden könnten: „Given the weak economic conditions, relatively fragile governance and underdeveloped administrative capacities in some beneficiaries, adopting EU standards at this stage may add significant costs to public activities and can inhibit the short-term competitiveness of productive activities” (\cite{epec11} : 3).\par
Eine weitere Problematik im Zusammenhang mit der IPA-Förderung generell, aber besonders für kleinere Länder wird in einer Entschließung des Europäischen Parlaments von 2011 zu Montenegro deutlich: „Das Europäische Parlament nimmt mit Zufriedenheit zur Kenntnis, dass die IPA-Hilfe in Montenegro gut funktioniert; fordert sowohl die Regierung Montenegros als auch die Kommission auf, das Verwaltungsverfahren für die Beantragung von IPA-Mitteln zu vereinfachen, damit diese für kleinere und dezentral organisierte Bürgerorganisationen, Gewerkschaften und andere Empfänger einfacher zugänglich sind“ (\cite{eurpar11} : o.S.). 

\subsection{Die SIGMA-Initiative der OECD }
\label{subsec:Die SIGMA-Initiative der OECD}

Von besonderer Bedeutung im Hinblick auf das Thema „Administrative Kapazitäten von Kandidatenländern“ ist die gemeinsame Initiative der Organisation für Ökonomische Zusammenarbeit und Entwicklung (OECD) und der Europäischen Kommission mit dem SIGMA-Programm (Support for Improvement in Governance and Management in Central and Eastern European Countries). SIGMA wurde 1992 gegründet, co-finanziert durch die EU. Ziel war es, die Länder Mittel- und Osteuropas bei der Modernisierung ihrer öffentlichen Verwaltungen zu unterstützen. In den späten 1990er Jahren entwickelte SIGMA Baseline-Kriterien, die Grundlage sind für die Betrachtung und Beurteilung der horizontalen administrativen Kapazitäten der Kandidatenländer. Diese OECD/SIGMA Baseline-Kriterien für die öffentliche Verwaltung sind: 
\begin{enumerate}
\item Policy-Entwicklung und Koordination 
\item Civil service und Verwaltungsrecht
\item Management der öffentlichen Ausgaben
\item Beschaffungswesen im öffentlichen Bereich
\item Interne Finanzkontrolle
\item. Externe Rechnungsprüfung (vgl. \cite{cardona09} : 3).
\end{enumerate}
In der Arbeit legt SIGMA besonderen Wert darauf, den Austausch und die Zusammenarbeit von Regierungen zur Verwaltungsentwicklung zu fördern. Dazu gehörte u.a. logistische Unterstützung für die Gründung von Netzwerken regionaler Verwaltungsspezialisten in Mittel- und Osteuropa und der Austausch dieser Experten mit solchen aus etablierten Demokratien. Weiterhin wird Informationsmaterial und Berichte über bestimmte Themen und Länder auf der OECD/SIGMA Website veröffentlicht. Es geht darum:
\begin{itemize}
\item den Reformfortschritt zu messen und Prioritäten zu identifizieren anhand Europäischer guter Praxis und bestehendem EU-Recht (Acquis Communitaire). 
\item Unterstützung für Entscheidungsträger und Administratoren zur Verfügung zu stellen, für die Einrichtung der rechtlichen Rahmenbedingungen und Prozesse, um europäischen Standards und guter Praxis zu entsprechen.
\item EU-Finanzierung durch Hilfe beim Projektdesign und der Implementierung zu unterstützen (vgl. \cite{oecd99} : 2).
\end{itemize}
SIGMA veröffentlicht seit 1999 Länderberichte zu bestimmten Themen der Verwaltungsentwicklung. Diese werden von der Europäischen Kommission für ihre jährlichen Fortschrittsberichte zu den Kandidaten- und potenziellen Kandidatenländern verwendet.\par
Ohne konkretes EU-Modell zur Verwaltungsmodernisierung füllte die SIGMA-Initiative in gewisser Weise das Vakuum, das aufgrund fehlender klarer Bestimmungen auf EU-Ebene und der sich entwickelnden Konditionalität im administrativen Bereich entstanden war. Da die EU im Rahmen der Konditionalität kein allgemein gültiges Modell der öffentlichen Verwaltung zugrunde legt, bleibt es bei der allgemeinen Forderung der „ability to implement the acquis“. Dimitrova kritisiert, dass die EU nicht das Modell des New Public Management (NPM) zugrunde legt, das einflussreichste Paradigma der letzten Jahrzehnte in der Debatte um Verwaltungsmodernisierung. Stattdessen dient der EU ein weitgehend klassisches Weberianisches Modell der Bürokratie mit Etablierung eines professionellen (Berufs-) Beamtenapparates, der politisch unabhängig ist, als wesentlicher Orientierungspunkt (vgl. \cite{dimit05} : 81).

\subsection{Abgeschlossene Programme für den Westbalkan}
Herausragende bereits abgeschlossene Programme zur Heranführung beitrittswilliger Staaten an die EU-Standards sind die Tätigkeit der European Agency for Reconstruction (EAR) sowie die Programme PHARE und CARDS, die nachfolgend jeweils bezüglich der Verwaltungsmodernisierung kurz dargestellt werden.

\subsubsection{European Agency for Reconstruction}
Die European Agency for Reconstruction war für die Verwaltung der wesentlichen Unterstützungsprogramme der EU für Serbien, Kosovo (unter UN-Verwaltung), Montenegro und Mazedonien zuständig. Die EAR wurde gegründet, um die Wiederaufbauhilfe der EU für Kosovo zu koordinieren. Nach dem Fall des Milosevic-Regimes im Jahr 2000 wurde das Mandat auf Serbien und Montenegro und 2002 auf Mazedonien erweitert. Die EAR hatte ihre Zentrale in Thessaloniki (Griechenland) und Büros in Pristina (Kosovo), Belgrad (Serbien), Podgorica (Montenegro) und Skopje (Mazedonien). Bis zum Juli 2007 hat die EAR ca. 2,3 Milliarden Euro an Hilfe in den von ihr unterstützten Ländern ausgezahlt, wie aus folgender Tabelle ersichtlich:
\begin{table}[H]
\center
\caption{Die Agency for Reconstruction (EAR). Zuwendungen bis Ende Juli 2007}
\small{
\begin{tabular}{|L{20mm}|R{36mm}|R{36mm}|R{16mm}|}\hline
&
Bereitgestellt&
Ausgezahlt&
Quote\\\hline
Serbien&
1,3 Milliarden Euro&
921 Millionen Euro&
71\%\\\hline
Montenegro&
130 Millionen Euro&
104 Millionen Euro&
80\%\\\hline
Kosovo&
1,11 Milliarden Euro&
998 Millionen Euro&
90\%\\\hline
Mazedonien&
327 Millionen Euro&
259 Millionen Euro&
79\%\\\hline
Gesamt EAR&
2,86 Milliarden Euro&
2,3 Milliarden Euro&
80\%\\\hline
\end{tabular}
}\\
\scriptsize{Quelle: \cite{zink} : 8 (eigene Übersetzung aus dem Englischen)}
\end{table}


Aus dieser Tabelle ist ersichtlich, dass der überwiegende Teil der durch die EAR ausgezahlten Hilfe dem Kosovo zugute kam. Neben den Hauptempfängerländern Kosovo und Serbien erhielten Mazedonien und Montenegro 259 und 104 Millionen Euro respektive an Hilfsleistungen.\par
Außer zum Wiederaufbau hat die EAR im Laufe der Zeit auch für andere Bereiche Unterstützung zur Verfügung gestellt, z.B. für Projekte zur wirtschaftlichen Entwicklung, der Stabilisierung der administrativen Kapazitäten der geförderten Länder, im Justizsektor, der Zivilgesellschaft und im Bereich der Medien. Die EAR war dem Europäischen Parlament und dem Rat der Europäischen Union verantwortlich und arbeitete eng mit der Europäischen Kommission und ihren Vertretungen vor Ort zusammen. Im Dezember 2008 endete das Mandat der EAR und die Weiterführung der Förderprogramme ging an die EU-Delegationen in den unterstützen Ländern über (vgl. \cite{zink} : 10).

\subsubsection{PHARE}
Als Hauptinstrument der EU zur Heranführung der Kandidatenländer aus Zentral- und Osteuropa an die EU diente das PHARE-Programm\footnote{Le phare (franz.) bedeutet Leuchtturm.}. Dieses Programm war ursprünglich entwickelt worden, um Polen und Ungarn in Form von Wirtschaftshilfe bei der ökonomischen Umstrukturierung zu helfen, wurde aber später auf die anderen Beitrittskandidaten ausgedehnt (Bulgarien, Tschechische Republik, Estland, Lettland, Litauen, Slowakei, Slowenien und Rumänien). Einige Pilotprojekte in Polen und Ungarn zwischen 1995 und 1997 betrafen auch schon die administrative Kompetenz (\cite{tomtul} : 380). In einer schrittweise sich entwickelnden Heranführungsstrategie standen in der Zeit ab 1996 PHARE, Beitrittspartnerschaften, und Nationale Programme für die Anpassung an den Acquis zur Verfügung. Von dieser Phase an waren neben der Infrastruktur, rechtlicher und ökonomischer Angleichung auch die administrativen Kapazitäten der Beitrittsländer im Blick der EU. 1997 beschloss die EU-Kommission die „Agenda 2000“ und schlug vor, die Hilfen zu 30\% für Institutionenbildung und 70\% für Investitionen zur Verfügung zu stellen. Die Abordnung nationaler Experten der Kandidatenländer zu Schulungszwecken (TAIEX) und von Beamten der EU-Mitgliedsländer in die Kandidatenstaaten (Twinning) wurde als Instrument der Verwaltungsentwicklung und -unterstützung entwickelt (\cite{lipumb04} : 60).\par

Als Ziele des PHARE-Programmes nennt die EU auf ihrer Website:
\begin{itemize}
\item „helping the administrations of the candidate countries to acquire the capacity to implement the Community acquis. PHARE also helps the national and regional administrations, as well as regulatory and supervisory bodies, in the candidate countries to familiarise themselves with Community objectives and procedures;
\item helping the candidate countries to bring their industries and basic infrastructure up to Community standards by mobilising the investment required, particularly in areas where Community rules are increasingly demanding: environment, transport, industry, product quality, working conditions etc.”\footnote{\url{http://europa.eu/legislation_summaries/enlargement/2004_and_2007_enlargement/e50004_en.htm} (Aufgerufen: 19.8.2012).}
\end{itemize}
In einer Evaluierung von PHARE-Projekten zur Verwaltungsmodernisierung in fünf Beitrittsländern\footnote{Estland, Lettland, Litauen, Polen und Slowakei.} wurden Projekte aus den Bereichen gesetzgeberische und organisatorische Reformen des civil service, Training für öffentlich Bedienstete und Verbesserung der IT-Infrastruktur untersucht. Die Ergebnisse sind schematisch in folgender Tabelle dargestellt:
 \begin{table}[H]
\center
\caption{Evaluierung von PHARE-Projekten zur Verwaltungsmodernisierung in CEE}
\small{
\begin{tabular}{|L{40mm}|C{14mm}|C{14mm}|C{14mm}|C{14mm}|C{14mm}|C{14mm}|}\hline
&Anzahl Projekte&Efficiency&Effectiveness&Impact&Nachhaltigkeit&Durchschnitt\\\hline\hline
{\bf Land}&\multicolumn{6}{|r|}{}\\\hline
Estland&3&2,3&2,3&2&1,7&2,1\\\hline
Lettland&9&3,6&3,2&2,9&2,3&3\\\hline
Litauen&10&2,7&2,1&1,8&1,5&2\\\hline
Polen&15&3,8&3,5&2,3&2,1&3\\\hline
Slowakei&3&3&2,7&2,7&2,3&2,7\\\hline\hline
{\bf Projektarten}&\multicolumn{6}{|r|}{}\\\hline
Rechtliche und orga- nisatorische Reformen&30&3,3&2,9&2,3&1,9&2,6\\\hline
Weiterbildung/Training&5&3,6&3,4&3,2&3&3,3\\\hline
Informationstechnologie&5&3,2&2,8&1,8&1,6&2,4\\\hline\hline
{\bf Zuwendungsempfänger}&\multicolumn{6}{|r|}{}\\\hline
Zentrale Exekutive&18&2,9&2,1&1,8&1,7&2,1\\\hline
Lokale Exekutive&13&3,7&3,7&2,8&2,2&3,1\\\hline
Lokale und Zentrale Exekutive&6&3,5&3,5&2,5&2,2&2,9\\\hline
Parlament&3&3,7&3,7&2,7&2,7&3,2\\\hline\hline
{\bf Durchschnitt (alle Projekte)}&40&3,3&3&2,3&2&2,6\\\hline\hline
\multicolumn{7}{|r|}{1 (sehr schlecht), 2 (eher schlecht), 3 (angemessen), 4 (gut), 5 (sehr gut)}\\\hline
\end{tabular}
}\\
\vspace{3pt}
\scriptsize{Quelle: \cite{ips} : 96 (eigene Übersetzung aus dem Englischen)}
\end{table}


Aus dieser Tabelle ist ersichtlich, dass im Bereich der größten Anzahl der Projekte, bei rechtlichen und organisatorischen Reformen des civil service (30 Projekte), die Nachhaltigkeit nicht ausreichend gegeben war.

Zur Analyse der Gründe für das generell schlechte Abschneiden der untersuchten Projekte werden mehrere Faktoren benannt. Ein wesentlicher Faktor wird von den Evaluatoren darin gesehen, dass es entweder keine strategische Ausrichtung der PAR-Projekte gab, oder diese häufigen Veränderungen unterlag. Weiterhin war die Projektentwicklung oft extern vergeben und hat die konkreten Bedingungen vor Ort nicht ausreichend berücksichtigt. Die Auswahl der Projekte war ad-hoc und „demand driven“, PAR war dagegen „project driven” mit starkem Gewicht auf inputs (Unterstützung durch Experten und Bereitstellung von Equipment) und wenig Augenmerk auf outputs und impact. Während die Autoren der Evaluierung viele der identifizierten Probleme den Turbulenzen der frühen Jahre der Transition zurechnen, mahnen sie Verbesserungen im Management der Unterstützungsprogramme an (vgl. \cite{ips} : 98).

Die Autoren der Evaluation schließen mit einer Empfehlung an die EU-Kommission: 

„The Commission is urgently in need of some criteria:

against which a rational discussion of PAR issues can be held, even if these discussions cannot form a formal part of the accession negotiations;
that would offer some guidance and policy focus for PHARE PAR programmes” (\cite{ips} : 103).\par
Für die Länder des Westbalkans wurde das PHARE-Programm im Jahr 2000 durch ein neues Instrument, CARDS, abgelöst.


\subsubsection{CARDS} 

Die wirtschaftliche, politische und soziale Zusammenarbeit der EU mit den Ländern des Westlichen Balkans wurde mit dem Hilfsprogramm „Community Assistance for Reconstruction, Democratization and Stabilization“ (CARDS) als neuem Instrument umgesetzt. CARDS war Teil der Stabilisierungs- und Assoziierungsstrategie der Europäischen Union gegenüber dem Westlichen Balkan, und im Rahmen des SAP wurden Mittel unter CARDS abrufbar. Ab dem Jahr 2000 wurden Mittel der EU bereitgestellt, um Reformprozesse in den Zielländern zu unterstützen (vgl. \cite{calic01} : 12).\par

Die drei Untersuchungsländer der vorliegenden Arbeit (Albanien, Mazedonien und Montenegro) profitierten ab 2000 von CARDS, das bis 2006 zur Verfügung stand. Die Zuständigkeit für das CARDS-Programm wechselte im Jahr 2005 von einer gemeinsamen Zuständigkeit der Generaldirektion Außenbeziehungen und des Europäischen Amtes für Entwicklungszusammenarbeit „EuropeAid“ hin zur Generaldirektion Erweiterung (vgl. \cite{eurrh} : C285/5).\par
Die Gelder aus dem CARDS-Programm standen für verschiedenste Zwecke zur Verfügung, z.B. für Infrastrukturprojekte, Hilfe für Flüchtlinge, Institutionenbildung und Polizeikooperation. Fast die Hälfte des Geldes war für Serbien/Montenegro bestimmt, vor allem aufgrund der Situation im Kosovo. In Mazedonien und Montenegro war ab 2003 die European Agency for Reconstruction (EAR) für die Verwaltung des Programmes zuständig. In Albanien wurde CARDS von der Delegation der Europäischen Kommission in Tirana verwaltet. Regionale Projekte wurden direkt von Brüssel unterstützt. Der Anteil des tatsächlich verausgabten Geldes unterscheidet sich für die Untersuchungsländer. Montenegro hatte eine Absorptionsrate von 86\%, Mazedonien 52\% und Albanien rangiert am unteren Ende mit 29\%. (vgl. Inotai 2007 : 42)TODO:find. Insgesamt wurden von der EU in den Jahren 2000–2006 unter CARDS 4,6 Milliarden Euro zur Verfügung gestellt (vgl. \cite{mus} : 13). Der Europäische Rechnungshof kommt in einer Evaluierung des CARDS-Programmes zu dem Ergebnis, dass vor allem Infrastrukturmaßnahmen erfolgreich umgesetzt wurden, während CARDS bei der Verbesserung der staatlichen Verwaltungskapazitäten weniger effektiv war. Gründe werden vor allem darin gesehen, dass der Schwerpunkt ursprünglich nicht auf dem Institutionenaufbau lag und die Empfängerländer keine ausreichenden Kapazitäten zur Absorption der Hilfe hatten (vgl. \cite{eurrh} : C285/15).
\begin{table}[H]
\caption{CARDS Mittelzuweisungen Albanien, Mazedonien (2002-2006) und Montenegro (2005-2006), nach Sektoren in Millionen Euro}

\small{
\begin{tabular}{|L{60mm}|R{12mm}|R{12mm}|R{12mm}|R{12mm}|}\hline
{\bf Albanien} &{\bf 2002} &{\bf 2003} &{\bf 2004} &{\bf 2005-6}\\\hline
Justice \& Home Affairs&21&20&35&27\\
Administrative Capacity Building&6&8&4&23\\
Economic \& Social Development&12,9&17,5&12&31\\
Environment, Natural Resources&4&-&10&-\\
Democratic Stabilisation&1&1&2,5&4\\\hline
\end{tabular}

\begin{tabular}{|L{60mm}|R{12mm}|R{12mm}|R{12mm}|R{12mm}|}\hline
{\bf Mazedonien} &{\bf 2002} &{\bf 2003} &{\bf 2004} &{\bf 2005-6}\\\hline
Justice \& Home Affairs&7&12,5&24&17\\
Administrative Capacity Building&14&9&8,5&24\\
Economic \& Social Development&11,5&11&15&20\\
Environment, Natural Resources&-&1&2&3\\
Democratic Stabilisation&3&3&3&2\\\hline
\end{tabular}

\begin{tabular}{|L{60mm}|R{12mm}|}\hline
{\bf Montenegro} &{\bf 2005-6}\\\hline
Justice \& Home Affairs&3\\
Administrative Capacity Building&11\\
Economic \& Social Development&16,3\\
Environment, Natural Resources&6\\
Democratic Stabilisation&3,7\\\hline
\end{tabular}
}\\
\scriptsize{Quelle: \url{http://ec.europa.eu/enlargement/how-does-it-work/financial-assistance/cards/statistics2000-2006_en.htm} Aufgerufen: 5.5.2010}
\end{table}

Aus den Tabellen wird ersichtlich, dass für die beiden Jahre 2005 und 2006 in Albanien und Mazedonien ein starker Anstieg der Ausgaben im Bereich Administrative Capacity Building stattgefunden hat. In einer Evaluation des CARDS-Programmes für Albanien wird dennoch konstatiert, dass die die Erfolge des Programmes im Bereich Verwaltungsunterstützung moderat waren: „Support to public administration has been limited in terms of both the number of projects and project size, and a proprer public administration reform and civil service reform have not been implemented“ (\cite{cowi} : ii). Für Montenegro ist die Darstellungsweise erst ab 2005/6 gegeben, wohl aufgrund der Darstellung zusammen mit Serbien bis zur staatlichen Unabhängigkeit 2006. 

\subsection{Zwischenergebnis für die Verwaltungsmodernisierung}
Betrachtet man die theoretischen Ansätze, die praktische Entwicklung und die gezielten Förderprogramme im Zusammenhang, so wird erkennbar, dass trotz der Erfahrungen der EU mit der Osterweiterung, die Verwaltungsmodernisierung und der Status der Verwaltung in den Beitrittsländern nicht angemessen berücksichtig wird. In den Förderprogrammen für den Westlichen Balkan werden regelmäßig Mittel auch für die Modernisierung der öffentlichen Verwaltung bereitgestellt. In Abwesenheit einer glaubhaften Konditionalität und einer Verankerung des Themas im Acquis communautaire sind kritische Evaluationen der bisherigen Förderprogramme in Bezug auf Verwaltungsentwicklung im Balkan allerdings nicht verwunderlich. Es entsteht der Eindruck, dass die EU das Thema Verwaltungsmodernisierung zwar immer wieder als wichtig darstellt, z.B. in den jährlichen Fortschrittsberichten, aber keine operationalisierbaren Instrumente entwickelt hat zu einer nachhaltigen Förderung einer modernen Verwaltung. \par
Dies ist umso erstaunlicher, als die fehlende Modernisierung der öffentlichen Verwaltung in den Ländern der letzten Erweiterungswelle wie gezeigt wurde von der Forschung deutlich herausgearbeitet wurde.\par
Um ein möglichst genaues Bild des Status quo des aktuellen Standes der Verwaltungsentwicklung in den drei Untersuchungsländern zu erhalten, wird im Folgenden der Blick erweitert um die historische Entwicklung der öffentlichen Verwaltung.




\include{chapter03}
\chapter{Die Ausgangslage auf dem Westbalkan }
Voraussetzung der Aufnahme eines Staates in die EU ist das vorherige Erreichen verschiedener Standards. In Abhängigkeit von der bisherigen Entwicklung des beitrittswilligen Staates kann sich das Erreichen dieser Standards über einen längeren Zeitraum erstrecken.\par
Für den hier bestehenden Untersuchungszweck sind insbesondere die Standards im Bereich der öffentlichen Verwaltung von Bedeutung. Die Verwaltungspraxis und somit die Verwaltungsleistungen erfordern zur Realisierung definierter Programme geeignete Strukturen, Personen und Mittel. Mit Blick auf die drei hier näher betrachteten beitrittswilligen Staaten wird daher zum einen geprüft, in welcher Weise und mit welchem Ergebnis die drei Verwaltungssysteme auf den Beitritt zur EU vorbereitet werden. Zum anderen ist aber zu bedenken, dass Veränderungen der Verwaltungspraxis kontextgebunden sind. Es wirken sowohl kulturelle Einflüsse mit als auch die bisherige Praxis einer Verwaltungstradition. Um diese „nachwirkende Tradition“ genauer zu erfassen und in ihrer Bedeutung einschätzen zu können, werden daher zunächst die Entwicklungswege der drei Staaten und ihrer Verwaltungen nachgezeichnet.\par


\section{ Historische Verwaltung in den Untersuchungsländern }
In diesem Abschnitt der Arbeit wird die Verwaltungsentwicklung in geschichtlicher Perspektive für die drei Untersuchungsländer betrachtet. Begonnen wird mit einer historischen Darstellung der Verwaltungsentwicklung in Montenegro und Mazedonien. Anschließend wird die historische Verwaltung in Albanien dargestellt. \par
Bei der historischen Darstellung kommen vier grobe Analyseraster zur Anwendung, die allerdings für jedes Land leicht angepasst werden. 
Auch die politische Entwicklung und Einbindung in regionale und globale Entwicklungen wird berücksichtigt, soweit für die Darstellung der Verwaltung notwendig. Die Analysekriterien sind dabei eine Orientierungshilfe für den Leser, finden sich aber nicht trennscharf in der Beschreibung der historischen Verwaltung in den Untersuchungsländern wieder. Die übergeordneten Analysekriterien für vergleichende Verwaltungsbetrachtungen von Kuhlmann und Wollmann entwickelt, sind: 
\begin{itemize} \itemsep1pt \parskip0pt \parsep0pt
\item Basismerkmale des Regierungssystems,
\item Staatsaufbau und nationales Verwaltungsprofil, 
\item Subnational-dezentrale Verwaltungsebene,
\item Öffentlicher Dienst (\cite{kuhwol}: 45).
\end{itemize}
Für den historischen Abschnitt wurde Literatur ausgewertet und es konnte auch auf zeitgenössische Darstellungen zurückgegriffen werden, die zum Teil im Original eingesehen wurden. Die zeitgenössischen Darstellungen wurden einerseits in der Österreichischen Nationalbibliothek eingesehen, andererseits handelte es sich um Akten des Österreichischen Staatsarchivs, die vor Ort im Original gesichtet wurden. Die vorgefundene Literatur zu Jugoslawien hat fast ausschließlich politischen Bezug. Dies ist einerseits verständlich angesichts der Notwendigkeit, Erklärungen zu suchen für das Ausbrechen offener ethnische Konflikte in Europa Ende des 20. Jh.s nach dem Auseinanderfallen Jugoslawiens. Andererseits ist im Hinblick auf die Aufnahme der Länder des Westlichen Balkans in die EU eine genauere Betrachtung und Einordnung der Verwaltungsentwicklung auch aus historischer Sicht nicht nur wünschenswert, sondern notwendig und überfällig. 

\subsection{Historische Verwaltung Montenegros }

Im Gebiet des heutigen Montenegro haben sich im 6. und 7. Jahrhundert slawische Stämme angesiedelt. Walachen und die autochthone christianisierte Bevölkerung wurden weitgehend slawisiert. Die ersten organisierten mittelalterlichen Staaten wurden im Südwesten des heutigen Serbien sowie im Gebiet des heutigen Kosovo und Nord-Montenegro gegründet. Der Serbische Staat erreichte den Höhepunkt seiner Macht im 14. Jahrhundert unter Zar Dušan, der einen modernen Verwaltungsentwurf einführte mit unabhängigen Gerichten und der Beschreibung einer besonderen Form von Bediensteten des Hofes, einem embryonalen civil service. Auch gab es eine starke Armee und Seestreitkraft. Nach dem Tode des Zaren zerfiel der Serbische Staat in einzelne Nachfolgestaaten, die vom Osmanischen Reich annektiert wurden (vgl. \cite{beckm90}: 28). Ende des 15. Jh.s. wurde auch Montenegro annektiert und an das Gebiet Skutari (Shkodra im heutigen Albanien) angeschlossen. Die orthodoxe Kirche hatte historisch großen Einfluss und die orthodoxen Bischöfe von Cetinje standen formell an der Staatsspitze. Erst im Jahr 1852 wurde das Fürstbistum abgeschafft und in der Folgezeit hatten die Fürsten nur noch weltliche Macht. Montenegro verfügte im 15. Jh. über fünf administrative territoriale Einheiten, die die Osmanen nach der Eroberung beibehielten, aber umbenannten. Im Laufe der osmanischen Zeit folgten mehrere territoriale Umgliederungen, wobei Montenegro einen rechtlichen und sozialen Sonderstatus innehatte, der mit seinen geografischen Besonderheiten zusammenhing. Die Bergregionen waren sehr dünn besiedelt und arm und konnten die üblichen Abgaben nicht leisten. Die osmanischen Besatzer hielten sich vorrangig in Städten auf und erlaubten der ländlichen Bevölkerung ihre eigene lokale Verwaltung zu organisieren. Auch aufgrund dieser geografischen Bedingungen konnte Montenegro nie völlig vom Osmanischen Reich kontrolliert werden (vgl. \cite{boeck}: 20). Mit dem Niedergang der türkischen Herrschaft und infolge eines Aufstandes in den 1830er Jahren wurde Montenegro eine begrenzte Autonomie zugestanden und in den späten 1850er Jahren volle Autonomie ohne türkische Beamte. Der Einfluss des Osmanischen Reiches blieb peripher. Unter dem Schutz der Kirche hielten viele Montenegriner an ihrer eigenständigen Identität fest (vgl. \cite{sevic}: 49).\par

Eine patriarchalische Stammesordnung in Montenegro war bestimmend für die Regelung von Konflikten. 
Die wichtigste politische Institution war bis Ende des 18. Jh.s die Allmontenegrinische Versammlung der montenegrinischen Stämme, auf der bis zu 2.000 Menschen über wichtige Fragen oder auch Stammeskonflikte berieten. Sie wählte auch den Bischof und entschied über Krieg und Frieden. Verbündete in dieser Zeit waren Russland, aber auch das Habsburger Reich. Bis in das 19. Jahrhundert ist der Einfluss Russlands groß, nicht zuletzt durch Zuflüsse zum Staatshaushalt (vgl. \cite{weithmann}: 202).\par

\subsubsection{Staatsgründung und staatliche Verwaltung }

Im Berliner Kongress wurde 1878 die Unabhängigkeit Montenegros anerkannt. Die Regierung blieb dem Fürsten verantwortlich, der auch das alleinige Recht der Beamtenernennung hatte. Nach der Säkularisierung des Landes Mitte des 19. Jh.s begann der erste weltliche Fürst (Danilo) mit der Modernisierung von Verwaltung und Militärwesen. Um die seit Generationen an der Spitze ihrer Stämme stehenden Familien auszuschalten, unterteilte er Montenegro in 40 Kapetanien, deren Grenzen sich nicht mehr mit denen der einzelnen Stämme deckte. Eine Militärreform und Militärpflicht wurden eingeführt, nach der die Montenegriner nicht mehr ihren Stammesverbänden, sondern festen militärischen Einheiten zugeordnet wurden. Die Regelung der Staatsfinanzen gestaltete sich schwieriger. Es konnten zwar Einfuhrzölle durchgesetzt werden, doch regelmäßige Steuerabgaben wären nur mit Gewalt durchzusetzen gewesen, so dass die Staatsausgaben weiterhin vor allem aus russischer Finanzhilfe bestritten wurden. Höhepunkt der Reformen war ein neues Gesetzbuch 1855, das eine Mischung aus kodifiziertem Gewohnheitsrecht und modernen Bestimmungen darstellte. Zeitgenössische Beobachter berichteten jedoch, dass es durchaus vorkam, dass sich Gerichte nicht an die Gesetzesvorgaben hielten, wenn sie ihnen unverhältnismäßig erschienen (vgl. \cite{dickel}: 110). Montenegro sind bei seiner Staatsgründung auch einige albanische Gebiete im Grenzland zu Albanien zugefallen, in denen bis heute albanische Minderheiten leben (vgl. \cite{hoenehhol}).\par
1905 erhielt Montenegro eine Verfassung und 1910 wurde eine konstitutionelle Monarchie ausgerufen mit einem Parlament in der Hauptstadt Cetinje. Faktisch handelte es sich um eine absolutistische Monarchie. Das Parlament bestand aus 62 Vertretern, einer Gruppe von durch den König designierten Mitgliedern und einer Mehrheit, die durch gleiches und allgemeines Wahlrecht bestimmt wurde (vgl. \cite{brepohl}: 14). Es wurden Ministerien eingerichtet, die dem Parlament und der Krone rechenschaftspflichtig waren und die Verfassung garantierte die Presse-, Rede- und Religionsfreiheit sowie die Versammlungsfreiheit. Überreste des osmanischen Feudalsystems wurden durch eine Landreform beseitigt. Neben der Zentralregierung wählten die Montenegriner die Verwaltung von Städten und Dörfern. Erste Industrieunternehmen entstanden in der Forst- und Holzindustrie sowie der Bier- und Tabakproduktion (vgl. \cite{beardradin}: 25).\par
Während die konstitutionelle Monarchie von oppositionellen Gruppen bekämpft wurde und politische Konflikte an der Tagesordnung waren, wurden Staat und Verwaltung weiter modernisiert, mit Gesetzgebungstätigkeit und Vereinheitlichung auf vielen Gebieten. Versuche einer Modernisierung des Landes mit Ansätzen einer modernen Staatsverwaltung scheiterten vor allem an den konkurrierenden Machtansprüchen der verschiedenen Clans. Anlässlich von Bestechungsvorwürfen gegen Minister in Bezug auf Konzessionen für elektrische Beleuchtung konstatiert der k.u.k. Gesandte in Cetinje im Jahr 1911: „Der circulus vitiosus ist immer derselbe: die Beamten sind nicht gezahlt und können von ihren Gehalten nicht leben, müssen daher zu unlauteren Mitteln ihre Zuflucht nehmen. Eine Erhöhung der Gehalte, um den Beamten ein integres Leben zu garantieren, verträgt das Budget, respektive die Armut des Landes und die schon aufs äußerste angespannte Leistungsfähigkeit der Steuerzahler nicht. Daher die Korruption aller Gerichtsbeamten, respektive so ziemlich aller Funktionäre des Landes.“\footnote{HHSTA, PA XVII Montenegro, Kt. 29, Berichte Weisungen 1910-1911, Bericht Freiherr von Giesl an österreichisches Außenministerium 24. Oktober 1911.}\par

Im Laufe des ersten Balkankrieges 1912 vergrößerte Montenegro sein Territorium erheblich. Weitreichende Gebiete konnten dem geschwächten Osmanischen Reich ausgliedert werden und Montenegro hatte bis 1913 sein Staatsgebiet verdoppelt, vor allem mit muslimischer und katholischer albanischer Bevölkerung im Grenzland zu Albanien. In dem Regierungsprogramm der 1914 neugewählten Skupstina (Parlament) wird angemahnt: „Die persönliche und die Preßfreiheit, das Versammlungs- und Eigentumsrecht garantierenden Gesetze sollen wahrhaft angewendet, die Administration vereinfacht werden. Der Zeitgeist verlangt es, dass in der Organisation der staatlichen Verwaltung auf der Basis der Stämme mit der bisherigen Einrichtung gebrochen werde.“\footnote{HHSTA PA XVII Montenegro, Kt. 30 Berichte, Weisungen 1914, Übersetzung Regierungsprogramm, Beilage zu Bericht vom 6. Februar 1914.}

\subsubsection{K.u.k. Militärverwaltung 1916–1918}

Montenegro wurde im Ersten Weltkrieg 1916 von österreichisch-ungarischen Truppen besetzt, und es wurde ein Militärgouvernement aufgebaut. Es folgten Jahre der Besatzung, die 1918 von Partisanen und Truppen der Entente beendet wurde. Während der Besatzungszeit wurde die Administration von der Besatzungsmacht ausgeübt mit entsprechenden Anweisungen des Armeeoberkommandos, die unter dem Titel „Allgemeine Grundzüge für die k.u.k. Militärverwaltung in Montenegro“ eine Reihe von Durchführungsanweisungen zu Steuerverwaltung, der Gendarmerie, der Gerichtsbarkeit der Gemeinden, aber auch hinsichtlich der Verwertung der Ernte erließ.\par
Während man die höheren Beamten austauschte, wurde auf Gemeindeebene die Anweisung erlassen, mit den vorgefundenen Administratoren weiterzuarbeiten. „Staatliche Funktionäre und Mitglieder der Gemeindebehörden sind [...] nicht zu internieren, vielmehr für das Weiterfunktionieren der militärischen Verwaltung im montenegrinischen Gebiet auszunützen. Einerseits sollten Stammeshäuptlinge zu Mitarbeitern ‚herangezogen’ werden, ihre Ämter aber nur unter der ständigen Beobachtung eines Vertreters der Monarchie ausüben, der ein genauer Kenner der dortigen Verhältnisse sein müsste“ (zit. in: \cite{scheer}: 4).\par
Mitte Dezember 1916 wird die Zwischenbilanz der einjährigen Militärherrschaft gezogen, in der Licht auf die Verhältnisse in der Verwaltung geworfen wird:\\
„Die Montenegrinischen Minister, höheren Funktionäre, Politiker und Notablen hatten aus der Unterwerfung Montenegros und aus dessen Bitte um Frieden, ferner aus dem Umstande, dass [...] sie selbst, sowie überhaupt die gesamte Beamtenschaft und die ganze Armee, im Gegensatze zu dem, was unmittelbar zuvor in Serbien beobachtet werden konnte, vertrauensvoll im Lande verblieben waren, die Hoffnung geschöpft, dass man sie bei der neuen Verwaltung des Letzteren nicht vollständig bei Seite schieben, sondern wenigstens in den wichtigeren, die Kenntnis der besonderen hiesigen Verhältnisse erfordernden Fragen zu Rate ziehen und anhören wird [...] Umso enttäuschter waren sie, als das Generalgouvernement, von Anfang an, nicht die geringste Notiz von ihnen nahm und als sogar eine von ihnen an den Gouverneur gerichtete Eingabe, mit welcher sie baten empfangen zu werden, unberücksichtigt blieb.“ In seinem ausführlichen Bericht zur Situation in Montenegro macht der k.u.k. Gesandte diese Missachtung der bisherigen Elite des Landes stark, wenn nicht hauptverantwortlich für die negative Stimmung der Montenegriner gegenüber der Militärverwaltung, was sich u.a. in zunehmendem Bandenwesen äußerte.\footnote{HHSTA, PA I, Kt. 998, 49f, Bericht des k.u.k. Gesandten an Außenministerium Mitte Dezember 1916.}\par
Dass die Militärverwalter der einheimischen Elite nicht trauten, wird in einem Bericht des k.u.k. Militärgouverneurs an den österreichischen Außenminister deutlich: „Es ist zur Genüge bekannt, dass sich in Montenegro gerade die Träger so genannter Intelligenz, aber auch die Träger hoher Staatsämter durch allerhand Spekulationen wirtschaftlicher oder kommerzieller Art zu bereichern bestrebt waren und sich in diesem Bestreben solidarisch unterstützen. Diese Klasse, sowie jene, die gewohnt war, aus dem politischen Getriebe allerhand persönlichen Vorteil zu ziehen, dann die Offiziere, von denen viele auch im politischen Verwaltungsdienst standen, sind begreiflicher Weise mit der gegenwärtigen Verwaltung nicht zufrieden, welche auf die früheren usurpierten Vorrechte und Vorteile keine Rücksicht nimmt.“\footnote{HHSTA, PA I, Kt. 998, 49g, k.u.k. Militärgouverneur von Weber an das k.u.k. Armeekommando, Cetinje, 6. Juni 1916.} \par
Die historische Verwaltung in Montenegro war geprägt von dem großen Einfluss der orthodoxen Kirche, die bis in das 19. Jahrhundert auch die weltliche Macht ausübte. An dieser Struktur änderte auch die Besatzung durch das Osmanische Reich wenig. Nicht zuletzt durch die dünne Besiedlung und ausgedehnte Bergregionen konnte das Osmanische Reich nur begrenzt Kontrolle ausüben und Montenegro blieb geprägt von der Tradition der kirchlichen Herrscher und dem Einfluss der Clans in der Verwaltung des Staates. Innerhalb einer absolutistischen Monarchie vor dem Ersten Weltkrieg wurden Ansätze einer modernen Staatlichkeit angelegt. Österreichisch-ungarische Besatzer, die die Verwaltung des Landes von 1916–18 weiter zu modernisieren suchten, stellten fest, dass die Eliten in der Verwaltung des Landes vor allem persönliche Interessen oder die ihres Clans verfolgten. Zunehmende Partisanentätigkeit seitens der Montenegriner machte den Besatzern zu schaffen und Ansätze zu weiterer Modernisierung der Verwaltung traten in den Hintergrund.\par

Im Folgenden wird auf die historische Entwicklung Mazedoniens eingegangen für die Zeit vor 1918, bevor Mazedonien als Südserbien Teil des Königreiches der Serben, Kroaten und Slowenen wurde. Politisch war Mazedonien in dieser Zeit vorwiegend geprägt von seiner Zugehörigkeit zum Osmanischen Reich; in den nächsten Abschnitten der Arbeit wird die Entwicklung der Verwaltung Mazedoniens in dieser Zeit genauer beleuchtet.

\subsection{Historische Verwaltung Mazedoniens}

Das geografische Gebiet Makedoniens wurde ab dem 14. Jh. in das Osmanische Reich eingegliedert und verblieb bis 1913 im Reich. Die reale Macht im Osmanischen Reich war in den Händen der Pashas (normalerweise Generäle der Armee), die die Provinzen, die Vilayets, verwalteten (vgl. \cite{toepfer}). Makedonien als geografische Bezeichnung fand erst im 19. Jh. Eingang in die europäische Kartographie. Das Osmanische Reich benutzte diesen Begriff nicht. Das Gebiet, das unter osmanischer Herrschaft stand, wurde als die „drei Vilayets“\footnote{Verwaltungseinheit des Osmanischen Reiches ab 1845, angelehnt an die französischen Departements. Ein Vilayet bestand aus zwei oder mehr Sandschaks.} Sealnik (Thessaloniki), Manastir (Bitola) und Üsküb (Skopje) bezeichnet. Im 19. Jh. umfasste das Gebiet Türken, Griechen, Slawen und Albaner. Aber auch Armenier, sephardische Juden, Tartaren und Aromunen, Roma und Kaukasier lebten dort. In den zeitgenössischen Werken wird Makedonien oft Bulgarien zugeordnet, doch die Bevölkerung definierte eine Zugehörigkeit meist über die Konfession, da diese die gesellschaftliche Stellung im Osmanischen Reich entschied. „Ein türkisch sprechender Makedonier verstand sich ebenso wenig als ethnischer Türke wie ein muslimischer Bulgare als Angehöriger einer bulgarischen Nation, sondern sie sahen sich als Osmanli, als Untertanen des osmanischen Sultans und Angehörige der Umma. Die nichtmuslimische Bevölkerung wurde im sogenannten Millet-System untergliedert, gemäß dem sich jede konfessionelle Gruppe unter ihrem religiösen Oberhaupt als autonome ‘nationale’ Gruppe organisierte“ (\cite{opfer}: 17).

\subsubsection{Verwaltung im Osmanischen Reich}

Die gesellschaftlich-politische Verfassung des Osmanischen Reiches war nach dem Personalprinzip und nicht nach dem Territorialprinzip organisiert. Neben dem Sultan und den Ministern der Zentralregierung waren die Provinz-Statthalter im riesigen Reich die wichtigste Autorität und wesentlich in der Administration des Reiches. Sie waren vom Sultan eingesetzt, nur ihm rechenschaftspflichtig und ihnen unterstanden untere Provinzverwalter. Die Provinzverwalter waren zuständig für die Einhaltung und Ausführung der Gesetze, die Gerichtsbarkeit und die Erhebung von Steuern. Eine zeitgenössische Analyse der finanziellen Praktiken des Osmanischen Reiches kommt zu folgender Einschätzung: „The dishonest administration of the taxes in the Provinces was matched by the lack of organization in the central ministry of Finance […] the faults and inherent qualities of the financial system encouraged dishonesty and deceit“ (\cite{blaisd}: 14).\par
Auch die diskriminatorische Erhebung von Steuern anhand von Sprache, Rasse, Religion und kulturellen Traditionen hinsichtlich der Höhe der Steuern war in den Augen des westlichen Beobachters ein Problem. Besonders deutlich wurde dies in dem Bereich des mazedonischen Teils des Reiches. Die dortige Armut, die heterogene Bevölkerung mit vielen Christen, führten zu dem Vorschlag der europäischen Mächte, eine lokale finanzielle Selbstverwaltung in diesem Gebiet einzuführen, was seitens des Osmanischen Reiches 1905 unter internationalem Druck angenommen wurde. Dabei waren die wesentlichen Beweggründe der europäischen Mächte der Wunsch, ihre finanziellen Investitionen im Osmanischen Reich abzusichern. Wesentliches Instrument war die Ottoman Public Debt Administration (OPDA), eine 1881 gegründete eigene Finanzadministration mit Beteiligung europäischer Kontrolleure innerhalb der osmanischen Bürokratie. Diese Institution hatte das Ziel, die Rückzahlung von Geldern zu organisieren, die von europäischen Geldgebern eingebracht worden waren (vgl. \cite{blaisd}: 164ff.). Mit der zunehmenden Verschuldung des Osmanischen Reiches und der Einrichtung der OPDA hatten verstärkt externe Akteure Einfluss auf Entscheidungen im Reich (vgl. \cite{toepfer}: 171). Ein Nebeneffekt der Aktivitäten dieser Organisation, die bis 1915 erfolgreich arbeitete, war der Druck, korrupte Praktiken in der Steuerpolitik zumindest einzuschränken (vgl. \cite{blaisd}: 164ff.).\par

\subsubsection{Auflösungstendenzen in der Osmanischen Verwaltung}	
Gegen Ende des 19. Jh.s wurde Makedonien zu einem „klassischen Beispiel innerbalkanischen nationalistischen Irredentismus“ (\cite{tzermisa}: 133). Griechen, Bulgaren und Serben machten Ansprüche auf das Gebiet geltend. Im Zuge des zunehmenden Machtverlustes des Osmanischen Reiches versuchten die Großmächte auf dem Balkan Einfluss zu gewinnen (vgl. \cite{bark01}: 8).
\par
Der Russisch-Türkische Krieg 1877/1878 führte dazu, dass das geografische Gebiet Makedonien im vorläufigen Vertrag von San Stefano zum größten Teil Bulgarien zugeschlagen wurde. Einige Monate später auf der Konferenz der Großmächte in Berlin, im Sommer 1878, legten diese allerdings fest, dass Makedonien wieder dem Osmanischen Reich zugesprochen wurde, womit der Einfluss Russlands, das Bulgarien in seinen Gebietsansprüchen unterstützt hatte, in Südosteuropa begrenzt werden sollte. Im Gebiet von Makedonien selbst führte der Vertrag von Berlin zu starker Unzufriedenheit, was im Oktober 1878 zu einem bewaffneten Aufstand führte, der niedergeschlagen wurde (vgl. \cite{bech09}:  lvii).\par
Der Niedergang des Osmanischen Reiches, der von Widerstand und gewaltsamen Aufständen begleitet war, fand im Gebiet von Makedonien seinen Ausdruck in der Gründung der Inneren Makedonisch-Adrianopolitanischen Revolutionären Organisation (IMARO) 1893. Diese sollte sich auf die „drei Vilayets“ konzentrieren mit dem Ziel einer Autonomie für Makedonien unter sozialistischer Orientierung (vgl. \cite{boeck}: 333). Blutige Konflikte führten zu Aufmerksamkeit der internationalen Diplomatie für das Gebiet, und Österreich-Ungarn sowie russische Diplomaten forderten 1903 die Hohe Pforte auf, Reformmaßnahmen einzuleiten mit Generalinspektoren in den drei Vilayets unter Aufsicht europäischer Offiziere.\par
 Ein Aufstand der IMARO wurde 1903 niedergeschlagen (vgl. \cite{tzermisa}: 134). Im ersten Balkankrieg im Oktober 1912 erreichten Serbien, Griechenland, Bulgarien und Montenegro gemeinsam, dass der Einfluss des Osmanischen Reiches stark zurückgedrängt wurde. Doch für Makedonien stellte sich die Situation schwieriger dar. Der zweite Balkankrieg wurde im Sommer 1913 zwischen Bulgarien und seinen vormaligen Verbündeten, die von Rumänien und dem Osmanischen Reich unterstützt wurden, ausgetragen. Bulgarien verlor und konnte nur 10\% des makedonischen Gebietes behalten, was verblieb wurde zwischen Serbien, Bulgarien und Griechenland aufgeteilt (vgl. \cite{bech09}: lx).\par
 Erkennbar wird, dass das Gebiet des heutigen Mazedonien sowie Albanien länger als andere Gebiete in der Region osmanischer Herrschaft unterstanden; diese endete dort erst nach dem Balkankrieg 1912/13 (vgl. \cite{batal98}: 120). Im Jahr 1915 stießen bulgarische Streitkräfte nach Mazedonien vor und begannen kurze Zeit später Verwaltungsstrukturen aufzubauen, die aber im Wesentlichen eine Militärverwaltung mit Ausbeutung der Ressourcen der besetzten Gebiete darstellten. Im Bereich der Bildungspolitik wurde versucht eine „Bulgarisierung“ vor allem über Sprache und Kultur zu erreichen. Zum Ende der bulgarischen Besatzung am Ende des Ersten Weltkrieges hatte die Region fast sieben Jahre unter Krieg und Ausnahmezustand verbracht (vgl. \cite{opfer}: 156).\par
Es wird deutlich, dass das Osmanische Reich bis 1912 Einfluss auf Mazedonien hatte und daher auch die Verwaltungsentwicklung stark mitbestimmte. Der Vormachtanspruch des Osmanischen Reiches wurde allerdings seit Mitte des 19. Jahrhunderts von unterschiedlicher Seite immer wieder in Frage gestellt. Auf Mazedonien bezogen sich die Interessen der Großmächte, die sich Einfluss in der Region sichern wollten. Weiterhin waren die direkten Nachbarn interessiert, sich Zugriff auf das Gebiet zu verschaffen. Dies führte zu wechselnden Gebietsaufteilungen und Besatzungen mit starkem Einfluss Bulgariens und später Serbiens auch in Bezug auf die Verwaltungsentwicklung. 

%Mit der Eingliederung Makedoniens in das Königreich der Serben, Kroaten und Slowenen wurde auch im Bereich der öffentlichen Verwaltung der Einfluss Jugoslawiens bestimmend.\par
%Die jugoslawische Zeit hat eine wesentliche Rolle gespielt bei der Konstituierung eines mazedonischen Bewusstseins. In der Mitte der 40er Jahre war eine standardisierte mazedonische Schriftsprache etabliert, die vom Serbischen und Bulgarischen als wesentlich unterschieden wahrgenommen wurde. Und auch administrativ wurde Mazedonien erst nach dem Zweiten Weltkrieg eine eigene Einheit. In den 50er Jahren, als Tito die Massenindustrialisierung und Enteignungen nach Widerstand der Bevölkerung aufgab, entstand sogar im traditionell armen Mazedonien bescheidener Wohlstand, der wiederum dem Regime Befürworter brachte und das so auch die pro-bulgarischen Bevölkerungsteile für sich gewann. Dennoch wurde Mazedonien nicht als eigenständiges nationales Gebilde wahrgenommen. Innerhalb der jugoslawischen Krise in den 1980er Jahren war Mazedonien dadurch gekennzeichnet, dass es als die ärmste Teilrepublik ökonomisch am stärksten auf die anderen Teilrepubliken als Markt angewiesen war. Weiterhin war Mazedonien ein Vielvölkerstaat, der sich dennoch ohne Krieg von Jugoslawien abspaltete (vgl. \cite{dobr}: 84).\par




\subsection{Königreich der Serben, Kroaten und Slowenen}

Nach dem Ersten Weltkrieg wurde das Königreich der Serben, Kroaten und Slowenen am 1.12.1918 als parlamentarische Monarchie gegründet. Der Staat war als Nationalstaat angelegt, doch handelte es sich um einen Vielvölkerstaat, wie aus folgender Tabelle deutlich wird.\footnote{Laut Volkszählung von.1921 lebten außerdem ca. eine halbe Million Deutsche, Ungarn und Albaner sowie kleinere ethnische Gruppen wie Rumänen, Türken, Juden, Slowaken, Tschechen, Italiener und Russen auf dem Staatsgebiet. Etwa 47\% der Bevölkerung bekannte sich zur Ostkirche, 39\% zum Katholizismus und 11\% zum Islam. 2/3 der Bevölkerung lebten in den vormals habsburgischen Gebieten. (vgl. \cite{hoenehhol}: 320).}
\renewcommand{\arraystretch}{1}
\begin{table}[H]
\caption[Bevölkerungsanteile im Königreich der Serben, Kroaten und Slowenen]{Bevölkerungsanteile im Königreich der Serben, Kroaten und Slowenen gemäß Volkszählung vom 31.1.1921}
\center
\small
\begin{tabular}{|R{46mm}|R{25mm}|}\hline
&Bevölkerung\\\hline
Serbien &4.129.638\\\hline
Montenegro&199.857\\\hline
Bosnien-Herzegowina&1.889.929\\\hline
Dalmatien&650.139\\\hline
Kroatien&2.710.883\\\hline
Slowenien&1.056.464\\\hline
Vojvodina&1.380.413\\\hline
Gesamt&12.017.323\\\hline
\end{tabular}\\
\vspace{0,5cm}
{\normalsize Quelle: \cite{beardradin} (eigene Übersetzung aus dem Englischen).}
\end{table}
Montenegriner waren die kleinste Bevölkerungsgruppe mit knapp 200.000 Einwohnern, während Serbien die meisten Einwohner (4 Millionen) in dem Königreich mit 12 Millionen Einwohnern stellte.

Das Gebiet Makedonien wurde 1918 unter der Bezeichnung „Südserbien“ dem Königreich der Serben, Kroaten und Slowenen angegliedert. Mehrere Tausend Serben wurden in Makedonien angesiedelt, um den Machtanspruch Belgrads zu bekräftigen. Das Wiedererstarken der nun IMRO genannten Widerstandsbewegung gegen die serbische Oberhoheit im Staat führte nach 1920 dazu, dass große Teile der Gendarmerie des Staates in Südserbien stationiert waren (\cite{bech09}:  lx). Auch die Administration und die Bildungspolitik waren serbisch dominiert. Doch „einer Anstellung in Makedonien haftete noch immer der Ruf einer Strafversetzung an. Auf diese Weise waren Verwaltung und Bildungswesen in ‚Südserbien’ mit unqualifiziertem Personal durchsetzt“ (\cite{opfer}: 164). In den 1920er und 1930er Jahren wuchs eine neue Generation von Mazedoniern heran, die nicht das bulgarische Bildungssystem durchlaufen hatten und kein Bulgarisch mehr sprachen. In den 1930er Jahren waren Belgrad und Zagreb und nicht Sofia die Zentren, die junge Mazedonier anzogen. In der gleichen Zeit bildete sich eine nationale Idee heraus, die von Mazedoniern als einem eigenständigen Volk von Slawen ausging, weder den Serben noch den Bulgaren zugehörig (vgl. \cite{bech09}: lxi).\par

Das neue Königreich war so vielgestaltig wie die verschiedenen nationalen Identitäten innerhalb des Staates. Der Zusammenhalt des Staatengebildes wird in den zeitgenössischen Berichten der Sächsischen Gesandtschaft in Wien in Frage gestellt. „Die Quellen der inneren Zwietracht entspringen vorwiegend der Verschiedenheit der Glaubensbekenntnisse der südslawischen Stämme“ (zit. n.: \cite{opitz}: 221). Nicht nur das Völkergemisch stellte den neuen Staat vor Herausforderungen. Auch hatten die einzelnen Nationen ganz unterschiedliche historische Erfahrungen gemacht und unterschiedliche kulturelle Prägungen erfahren. Die Serben hatten vierhundert Jahre unter osmanischer Herrschaft gelebt, hatten allerdings auch als erstes der christlichen Balkanvölker ihre Unabhängigkeit erkämpft, zunächst als Autonomie und später in staatlicher Selbstständigkeit. Die südlichen und östlichen Gebiete, einschließlich Makedonien waren sogar bis zu den Balkankriegen 1912/13 Teil des Osmanischen Reiches geblieben (vgl. \cite{libal}: 15).
\par

\subsubsection{Verwaltungsaufbau}

Das Königreich wurde administrativ in Herzogtümer sowie die Stadt Belgrad (mit speziellem Status) eingeteilt. Dabei war nur Kroatien mit stärkeren politischen Rechten ausgestattet, alle anderen Herzogtümer waren die nächstuntere Verwaltungsebene (vgl. \cite{sevic}: 53).\par
Der Aufbau des Staatswesens unterlag einer schwierigen Kompromissfindung zwischen den unterschiedlichen Interessen von Zentralisten und Föderalisten. Im Jahr 1921 wurde eine neue Verfassung auf Basis der serbischen Verfassung von 1903 angenommen. Die Verfassung legte den Grundstein zu einer Landreform und zur Verstaatlichung der Wälder. Eigentumsrechte und -pflichten sowie die Besteuerung wurden festgelegt (vgl. \cite{beardradin}: 56). Das Königreich der Serben, Kroaten und Slowenen war als Einheitsstaat konzipiert und wurde in 33 Gebiete (oblasti) eingeteilt (vgl. \cite{beckm90}: 32).\par
Das parlamentarische System verfügte über keinen tragfähigen Grundkonsens der verschiedenen Gruppen und in den 10 Jahren nach Staatsgründung lösten sich 24 Kabinette ab. Nach nationalen und regionalen Sonderinteressen ausgerichtete Parteien, Korruption, Vetternwirtschaft und Wahlmanipulationen bestimmten das Geschehen. Trotz dieser wechselvollen innenpolitischen Entwicklungen schien die Arbeit der Verwaltung weitgehend reibungslos und unbeeindruckt vom politischen Wechselspiel zu funktionieren. „Owing to the permanence and discipline of the bureaucracy, the functions of the government continue unbroken amid all the storms of Parliament. The budget system is well organized and operates smoothly” (\cite{beardradin}: 172). Zu dieser Stabilität der Verwaltung hat auch das Beamtentum beigetragen, das im Königreich fast ausschließlich mit dauerhaften Positionen organisiert war.\par
Nach zunehmenden Spannungen und innenpolitischen Krisen wurde 1929 eine Königsdiktatur unter der Bezeichnung Königreich Jugoslawien errichtet. Die Verfassung wurde suspendiert, Parteien verboten und das Parlament aufgelöst. Nach der Errichtung der Königsdiktatur 1929 wurden in Belgrad, Zagreb und Ljubljana (Laibach) durch Dekret neue Verwaltungen eingesetzt, für die übrigen Städte und Gemeinden war der Innenminister zuständig (vgl. \cite{libal}: 24). An Stelle von bisher 33 Gebieten wurden 9 Verwaltungseinheiten eingerichtet, „die mit Ausnahme Sloweniens keine Rücksicht auf nationale Gegebenheiten nahmen“ (\cite{beckm90}: 33) Die Diktatur wurde 1931 formal wieder aufgehoben. Die Verfassung von 1935 legte fest, dass Ministerien ein integraler Bestandteil des Staatsrates waren und die Minister dem Staatsrat unterstanden. Während der Staatsrat die exekutive, legislative und judikative Macht hatte, waren die Minister für die Ausarbeitung der Gesetzesvorlagen zuständig (vgl. \cite{kinhil}: 441).

\subsubsection{Beamtentum}
Die Position des Staatsbediensteten (Drzavni Cinovnik) war mit einigem sozialen Prestige verbunden. Zwischen den beiden Weltkriegen waren die meisten Staatsbediensteten aus dem serbischen Beamtenapparat übernommen worden, wobei auch eine Anzahl aus dem früheren Österreich-Ungarn kamen, vor allem aus der Region von Kroatien und Slowenien (vgl. \cite{sevic}: 52). In der Verwaltung des neuen Staates, der stark zentral ausgerichtet war, dominierten die vorwiegend in Frankreich ausgebildeten Serben, was sich auch in den Verwaltungsstrukturen niederschlug, die oft dem französischen Vorbild folgten (vgl. \cite{beardradin}: 27). Staatsbedienstete wurden von den Ministern ausgewählt, aber vom König ernannt (vgl. \cite{sevic}: 51). Die ökonomischen Zentren des Staates waren die Städte Belgrad und Zagreb. In den südserbischen Gebieten Kosovo und Mazedonien wurde vor allem Landwirtschaft betrieben und dort fanden sich die serbischen Beamten einer mehrheitlich anderssprachigen Bevölkerung gegenüber (vgl. \cite{beardradin}: 27).\par
Das Königreich war zentralistisch organisiert und erst kurz vor dem Zweiten Weltkrieg begann eine Dezentralisierung mit der Einrichtung eines Herzogtums Kroatien, womit kroatische Bestrebungen nach Unabhängigkeit eingedämmt werden sollten (vgl. \cite{sevic}: 52).\par

\begin{wraptable}{r}{88mm}

%\begin{table}[H]
\caption[Höhere Staatsbedienstete im Königreich Jugoslawien ]{Höhere Staatsbedienstete im Königreich Jugoslawien gemäß Haushalt 1928–29}
\label{tab:Haushalt}
\center
\footnotesize
\begin{tabular}{|L{46mm}|R{15mm}|}\hline
\multicolumn{2}{|l|}{Allgemeine Verwaltung}\\\hline\hline
Höhere Staatsbeamte&418\\\hline
Justiz&8.316\\\hline
Bildung&29.066\\\hline
Religion&1.127\\\hline
Inneres (inkl. Polizei)&25.238\\\hline
Öffentliche Gesundheit&3.048\\\hline
Äußere Beziehungen&597\\\hline
Finanzen&13.859\\\hline
Militär&21.009\\\hline
Öffentliche Gebäude&2.701\\\hline
Kommunikation&545\\\hline
Landwirtschaft und Wasserwege&985\\\hline
Handel und Industrie&1.122\\\hline
Öffentliche Fürsorge&445\\\hline
Landwirtschaftsreform&283\\\hline
Vereinheitlichung von Gesetzen&1\\\hline
Gesamt&108.760\\\hline
\multicolumn{2}{c}{}\\
\multicolumn{2}{C{61mm}}{\normalsize Quelle: \cite{beardradin}: 185 (eigene Übersetzung aus dem Englischen).}
\end{tabular}\\
\end{wraptable}
%\end{table}
\renewcommand{\arraystretch}{1.2}
Das System des Staatsbeamtentums wurde stark an die austro-germanische Tradition angelehnt. Staatsbediensteter wurde man nach Durchlaufen einer Vorbereitungszeit und einer Bewährungszeit von bis zu drei Jahren. Es folgte eine Lebenszeitstellung mit Karriereschritten, die gesetzlich festgelegt waren. Jährliche Beurteilungen waren ebenfalls vorgeschrieben. In Tabelle \ref{tab:Haushalt} sind die Bereiche aufgelistet, in denen Staatsbedienstete eingesetzt waren gemäß dem Haushalt von 1928–29. 
Die Tabelle zeigt, dass im Bildungssystem, der Polizei und dem Militär zusammen ca. 75\% der Staatsbediensteten eingesetzt waren.\par
Probleme des Beamtentums im Königreich werden in der Studie zweier zeitgenössischer amerikanischer Verwaltungsexperten benannt: 1. Das Fehlen einer zentralen Stelle, die einheitliche Verfahren zu Anforderungen, wie z.B. Bildung im Detail entwickelt hätte. 2. Das Beurteilungswesen war nicht einheitlich geregelt in den unterschiedlichen Ministerien. 3. Es gab zu viele Beamte. 4. Eine starke Ausrichtung auf Juristen bei der Einstellungspraxis. 5. Keine Eingangstests. 6. Minister hatten das Recht auf Personalentscheidung bei Besetzungen. 7. Die Entfernung aus einer Position war so kompliziert, dass sich Permanenz anstelle von Effizienz durchsetzte. Insgesamt war so eine machtvolle Bürokratie entstanden, ganz analog den kontinentaleuropäischen westlichen Vorbildern der Zeit (vgl. \cite{beardradin}: 191).

\subsubsection{Verwaltungsgerichtsbarkeit und Verwaltungskontrolle}

In den Gebieten, die später Teil Jugoslawiens wurden, bestand seit Ende des 19. Jh.s eine rechtliche Verwaltungskontrolle mit verschiedenen Ausgestaltungen und organisatorischen Lösungen. Das Verwaltungsgericht in Wien wurde 1875 für den österreichischen Teil der Monarchie gegründet. Das Gericht war auch in Slowenien und Teilen Kroatiens (Istrien und Dalmatien) zuständig. Beschwerde konnte eingelegt werden gegen einzelne administrative Maßnahmen. Entscheidungen in polizeilichen Kriminalfällen konnten dagegen nicht überprüft werden. Im ungarisch-kroatischen Teil der Monarchie war die Situation anders. Das Finanzverwaltungsgericht wurde 1883 gegründet und befasste sich ausschließlich mit finanziellen Fragen. Ein Verwaltungsgericht wurde 1896 eingerichtet. Im Gebiet des heutigen Kroatien hatte das Verwaltungsgericht nur die Zuständigkeit für die im engeren Sinne administrativen Fälle, die nicht von der kroatischen Autonomie abgedeckt waren. Im Gebiet des damaligen Bosnien-Herzegowina gab es keine juristische Überprüfbarkeit des Verwaltungshandelns, nicht einmal als es Teil der österreichisch-ungarischen Monarchie wurde. Dem Staatsrat in Montenegro wurde durch konstitutionelle Änderungen 1905/6 in bestimmtem Maße Kompetenz über ministerielle Entscheidungen in umstrittenen administrativen Fragen zugestanden (vgl. \cite{kopric}: 2).\par
Trotz politischer Instabilität kann das Verwaltungsrecht im Königreich als relativ entwickelt gelten. Aufbauend auf dem nach französischem Modell im Königreich Serbien entstandenen Staatsrat wurde 1922 eine zweistufige Verwaltungsgerichtsbarkeit eingeführt und 1930 das Verwaltungsverfahren kodifiziert. Ein zweistufiges Widerspruchsverfahren wurde festgelegt (vgl. \cite{lucht}: 24).
\par
Nach der Errichtung des Königreiches der Serben, Kroaten und Slowenen im Jahr 1918 und bis 1922 hatte der Staatsrat von Serbien weiterhin die Kompetenz zur rechtlichen Überprüfung königlicher Anweisungen und ministerieller Entscheidungen. In der Verfassung von 1922, die eine zentralistische Verwaltungsgliederung vorsah, wurde der Staatsrat als Verwaltungsgericht der zweiten Instanz weitergeführt. Erstinstanzliche Verwaltungsgerichte konnten nun in Belgrad, Zagreb, Celje, Sarajevo, Skopje and Dubrovnik eingerichtet werden und der Ausbau der Verwaltungsgerichtsbarkeit wurde bis 1929 weitergeführt (vgl. \cite{kopric}: 3). Dabei stand die in der Zwischenkriegszeit noch weitgehend bestehende Agrargesellschaft mit oft ungebrochen feudalistischen Herrschaftsweisen dem eigentlich fortschrittlichen und rechtsstaatlich orientierten Verwaltungsstaat entgegen (vgl. \cite{lucht}: 24).

Juristisch gesehen gab es die Möglichkeit, sich gegen Entscheidungen beim Staatsrat zu beschweren oder bei einem Distriktgericht ein Verfahren anzustrengen. Während die Macht des Staatsrates aufgrund weiterer Änderungen abnahm, blieb die Verwaltungsgerichtsbarkeit bis zur Besetzung Jugoslawiens 1941 bestehen. Alle serbischen Verfassungen enthielten das Recht auf Anfechtung. Dieses Recht wurde umfassend ausgelegt, so konnten sowohl Rechtsakte als auch Entscheidungen der Verwaltung angefochten werden. Alle Regierungsentscheidungen mussten veröffentlicht werden (vgl. \cite{sevic}: 51).\par

Während des Zweiten Weltkrieges wurde 1941 der größte Teil des mazedonischen Gebietes von Bulgarien okkupiert. Die westlichen Gebiete wurden dagegen von Italien in sein Protektorat Albanien eingegliedert. Die Aufrufe Titos an die Mazedonier, sich am Partisanenkampf zu beteiligen, wurden zunächst nur zögerlich aufgenommen. Erst nachdem die serbischen Bevölkerungsteile von der bulgarischen Besatzungsmacht benachteiligt wurden, vor allem im Bereich der Beschäftigung im öffentlichen Dienst und mit Beschlagnahmungen, hatte die kommunistische Idee seit 1943 Zulauf. In der Folge führte die Kombination von Kommunismus und Nationalismus mit der Idee einer Republik Mazedonien innerhalb eines Staates Jugoslawien zu einer starken Mobilisierung in der Bevölkerung (vgl. \cite{bech09}: lxiii).\par

In der historischen Betrachtung zeigt sich, dass Mazedonien noch stärker als Peripherie bezeichnet werden konnte als das im Königreich ebenfalls randständige Montenegro. Während die Verwaltungsstruktur sowohl Montenegros als auch Mazedoniens vom jugoslawischen System geprägt war, kann davon ausgegangen werden, dass Montenegro stärker den kontinentaleuropäischen Traditionen der Verwaltung verhaftet war. Montenegro hatte sich dem Einfluss des Osmanischen Reiches weitgehend entzogen und stand 1916–18 direkt unter k.u.k. Verwaltung. In Mazedonien dagegen war die Verwaltung maßgeblich vom Osmanischen Reich geprägt, mit kurzfristigeren Einflüssen auch durch Bulgarien und Griechenland.\par

Zusammenfassend lässt sich sagen, dass die Gründung des Königreiches der Serben, Kroaten und Slowenen nach dem Ersten Weltkrieg zu einem Vielvölkerstaat mit starker serbischer Dominanz führte. In diesem Staat waren Montenegro und Mazedonien (Südserbien genannt) periphere und rückständige Gebiete, die hauptsächlich von Landwirtschaft lebten. Für die weitere Untersuchung festzuhalten ist, dass eine landesweite Verwaltungsstruktur mit Berufsbeamtentum eingeführt wurde, die sich am serbischen Vorbild orientierte. Die serbische Verwaltungsstruktur mit einem Staatsrat wies viele französische Einflüsse auf und verfügte über ein entwickeltes Verwaltungsrecht und eine eigenständige Verwaltungsgerichtsbarkeit. Diese Einflüsse finden sich, wie im Folgenden zu zeigen ist, zum Teil noch im sozialistischen Jugoslawien wieder. Dabei sei vor allem die, zumindest nominelle Möglichkeit der Überprüfung von Verwaltungshandeln über das gesamte Bestehen der SFRJ hinweg genannt. Im Gegensatz dazu wurde das Berufsbeamtentum im sozialistischen Jugoslawien sukzessive abgebaut und umgewandelt und das generelle Arbeitsrecht auch für öffentliche Bedienstete angewandt.


\subsection{Sozialistische Verwaltung Jugoslawiens}
Der folgende Unterabschnitt widmet sich kursorisch dem Staatsaufbau und der Verwaltungsentwicklung in der Sozialistischen Föderativen Republik (SFR) Jugoslawien. Auffallend ist, dass sich zum Thema öffentliche Verwaltung im sozialistischen Jugoslawien keine zusammenfassenden oder überblicksartigen Darstellungen in westlichen Sprachen in der Literatur finden ließen.\footnote{Dieser Eindruck wurde der Autorin von einem ausgewiesenen Südosteuropa-Experten, Prof. Dr. Sundhaussen, im November 2012 bestätigt.} Dennoch wird eine skizzenhafte Darstellung der Thematik versucht, anhand der allgemeinen Literatur zum Selbstverwaltungssozialismus Jugoslawiens, und der dort aufgefundenen Beschreibungen der öffentlichen Verwaltung. \par
Bei der Darstellung wird kurz auf die politische Verfasstheit des Staates eingegangen. Besonderes Augenmerk liegt auf der Herausarbeitung der spezifischen Ausprägung der Verwaltung. Dies ist notwendig als Basis für die Beantwortung der Frage, ob die vordemokratische Verwaltungsentwicklung Einfluss hat auf die heutige Situation der öffentlichen Verwaltung in Mazedonien und Montenegro, die beide Teilstaaten Jugoslawiens waren. Die Betrachtung der Verwaltungsentwicklung zur Zeit der SFR Jugoslawien ist auch wichtig, um Einflüsse und Folgen erkennen zu können, die sich möglicherweise unterscheiden von anderen Ländern, die Nachfolgestaaten der Sowjetunion waren. Es wird zu zeigen sein, dass das Verwaltungssystem ein ganz spezifisches war, das nicht dem in der Sowjetunion entsprach. Für die Frage, ob die Erfahrungen mit der Verwaltungsmodernisierung der Länder der Osterweiterung für die Analyse der aktuellen Erweiterungsrunde angewandt werden können, sind diese historischen Unterschiede von Bedeutung.

\subsubsection{Staatsorganisation }

Jugoslawien entstand nach dem Zweiten Weltkrieg als sozialistischer Bundesstaat mit den sechs Teilrepubliken Slowenien, Kroatien, Bosnien-Herzegowina, Montenegro, Serbien und Mazedonien sowie den beiden Provinzen Vojvodina und Kosovo. Der Präsident war oberstes Staatsorgan. Daneben bestand das Präsidium der SFR Jugoslawien aus einem Vertreter pro Republik und Provinz sowie dem Präsidenten auf Lebenszeit (Josip Broz Tito, Partisanenkämpfer und Generalsekretär des Bundes der Kommunisten Jugoslawiens). Weiterhin gab es ein bikammerales Parlament. Zwingende Voraussetzung, um Delegierter werden zu können, war die Zugehörigkeit zur Kommunistischen Partei oder einer ihrer gesellschaftspolitischen Organisationen (vgl. \cite{toepfer}: 190). Seit dem Bruch mit der Sowjetunion 1948 verfolgte Jugoslawien einen eigenen jugoslawischen Kommunismus, der ein kompliziertes Selbstverwaltungssystem beinhaltete, engagierte sich in der Bewegung der blockfreien Staaten und nahm wirtschaftliche Beziehungen auch zu Westeuropa auf.\par
Im Jahr 1948 trat eine sozialistische Verfassung in Kraft, mit eigenen Verfassungen für die Teilrepubliken. Nach einem Gründungsboom verlangsamte sich das Wirtschaftswachstum in den 1960er Jahren und die ungleiche Verteilung der Mittel zwischen den einzelnen Republiken wurde zum Thema. „Auf der einen Seite drohte der ‚Lokalismus’ (das meinte Nationalismus und Separatismus) der Republiken den Staat von innen zu zerreißen. Auf der anderen Seite tendierten Unitarismus und Etatismus der Zentralgewalten zur Hegemonie der größeren über die kleineren Volksgruppen“ (\cite{calic10}: 228). Mit einer sozialistischen Marktwirtschaft und zunehmender Verlagerung von Verantwortung in die Republiken versuchte man die sich abzeichnenden Probleme in den Griff zu bekommen (vgl. \cite{ramet}: 297). Während in anderen osteuropäischen Staaten wie der Sowjetunion, Bulgarien und Polen kaum Verfassungsänderungen vorgenommen wurden, erfuhr die jugoslawische Verfassung mehrere Veränderungen, in deren Folge auch die Staatsorganisation angepasst wurde. Für diese bis 1974 häufigen und umfangreichen Veränderungen sieht Roggemann im Wesentlichen acht Gründe:
\begin{itemize}
\item Starke zentrifugale nationale Kräfte, die ein weitgehend dezentralisierter, und im europäischen Verfassungsrecht wohl einmaliger Föderalismus zu binden versuchte.
\item Ökonomische Ungleichgewichtigkeiten und Entwicklungsstörungen eines ‚halbentwickelten’ Industriestaates im Spannungsverhältnis zweier Wirtschaftssysteme (RGW\footnote{Rat für gegenseitige Wirtschaftshilfe, oft als Comecon (aus der englischen Übersetzung Council for Mutual Economic Assistance) bezeichnet, war der wirtschaftliche Zusammenschluss der sozialistischen Staaten unter Führung der Sowjetunion. Er löste sich im Jahr 1991 infolge der politischen Umwälzungen des Jahres 1989 auf.} und EWG).
\item Die spezifischen Schwierigkeiten des ökonomischen Mischsystems einer ‚sozialistischen Marktwirtschaft’.
\item Der Versuch den ‚Selbstverwaltungssozialismus’ als eigenständige Form gegenüber dem zentralistischen Leninismus weiterzuentwickeln und dennoch mit dem Lenkungsanspruch der BdKJ zu verbinden.
\item Die maritime Randlage, mit historischer Öffnung nach Westen (Weltmarkt und gleichzeitige Assoziierung RGW).
\item Historisch und geografisch bedingtes Ungleichgewicht der industriellen Entwicklung, Infrastruktur und Sozialstruktur in den Landesteilen.
\item Abgrenzung gegen imperiale Ansprüche der UdSSR in Südosteuropa.
\item Militärische Selbständigkeit und Selbstbehauptungswille (vgl. \cite{roggemann}: 260).
\end{itemize}
Bis ca. 1984/85 hatte die Kommunistische Partei einen festen Zugriff auf die Macht, der über ein spezifisches System bürokratischer Herrschaft funktionierte im Zusammenspiel von wirtschaftlichen, sozialen und politischen Instanzen. Močnik entwickelte dazu ein Modell dreier gesellschaftlicher Untersysteme mit Selbstverwaltung durch Arbeiter, sozialistischem Markt durch Manager und Einparteienstaat durch Bürokraten. Im Folgenden eine systematische Darstellung, wie sich die Partei in diesem System die Kontrolle über alle gesellschaftlichen Bereiche sicherte:
\begin{figure}[H]
\setlength\belowcaptionskip{10pt}
 \caption{Einfluss der Partei im sozialistischen Jugoslawien}
  \centering
  \includegraphics[width=4in]{Material/Parteienbuerokratie}\\
\vspace{0,5cm}
Quelle: nach \cite{mocnik}: 137.
\end{figure}

\subsubsection{Verwaltungsaufbau}

Die öffentliche Verwaltung in Jugoslawien entsprach zunächst der Verwaltung eines sozialistischen Staates mit Planungsfunktion. Die Staatsverfassung kannte keine klassische Gewaltenteilung, so dass Verwaltungsmaßnahmen nicht nur von den eigentlichen Verwaltungsorganen getroffen wurden, sondern auch von Gesetzgebungs- und Repräsentationsorganen. Dennoch wies die Verwaltung Jugoslawiens im Vergleich zu anderen sozialistischen Verwaltungen einige Besonderheiten auf. Seit Ende des Zweiten Weltkrieges waren vier Phasen (bis 1975) erkennbar.
\begin{enumerate}[label={\Roman*}:,align=left,  leftmargin=*]
\item  In der ersten Periode wurde das sowjetische Verwaltungskonzept nachgebildet, diese Periode ging formell 1952 zu Ende. Bis 1952 waren die staatliche Administration und der civil service stark zentralistisch und hierarchisch organisiert. Staatseigene Betriebe, die öffentliche Aufgaben wahrnahmen, wie Gesundheits- und Bildungseinrichtungen, hatten auch den Status von Verwaltungseinheiten (vgl. \cite{sevic}: 54). Während alle exekutiven und administrativen Kompetenzen beim Zentralstaat lagen, hatten die Republiken in ihrem Bereich Befugnisse, allerdings nur zur Ausführung zentralstaatlicher Gesetze.
\item Die Periode der Abwendung vom sowjetischen Vorbild, mit Abbau des demokratischen Zentralismus, Übergang von staatlichen Plänen zu gesellschaftlichen Plänen, Ausbau der Selbstverwaltung der Republiken, territorialer Verwaltungsdezentralisation. Mit einer Verfassungsänderung 1953 wurden diese Veränderungen umgesetzt (vgl. \cite{beckm90}: 52).
\item Der spezifische „Selbstverwaltungssozialismus“ ist in der Verfassung von 1963 verankert. Der zentralistische Parteiapparat diente als Gegengewicht zur föderalen Staatsstruktur (vgl. \cite{roggemann}: 258). Bis 1967 wurden Gesetze an die neue Verfassung angeglichen, auch das im Jahr 1957 erlassene Verwaltungsrecht musste angepasst werden. Die neue Verfassung hat den Begriff „Volksdemokratie“ durch den Begriff „sozialistisches Land“ ersetzt. Die Volkskomitees wurden zugunsten lokaler territorialen Einheiten ersetzt. Das wesentliche Organ wurden die Versammlungen, die ihre eigene Verwaltung und Räte bekamen (vgl. \cite{blago69}: 5). 
\item Die Periode mit neuem Verfassungskonzept sowie neuer Kompetenzabgrenzung. Seit 1967 wurden immer mehr Verwaltungskompetenzen auf die Teilrepubliken übertragen (vgl. \cite{grot}: 134). Im Jahr 1974 trat eine neue Verfassung in Kraft, die weitgehende Kompetenzen der Republiken und der Selbstverwaltungsrechte der unteren Verwaltungsebenen und gesellschaftlicher Gruppierungen festschrieb. Weiterhin sollten Wirtschaftsorganisationen direktere Mitspracherechte haben (vgl. \cite{beckm90}: 80ff.). 
\end{enumerate}

Es entstanden vier Ebenen der Verwaltung: 1. Die Föderation, 2. die Republiken, 3. die autonomen Provinzen und 4. die territorial-administrativen Einheiten (Gemeinden, Städte, Distrikte, Regionen). Distrikte und Städte hatten lokale Selbstverwaltungsfunktion, analog ihren Aufgaben im 19. Jh., waren aber gegenüber der lokalen Legislative verantwortlich und die zentralstaatliche Ebene konnte die unteren Ebenen der Administration anweisen. (vgl. \cite{sevic}: 47). Neben der neuen Verfassung für Jugoslawien 1974 verabschiedeten die sechs Republiken eigene Verfassungen mit genauen Aufgabenbeschreibungen und Kompetenzzuweisungen. Dabei wurde der Bundesstaat als „staatliche Gemeinschaft freiwillig vereinter Völker und ihrer sozialistischen Republiken“ (Art. 1 Verf. 1974) beschrieben. Die sozialistischen Republiken wurden ausdrücklich als Staaten verstanden mit dem Volk als Träger der Souveränität (Art. 3 \& 4 Verf. 1974). Gleichzeitig enthielt die Verfassung keinen eindeutigen Bezug auf eine jugoslawische Nation. Der Bund bearbeitete einige wesentliche ihm zugewiesene Bereiche, wie Auswärtige Beziehungen, Landesverteidigung und Staatssicherheit und war für die Aufrechterhaltung des einheitlichen Rechts- und Wirtschaftssystems verantwortlich. Die Republiken waren in allen anderen Belangen zuständig und auch für den Vollzug der Bundesgesetze, außer für Fälle, in denen ein Bundesorgan ermächtigt war. Den Bundesverwaltungsorganen stand für den Vollzug von Bundesgesetzen kein unbeschränktes Durchgriffsrecht zu, sondern sie hatten den föderativen Dienstweg über die entsprechenden Organe der Republiken einzuhalten (vgl. \cite{roggemann}: 286).\par
Die wesentlichen Ausführungsorgane hinsichtlich der Verwaltung waren die Gemeinden. Sie hatten grundsätzlich die Rechtsnormen der Gemeinden, der Republiken und autonomen Provinzen sowie des Bundes auszuführen. Weiterhin oblag ihnen die Durchführung des Verwaltungsverfahrens. Die Verwaltungsorgane der Republiken und autonomen Provinzen waren im Wesentlichen in einer Aufsichts- und Koordinierungsfunktion tätig. Die Ausführung durch die Gemeindeebene und die Aufsichtsfunktion durch die Ebene der Republiken und autonomen Provinzen haben viel Kritik auf sich gezogen und wurden von der Bundesebene als nicht effizient dargestellt (vgl. \cite{beckm90}: 209ff.). Die komplizierte Machtbalance zwischen der Föderation und den Republiken und autonomen Provinzen, ab 1974 angereichert um die verstärkten Selbstorganisationsbefugnisse, führte zu einer zunehmenden Lähmung des Staatsapparates. Nach Einschätzung von zeitgenössischen Beobachtern wendeten die Republiken und Gemeinden eine Taktik der „Nichtausführung“ von Gesetzen an. „Es ist vielmehr wesentlich bequemer, auf der Bundesebene für diese Maßnahme einzutreten – mit dem geheimen Vorbehalt, sie nicht auszuführen und so kurzfristige egoistische Ziele zu verfolgen. Ein weiterer Grund für den mangelhaften Gesetzesvollzug liegt in vielen Gesetzen selber: sie sind zum Teil überfrachtet mit Absichtserklärungen, unnötigen Wiederholungen und zu allgemein gefasst, so dass es nahezu unmöglich ist, ihnen eine klare Verhaltensnorm zu entnehmen“ (vgl. \cite{beckm90}: 224). Elemente der französischen Verwaltungstradition, die im 19. Jh. Einfluss auf Slowenien und Teile Kroatiens hatten, bestanden auch im sozialistischen Jugoslawien weiter.

\subsubsection{Vom Beamten zum Staatsbediensteten}

Bis zum Jahr 1946 war das 1931 für das Königreich Jugoslawien erlassene Beamtengesetz in Kraft, das alle Grundelemente, die für den Beamtenstatus typisch sind, beinhaltete. Mit einem neuen Gesetz wurde 1946 den staatlichen Bediensteten das hoheitliche Entscheidungsrecht entzogen. Sie waren nun weder Träger der Staatsgewalt, noch konnten sie diese ausüben. Dieses Recht kam nur noch den gewählten Funktionären zu. Der Begriff Staatsbediensteter wurde geprägt, worunter umfassende Verwaltungs- und Facharbeit unter Anleitung der gewählten Funktionäre zu verstehen war. Staatsbedienstete übten also wesentlich umfassendere Aufgaben aus als Beamte, aber ohne deren Kompetenzen. Bis 1957 war diese Rechtsordnung in Kraft (vgl. \cite{vavpetic}: 108).\par
1957 ersetzte das Bundesgesetz über die Staatsbediensteten das gleichnamige Gesetz von 1946. Beide Gesetze gingen von einer besonderen Rechtsstellung der Staatsbediensteten aufgrund ihrer besonderen Aufgaben aus. Die unterschiedlichen Regelungen für Staatsbedienstete und das allgemeine Arbeitsrecht wurden in der neuen Verfassung von 1963 aufgegeben (vgl. \cite{sevic}: 55). So verlor das Gesetz über die öffentlichen Bediensteten 1965 seine Gültigkeit. Das neue Gesetz über die Arbeitsverhältnisse enthielt den Begriff öffentlicher Bediensteter nicht mehr. Man sprach nur noch vom Arbeiter, unterschieden nach Hand- oder Kopfarbeiter. Für alle galten die gleichen Bestimmungen. \par
Nun wurde das gleiche Recht, das Arbeitsrecht, auf alle öffentlich Bediensteten angewandt (vgl. \cite{markic}: 12). Die Leistungen, die von diesen öffentlichen Bediensteten erbracht wurden, kann man im Sinne eines „public service“ definieren. Die jugoslawische Besonderheit war, dass viele Organisationen nicht direkt dem Bundesstaat unterstanden, sondern im Selbstverwaltungssystem als Gesellschaftsorganisationen bestanden (z.B. Gesundheits-, Wissenschafts- oder Bildungseinrichtungen) (vgl. \cite{vavpetic}: 113).

\subsubsection{Verwaltungsgerichtsbarkeit }

Nach dem Zweiten Weltkrieg wurde 1952 wieder eine generelle administrativ-juristische Kontrolle der öffentlichen Verwaltung eingeführt. Damit war Jugoslawien das einzige kommunistische Land, das einen gerichtlichen Verwaltungsrechtsschutz vorsah. Die Verfassung Jugoslawiens von 1974 delegierte die Organisation der Justiz an die Republiken. Auseinandersetzungen über administrative Entscheidungen wurden meist von den höchsten Gerichten der föderalen Strukturen (Republiken und Autonomen Provinzen) entschieden (vgl. \cite{roggemann}: 317). Der meist letztinstanzlichen gerichtlichen Überprüfung musste ein behördliches Widerspruchsverfahren vorangegangen sein. Für die Anfechtung von Akten der Bundesverwaltungsorgane war das Bundesgericht zuständig. Wirtschaftsgerichte waren für die Anfechtung bestimmter Akte der Finanzverwaltung zuständig sowie Militärgerichte für Maßnahmen der Militärbehörden (vgl. \cite{beckm90}: 234). Rechtsstreite in Fragen des Dienstrechts der Verwaltungsangestellten wurden von den Arbeitsgerichten entschieden. Kontrolle der öffentlichen Verwaltung wurde durch eine Staatliche Kontrollkommission ausgeübt, die allerdings als unzulänglich eingeschätzt wurde. Während die individuelle Verfassungsbeschwerde 1974 in der Verfassung abgeschafft wurde, wurde jetzt den Gerichten diese Grundrechtskompetenz zugewiesen, die sie im Verwaltungsstreitverfahren wahrnehmen sollten (vgl. \cite{roggemann}: 318).

\subsubsection{Die Staatskrise Jugoslawiens}

Die Verfassungsreform von 1974 hat die Dezentralisierungstendenzen im jugoslawischen Staat fortgesetzt, womit Nationalitätenkonflikte innerhalb des Landes kanalisiert und aufgefangen werden sollten (vgl. \cite{roggemann}: 282). Gleichzeitig änderte sich an den wirtschaftlichen Disparitäten zwischen den Republiken und Provinzen im Laufe der Zeit wenig. Unterentwickelte, meist stark agrarisch geprägte Regionen, insbesondere Kosovo, Mazedonien und auch Montenegro konnten kaum aufholen und wurden wirtschaftlich zunehmend „abgehängt“. Auch ein Ausgleichsfonds, der Gelder von den wirtschaftlich besser gestellten Republiken (vor allem Slowenien und Kroatien, die 45\% der Mittel des Fonds bereitstellten) in die randständigen Gebiete umverteilen sollte, zeigte keine nachhaltige Wirkung (vgl. \cite{ramet}: 380).\par
Im Zuge der sich zuspitzenden Staatskrise nach Titos Tod 1980 erstarkten die Teilstaaten auch aufgrund des zentralstaatlichen Machtvakuums und übernahmen Machtbefugnisse der Zentralregierung. Sowohl eine 1983 eingeleitete Wirtschaftsreform als auch die später einsetzende Reform des politischen Systems wurden in langwierigen Prozessen von eigens eingesetzten Kommissionen in mehrjähriger Arbeit mit Abschlussberichten vorbereitet. Um zunehmende öffentliche Kritik aufzufangen, wurde 1982 eine Analyse des politischen Systems in Jugoslawien im Auftrag des Kongresses des Bundes der Kommunisten Jugoslawiens (BdKJ) erstellt. Nach umfassenden Beratungen wurden die Ergebnisse der Bevölkerung vorgestellt als Broschüre „Kritische Analyse“ in einer Sonderbeilage zu den überregionalen Tageszeitungen im Januar 1986. Einer der Ausgangspunkte für die Arbeit an der „Kritischen Analyse“ war die Lähmung des politischen Systems durch das faktische Vetorecht der Republiken und Provinzen bei allen Gesetzesinitiativen, das dazu führte, dass kaum noch Entscheidungen getroffen werden konnten (vgl. \cite{reuter}: 399).\par
Die sich zuspitzenden Schwierigkeiten in Jugoslawien schrieb 1985 der Präsidiumsvorsitzende des ZK des BdKJ auch der Staatsstruktur zu, welche die egoistische Verfolgung von Eigeninteressen der Teilrepubliken ermöglichte. „Der polyzentristische Etatismus ist die Hauptursache für [unsere] Wirtschaftskrise, technologische Stagnation und unsere finanzielle Abhängigkeit vom Ausland. […] Im vergangenen und im laufenden Jahr haben wir gravierende Probleme bei der Entscheidungsfindung im Bund, vor allem bei der Kompromissfindung zwischen den Republiken und Provinzen miterlebt. […] Es ist unannehmbar, dass Delegationen und oft auch den Abgeordneten in den Räten des Parlaments der SFRJ Anweisungen gegeben werden, wie lange sie bestimmte Standpunkte diskutieren sollen, wann und bis zu welchem Grad sie nachgeben sollen, um auch die belanglosesten Interessen ihrer Republiken und Provinzen zu befriedigen“ (Tanjug, 22.9.1985, zit. in:  \cite{ramet}: 449).\par
Im Zuge wirtschaftlicher Schwierigkeiten mit einer zunehmenden Schuldenkrise zerbrach Jugoslawien nicht zuletzt in Folge der politischen Umwälzungen von 1989. Lampe benennt als allgemeine Ursachen der jugoslawischen Krise am Ende der 1980er Jahre:
\begin{itemize} \itemsep1pt \parskip0pt \parsep0pt
\item Veränderung des internationalen Umfelds (Epochenwandel 1989/91) – Der Westen verliert das Interesse an einem eigenständigen Jugoslawien – Ende der „Alimentation“ Jugoslawiens;
\item allgemeine Diskreditierung aller Formen von Ein-Parteiensystemen;
\item innerwirtschaftliche Probleme (notwendige Investitionsanstrengungen zugunsten von Lohnsteigerungen eingeschränkt) (vgl. \cite{lampe}: 329).
\end{itemize}
Für den weiteren Fortgang der Untersuchung ist wichtig festzuhalten, dass sich Jugoslawien stark von der extrem zentralistisch ausgerichteten Verfasstheit in den ehemals kommunistischen Staaten Osteuropas unterschied. In Jugoslawien bestand nach dem Zweiten Weltkrieg die Tradition kontinentaleuropäischer Verwaltungsstrukturen zunächst fort, bevor diese im neuen sozialistischen Staat sukzessive umgebaut wurden. Das anfänglich noch bestehende System des Berufsbeamtentums wurde umgewandelt und ab den 1970er Jahren arbeitsrechtlich nicht mehr unterschieden zwischen Personen, die staatliche Aufgaben wahrnahmen und Arbeitern sowie anderen Angestellten. Im Gegensatz zu diesem Abbau des Berufsbeamtentums hatte das Prinzip der Überprüfbarkeit von Verwaltungshandeln, zumindest nominell, in Jugoslawien Bestand. Weiteres Kennzeichen der Entwicklung in Jugoslawien war die starke Zunahme der Kompetenzen der Republiken, die einherging mit der Abnahme der Gestaltungsmacht und Zuständigkeit der Zentralregierung. Dies fand seinen Höhepunkt in der Verabschiedung der Verfassung von 1974. Ab diesem Zeitpunkt beschleunigte sich die Abnahme der staatlichen Einheit mit dem weiteren Erstarken der Republiken, die den Einfluss der Zentralregierung zu minimieren versuchten. Allerdings ist zu beachten, dass durch den Zugriff der Partei auf die Administration aller Ebenen nicht von einer „echten“ Dezentralisierung gesprochen werden kann. Das endgültige Auseinanderfallen des Staates begann in Folge der politischen Wende in Osteuropa 1989 und die Neugründung von Staaten auf der Basis der bisherigen Republiken wurde begleitet von ethnischen Abgrenzungskonflikten. \par


\subsection{Historische Verwaltung Albaniens }
Zur Darstellung der Entwicklung in Albanien kann auf die Masterarbeit der Autorin zum Thema „Dezentralisierung des öffentlichen Sektors in Albanien im Spannungsfeld von Verwaltungsmodernisierung und EU-Annäherung“ zurückgegriffen werden, die im Sommersemester 2007 im Studiengang „Öffentliches Management“ zur Erlangung des Grades „Master of Public Administration“ (MPA) im Fachbereich „Wirtschaftswissenschaften“ der Universität Kassel vorgelegt wurde. Auszüge aus der Masterarbeit wurden in der vorliegenden Arbeit eingefügt und die entsprechenden Kapitel mit „Vollmer 2007“ kenntlich gemacht.\par
Nach dem chronologischen Ablauf wird die Entwicklung Albaniens in „Vor-osmanische und osmanische Zeit“ sowie weitere Unterabschnitte zu den einzelnen historischen Stationen unterschieden.
\subsubsection[Vor-osmanische und osmanische Zeit]{Vor-osmanische und osmanische Zeit\footnote{Vgl. \cite{vollmer07}: 7ff.}}
Das heutige geografische Albanien befand sich von der Antike bis zur osmanischen Eroberung im 14. Jahrhundert unter römischer, byzantinischer, später serbischer und bulgarischer, zeitweise auch normannischer und neapolitanischer Herrschaft. Bis zur Staatsgründung 2912 war das Gebiet fünfhundert Jahre lang Bestandteil des Osmanischen Reichs (vgl. \cite{winnifrith}: 16).\par
Die türkische Besatzung hatte akzeptiert, dass die Nordalbaner der Berge keine Steuern zahlten und räumte ihnen im Rahmen ihrer ungeschriebenen Gesetze, die sie seit Jahrhunderten befolgten, Selbstverwaltung ein. Es gab keine Straßen in das unwegsame Gebiet und die erste Straßenverbindung wurde unter der österreichischen Militärverwaltung 1916 von Shkodra nach Durres gebaut (vgl. \cite{hasluk}: 9). In den unzugänglichen Bergregionen wurde eine Autonomieregelung mit Verbindungsmännern zur zentralen Regierung und Eigenverantwortung für regionale administrative Aufgaben angewandt. Die Stammes- und Clanstrukturen blieben in diesen Landesteilen bestimmend für die Sozialisation. Interessengegensätze innerhalb und zwischen den Stammesgruppen und Familien unterlagen nach wie vor der Reglung durch das Gewohnheitsrecht. Eine zentrale politische Autorität konnte die osmanische Herrschaft in diesen Gebieten nicht durchsetzen (vgl. \cite{hens99}: 74).\par
Gegen Ende des 17. Jh.s führte eine Änderung in der osmanischen Finanzpolitik zu einem Islamisierungsschub in Albanien. Während bislang die Steuern für Nichtmuslime moderat waren und von der gesamten Gemeinde eingebracht wurden, erhob man nun Steuern auf individueller Basis. Die einsetzende Islamisierung albanischer Christen war dennoch oft eine formelle und die alten Stammesbräuche lebten weiter (vgl. Zhelyakova, 2002: 243). Mit dem zunehmenden Machtverlust des Osmanischen Reiches im 18. Jahrhundert, setzten auch in Albanien Sezessionsbestrebungen ein. Im Norden gelang es dem Provinzstatthalter von Shkoder eigene Handelsbeziehungen und diplomatische Beziehungen zu europäischen Mächten aufzubauen (vgl. \cite{vickers}: 18f).\par
Mit umfangreichen Reformen versuchten die osmanischen Machthaber ab den 30er Jahren des 19. Jahrhundert dem drohenden Zerfall des Imperiums entgegenzuwirken. Nichtmuslime wurden mit Muslimen gleichgestellt, ein neues Justizsystem wurde eingeführt und das Steuersystem neu organisiert. Die Reformen hatten Auswirkungen vor allem in Flachlandgebieten, wo Formen einer modernen Verwaltungsführung entstanden. In den Bergregionen und aufgrund von fehlender Infrastruktur in unzugänglichen Gebieten stellten nach wie vor die Familie und der Clan den Hauptbezugspunkt der persönlichen und kollektiven Identität (vgl. \cite{kaser}: 292).\par
Nach einem Aufstand erkannte die türkische Herrschaft Albanien als autonome Administration mit den vier Provinzen Scutari, Kosovo, Monastir und Janina an und es wurden lokale Verwalter eingesetzt. Schulen konnten nun in albanischer Sprache unterrichten. Dieser Aufstieg eines autonomen Albanien kollidierte mit Gebietsansprüchen Serbiens auf das Kosovo und Montenegros auf Scutari (Shkodra), die auch die europäischen Mächte auf den Plan riefen. In einem Krieg der Balkanstaaten 1912 gegen die Türkei wurden albanische Gebiete von anderen Nachbarländern (Montenegro, Serbien und Griechenland) besetzt. Die Interessen Italiens und Österreich-Ungarns in der Region und der Antagonismus mit Russland, das Serbien unterstützte, ließen einen internationalen Konflikt am Horizont aufscheinen (vgl. \cite{chek}: 63).\par
 In der folgenden Landkarte ist ersichtlich, dass zwar Gebiete der späteren Staaten Mazedonien
und Albanien um 1912 noch unter Einfluss des Osmanischen Reiches waren,
Montenegro dagegen nicht.
\begin{figure}[H]
  \centering
\setlength\belowcaptionskip{10pt}
  \caption{Einfluss des Osmanischen Reiches im Balkan 1912}
  \includegraphics[width=5in]{Material/Balkan_1912}\\
\vspace{10pt}
Quelle: http://de.wikipedia.org/w/index.php?title=Datei:Balkan\_1912.svg\\ (Aufgerufen: 22.8.2012).
\end{figure}

\subsubsection[Staatsgründung 1912 ]{Staatsgründung 1912\footnote{Vgl. Vollmer 2007: 10.}}

Im Vertrag von Bukarest, mit dem der Zweite Balkankrieg zu Ende ging, legten die führenden Mächte die Grenzen Albaniens fest. Die Gründung des albanischen Nationalstaates am 28. November 1912 nach fast 500 Jahren unter osmanischer Herrschaft war im Interesse der Großmächte. Angesichts des Machtverlustes des Osmanischen Reiches sollte ein Gegengewicht geschaffen werden gegen eine slawische Expansion (vgl. \cite{biber}: 18). In der Beilegung des Konfliktes der Balkanländer und ihrer jeweiligen Verbündeten durch eine internationale Konferenz wurde erstmals ein unabhängiges Albanien anerkannt. Die Abtrennung von Albanern bewohnter Gebiete wurde beschlossen\footnote{Von Albanern bewohnte Gebiete, die von Serbien (Kosovo), Montenegro (ein Gebiet am Mittelmeer) und Griechenland (Cameria) besetzt waren, wurden den Besatzern zugesprochen.} und Albanien sollte von einem Prinzen\footnote{Der Deutsche Verwalter, Prinz Wilhelm zu Wied verließ aufgrund lokaler Widerstände bei Ausbruch des ersten Weltkriegs das Land (\cite{biber}: 18). }, auf den sich die europäischen Mächte einigten, regiert werden. Alle sechs europäischen Mächte (Österreich-Ungarn, Deutschland, Italien, Russland, Frankreich, Großbritannien) sollten in einer internationalen Kontrollkommission für Albanien präsent sein. Die Kontrollkommission hatte die Aufgabe, die Ausgaben überprüfen, der Regierung bei der Administration des Landes zu assistieren und einzuschreiten bei juristischen Übertretungen der albanischen Regierung. Die Kommission war einflussreich aber uneffektiv, da ihre Mitglieder hauptsächlich die Interessen ihrer Entsendungsregierungen vertraten, die sich zum Teil diametral gegenüberstanden (vgl. \cite{chek}: 127).\par
Im Anschluss an die Staatsgründung wurde das politische Geschehen dominiert von dem Wunsch nach Wiedergewinnung der verlorenen Gebiete und der Sorge um die instabile Lage auf dem Balkan. Der Aufbau und die Modernisierung des Staatswesens kamen nicht voran (vgl. \cite{hens99}: 82).
\subsubsection{K.u.k. Militärverwaltung 1916-1918}

Während des Ersten Weltkrieges befanden sich Nord- und Mittelalbanien von 1916 bis 1918 unter österreichisch-ungarischer Militärverwaltung. Wie der Entwicklungsstand des besetzten Landes eingeschätzt wurde, wird in einer geheimen Note des österreichisch-ungarischen Generalstabes exemplarisch deutlich: „Das albanische Volkskonglomerat bedarf einer Generationen dauernden kultivierenden und erzieherischen Tätigkeit mit straffer Hand im Rahmen unserer Monarchie, um zu jener Blüte zu gelangen, zu der es vom wirtschaftlichen Standpunkt fähig wäre.“\footnote{HHSTA, PA I, Karton 1001, Geheime Note des k.u.k. Chefs des Generalstabes, 28.Mai 1916, OP.No.25.492.} Zum Zustand der Administration wird ebenfalls Stellung genommen: „Die lang anhaltende türkische Herrschaft ist zweifellos die Hauptträgerin der Schuld, sie hat das Land nicht gedeihen lassen, das Volk aber verdorben. Fast methodisch wurde Albanien von allen äußeren kulturellen Einflüssen ferngehalten [...] die Regierung wollte einfach die primitiven Verhältnisse, weil sie glaubte, auf diese Weise am wenigsten Scherereien zu haben. Freiheiten ließ sie genug, sie schuf Privilegien, hob fast keine Staatssteuern ein etc., lange ging die Sache so hin, endlich drangen aber doch andere, moderne, wenn auch unausgegorene Ideen ins Land, es kamen die Balkanwirren und jetzt erscheinen kaleidoskopartig die verschiedenen Verwaltungen und Regierungen: die des Ismail, die internationale, die Wiedsche, jene Essads, von Serben und Montenegrinern (in Teilen), endlich die österreich-ungarische.“\footnote{HHSTA, PA I, Nr. 1001, Geheime Note des k.u.k. Chefs des Generalstabes, 28.Mai 1916, OP.No.25.492, Beilage 1.} Anders als in Montenegro, wo von der österreichisch-ungarischen Besatzungsmacht eine funktionierende Verwaltung vorgefunden wurde, gab es in Albanien allenfalls archaische Verwaltungsstrukturen.\par
Aus einem militärischen Lagebericht wird die Problematik einer zu ambitionierten Verwaltungsreform seitens der Besatzer selbstkritisch angemerkt: „Bezüglich der Behandlung der albanischen Verwaltungsangelegenheiten wird vorläufig an dem Grundsatz festgehalten, nicht allzu reformierend aufzutreten, da ansonsten die alte eingelebte, wenn auch schwerfällige und verzopfte, aber doch zur Not entsprechende Regierungsform genommen und wegen Mangel an geeigneten Personen keine sofort brauchbare neue gegeben wird. Aus diesem Grund werden im Allgemeinen die alten türkischen Gesetze und Vorschriften beibehalten und nur die notwendigsten Änderungen vorgenommen.“\footnote{HHSTA, PA I, Karton 1001, Lagebericht Feldpostamt 140, 9.September 1916, Trollmann, k.u.k. XIX Korpskommando, E.V. Nr. 962/IX.}\par
In der Folge waren erhebliche Fortschritte in der Reformierung bzw. dem Aufbau der Zivilverwaltung zu verzeichnen, mit Ausgestaltung der verschiedenen Zweige der Administration und der Einteilung aller Landesangestellten in bestimmte Rangklassen mit festgelegtem Gehaltsschema. Es wurde eine Dienstpragmatik erlassen sowie die Arbeitszeit und Öffnungszeiten der Ämter festgelegt.\footnote{HHSTA, PA I, Karton 1006, Z. 56/P.Kral, Shkodra, 5. April 1917. } Weiterhin wurden Verfahren vor den Zivilgerichten kodifiziert, basierend auf türkischem Recht. Im Sommer 1916 wurde ein Gerichtssystem eingerichtet mit Friedensgerichten für kleinere Zivil- und Handelsstreitigkeiten, Berufungsgerichten, und für größere Streitigkeiten in vier Städten und dem Appellationsgericht in Shkodra. In den Jahren 1916 und 1918 wurde eine Volks- und Viehzählung durchgeführt, die offenbar mit einigem Aufwand verbunden war. „Mit Rücksicht darauf, dass in diesem Lande, speziell in den schwer zugänglichen Gebirgsdistrikten eine derartige Zählung noch nie stattgefunden hatte, muss die einschlägige Aktion, die verhältnismäßig gut gelungen ist, als ein bedeutungsvoller Fortschritt in der Verwaltung Albaniens bezeichnet werden.“\footnote{HHSTA, PA I, Karton 1006, Z. 184/P.Kral, Shkodra, 16. November 1916.} Im Januar 1918 wurden die Erbschaftsangelegenheiten der Christen von den Sheriatsgerichten an die Zivilgerichte verwiesen und ein eigenes Erbrecht wurde in Auftrag gegeben. Weiterhin wurde die Neuorganisation des Schulwesens in einer Weise durchgeführt, die „durchaus modernen Anforderungen standhalten konnte und in gleicher Weise auf bodenständige Faktoren Rücksicht nahm“ (\cite{schwanke}: 177). Während der österreichisch-ungarischen Besatzungszeit sind umfangreiche Arbeiten am Straßennetz erfolgt und ca. 840 km Straßen und zahlreiche Brücken gebaut sowie Telefon- und Telegrafenverbindungen eingerichtet worden. Während diese vor allem militärischen Zwecken dienten, haben sie doch zu einer nachhaltigen Verbesserung der Infrastruktur beigetragen.\par
Die Organisation des Steuerwesens in der Okkupationszeit konzentrierte sich darauf, die alten Gesetze und Gewohnheiten beizubehalten und diese erst allmählich zu modernisieren und auf Gebiete auszudehnen, die zu türkischer Zeit keine oder nur unter großen Schwierigkeiten Steuern gezahlt hatten. So wurde die Zehntsteuer beibehalten und auch die Gebiete der Ebene um Shkodra erstmals einbezogen, während in der Gebirgsregion aufgrund von Armut und schlechter Ernte eine kumulative Kopfsteuer eingezogen wurde.\footnote{Vgl. HHSTA, PA I, Karton 1006, Nr. 163/P.Kral, Shkodra, 30. Juni 1918.}\par
Die Früchte dieser Arbeit zur Modernisierung der Verwaltung konnte die Monarchie am Ende des Weltkrieges nicht mehr ernten. Umso mehr baute der 1921 neu erstandene albanische Staat seine Verwaltung auf den von der k.u.k Militärverwaltung geleisteten Vorarbeiten auf. „Ohne die von Österreich-Ungarn geleistete Pionierarbeit auf dem Verwaltungssektor wäre eine beinahe dreijährige Verwaltung mit ihrer lang nachhaltigen Wirkung über den Ersten Weltkrieg hinaus nicht möglich gewesen“ (\cite{schwanke}: 89). Politischer Hintergedanke dieser Aufbauarbeit war sicherlich auch das Ziel, eine günstige Ausgangsstellung in Bezug auf politische und wirtschaftliche Beziehungen mit dem strategisch günstig gelegenen Albanien zu schaffen.

\subsubsection[Zwischenkriegszeit]{Zwischenkriegszeit\footnote{Vgl. \cite{vollmer07}: 11ff.}}

In den frühen 20er Jahren war Albanien zwischen zwei grundlegend gegensätzlichen Kräften aufgeteilt: Einerseits konservativen Lokalfürsten, die Albanien nach den alten osmanischen Regeln beherrschen wollten. Auf der anderen Seite fanden sich liberale Intellektuelle, demokratische Politiker und fortschrittliche Handeltreibende, die sich unter der Führung eines südalbanischen Bischofs der orthodoxen Kirche nach Westen orientierten.\par
In den 1920er Jahren erlangte ein lokaler Führer aus dem Norden mittels eines militärischen Coup die Macht und etablierte 1928 eine Monarchie. Eine Reform des Landes fand nicht statt und trotz immenser Nahrungsmittelimporte verließen tausende Albaner ihr Land auf der Suche nach einem besseren Leben. Es bestand zwar eine lokale Verwaltung, doch die von der Zentralregierung ernannten Präfekten übten politische Kontrolle über die Regionen und damit über die Distrikte und Kommunen aus. Lediglich die Distriktversammlungen (insgesamt 36 Distrikte) waren gewählt, alle anderen Funktionen inklusive der Bürgermeister wurden ernannt. In den Gebieten des Nordens fand weiterhin keine Eingliederung in staatliche Strukturen statt, es wurden keine Steuern gezahlt und Verordnungen des Staates konnten nur mit Zustimmung der lokalen Führer durchgesetzt werden (vgl. \cite{biber}: 19).\par
Die Unabhängigkeit Albaniens 1912 brachte keine nennenswerten Fortschritte für das wenig entwickelte Land. Traditionelle Gesellschaftsstrukturen und die schwierige ökonomische Situation behinderten eine Modernisierung. Albanien war fast vollständig von Wareneinfuhren und Hilfslieferungen abhängig, die vor allem von Italien zur Verfügung gestellt wurden. Italien stellte auch Anleihen und Investitionen zur Verfügung, was zu einem quasi-kolonialen Status führte. Nach Beginn des Zweiten Weltkrieges annektierte das faschistische Italien 1939 Albanien (vgl. \cite{fisch84}: 283).
\par
Zusammenfassend wird deutlich, dass einer Modernisierung des Staatswesens in Albanien historisch betrachtet mehrere Entwicklungen entgegenstanden. Zunächst führte die Jahrhunderte dauernde osmanische Herrschaft zur Abschottung des Landes gegen kontinentaleuropäische Einflüsse. Auch die Isolation der Bewohner in den zahlreichen Bergregionen des Landes mit Gewohnheitsrecht und traditionalen Vergesellschaftungsformen wirkte sich aus. Besonders im Norden des Landes waren bis Ende des Zweiten Weltkrieges praktisch alle Versuche zu einer Modernisierung der Verwaltung ergebnislos verlaufen. Die zweijährige Zeit der k. u. k. Militärverwaltung in der nördlichen Hälfte von Albanien während des Ersten Weltkrieges kann dennoch als eine Zeit moderner Einflüsse auch in verwaltungstechnischer Hinsicht angesehen werden, auch wenn diese Einflüsse regional begrenzt waren.\par
Die kurze Phase Albaniens als unabhängiger Staat fiel in die Zeit zwischen beiden Weltkriegen und führte zu weiteren Besatzungen und schließlich zur Annexion durch das faschistische Italien. Aufgrund äußerer Bedrohungen und fehlender Erfahrungen der Administratoren sind auch in dieser Phase keine effektiven und umfassenden Neuerungen in der Verwaltungsstruktur des Landes durchgeführt worden. Nach dem Zweiten Weltkrieg etablierte sich in Albanien ein kommunistischer Staat mit spezifischen Auswirkungen auf die Verwaltungsentwicklung, die im Folgenden skizzenhaft dargestellt werden.
\subsubsection[Das kommunistische Albanien]{Das kommunistische Albanien\footnote{Vgl. \cite{vollmer07}: 14ff.}}

Im Widerstand gegen die Besatzung durch zunächst Italien und dann Deutschland während des Zweiten Weltkrieges setzte sich die kommunistisch orientierte Partisanenbewegung durch. Eine kommunistische Partei wurde in Albanien 1941 gegründet und 1944 eine Verfassung verkündet.\footnote{Eine direkte Kopie der jugoslawischen Verfassung.} Im Dezember 1945 fanden Wahlen statt, in denen fast ausschließlich Kandidaten der Kommunistischen Partei angetreten waren. Die Partei erhielt 93\% der Stimmen und die Volksrepublik mit Enver Hoxha an der Spitze wurde ausgerufen. Verbliebene Kritiker der Überführung Albaniens in einen kommunistischen Staat wurden umgebracht, vertrieben oder interniert (vgl. \cite{vickers}: 164).\par
Das Ziel der Verstaatlichung des Agrarsektors war Mitte der 70er Jahre weitestgehend erreicht, wobei auch erstmals auch die Bergregionen im Norden erfasst wurden. Die Industrialisierung des Landes wurde nach dem Ende des Krieges mit Hochdruck betrieben, wobei der Ausbau der Industrieproduktion stark von ausländischen Geldern und Kooperationen abhängig war.\footnote{Hauptsächlich Jugoslawien, Sowjetunion, COMECON und China.} Der Rückgang der selbstständigen Betriebe und die massive Industrialisierung mit Staatsbetrieben führten dazu, dass der Staat zum Hauptarbeitgeber wurde (vgl. \cite{hens99}: 89). Bürokratisierung auch einfacher Aufgaben und Bestechlichkeit des Personals waren während des kommunistischen Regimes Kennzeichen staatlicher Herrschaftsausübung. Der Zugang zu allen attraktiven Führungs- und Entscheidungspositionen wurde von der Partei reguliert. Die in allen sozialistischen und kommunistischen Regimen starken personellen und ideologischen Loyalitätsbedingungen trafen in Albanien auf die durchaus ähnlichen Muster der Clan- und Stammesstrukturen, die über Jahrhunderte das Leben in der traditional geprägten Gesellschaft bestimmt hatten (vgl. \cite{vickers}: 189). Auch gegen die Gegner des Systems wurde die Tradition der Loyalität zum erweiterten Familienkreis instrumentalisiert. In Ungnade gefallene Funktionsträger wurden oft mitsamt ihrer Familien in Internierungslager verbannt. Ähnlich anderen kommunistischen Ländern, wurde in Albanien die öffentliche Verwaltung stark zentralistisch und vertikal ausgerichtet. Eine lokale Ebene bestand zwar und wurde durch Wahlen alle vier Jahre bestimmt, doch hatte diese Ebene keinerlei eigenständige Politikgestaltungsbefugnis. Das Budget, mit Instruktionen zu seiner Verwendung, wurde zentral zur Verfügung gestellt. 36 Distrikträte waren zuständig für die Überwachung der Einhaltung der Produktionspläne der industriellen und landwirtschaftlichen Staatsbetriebe im jeweiligen Distrikt (vgl. \cite{hoxha}: 5). Die von den Kommunisten verfolgte Modernisierung der Gesellschaft gelang nur oberflächlich.

\subsubsection[Der Weg in die außenpolitische Isolation]{Der Weg in die außenpolitische Isolation\footnote{\cite{vollmer07}: 15ff.}}

Außenpolitisch beschritt Albanien einen Weg, der zu einem gewissen Teil mit der Verhaftung in vormodernen Traditionen erklärt werden kann. Dabei folgte auf eine Zeit der Freund-Feind-Konstellation mit jeweils einem starken Verbündeten, ab Ende der 1970er Jahre eine Zeit der zunehmenden und selbst gewählten außenpolitischen Isolation. Von 1944-48 war es vor allem Jugoslawien, das den albanischen Interessen als eigenständiger Staat bei den Vereinten Nationen Nachdruck verlieh und umfangreiche finanzielle Hilfe für den Aufbau des Staatswesens zur Verfügung stellte. Auf der anderen Seite war Jugoslawien bestrebt, Albanien ökonomisch sehr eng an sich zu binden, woraus die Befürchtung entstand, Jugoslawien wolle die politische Kontrolle über Albanien erlangen. 1948 kam es daher zum Bruch mit Jugoslawien gefolgt von einer Hinwendung zur Sowjetunion (vgl. \cite{odonnell}: 29ff). Die außenpolitischen Beziehungen zur Sowjetunion stärkten Albanien, das nun eine Weltmacht zum Verbündeten hatte. Allerdings kühlten sich die Beziehungen nach Stalins Tod 1953 deutlich ab. Zum endgültigen Abbruch der Beziehungen zur Sowjetunion kam es 1961.\par
Ab Mitte der 1950er Jahre trat China in Vordergrund und stellte ab 1957 Finanzhilfe für Albanien zur Verfügung. In der Folgezeit konzentrierte sich Albanien auf die ökonomische und militärische Unterstützung durch China, bis auch diese Allianz 1978 von Albanien aufgekündigt wurde. Nach dem Bruch mit der Volksrepublik China bis zum Tode Enver Hoxhas 1985 war Albanien außenpolitisch vollständig isoliert. Nach Beginn der Systemwechsel in Osteuropa verlor die kommunistische Partei in Albanien Ende 1990 ihr Monopol und die ersten freien Wahlen fanden am 31.3.1991 statt (vgl. \cite{ammann}: 486). Der Beitritt zur OSZE 1991 beendete formell die Isolation Albaniens und ein Strom von Hilfsgeldern floss in das Land (vgl. \cite{deza}: 5).\par
Für die Entwicklung Albaniens ergibt sich also, dass die historische Entwicklung bis zum Ende des Zweiten Weltkrieges ähnlich anderen Gebieten in der Region verlief, mit starkem Einfluss des Osmanischen Reiches und entsprechenden Prägungen auch in der Verwaltungsstruktur. In dieser Hinsicht verlief die Entwicklung in Albanien ähnlich der in Mazedonien. Mit Montenegro hat Albanien die geografische Besonderheit von unzugänglichen Bergregionen gemeinsam, die historisch zu einer verstärkten Abschottung dieser Gebiete vor Modernisierung und dem Weiterbestehen von traditionellen Clanstrukturen führte. Nach dem Ende des Zweiten Weltkrieges waren Montenegro und Mazedonien Teil der SFRJ mit ihrem spezifischen System des Selbstverwaltungssozialismus. Während in den kommunistischen Ländern des Ostblockes zu dieser Zeit eine zentralistische Wirtschafts- und Verwaltungsweise vorherrschte, entwickelten sich in Jugoslawien marktwirtschaftliche Elemente und eine besondere, von kommunistischen Ostblockländern unterschiedene Ausprägung der Verwaltung. Albanien wiederum war ähnlich den Ostblockstaaten in seiner Verwaltung streng zentralistisch organisiert, wobei die rückständigen Bergregionen Modernisierungstendenzen nicht zugänglich waren.
\par
Festzuhalten bleibt, dass in Albanien geschichtlich nie ein Modell einer funktionsfähigen und allgemein durchgesetzten lokalen Verwaltungsstruktur bestanden hat, an das nach der demokratischen Neuorientierung angeknüpft werden konnte. Hier besteht ein Unterschied zu anderen Transformationsländern mit kommunistischer Vergangenheit. In den osteuropäischen und den anderen südosteuropäischen Ländern, einschließlich der Nachfolgestaaten Jugoslawiens, bestanden wie auch immer geartete vorkommunistische lokale Verwaltungsstrukturen, an die angeknüpft werden konnte.\par
\section{Die Bedeutung von Legacies}
Bezogen auf die letzten EU-Erweiterungen liegen einige Untersuchungen vor, in denen speziell die Bedeutung von Legacies betrachtet wird. Verwaltungsentwicklung kommt dabei allerdings meistens nur am Rande in den Blick. Es wird konstatiert, dass in vormals kommunistisch geprägten Verwaltungen eine Diskrepanz besteht zwischen formaler Übernahme von demokratischen Regeln und den Reformprozessen der Administration (\cite{dimgoe, meyersah06}). Institutionelle Instabilität und personalisierte Machtausübung sowie der Einfluss politischer Parteien auf die Personalpolitik werden als weiterhin bestehende Merkmale in den vormals kommunistischen Ländern identifiziert (\cite{goewoll, meyersah08a}). Dabei wird in den Untersuchungen in der Regel von „der“ kommunistischen Verwaltungsstruktur gesprochen und somit impliziert, dass es sich um ein in allen post-kommunistischen Ländern ähnliches System handelte. Angesichts der Unterschiede in der Ausgestaltung des kommunistischen Systems in Osteuropa als Teil der Sowjetunion und der in vielen Bereichen von diesem System unterschiedlichen Ausprägung in Südosteuropa, ist anzunehmen, dass sich dies auch auf die Ausgestaltung der öffentlichen Verwaltung auswirkte. In Albanien bestand, bevor es in die Demokratie wechselte, ein kommunistisches Regime, das aber spezifische Unterschiede zu der Verwaltung in der Sowjetunion aufwies. Im früheren Jugoslawien, zu dem sowohl Montenegro als auch Mazedonien vor der staatlichen Unabhängigkeit gehörten, bestand wiederum ein spezifisches Modell der sozialistischen Verwaltung unter dem Stichwort „Selbstverwaltungssozialismus“, das ebenfalls wesentliche Unterschiede zum Verwaltungssystem der Sowjetunion aufwies.\par

Vor der sozialistischen bzw. kommunistischen Zeit übten unterschiedliche Großmächte Einfluss auf die Region aus und bestimmten damit auch die Weichenstellung für die öffentliche Verwaltung. Spezifisch für die Staaten des Westbalkans ist der Einfluss des Habsburger bzw. Osmanischen Reiches bis in das 20. Jahrhundert hinein. In den Untersuchungsländern waren Albanien und Mazedonien historisch vom Osmanischen Reich geprägt, während Montenegro früher westlichen rechtlichen Traditionen ausgesetzt war.\par
Unter dem Einfluss des Habsburger Reiches erlangte ein westlicher Absolutismus in aufgeklärter Form Einfluss auf Teile der Region, der zumindest in Ansätzen der „rule of law“ entsprach. In dieser Hinsicht als wesentlich weniger günstig wird die absolutistische Herrschaft des Osmanischen Reiches mit seiner im Weberschen Sinne extremen Variante des Patrimonalismus mit stark personalisierter Form der Machtausübung eingeschätzt. Wesentliche Kennzeichen dieser Herrschaft waren das Fehlen einer klaren Trennung des Staatshaushaltes von dem des Herrschers, die persönliche Abhängigkeit der Administratoren vom Herrscher und der Tradition als Basis des Staates (vgl. Diamandouros/Larrabee, 2000: 30). Der Niedergang des Osmanischen Reiches war gekennzeichnet durch zunehmende Desintegration der Institutionen im 19. Jahrhundert sowie durch Bestrebungen der Territorien, eigenständige Nationalstaaten zu bilden. Weiterhin sind die Länder des Balkans Ende des 19. Jh.s in das weltpolitische Interesse getreten. Die Großmächte Russland, England, Deutschland und Frankreich verfolgten jeweils eigene Ziele auf dem Balkan (vgl. \cite{farada02}: 21).\par
Dass diese vor-kommunistische Verwaltungstradition durchaus Einfluss hat auf die EU-Perspektive der Länder des Westbalkans mit schneller EU-Integration für Länder mit österreichisch-ungarischer Legacy und nur verhaltener EU-Perspektive für Kandidatenländer, die geschichtlich osmanischen Einflüssen ausgesetzt waren, konstatieren Emerson/Noutcheva: „The EU and the states of the region, seemingly obeying laws of historical determinism, have quickly seen to the accession of the former Austro-Hungarian Slovenia and soon next Croatia, while taking their time over the former Ottoman empire“ (\cite{emenou}: 13). \par
Im Zusammenhang mit der Verwaltungsentwicklung in den Ländern des Westbalkans ist ferner die historische Bedeutung von Clanstrukturen nicht zu vernachlässigen. Clanstrukturen behinderten oder verhinderten teilweise die Durchsetzung rationaler administrativer Strukturen in den Untersuchungsländern. \par
Die einzelnen Analyseergebnisse für die drei Untersuchungsländer können tabellarisch wie folgt zusammengefasst werden: 
\begin{table}[H]
\setlength\belowcaptionskip{10pt}
\caption{Überblick über historische Einflüsse in den Untersuchungsländern }
\footnotesize
\begin{tabular}{|R{36mm}|R{30mm}|R{30mm}|R{30mm}|}\hline
&{\bf Montenegro}&{\bf Mazedonien}&{\bf Albanien}\\\hline
Osmanisches Reich&
bis 1878&
bis 1913&
bis 1912\\\hline
Clanstrukturen &
ja&
weniger&
ja\\\hline
KuK Militärverwaltung &
1916-18&
nein&
1916-18\\\hline
Andere Verwaltungseinflüsse &
Kontinentaleuropäische&
Bulgarien, Griechenland&
Italien\\\hline
Sozialistische Föderative Republik Jugoslawien&
ja&
ja&
nein\\\hline
Kommunistischer, zentralistischer Staat&
nein&
nein &
ja\\\hline
 \noalign{\smallskip}
\multicolumn{4}{r}{}\\
\multicolumn{4}{c}{\normalsize Quelle: eigene Zusammenstellung.}\\
\end{tabular}
\end{table}
Aus dieser Tabelle ist ersichtlich, dass die drei Untersuchungsländer zum Teil ähnlichen und zum Teil unterschiedlichen Einflüssen in der geschichtlichen Entwicklung ausgesetzt waren. Die bisherige Untersuchung zeigt, dass diese Einflüsse auch spezifische Entwicklungen der öffentlichen Verwaltung in den einzelnen Ländern nach sich zogen. Im Rahmen des oben skizzierten Legacy Ansatzes ist davon auszugehen, dass die historischen Bedingtheiten sich auch im Status quo der öffentlichen Verwaltung der Untersuchungsländer wieder finden lassen und Bedeutung erlangen bei der Frage der weiteren Entwicklung, bzw. Förderung der öffentlichen Verwaltung im Zuge der EU-Erweiterung.\par

\section{Verwaltung in der Demokratie in den Untersuchungsländern}
Im folgenden Abschnitt wird die Verwaltungsentwicklung in den Untersuchungsländern seit dem demokratischen Wandel nach 1990 betrachtet. Alle drei Länder vollzogen einen Systemwechsel. Mazedonien und Montenegro sind Nachfolgestaaten Jugoslawiens, eines sozialistischen Staates mit spezifischen Merkmalen, die ihn von dem sowjetischen Staatsmodell unterschieden. Während Mazedonien nach dem Zerfall Jugoslawiens direkt ein unabhängiger Staat wurde, verblieb Montenegro bis 2006 in einer Staatenunion mit Serbien. In einem Referendum bestätigte die Bevölkerung 2006 ihren Wunsch nach einem eigenen Staat und Montenegro wurde selbstständig.\par
Albanien vollzog ebenso wie Mazedonien und Montenegro im Zuge der Auflösung des Ostblockes eine Systemveränderung. Allerdings geschah der Übergang zu einer parlamentarischen Demokratie ausgehend von einem isolationistischen kommunistischen Regime, mit spezifischen Problemen auch für die Verwaltungsentwicklung.\par
Als Grundlage für die länderspezifische Darstellung wird auf die von Kuhlmann/Wollmann entwickelte Systematik für Ländervergleiche der Verwaltung zurückgegriffen. Anhand von vier Kriterien wird die Verwaltung in der Demokratie für die drei Untersuchungsländer dieser Arbeit dargestellt. Das Ziel ist eine vergleichende Betrachtung, die Hinweise liefern kann für die Beantwortung der Untersuchungsfrage und der weiterführenden Fragen dieser Arbeit. Nach einer kurzen Einleitung zu jedem Land wird die Entwicklung jeweils anhand der folgenden vier Kriterien dargestellt:
\begin{itemize} \itemsep1pt \parskip0pt \parsep0pt
\item Basismerkmale des Regierungssystems,
\item Staatsaufbau und nationales Verwaltungsprofil,
\item Subnational-dezentrale Verwaltungsebene,
\item Öffentlicher Dienst (vgl. \cite{kuhwol}: 45ff.).
\end{itemize}
\subsection{Montenegro}
Nach dem Zerfall Jugoslawiens Anfang der 1990er Jahre bildete Montenegro ab 1992 zusammen mit Serbien das verbleibende Rumpf-Jugoslawien. Nachdem alle anderen jugoslawischen Teilrepubliken selbstständig geworden waren, verblieben nur Serbien und Montenegro in der Föderation. Aufgrund der Übernahme vieler Kompetenzen durch die Teilrepubliken war die zentrale Verwaltung des neuen Jugoslawien eher schwach und nur mit wenigen Kompetenzen ausgestattet. Weiterhin war aus Sicht Montenegros die Dominanz der Serben in der Zentralverwaltung ein Problem, das sich allerdings zum Teil aus dem Bevölkerungsgefälle ergab (Serbien: 10,5 Mio. und Montenegro 0,6 Mio. Einwohner). Aufgrund der schwachen Zentralregierung hatte sich in Montenegro eine weitgehende Selbstverwaltung entwickelt und in beiden Republiken (Serbien und Montenegro) wurde der civil service wieder eigenständig aufgebaut (vgl. \cite{sevic}: 57).\par
Erst 2006 fand die endgültige formale Abspaltung Montenegros von Serbien statt. Doch schon in den 1990er Jahren verselbstständigte Montenegro sich gegenüber Serbien. Die Jugoslawienkriege in den 1990er Jahren belasteten die Beziehung zwischen Montenegro und Serbien, und die montenegrinische Regierung strebte die Trennung von Serbien an. Westliche Kritik blieb weitgehend aus, als sich im Zuge der UN-Sanktionen gegen Serbien in den 1990er Jahren eine Schattenökonomie in Montenegro herausbildete.\par
Seit Ende 1998 gab es keine finanziellen Transaktionen zwischen dem montenegrinischen und jugoslawischen Haushalt mehr. Seit Anfang August 1999 verfolgte Montenegro eine eigene Zoll-, Handels- und Visapolitik und am 02.11.1999 wurde die D-Mark als Zweitwährung eingeführt. Während des Kosovokrieges (1999) erklärte sich Montenegro offiziell als neutral, positionierte sich de facto aber im westlichen Lager. Die jugoslawischen Präsidentschaftswahlen im September 2000 wurden von Montenegro boykottiert. Sowohl Serbien als auch Montenegro wären gerne ihren eigenen Weg gegangen. Serbien hatte nach den Fehlschlägen der Kriege kein großes Interesse an einer staatlichen Verbindung mit Montenegro; der Teilstaat wurde als finanzielle Belastung gesehen. Allerdings war der Zugang zum Mittelmeer (Montenegro) für die Binnenrepublik Serbien wichtig. Montenegro dagegen meinte, ohne Serbien den Weg nach Europa schneller zu finden als zusammen mit Serbien (vgl. \cite{schmitz04}: 115).\par
Im Jahr 2003 entstand trotz dieser Abstoßungserscheinungen durch Druck und Vermittlung der Europäischen Union eine lose Staatenverbindung zwischen Serbien und Montenegro. Es wurde festgelegt, dass Jugoslawien in eine Union der beiden Staaten mit weitgehender gegenseitiger Unabhängigkeit umgewandelt werden sollte. Am 04.02.2003 wurde in Belgrad formell die Auflösung Jugoslawiens beschlossen und die Union als neuer Staat „Serbien und Montenegro“ gegründet, eine Union, die nach einer dreijährigen Frist in einem Referendum bestätigt oder beendet werden konnte (vgl. \cite{hoenehhol}: 464).\par
Das Referendum zur Frage der staatlichen Unabhängigkeit führte Montenegro 2006 durch. Als Bedingung für die Unabhängigkeit war die Zustimmung von mindestens 55 Prozent der Wahlberechtigten festgelegt worden. Diese wurden knapp überschritten und mit einer Unterbrechung von 88 Jahren wurde Montenegro wieder ein souveräner Staat. Die von der EU unterstützte Staatenunion mit Serbien wurde beendet. „The experiment with building a State Union of Serbia and Montenegro, one of the very first EU supported state-building projects in the Western Balkans, ended with a ‘velvet divorce’ after three years of existence, during which the common state failed to capture the imagination of its population” (\cite{noutcheva}: 3).\par
Die geografische Lage von Montenegro ist aus der folgenden Karte ersichtlich mit der Hauptstadt Podgorica (früher Titograd) und der historischen Hauptstadt Cetinje.
\begin{figure}[H]

  \centering
   \caption{Landkarte Montenego}
  \includegraphics[width=4in]{Material/Montenegro_de}\\
 Quelle: http://upload.wikimedia.org/wikipedia/commons/0/0f/\\
Montenegro\_de.png (Aufgerufen: 10.10.2013).
\end{figure}

Montenegro besteht im Wesentlichen aus Gebirgslandschaft, hat aber auch Zugang zum Mittelmeer. Die Küstenregion ist sehr touristisch geprägt. Direkte Nachbarländer sind Albanien, Kosovo, Serbien und Bosnien-Herzegowina.


\subsubsection{Basismerkmale des Regierungssystems }

Am 19. Oktober 2007 trat eine Verfassung im selbstständigen Staat Montenegro in Kraft.\par
In der parlamentarischen Demokratie ist der Präsident das Staatsoberhaupt; er wird in direkter und geheimer Wahl für fünf Jahre gewählt. Er schlägt den Premierminister vor, der anschließend von dem 81 Abgeordnete umfassenden Parlament gewählt wird. Parlamentswahlen finden alle vier Jahre statt (\cite{osceodihr09}: 4).\par
Mit einer Fläche von 13.812 km$^2$ und ca. 600.000 Einwohnern ist Montenegro eines der kleinsten Länder Europas. Als Währung führte Montenegro unilateral zunächst die D-Mark und dann den Euro ein.\par
Außerhalb von Serbien lebt nach Bosnien-Herzegowina die prozentual größte Anzahl von Serben (30\%) in Montenegro. Dies hat in den letzten Jahren immer wieder zu politischen Auseinandersetzungen geführt. So. z.B. anlässlich der Unabhängigkeitserklärung Kosovos, die seitens der serbischen Bevölkerung in Montenegro abgelehnt, aber von den ca. 10\% montenegrinischen Albanern befürwortet wird (vgl. \cite{stanislaw}: 53).\par
In der folgenden Tabelle sind die wichtigsten Wirtschaftsdaten für Montenegro von 2007 bis 2010 überblicksartig dargestellt:

\begin{table}[H]
\setlength\belowcaptionskip{10pt}
\caption{Wirtschaftsdaten Montenegro 2007-2010}
\footnotesize
\begin{tabular}{|R{46mm}|R{25mm}|R{15mm}|R{15mm}|R{15mm}|R{15mm}|}\hline
&&2007&2008&2009&2010\\\hline
Bruttoinlandsprodukt (BIP)&
Millionen Dollar&3.668,9&4.519,7&4.141,4&4.111,1\\\hline
BIP Wachstum&\%&10,7&6,9&-5,7&2,5\\\hline
Inflation &\%&4,3&8,8&3,5&0,7\\\hline
Ausländische Direktinvestitionen &\% des BIP&25,5&21,2&36,9&18,5\\\hline
Verschuldung öffentliche Hand&\% des BIP&27,5&31,9&40,7&44,1\\\hline
\multicolumn{6}{c}{}\\
\multicolumn{6}{c}{\normalsize Quelle: \cite{bert12a}: 17 (eigene Übersetzung aus dem Englischen).}
\end{tabular}

\end{table}
 Aus dieser Tabelle ist ersichtlich, dass Montenegro ganz erheblich von der Finanzkrise betroffen war, mit stark gestiegener Verschuldung der öffentlichen Hand und rapidem Rückgang des Bruttoinlandsproduktes.

\subsubsection{Staatsaufbau und nationales Verwaltungsprofil}

Die montenegrinische Regierung hat mit Unterstützung der EU verschiedene institutionelle Reformen auf den Weg gebracht. Diese bezogen sich auf die Erbringung von öffentlichen Aufgaben, Modernisierung der Energieversorgung und den Bereich Umweltschutz. Im Fokus war auch die Reform der zentralen und lokalen Verwaltung. Korruption und politische Einflussnahme auf Medien und Justiz wurden allerdings noch im Jahr 2008 vom Europarat kritisiert, der mahnte, dass mehr für die Unabhängigkeit von Gerichten, der Anti-Korruptionseinheit und der staatlichen Medien getan werden müsse (vgl. \cite{stanislaw}: 53). Montenegro war bis zu Beginn der Wirtschaftskrise 2008 von einem beispiellosen ökonomischen Boom erfasst. Dies war zum einen der Privatisierung von Großindustrien, vor allem der Aluminiumfabrik nahe Podgorica geschuldet und zum anderen dem Aufkauf touristisch interessanter Küstengebiete, vor allem durch russische Oligarchen. Angesichts der wenig entwickelten Instrumentarien zur Finanzkontrolle in Montenegro wird vermutet, dass Geldwäsche in diesem Zusammenhang eine Rolle gespielt haben könnte (vgl. Stanislawski 2008: 55).\par
Eine Strategie zur Verwaltungsmodernisierung in Montenegro für die Jahre 2002-2009 wurde im Rahmen des mit CARDS-Mitteln finanzierten Programms „Public Administration Reform in Montenegro“ (PARIM) der European Agency for Reconstruction (EAR) vom montenegrinischen Justizministerium entwickelt. Das erklärte Ziel dieser Strategie war die stärkere Angleichung an europäische Standards und Ausdruck des Wunsches, die EU-Integration schneller voranzutreiben, was wiederum, so die Einschätzung eines Balkan-Experten, der wesentliche Treiber für die Reformbemühungen war (vgl. \cite{dzamuk}: 15).\par

Es können drei Phasen der geplanten Umsetzung der PARIM-Strategie ausgemacht werden:
\begin{enumerate}[label=Phase {\Roman*}:,align=left,  leftmargin=*] \itemsep1pt \parskip0pt \parsep0pt
\item (2002-2004) Vorbereitung
\item (2004-2007) Entwicklung
\item  (2007-2009) Vollendung
\end{enumerate}

Mit diesen drei Phasen sollten acht Punkte umgesetzt werden, um eine effektive öffentliche Verwaltung zu erreichen:
\begin{enumerate}[label= {\arabic*}),leftmargin=*]\itemsep1pt \parskip0pt \parsep0pt
\item Dezentralisierung des administrativen Systems durch Delegation der Kompetenzen auf die unteren Ebenen. Damit sollte die Flexibilität der öffentlichen Verwaltung erhöht werden und ihr mehr Handlungsspielraum zukommen.
\item Einführung einer Qualitätskontrolle bei der Vergabe von Aufträgen und Ausführung der administrativen Aufgaben.
\item Einführung kompetitiver Strukturen, die Bürgern und Unternehmen ermöglichen, ihre bevorzugten Anbieter oder Verwaltungsangebote zu wählen.
\item Kundenorientierung bei der Erfüllung öffentlicher Aufgaben.
\item Etablierung eines Personalmanagement Systems für die öffentlichen Bediensteten.
\item Modernisierung durch Informationstechnologie.
\item Entwicklung eines rechtlichen Systems, das die wichtigsten Aufgaben standardisierte und überregulierte Verwaltungsstrukturen deregulierte.
\item Verbesserung der Steuerungs- und Monitoringprozesse. (vgl. \cite{govmont03}: 19)
\end{enumerate}

Im Jahr 2003 sind etliche für die öffentliche Verwaltung relevante Gesetze verabschiedet worden: das Gesetz über staatliche Verwaltung, das die Organisation, Arbeitsweise und Funktion der staatlichen Administration beschreibt und das Gesetz zu Streitfällen, die die öffentliche Verwaltung betreffen. Das Gesetz über die staatliche Verwaltung etablierte das Prinzip, dass die Aufgaben der staatlichen Verwaltung von civil servants ausgeführt werden müssen. Aufgaben staatlicher Verwaltung werden von der Regierung, Ministerien und anderen administrativen Organen oder „agencies“ ausgeübt.

Mit dem Ziel einer modernen Verwaltung näher zu kommen wurde 2004 das Gesetz zu civil servants und Staatsbediensteten und 2005 ein entsprechendes Besoldungsgesetz verabschiedet. Eine „Human Resources Management Agency“ (HRMA) wurde gegründet, zuständig für Personalauswahl und Weiterbildung der civil servants. Weiterhin wurde 2003 ein Ombudsman’s Office eingeführt als Beschwerdeinstanz für Bürger, die mit der Verwaltung unzufrieden sind (vgl. \cite{freund07}: 2) Im Jahr 2004 wurde ein nationaler Rechnungshof eingerichtet, der laut Verfassung von 2007 eine unabhängige Institution ist, die dem Parlament berichtet (vgl. \cite{oecd08b}: 2).\par
Die Verfassung von 2007 legt in Artikel 16 die Einrichtung und die Kompetenzen der Behörden fest, Art. 20 definiert ein individuelles und kollektives Recht auf Anfechtung bei Verletzung von Rechten durch Behörden (\cite{oecd08b}: 2). Das Verwaltungsgericht erhält regelmäßig gute Noten von SIGMA: „Based on its excellent performance, the Administrative Court has become a role model throughout the public sector of Montenegro, and probably in the whole region. Its remarkable success is nevertheless endangered by the fact that its rulings against public authorities frequently cannot be effectively enforced” (\cite{oecd11a}: o.S.). Die Tradition der Verwaltungsgerichtsbarkeit in Montenegro ist positiv hervorzuheben, wobei es bei der Umsetzung der Rechtsprechung Probleme gibt. \par
Die Verantwortung für die Reform der öffentlichen Verwaltung liegt in Montenegro beim Ministerium für Inneres in der Abteilung für öffentliche Verwaltung, die unterbesetzt ist für den Umfang ihrer Aufgaben (vgl. \cite{oecd11a}: o.S.).\par
Eine neue Strategie zur Verwaltungsreform wurde im Jahr 2011 verabschiedet unter der Bezeichnung „Public Administration Reform Strategy 2011-2016“ (AURUM). Die Strategie zielt ab auf eine effiziente, professionelle, leicht erreichbare und service-orientierte öffentliche Verwaltung, die den Bürgern sowie sozialen und ökonomischen Akteuren dient. Als gesonderte Ziele wurden festgelegt:
\begin{itemize} \itemsep1pt \parskip0pt \parsep0pt
\item Stärkung des Rechtsstaates und der Verantwortlichkeit der öffentlichen Verwaltung;
Institutionelle Stabilität, Funktionalität und Flexibilität des Systems der öffentlichen Verwaltung; 
\item Bessere Bedingungen für Wirtschaftsakteure durch Verbesserung der öffentlichen Leistungserbringung und Verringerung der administrativen Belastungen; 
\item Transparentes und ethisches Handeln in der öffentlichen Verwaltung; 
\item Weitere Eingliederung Montenegros in den Europäischen Verwaltungsraum (vgl. \cite{govmont11}: o.S.).
\end{itemize}

Noch bestehende Probleme, die einer vollständigen Umsetzung der vorherigen Strategie (PARIM) zur Verwaltungsmodernisierung im Wege standen, werden ebenfalls benannt:
\begin{itemize} \itemsep1pt \parskip0pt \parsep0pt
\item Widerstände in der Verwaltung zu Beginn der Reformen;
\item Die globale Krise, die zu einer Destabilisierung der öffentlichen Finanzen und zu Haushaltsdefiziten führt;
\item Fehlen von angemessenen Mechanismen zur Verbesserung des finanziellen Status von civil servants;
\item Zu wenige junge, kreative Bewerber mit entsprechender professioneller Qualifikation;
\item Die Wahrnehmung eines hohen Grades an Bestechlichkeit in bestimmten Bereichen und Positionen;
\item Ungenügende Werbung für die Reformbemühungen und ihre Notwendigkeit;
\item Fehlen einer kompetenten Institution, die den Reformprozess mit Blick auf Professionalität und Methodologie begleitet und logistische Unterstützung zur Verfügung stellt (vgl. \cite{govmont11}: o.S.).
\end{itemize}
Die neue Strategie zur Verwaltungsreform wird von SIGMA im Kontext der bisherigen Aktivitäten Montenegros zur Verwaltungsreform kritisch beurteilt, vor allem in Hinblick auf den politischen Willen zur tatsächlichen Umsetzung: „The development of this strategy was largely driven by the perception that it was requested by donors and primarily by the EU integration process. […] The drafting of AURUM was thus heavily dependent on input from outside sources and had limited inter-ministerial co-ordination. This generates doubts on its ownership by the Government of Montenegro, concerns on the will and capacity to implement it and - finally - on its sustainability” ( \cite{oecd11a}: o.S.).
\par
Die EU beurteilt die neue AURUM-Strategie in ihrem Progress Report 2011 verhalten positiv: „The strategy includes introducing European standards covering recruitment and promotion and measures to increase the efficiency of the State administration. It also envisages an overall reduction of employment in the public sector; yet, it does not specify how this would be achieved without affecting the performance and efficiency of services. Some measures have already been taken to introduce economies of scale and integrate bodies whose activities have been disparate and uncoordinated (e.g. the various State inspection services)” (\cite{eurcom11c}: 8).\par
Die Einrichtung von neuen Institutionen in der staatlichen Administration, die sogenannte „agencification”, ist kein spezifisch montenegrinisches Phänomen. Doch für ein so kleines Land ist dieser Trend besonders problematisch, wie auch SIGMA in einem Assessment feststellt und vor weitreichenden Auswirkungen warnt: “Sometimes newly created bodies remain completely understaffed; they might exist on paper but in reality are nearly empty shells. Sometimes new mechanisms have been established in parallel to already existing institutions (departments in ministries, administrative bodies). Further ‘agencification’ is weakening the rule of law in Montenegro” ( \cite{oecd11a}: o.S.).

\subsubsection{Subnational-dezentrale Verwaltungsebene}

Die lokale Verwaltung in Montenegro basiert im Wesentlichen auf dem „Gesetz zur Lokalen Verwaltung” von 1991, das die Kommune (municipality) als Träger der lokalen Verwaltung festlegt. Auch die Verfassung von 2007 definiert die territoriale Organisation des Staates auf der Grundlage der Kommunen. In Montenegro gibt es 21 Kommunen, wobei Podgorica den Status der administrativen Hauptstadt hat und Cetinje den Status der historischen Hauptstadt. Alle Kommunen sind nach den gleichen Prinzipien organisiert, mit einem Rat und einem Bürgermeister, und haben die gleichen Kompetenzen hinsichtlich der Erbringung der lokalen Dienstleistungen. Die Kommunen werden vom Staat kontrolliert, mit der Möglichkeit den Rat und den Bürgermeister abzusetzen, wenn sie ihre Aufgaben nicht erfüllen (vgl. \cite{oecd08b}: 8). Die Kommunen haben jeweils einen für vier Jahre direkt gewählten Rat. Der Bürgermeister wird vom Gemeinderat aus seiner Mitte gewählt.\par
Die Kommune hat gesetzlich definierte Funktionen, und solche, die von der Republik zugewiesen werden können. Spezifische Aufgabenbereiche der Kommunen sind:
\begin{itemize} \itemsep1pt \parskip0pt \parsep0pt
\item Erstellung von Bebauungsplänen;
\item Bereitstellung öffentlicher Dienstleistungen;
\item Management der lokalen Wasserversorgung;
\item Zuteilung von Bauland und Erteilung von Nutzungsrechten für Büroraum;
\item Baugenehmigungen und technische Inspektionen;
\item Bau und Unterhalt von lokalen Straßen und öffentlichen Gebäuden;
\item Budgeterstellung;
\item Organisation öffentlicher PNV und Parkplätze\footnote{http://ec.europa.eu/enlargement/archives/ear/montenegro/montenegro.htm.}. 
\end{itemize}
Der wirtschaftliche Niedergang Montenegros in den 1990er Jahren als Teil des verbliebenen Rumpf-Jugoslawien betraf vor allem die Infrastruktur mit ausbleibenden Investitionen für Straßenbau, Energie-, Wasser- und Abwasserversorgung sowie Telekommunikation. Im Jahr 2000 betrug das kommunale Budget für das Jahr ca.  \euro{}  1,3 Millionen pro Kommune, wobei die Hauptstadt Podgorica  \euro{}  7,0 Millionen zur Verfügung hatte und die am wenigsten entwickelte Kommune  \euro{} 200.000.\footnote{http://ec.europa.eu/enlargement/archives/ear/montenegro/montenegro.htm.}\par
Während in den Republiken der SFR Jugoslawien eine Tradition der Selbstverwaltung bestand, führte die starke Kontrolle durch die Kommunistische Partei dazu, dass die Kommunen administrativ nicht in der Lage waren, eigenständig Aufgaben zu planen und auszuführen. Die schnelle Dezentralisierung seit der Demokratisierung, oft durch internationalen Einfluss in den neu entstandenen Ländern durchgesetzt, wurde nicht durch entsprechende fiskale Dezentralisierung begleitet, und die Kontroll- und Prüfmechanismen auf lokaler Ebene waren nicht ausgebildet. „Over-hasty and ambitious decentralisation has increased the risk of corruption and misuse of public monies across the region, without really increasing local autonomy and accountability” (\cite{oecd04}: 10).\par
Institutionell wird die Umsetzung der Reform der lokalen Verwaltung von einer Abteilung im Innenministerium begleitet und überwacht. Das Committee for Coordination of Local Self-Government Reform (CCLSGR) wurde 2007 etabliert und sorgt für Dialog, Kooperation und Koordination zwischen zentraler und lokaler Verwaltung. Darin vertreten sind das Finanzministerium, das Innenministerium, die Vereinigung der Kommunen, die seit 1972 besteht, sowie fünf Kommunen (rotierend). Drei Komitees arbeiten zu den Themen Internationale Kooperation, Dezentralisierung der Finanzen und Administrative Dezentralisierung. \par
Montenegro hat die Europäische Charta der Kommunalen Selbstverwaltung im Jahr 2008 unterzeichnet und die Regierung führte im Sommer 2009 eine 60-tägige öffentliche Debatte durch zum Gesetzesentwurf „on Territorial Organisation“. Das Gesetz trat 2011 in Kraft.\footnote{http://assembly.coe.int/ASP/Doc/XrefViewPDF.asp?FileID=18947 (Aufgerufen: 1.9.2012).} 


\subsubsection{Öffentlicher Dienst }
1999 wurden offizielle Zahlen für den föderalen civil service des damaligen Rumpf-Jugoslawien mit 12.000 Beschäftigten angegeben. Daneben waren 80.000 Beschäftigte im civil service Serbiens tätig. Zahlen für Montenegro wurden nicht veröffentlicht, beliefen sich aber geschätzt auf 20.000 civil servants in Montenegro (vgl. \cite{sevic}: 57).\par
Das Gesetz „on Civil Service and State Employees“ von 2004 definiert civil servants und state employees als Beschäftigte, die administrativ für die Umsetzung der Kompetenzen einer staatlichen Institution tätig werden. Somit sind beide im Prinzip von der Tätigkeit her gleichgestellt, mit der Ausnahme, dass state employees (die z.B. im staatlichen Pensionsfonds oder dem staatlichen Gesundheitssystem arbeiten) fiskalische und technische Funktionen haben, die für civil servants nicht erwähnt sind. Der civil service ist als System nach der Position (position based) etabliert und nicht am Leistungsgedanken (merit based) orientiert. \par
Auch die neue Verfassung von 2007 differenziert nicht zwischen Beamten und anderen öffentlichen Angestellten (vgl. \cite{oecd08c}: 1). Im Gegensatz zu der Praxis in den meisten EU-Ländern enthält die Verfassung keine Referenz zu Unparteilichkeit der staatlichen Bediensteten, der Einstellung aufgrund von Fähigkeiten sowie einem transparenten und kompetitiven Einstellungssystem. Andererseits gibt es in der Verfassung einige wichtige Vorschriften, die im Zusammenhang mit der weiteren Reform der öffentlichen Verwaltung bedeutsam sein könnten. So ist beispielsweise jegliche politische Organisation in der staatlichen Verwaltung verboten. Allerdings wurde im Gegensatz zu der vorherigen Verfassung von 1992 nicht aufgenommen, dass öffentliche Bedienstete ihre Aufgaben verantwortlich und ehrlich erfüllen müssen und für ihr Handeln zur Rechenschaft gezogen werden können (vgl. \cite{dzamuk}: 19). \par
Für die Umsetzung der Personalangelegenheiten des civil service und der state employees wurde die Human Resource Management Agency (HRMA) als staatliche Institution im Jahr 2004 gegründet. Die HRMA ist zuständig ist für die Veröffentlichung von Stellenanzeigen, die Selektion von Personal nach etablierten Regeln, für Stellenbeschreibungen, ein Register von civil servants und deren Weiterbildung (\cite{vukovic}: o.S.)\par	
Die Gesetzgebung zum civil service wurde von SIGMA und der EU kontinuierlich kritisiert. Exemplarisch sei dazu folgende Einschätzung wiedergegeben: „The current Law on Civil Servants and State Employees does not provide a clear legal definition of the scope of the civil service, which would help to establish rights and obligations as well as the accountability and liability of all civil servants at both national and municipal levels. Furthermore, it does not fully reflect the merit principle in recruitment and promotion. Finally, a new civil service law needs to include a special regulation for civil servants on conflict of interest, incompatibilities, gift policies, and whistle blowing. The current Law on Conflict of Interest does not provide a satisfactory solution, mainly because it encompasses both politicians and civil servants, which are two groups representing different realities and having different regulatory requirements” (\cite{oecd10a}: 3).\par
Ein neues meritokratisches Gesetz zu civil servants und state employees wurde Mitte 2011 verabschiedet und soll zusammen mit einem transparenteren Besoldungssystem 2013 in Kraft treten. Auch die HRMA wurde gestärkt und ihr wurde eine aktive Rolle in der Entwicklung von Personalplanungskonzepten zugewiesen. 2011 hatte die HRMA 36 Beschäftigte, soll aber zur effektiven Umsetzung der neuen Gesetzgebung zum civil service und state employees auf 52 Beschäftigte aufgestockt werden.\par
Zusammengefasst zeigt sich, dass in Montenegro die öffentliche Verwaltung in der Demokratie zu einem erheblichen Teil von der Praxis in Jugoslawien geprägt war und ist. In Montenegro wurde der Umbau der öffentlichen Verwaltung seit dem demokratischen Wechsel Anfang der 1900er Jahre ab 2002 von einer umfassenden Strategie (PARIM) begleitet, die unter Federführung der EU entwickelt wurde. Wesentliche Modernisierungsbemühungen haben stattgefunden, allerdings mit mäßigem Erfolg. Eine neue Strategie zur Verwaltungsmodernisierung im Zusammenhang mit der EU-Annäherung trat 2011 in Kraft. \par
Die bestehenden Probleme der Modernisierung der öffentlichen Verwaltung in Montenegro sind zu einem erheblichen Teil mit der Entwicklung in Jugoslawien zu erklären. In der Demokratie war das bisherige jugoslawische Konzept der public sector employees weiterhin bestimmend. Historisch gesehen (vor der SFR Jugoslawien) bestand jedoch in Montenegro eine klare Trennung von Beamten und anderen öffentlich Angestellten und eine Wiederannäherung an diese kontinentaleuropäische Tradition scheint daher möglich. \par
Das starke Weiterwirken der jugoslawischen Traditionen zeigte sich auch im von Verhältnis Zentralregierung und lokaler Ebene. Während der lokalen Ebene in der jugoslawischen Zeit einerseits große Bedeutung zukam (Stichwort: Selbstverwaltungssozialismus), war diese Ebene der Verwaltung andererseits durch den starken Einfluss der Kommunistischen Partei, die stark zentralistisch organisiert war, nicht eigenständig handlungsfähig. Dieses Erbe wurde in der demokratischen Zeit deutlich, als die lokale Ebene im Zuge der Dezentralisierung und im Rahmen der Annäherung an europäische Standards vermehrt Kompetenzen zugesprochen bekam. Die verabschiedeten Konzepte der Dezentralisierung fruchteten in der Praxis allerdings kaum, aufgrund von fehlender Erfahrung der Zusammenarbeit sowie Aufgabenteilung zwischen lokaler und zentraler Verwaltung. Weiterhin hat die fiskalische Dezentralisierung nicht in gleichen Maß stattgefunden wie die Übertragung von Aufgaben.
\par
Ein spezifisches Phänomen in Montenegro, das sich so nicht in den anderen beiden Untersuchungsländen wiederfindet, ist die Bedeutung von unregulierten Finanzflüssen auf Wirtschaft und Gesellschaft. In den 1990er Jahren während der Jugoslawienkriege entwickelte sich eine Schattenökonomie und in den 2000er Jahren bis zur Finanzkrise 2008 investierten vor allem russische Oligarchen in der montenegrinischen Küstenregion mit entsprechendem, zum Teil unregulierten Geldzufluss.
\subsection{Mazedonien} 
Mazedonien erklärte nach einem Referendum 1991 die Unabhängigkeit von Jugoslawien und eine neue Verfassung legte im gleichen Jahr die Grundlage für den Staat. Mazedonien war in Bezug auf Infrastruktur und Wirtschaftskraft der am wenigsten entwickelte Staat, der aus der SFR Jugoslawien hervorging. Obwohl Mazedonien nicht direkt an den Balkankriegen (1991 bis 1995) beteiligt war, hatten diese doch einen enormen ökonomischen Effekt auf das Land. Eine Wirtschaftskrise in Mazedonien von 1991-1993 resultierte zum großen Teil aus den Folgen der Sanktionen der Vereinten Nationen gegenüber Serbien, das Mazedoniens Haupthandelspartner war. Ein ökonomischer Stabilisierungsplan des IMF und der Weltbank stellten die makroökonomische Stabilität Mitte der 1990er Jahre wieder her (vgl. \cite{bech09}: lxix). 
\begin{figure}[H]
\setlength\belowcaptionskip{10pt}
 \caption{Landkarte Mazedonien}
  \centering
  \includegraphics[width=4in]{Material/Macedonia-CIA_WFB_Map}\\
 Quelle: http://commons.wikimedia.org/wiki/\\
File:Macedonia-CIA\_WFB\_Map.png (Aufgerufen: 10.10.2013).
\end{figure}

Aus der Karte ist die geografische Lage Mazedoniens ersichtlich als ein Land ohne Meereszugang. Die Hauptstadt ist Skopje. Mazedonien hat fünf direkte Nachbarstaaten: Serbien, Bulgarien, Griechenland, Albanien und Kosovo. Das Land ist 25,713 km$^2$ groß und hat etwas über 2 Millionen Einwohner (vgl. \cite{ramet}: 785). 

\subsubsection{Basismerkmale des Regierungssystems}

Mazedonien ist eine parlamentarische Demokratie. Staatsoberhaupt ist der Staatspräsident, der vom Volk für fünf Jahre gewählt wird, allerdings wesentlich weniger Machtbefugnisse hat als der Premierminister. Der Präsident beruft den Premierminister aus der Partei oder Koalition, die die Mehrheit im Parlament stellt. Das Parlament hat 123 Abgeordnete, die mittels Verhältniswahl gewählt werden (vgl. \cite{osceodihr11}: 2). Unter den 18 Ministerien ist eines für die Umsetzung des Rahmenabkommens von Ohrid zuständig (vgl. \cite{markic}: 3).\par
Mehrere 100.000 Mazedonier leben im Ausland. Neben Mazedoniern, die ca. 65\% der Bevölkerung stellen, leben ca. 25\% Albaner in Mazedonien, was immer wieder zu Konflikten zwischen der Mehrheits- und der Minderheitsgruppe führt.\footnote{Andere Minderheiten umfassen Türken (3,85 \%), Roma (2,66 \%), Serben (1,78 \%), Bosniaken (0,84 \%), Aromunen/Meglenorumänen (0,48 \%) und andere (1,04 \%).} Gewalttätige Ausschreitungen im Jahr 2001 durch albanische Guerilla-Gruppen im Norden Mazedoniens lösten eine Krise aus, die nur mit internationaler Vermittlung entschärft werden konnte. Es kam zur Unterzeichnung des Ohrid-Rahmenabkommens im August 2001 in der mazedonischen Stadt am Ohrid-See.\footnote{Das Ohrid-Abkommen hat zum Inhalt: 1. Friedenssicherung (Artikel 1 und 2), 2. Dezentralisierung und Symbolgebrauch; 3. Minderheitenregelungen; sowie 4. Bildung und Sprachgebrauch.} Das Abkommen führte zu größeren politischen Beteiligungsrechten der Albaner mit einer doppelten Mehrheit für Parlamentsentscheidungen zu bestimmten Themen, die von besonderem Interesse für Minderheiten sind. In diesen Fällen ist eine Mehrheit aller Abgeordneten und eine Mehrheit der nicht-mazedonischen Abgeordneten notwendig (Badinter-Prinzip)\footnote{Benannt nach dem früheren französischen Justizminister Robert Badinter, der an den Ohrid-Verhandlungen beteiligt war.} (vgl. \cite{szpala}: 60).\par
In der folgenden Tabelle sind die wichtigsten Wirtschaftsdaten für Mazedonien von 2007 bis 2010 überblicksartig dargestellt:
\begin{table}[H]
\caption{Wirtschaftsdaten Mazedonien 2007-2010}
\footnotesize
\begin{tabular}{|R{46mm}|R{25mm}|R{15mm}|R{15mm}|R{15mm}|R{15mm}|}\hline
&&2007&2008&2009&2010\\\hline
Bruttoinlandsprodukt (BIP)&
Millionen Dollar&8159,8&9834&9313,6&9189,5\\\hline
BIP Wachstum&\%&6,1&5&-0,9&1,8\\\hline
Inflation &\%&3,6&7,2&-0,3&2,1\\\hline
Ausländische Direktinvestitionen &\% des BIP&8,6&6&2,1&3,2\\\hline
Verschuldung öffentliche Hand&\% des BIP&24&20,6&23,9&24,8\\\hline
\multicolumn{6}{c}{}\\
\multicolumn{6}{c}{\normalsize Quelle: \cite{bert12b}: 18 (eigene Übersetzung aus dem Englischen).}
\end{tabular}\\
\end{table}

In der Tabelle ist erkennbar, dass Mazedonien trotz Finanzkrise, die das Land in der Wirtschaftsentwicklung verlangsamt hat, dennoch die Staatsverschuldung stabil halten konnte.
\subsubsection{Staatsaufbau und nationales Verwaltungsprofil}
Folgende Funktionen sind in der Verfassung festgelegt: der Staatliche Justizrat mit sieben vom Parlament ernannten Mitgliedern, der für die Verwaltung der Justiz zuständig ist und dem Parlament die Richter vorschlägt; das Büro der Ombudsperson, die vom Parlament für acht Jahre (erneuerbar) ernannt wird; der Rechnungshof, der dem Parlament berichtet; die Agentur für Civil Servants (CSA) mit einem Direktor, der/die für sechs Jahre ernannt wird. \par

Der bisherige Reformprozess der öffentlichen Verwaltung in Mazedonien kann in zwei Phasen eingeteilt werden: 
\begin{enumerate}[label= {\alph*}), leftmargin=*] \itemsep1pt \parskip0pt \parsep0pt
\item Bis zum Jahr 1998 wurden bestimmte Maßnahmen eingeführt, die das System im Rahmen eines systematischen und kohärenten Ansatzes verbessern und effizienter machen sollten. Zu diesen Entwicklungen gehörten u.a. die Einführung eines Rechnungshofes (State Audit), die Einrichtung des Büros der Ombudsperson, Regeln für öffentliche Anschaffungen und ein Budgetgesetz.
\item Nach der Entwicklung der ersten Strategie zur Reform der Öffentlichen Verwaltung im Jahr 1999. 
\end{enumerate}

Die Veränderung der öffentlichen Verwaltung von einem Ausführungs- und Kontrollorgan des Staates hin zu eher regelnder Funktion wurde vorangetrieben durch die Erfordernisse der Marktwirtschaft und die Einleitung einer Annäherung an die EU. Eine „Strategie der Reform der Öffentlichen Verwaltung“ wurde im Mai 1999 angenommen und definierte Prinzipien für die öffentliche Verwaltung, Reformziele und Sektorprioritäten. Als Prinzipien für die öffentliche Verwaltung wurden festgelegt: Rechtsstaatlichkeit, Transparenz, Stabilität, Verantwortlichkeit, Verlässlichkeit, Gleichbehandlung, Effizienz und Ethik.
Als wesentliche Bereiche für die Reform wurden definiert:
\begin{itemize} \itemsep1pt \parskip0pt \parsep0pt
\item „State administration system;
\item Public administration system;
\item Local self-government system;
\item Redefining the role of the State;
\item Exercise and protection of citizen’s rights;
\item Public finance restructuring; and
\item IT system development” (\cite{analyt07}: 7).
\end{itemize}
Ende der 1990er Jahre dominierten zunächst andere Themen Politik und Verwaltung in Mazedonien. Die Kosovo-Krise 1999 brachte Flüchtlingsströme, die versorgt werden mussten, über die Grenze. Im Jahr 2001 brachen dann gewalttätige ethnische Konflikte im Norden Mazedoniens aus, zu deren Beilegung internationale Vermittlung notwendig wurde. \par
Das Ohrid-Rahmenabkommen von 2001 hatte umfassende Auswirkungen auf die weitere Reform des Staates und der öffentlichen Verwaltung. Darin enthalten waren Lösungen für den Zugang zu öffentlichen Ämtern, den Sprachgebrauch, den Status der albanischen Minderheit und die Dezentralisierung.\par
Zur Überwachung der Umsetzung der Reform der öffentlichen Verwaltung wurde eine Kommission für Verwaltungsreform mit mehreren Arbeitsgruppen im Justizministerium eingesetzt. Diese Kommission wurde im Jahr 2002 durch eine Abteilung für Verwaltungsreform im Sekretariat der Regierung ersetzt (vgl. \cite{analyt07}: 7). Eine erneute Änderung der Zuständigkeiten führte 2011 dazu, dass die Koordination für die Verwaltungsreform nun im Ministerium für Verwaltung und Information angesiedelt ist.\par
Projekte zur Verwaltungsreform und insbesondere zur Umsetzung des Ohrid-Abkommens wurden von der EU sowie von einer Vielzahl anderer Geber und Akteure vorangetrieben. Ebenso ist eine erneute strategische Standortbestimmung zur Reform der öffentlichen Verwaltung für die Jahre 2010-2015 mit finanzieller Unterstützung der EU verabschiedet worden. Wesentliche Ziele der neuen Strategie sind Verbesserungen in den Bereichen öffentliche Finanzen, Human Resources Management, e-government und Korruptionsvermeidung (vgl. \cite{dimeski}: 7). \par
Die Qualität der neuen Strategie für Verwaltungsreform wird von einheimischen Experten gewürdigt. Jedoch wird auf die Probleme mit der Umsetzung der bisherigen Strategie verwiesen und mit dem Bekenntnis der verschiedenen Regierungen zu Verwaltungsmodernisierung streng ins Gericht gegangen. Es wird konstatiert, dass es an Umsetzungswillen und effektivem Management des Prozesses der Verwaltungsmodernisierung mangelte (vgl. \cite{dimeski}: 11).

\subsubsection{Subnational-dezentrale Verwaltungsebene}
Mazedonien hat eine bipolare Verwaltungsstruktur mit Kommunen und Zentralregierung, ohne intermediäre Ebene. Während die Dezentralisierung schon in der Strategie für die Verwaltungsreform von 1999 als Ziel formuliert war, führte das Ohrid-Rahmenabkommen zu einem Schub der Dezentralisierungsbemühungen mit der Einbeziehung von Minderheiten auf der lokalen Ebene. Das Ministerium für Lokale Selbstverwaltung war zuständig für die Koordination der Dezentralisierung, inklusive der finanziellen Aspekte. Auch der Zusammenschluss der Kommunen „Association of the Units of Local Self-Government“ (ZELS) ist ein wichtiger Akteur und Ansprechpartner für die Kommunen im Prozess der Dezentralisierung (vgl. \cite{analyt07}: 10).\par
Wie in Artikel 3 des Ohrid-Rahmenabkommens vorgesehen, verabschiedete das Parlament ein neues Gesetz zur lokalen Selbstverwaltung (2002). In Artikel 22 dieses Gesetzes sind die Bereiche aufgeführt, die an die Ebene der Gemeinden abgegeben werden sollen: (i) Stadtplanung und ländliche Entwicklung; (ii) Umweltschutz; (iii) lokale Wirtschaftsentwicklung; (iv) kommunale Dienstleistungen (unter anderem: Wasserversorgung, Abwasserbehandlung, Müllbeseitigung und -behandlung, Zurverfügungstellung von Wärmeenergie und Straßenbau; (v) Kultur; (vi) Sport- und Freizeiteinrichtungen; (vii) Sozialwesen; (viii) Bildung; (ix) Gesundheit; (x) Zivilschutz; (xii) Feuerwehr; (xii) Überwachung von kommunalen Aufgaben und (xiii) andere gesetzlich definierte Aktivitäten.\par
In Ergänzung zu dem Gesetz entwickelte die Regierung einen Ausführungsplan für die Dezentralisierung für 2003-2004. Für die Umsetzung war das Ministerium für Lokale Verwaltung zuständig. Im Jahr 2004 trat dann das neue Gesetz zur Territorialverwaltung in Kraft, das Mazedonien in acht Regionen mit 84 Gemeinden einteilte. Die bisherigen 123 Gemeinden wurden teilweise zusammengefasst, nur im Verwaltungsraum der Hauptstadt Skopje wurden die bisher acht Gemeinden auf zehn erhöht. Die Umsetzung der Dezentralisierung begann offiziell im Juli 2005 mit der vorgesehenen territorialen Reorganisation. \par
Wie in anderen vormals zentralistisch organisierten Ländern, verlief der Prozess der Dezentralisierung in Mazedonien nicht reibungslos. Der Transfer von Kompetenzen von der zentralen Verwaltung auf die Kommunen begann für verschiedene Bereiche im Jahr 2005. In einem ersten Schritt wurde die Steuererhebung von Einkommenssteuer, Mehrwertsteuer und Grundsteuer an die Kommunen abgegeben. In einem zweiten Schritt, der auf die Zeit ab 2008 verschoben wurde, sollten Zuweisungen z.B. für Bildung an die lokale Verwaltungsebene fließen. Problematisch für die lokale Ebene sind auch die geringen eigenen Steuereinnahmen (vgl. \cite{repofmac12}: 55). Die Europäische Kommission kommentierte 2006: „In many cases, municipalities are far from ready to take on their newly devolved responsibilities and central ministries remain unsure of their role in the process. Therefore, operational and financial support to line ministries and municipalities will remain crucial. Weaker municipalities will need greater direction and support with an equalisation procedure to correct any imbalance. Bringing the process of decentralisation to a good end is without doubt one of the main challenges the country faces” (\cite{ear}: o.S.). Hingewiesen wurde von der EU auch auf die nicht entwickelten Systeme der Finanzkontrolle, die nötig wären, um die Veruntreuung öffentlicher Gelder auf kommunaler Ebene zu verhindern. Unter CARDS 2004 waren einige Überprüfungen der Finanzen auf lokaler Ebene als Pilot-Projekte geplant, doch fehlende politische Unterstützung durch die Gemeinden, die keine lokalen Prüfer ernannten, und Widerstände der Mitarbeiter der lokalen Verwaltung verhinderten eine erfolgreiche Umsetzung unter „local ownership“ (vgl. \cite{eurcom06a}: 3).

\subsubsection{Öffentlicher Dienst}
Schon 1990 im Systemübergang wurde das Gesetz für die Verwaltungsorgane erlassen, das den Status der Beschäftigten in der staatlichen Verwaltung definierte. In der sozialistischen Geschichte ist der besondere Status der öffentlich Bediensteten zugunsten einer Gleichstellung mit anderen Arbeitnehmern befördert worden, was unter anderem zu einer Entprofessionalisierung führte. Dem versuchte im Jahr 2000 das Gesetz über civil servants entgegenzuwirken, mit dem Ziel die Bedingungen für einen professionellen, politisch neutralen, kompetenten, verantwortlichen und stabilen civil service herzustellen, der effizient im Sinne der Bürger und Wirtschaft arbeitet. Das Gesetz beschreibt den Begriff der civil servants. Diese sind bei den staatlichen Institutionen der Exekutive, Legislative oder Judikative angestellt, arbeiten als Experten und entscheiden über administrative Vorgänge, basierend auf der Verfassung und den Gesetzen. (vgl. \cite{markic}: 12). Im Gegensatz zu den meisten Ländern in West- und Osteuropa handelt es sich nicht um ein Karrieremodell, sondern ist „position based“. Das Gesetz wurde mehrmals geändert, unter anderem, um die Bestimmungen des Ohrid-Abkommens zur gleichberechtigten Teilhabe der Minderheiten umzusetzen. \par
Ebenfalls im Jahr 2000 wurde eine  „Strategy for Civil Servants Training in the Process of Macedonia's Approximation to the European Union“ angenommen. Auch die Civil Servants Agentur (CSA) wurde im Jahr 2000 als eigenständige Institution aufgebaut. Sie ist verantwortlich für die Personalentwicklung, das Vorbereiten von Sekundärgesetzgebung im Bereich civil servants, die Koordination von Schulungen, Systematisierung von Stellenbeschreibungen und Stellenanzeigen. Außerdem ist die CSA zweite Instanz in Beschwerdeverfahren von civil servants. Der Direktor der CSA muss dem Parlament jährlich einen Tätigkeitsbericht vorlegen. Weiterhin ist es Aufgabe der CSA, ein nationales Register für civil servants zu führen. Laut CSA gab es im Jahr 2011 auf nationaler Ebene 10.000 und auf lokaler Ebene 1.800 civil servants. Diese Zahl erhöht sich insgesamt auf ca. 32.000, wenn man Armee und Polizei mitrechnet (\cite{repofmac04}: 201). \par
Zurzeit arbeiten insgesamt etwa 110.000 Beschäftigte in der öffentlichen Verwaltung (ausgenommen Beschäftigte in Staatsbetrieben), davon sind ca. 42.500 in der Staatsverwaltung, ca. 33.500 in Bildung und Wissenschaft und ca. 33.000 im Gesundheitssystem tätig. Damit macht die Beschäftigung im öffentlichen Sektor rund 12\% der Erwerbsbevölkerung aus, eine Quote vergleichbar mit anderen Ländern Zentral- und Osteuropas. In Mazedonien werden ca. 6\%-7,5\% des Haushaltes für Gehälter der öffentlichen Verwaltung ausgegeben (vgl. \cite{analyt09}: 4).\par
Neben der Definition von civil servants gab es keine Definition oder Regelung für die ‘public employees’, das sind Personen, die im öffentlichen Sektor arbeiten, aber keine civil servants sind. Auf die public employees, die die Mehrheit der im öffentlichen Sektor Tätigen stellen, trifft lediglich das Arbeitsgesetz zu, was Tür und Tor öffnete für Missbrauch der Regelungen und für das Einsetzen politisch opportuner Kandidaten der jeweiligen Regierung (vgl. \cite{analyt07}: 16). Die EU beschrieb darüber hinaus immer wieder die problematische Praxis von temporären Verträgen, unter Umgehung der vorgeschriebenen Rekrutierungsverfahren. In den letzten Jahren wurden viele dieser temporären Positionen in permanente umgewandelt, was der Kritik weiter Vorschub leistete, da oft keine ausreichenden Begründungen und Qualifikationsnachweise zugrunde gelegt wurden (vgl. \cite{eurcom11b}: 10). Da die Anzahl der Beschäftigten überdies die Zahl der real existierenden Arbeitsplätze übersteigt, gelangten immer öfter Meldungen an die Öffentlichkeit über Verwaltungsangestellte, die formal eingestellt sind und regelmäßig Lohn beziehen, jedoch keine Arbeit leisten müssen (vgl. \cite{malahova}: 8).\par
Die oft kritisierten Lücken in der Gesetzgebung wurden nun teilweise geschlossen durch ein Gesetz zu „public employees“, seit April 2011 in Kraft, das sich der großen Gruppe der öffentlich Beschäftigten annimmt, die keine civil servants sind. SIGMA bleibt jedoch kritisch: “The law suffers from serious shortcomings in terms of both methodology and content. The structure is unsystematic and inconsistent. The regulatory content fails to make a clear distinction between civil servants and public employees and, more over contains numerous substantive (material) regulations that are inappropriate and will cause problems and confusion when the law has to be implemented. The provisions on recruitment procedures do not exclude arbitrariness; rules on promotion do not exist; mobility is badly regulated; and the six biannual appraisals will impose an unbearable administrative burden on managers” (\cite{oecd10c}: 5).\par
Nach Kritik an der halbherzigen Umsetzung der Vereinbarungen des Ohrid-Rahmenabkommens zum Minderheitenzugang zur öffentlichen Verwaltung wurde auch in dieser Hinsicht nachgebessert und 1.600 civil servants von Nicht-Mehrheitskommunen eingestellt. Die Europäische Kommission kommentierte dies als Erfüllung der Vorgaben in quantitativer Hinsicht, ohne Einbeziehung der Bedarfe in den entsprechenden Institutionen: „However, the trend of recruiting employees from these communities on a quantitative basis without regard to the real needs of the institution continued”, und ergänzt: „The recruitment procedure remains vulnerable to undue political influence“ (\cite{eurcom11b}: 10).\par
Zusammengefasst zeigt sich, dass Verwaltungsmodernisierung in Mazedonien in den Anfangsjahren der Demokratie zwar ab 1999 von einer Strategie geleitet war, es aber andere Themen gab, die politische Priorität hatten. Eine spezielle Gesetzgebung führte zur Einstellung von Minderheiten in der öffentlichen Verwaltung. Insgesamt zeigt sich, dass das Bekenntnis des Landes zur Reform der öffentlichen Verwaltung von nationalen und internationalen Beobachtern weitgehend als Lippenbekenntnis eingeschätzt wird.\par
Einer der mazedonischen Think Tanks stellt die Anstrengungen Mazedoniens zu der Modernisierung der öffentlichen Verwaltung in Frage und sieht die EU als wesentlichen Treiber der Verwaltungsmodernisierung. Gleichzeitig werden die etablierten Praxen, die vor allem den politischen Eliten dienen, beibehalten: „The Public Administration (PA) in Macedonia, even after its succession from Yugoslavia, has functioned according to the same old principles, which have served well the interests of the governing political parties. Efforts to reform the PA have come only due to the fact that Public Administration Reform (PAR) stands as an important precondition for Macedonia’s accession in the EU” (\cite{analyt11}:  o.S.).\par
Ob konkrete Schritte der erneuerten Strategie zur Verwaltungsreform von 2011 folgen werden, bleibt abzuwarten. Der Missbrauch von Stellenbesetzungen im öffentlichen Bereich wurde bislang als besonders problematisch erlebt. 

\subsection{Albanien}
Anders als Jugoslawien, das nach der sozialistischen Staatsgründung eine graduelle, vor allem wirtschaftliche Öffnung praktizierte, hatte das kommunistische und international vollständig isolierte System in Albanien ein wirtschaftlich vollkommen rückständiges Land hinterlassen. Albanien war nach dem Ende des kommunistischen Regimes von Hilfsleistungen und internationaler Unterstützung abhängig. Die wirtschaftlich prekäre Lage führte dazu, dass viele Albaner versuchten ihr Land zu verlassen. Betrügerische Systeme zur Geldanlage, in die mehr als die Hälfte der albanischen Bevölkerung investiert hatte, brachen 1997 zusammen und große Teile der Bevölkerung verloren alle Ersparnisse. In der Folge kam es zu bürgerkriegsähnlichen Ausschreitungen. Die Situation konnte nur mit internationaler Hilfe entschärft werden und stellte eine schwerwiegende Krise für die neue Demokratie dar (vgl. \cite{jarvis}: 11).
\begin{figure}[H]

  \caption{Landkarte Albanien}
  \centering
  \includegraphics[width=4in]{Material/AlbanianMapShqip}\\
  Quelle: http://de.wikipedia.org/wiki/Datei:AlbanianMapShqip.png (Aufgerufen: 10.10.2013).
\end{figure}
Aus dieser Karte ist ersichtlich, dass Albanien eine lange Küste am Mittelmeer hat. Diese ist aber touristisch noch wenig erschlossen. Albanien hat 28.748 km$^2$ und ca. 3,2 Millionen Einwohner. Die Hauptstadt ist Tirana. Direkte Nachbarstaaten von Albanien sind Griechenland, Mazedonien, Kosovo und Montenegro.

\subsubsection{Basismerkmale des Regierungssystems}
Seit den ersten freien Wahlen von 1991 ist Albanien eine parlamentarische Republik mit einer zentralen Regierung und lokalen Regierungsstrukturen. Gesetzgeber ist die Versammlung der Republik Albaniens, deren 140 Abgeordnete alle vier Jahre gewählt werden. Staatsoberhaupt ist der vom Parlament auf fünf Jahre gewählte Präsident. Die dem Parlament verantwortliche Regierung wird vom Ministerpräsidenten geführt. Dieser ernennt die Minister, die vom Präsidenten bestätigt werden müssen. Die derzeit gültige Verfassung wurde am 28. November 1998 durch eine Volksabstimmung angenommen (vgl. \cite{oeza06}: 8). Kennzeichnend für die politische Kultur in Albanien ist das Fehlen einer konstruktiven Zusammenarbeit zwischen den beiden größten politischen Parteien, was vielfach zu Blockadesituationen auch im Parlament führt. Im legislativen Prozess ist für bestimmte wichtige Gesetze eine 3/5-Mehrheit der Abgeordneten nötig, die nur durch Zusammenarbeit und Kompromisse der zwei fast gleich starken Parteien zustande kommen kann. Auch auf die lokale Ebene hat die Polarisierung zwischen den beiden größten Parteien Auswirkungen, wie immer wieder an Kontroversen zwischen oppositionsgeführten Kommunen und der Zentralregierung deutlich wird. Ein Projekt allerdings wurde und wird von allen politischen Kräften in Albanien vertreten: Das Ziel der Aufnahme Albaniens in die EU.\par
In der folgenden Tabelle sind die wichtigsten Wirtschaftsdaten für Albanien von 2007 bis 2010 überblicksweise dargestellt:
\begin{table}[H]

\caption{Wirtschaftsdaten Albanien 2007-2010}
\footnotesize
\begin{tabular}{|R{46mm}|R{25mm}|R{15mm}|R{15mm}|R{15mm}|R{15mm}|}\hline
&&2007&2008&2009&2010\\\hline
Bruttoinlandsprodukt (BIP)&
Millionen Dollar&10704,7&12968,7&12044,9&11786,1\\\hline
BIP Wachstum&\%&5,9&7,7&3,3&3,5\\\hline
Inflation &\%&2,9&3,4&2,3&3,5\\\hline
Ausländische Direktinvestitionen &\% des BIP&6,2&7,4&8,0&9,4\\\hline
Verschuldung öffentliche Hand&\% des BIP&53,8&55,2&60,2&59,7\\\hline

\multicolumn{6}{c}{}\\
\multicolumn{6}{c}{\normalsize Quelle: \cite{bert12c}: 16 (eigene Übersetzung aus dem Englischen).}
\end{tabular}
\end{table}

Aus dieser Tabelle ist erkennbar, dass die Staatsverschuldung Albaniens seit 2007 bei über 50\% liegt und sich in Folge der Finanzkrise noch etwas erhöht hat, sich aber auf diesem hohen Niveau zu konsolidieren scheint. Das Bruttoinlandsprodukt, das 2008 noch 7,7\% betrug, hat sich infolge der Finanzkrise halbiert, ist aber immer noch positiv.

\subsubsection{Staatsaufbau und nationales Verwaltungsprofil}
Der Aufbau einer funktionsfähigen Verwaltung war eines der Hauptziele nach dem Systemwechsel. In den 1990er Jahren lag das Hauptaugenmerk vor allem auf der Entwicklung einer funktionsfähigen Zentralregierung nach demokratischen Prinzipien. Auch die Neuausrichtung im makroökonomischen Bereich und im Bankwesen, sowie die Privatisierung waren vorrangige Ziele dieser Zeit. Albanien erhielt Unterstützung durch das EU-Programm PHARE. Prioritäten waren die Entwicklung eines Dienstrechts für den öffentlichen Dienst und ein Management-System für öffentliche Ausgaben. Während an gesetzlichen Grundlagen für eine moderne öffentliche Verwaltung gearbeitet wurde, blieb die Umsetzung und Durchsetzung der Standards gering. Auch gab es in Albanien nach der Demokratisierung des Landes keine unabhängige Kontrolle von Regierung und Verwaltung. Die Verfassung überträgt dem Ministerrat (council of ministers) umfangreiche Kompetenzen unter anderem dann, wenn eine Funktion nicht an ein anderes Organ oder die lokale Verwaltung zugewiesen wird. Dies wird oft als verfassungsmäßige Grundlage für die Ermächtigung der Verwaltung angewandt (vgl. \cite{oecd08a}: 2). Erst 1999 wurde ein Verwaltungsverfahrensgesetz eingeführt mit neuen Kontrollmechanismen für die öffentliche Verwaltung, das zu mehr Verantwortlichkeit führen sollte, ein Anspruch, der in der Praxis allerdings oft keinen Bestand hatte (vgl. \cite{oecd04}: 3).\par
Die Verfassung sieht Gewaltenteilung vor, lokale Selbstverantwortung sowie die Unabhängigkeit wichtiger Institutionen, z.B. einer speziellen Institution zur Ernennung von Richtern, den Ombudsmann und das Verfassungsgericht. Weiterhin ist das Recht auf Anfechtung und die Wiedergutmachung von durch die Verwaltung verursachte Schäden in der Verfassung vorgesehen (vgl. \cite{oecd10b}: 6). Die gerichtliche Überprüfung von Verwaltungsakten wurde bisher durch das ordentliche Gerichtswesen wahrgenommen, erst im Mai 2012 wurde ein Gesetz verabschiedet, das den Aufbau von Verwaltungsgerichten regelt (vgl. \cite{oscepres}: 1).\par
Ausgehend von dem Befund, dass die langsame Erteilung von Genehmigungen die Wirtschaftstätigkeit beeinträchtigt, wurden durch den „Regulatory Reform Action Plan 2007“ per Gesetz „one-stop-shops“ eingerichtet. Weiterhin entstand ein „National Center for Licensing“ unter Aufsicht des Finanzministeriums (vgl. \cite{oecd08a}: 5).\par
Das Department of Public Administration (DoPA) wurde 1994 gegründet und ist verantwortlich für die Formulierung und Umsetzung von Verwaltungsreformen und Personalmanagement des civil service. Organisatorisch war DoPA zunächst beim Premierminister angesiedelt und ist seit 2005 dem Innenministerium zugeordnet (vgl. \cite{selenica}: 185).\par
Die Abteilung hat folgende Aufgaben:
\begin{itemize} \itemsep1pt \parskip0pt \parsep0pt
\item manages the civil service in all the institutions of the central administration;
\item leads and implements the functional and structural reform in the institutions of public administration;
\item creates and implements the reform in the salary field;
\item coordinates the reform for the implementation of the information technologies and in the e-government field.\footnote{http://www.pad.gov.al/en/dap.html (Aufgerufen: 24.10.2012).}
\end{itemize}
Eine erste Strategie für staatliche Institutionen und öffentliche Verwaltung wurde von der Regierung 1997 angenommen und ein Gesetz zum civil service 1999 verabschiedet.\par
Eine neue Strategie zur Verwaltungsreform wurde 2009 unter dem Titel „Intersectoral Strategy of Public Administration Reform 2009-2013” von der Regierung beschlossen. Entgegen dem umfassenderen Titel beschäftigt sich die Strategie vornehmlich mit dem civil service. 

\subsubsection{Subnational-dezentrale Verwaltungsebene }
In Albanien begann der Prozess der Dezentralisierung 1998. Die Verfassung legt fest, dass die Beziehung zwischen dem Staat, den Regionen und den lokalen Verwaltungen auf Autonomie, Legalität und Kooperation gegründet ist. Ein Gesetz zur lokalen Verwaltung wurde 2001 erlassen. (vgl. \cite{refworld10}: 59). Die Regierung hat eine Dezentralisierungsstrategie aufgelegt, die im Wesentlichen mit der Europäischen Charta of Local Self Government übereinstimmt, die Albanien am 21. Oktober 1998 ratifizierte. Es gibt in Albanien 373 local government units (LGU), wovon 65 Städte (municipalities) sind und 308 Kommunen (communes). Städte und Gemeinden bilden 12 Regionen. Die Regionen haben vage definierte Kompetenzen vor allem in der Koordination der ökonomischen Entwicklung und der Förderung der öffentlichen Investitionen. In der Vergangenheit wurden ca. 4\% des kommunalen Budgets an die Regionen ausgewiesen (vgl. \cite{oecd08a}: 2). \par

Der lokalen Verwaltung sind vier Bereiche zugewiesen, in denen sie eigenständige Funktionen hat: 
\begin{enumerate}[label={\arabic*}.] \itemsep1pt \parskip0pt \parsep0pt
\item Infrastruktur und öffentliche Dienstleistungen
\item Soziale, kulturelle und für den Tourismus relevante Aktivitäten
\item Lokale ökonomische Entwicklung
\item Zivilschutz
\end{enumerate}
Weiterhin können Städte und Gemeinden seit 2002 mit der Zentralregierung gemeinsame Aufgaben wahrnehmen (Shared Functions): 

\begin{enumerate}[label={\arabic*}.] \itemsep1pt \parskip0pt \parsep0pt
\item Vorschul- und voruniversitäre Bildung
\item Gesundheitsversorgung 
\item Soziale Unterstützung
\item Öffentliche Ordnung und Zivilschutz
\item Umweltschutz
\item Andere gesetzlich definierte Funktionen
\end{enumerate}
Bislang haben solche Kooperationen vor allem für Infrastrukturmaßnahmen bei Schulen und Krankenhäusern, stattgefunden, wobei die finanziellen Mittel von der Zentralregierung zur Verfügung gestellt wurden. Weiterhin legt das Gesetz fest, dass die lokale Ebene delegierte Funktionen wahrnehmen kann. Das sind Aufgaben, die die zentrale Ebene zu erbringen hat, die aber von der Regierung oder den zentralen Institutionen per Gesetz oder vertraglicher Vereinbarung delegiert werden können. Dabei werden Art und Weise der Leistungserbringung beschrieben und die nötigen Finanzmittel von der zentralen Ebene bereitgestellt. Die Kommunen können eigene Mittel einbringen, um einen besseren Grad der Leistungserbringung zu erreichen (vgl. \cite{did}: 12).\par
Der Prozess des Aufgabentransfers von der Zentralregierung auf die lokale Ebene kommt nur stockend voran, was zum Teil an fehlender finanzieller Ausstattung der Gemeinden liegt, beispielsweise im Bereich der Infrastruktur für die Wasserversorgung. Im Jahr 2008 wurde ein Gesetz verabschiedet, das den Gemeinden ermöglicht, Geld für öffentliche Zwecke aufzunehmen. Andererseits wurde ein Gesetz verabschiedet, das die Steuern für Kleingewerbe reduzierte, was sich auf die Einnahmen vor allem der lokalen Ebene auswirkte (vgl. \cite{refworld10}: 60). Die Überprüfung der Arbeit der Kommunen auf ihre Legalität hin ist Aufgabe des Präfekten, der von der Zentralregierung in den 12 Regionen eingesetzt ist. Die Association of Municipalities, der Zusammenschluss der Städte, hat bereits mehrfach Gerichtsverfahren gegen die Zentralverwaltung angestrengt, wobei es vor allem um Fragen der Stadtplanung ging. Meist wurde der Zentralregierung vorgeworfen, in die rechtliche Zuständigkeit der kommunalen Ebene einzugreifen (vgl. \cite{oecd08a}: 2). Während die Dezentralisierung ein kleines Land wie Albanien ohnehin vor Probleme stellt, ist die fehlende historische Erfahrung von lokaler Verwaltung und Erbringung von Aufgaben in dem vormals streng zentralistisch organisierten Staat ein besonderes Problem. In der gegenwärtigen politischen Situation kommt die starke politische Polarisierung hinzu, die oft dazu führt, dass die Kooperation zwischen Zentralregierung und lokaler Ebene nur mit großen Reibungsverlusten funktioniert, wenn es sich bei der jeweiligen Regierungsmehrheit nicht um die gleiche Partei handelt.\par
In ihrer Antwort auf einen vorläufigen Bericht des Europarates zu Problemen, die im Zusammenhang mit der Dezentralisierung in Albanien identifiziert wurden, stellt die Regierung von Albanien die Verbesserung der finanziellen Situation der Kommunen, die aktiv betrieben worden sei, heraus: „We would like to point out that intensive work has been carried out to strengthen the first level local government via the increase of the financial autonomy of local government. The legal frame on local taxes has been improved, as have the Guidelines on the administration of local taxes and tariffs. The State Budget has tripled the grants for Municipalities and Communes, and each year the formula for the details of the unconditioned transfers has been improved. Investment grants have more than quadrupled“ (\cite{govalb09}: 4). Die Verwendung von Transfers (grants) der Zentralregierung an die lokale Verwaltung ist einerseits notwendig angesichts der geringen finanziellen Autonomie der kommunalen Ebene. Mindestleistungen der Kommunen können auf diese Weise sichergestellt werden. Andererseits können Transfers in einem Land ohne Tradition der Kommunalverwaltung die Kontrolle, bzw. politische Einflussnahme durch die höhere Ebene zumindest begünstigen. \par 
Die Betrachtung der Dezentralisierung in Albanien zeigt, dass durch die EU-Perspektive Regionalisierungsprozesse verstärkt wurden. Dies geschieht vor allem im Zusammenhang mit der Entscheidung der EU, die in der EU praktizierte Strukturpolitik auf die Beitrittskandidaten zu übertragen und damit zum wichtigsten Unterstützungsinstrument für die wirtschaftliche Angleichung zu machen. In der EU-Strukturpolitik, die sowohl den sozialen als auch den wirtschaftlichen Zusammenhalt in der Union verbessern soll, und etwa ein Drittel der EU-Haushaltsausgaben ausmacht, spielen Regionen eine wichtige Rolle. Die Programme der Förderung rückständiger Regionen in der EU („Ziel-1-Gebiete“: weniger pro-Kopf Einkommen als 75\% des EU-Durchschnittes) werden zusammen mit regionalen und nichtstaatlichen Gebietskörperschaften geplant und implementiert. Zu statistischen und Planungszwecken hat die EU die regionalen Gebietseinheiten der Mitgliedstaaten in der „Nomenclature des Unités territoriales de statistique“ (NUTS) klassifiziert (vgl. \cite{brusis09}: 203).\par
Für Albanien bedeutet dies, dass Regionen im Sinn der EU als zusätzliche Ebene zum komplizierten Austarieren der unterschiedlichen Verwaltungsebenen hinzukommen.
\subsubsection{Öffentlicher Dienst} 

Die Verfassung von 1998 legt fest, dass öffentlich Bedienstete ihre Tätigkeit auf der Basis von Gesetzen ausführen und den Bürgern dienen. Das Gesetz zum civil service, das im Jahr 2000 in Kraft trat, ist das wesentliche rechtliche Instrument für civil servants. Das allgemeine Arbeitsrecht gilt für öffentliche Angestellte, die keine civil servants sind und es gilt auch für civil servants für die Fälle, in denen kein Sonderrecht besteht. Das Gesetz definiert den civil service als „positions exercising public authority“ oder „directly involved in policy making at central and local self-government level“. Insgesamt sind nur etwa 6\% (5.000) aller öffentlichen Beschäftigten von dem Gesetz zum civil service erfasst, auf zentraler Ebene ca. 2.600, davon ca. 1.500 in den Ministerien. (vgl. \cite{skarica}: 374).\par
Als public employees gelten etwa 90.000 Beschäftigte. Davon sind ca. 15.000 in der lokalen und regionalen Verwaltung tätig. Beschäftigte im Gesundheits- und Bildungssektor zählen ebenfalls zu den public employees, gelten aber nicht als civil servants (vgl. European Commission 2010: 15). Das Gesetz zu civil servants gilt nicht für eigenständige Agenturen oder die Administration der Präfekten. Auf lokaler Ebene sind die ca. 9.600 Angestellten der 308 Kommunen (communes) ebenfalls nicht von dem Gesetz erfasst, hingegen unterliegen die civil servants der 65 Städte (municipalities) und civil servants der 12 Regionen (regions) dem Gesetz (vgl. \cite{oecd09}: 5).\par
Das System der Bezüge ist zentralisiert und wird vom Ministerrat beschlossen. Seit 2002 gibt es ein neues Entgeltsystem mit Basisanteil, Zulagen für Beschäftigungsdauer, Qualifikation und Arbeitsbedingungen sowie Zulage für die erreichte Position. Das Zulagensystem macht den größten Teil der Bezüge aus, ein System, das von SIGMA als relativ transparent eingeschätzt wird (vgl.  \cite{oecd09}: 15). Der rechtliche Rahmen für den civil service stimmt in weiten Teilen mit europäischen Standards überein, beinhaltet aber keine leistungsabhängigen Elemente. Es gibt die rechtliche Möglichkeit der Einsparung und Re-Strukturierung, die in der Vergangenheit oft zur Entlassung öffentlicher Bediensteter mit unklarer Begründung geführt hat. Weiterhin gibt es die Praxis der Umgehung der gesetzlichen Bestimmungen durch zeitlich befristete Anstellung im Rahmen des allgemeinen Arbeitsgesetzes. Dies geschieht meist durch die Möglichkeit, dringende Einstellungen ohne Einstellungstests auf der Basis des Arbeitsrechtes durchzuführen. Oft werden diese temporär Eingestellten nach sechs Monaten zu civil servants, ohne weitere Überprüfung, oder es finden nachträgliche formale Einstellungstests statt, bei denen die derzeitigen Stelleninhaber im Vorteil sind, Praxen, die politische Ernennungen begünstigen und das öffentliche Vertrauen in die Verwaltung belasten. (vgl. \cite{oecd11a}: 5). Besonders nach dem Regierungswechsel in Folge der Parlamentswahl 2005 war diese Praxis zu beobachten und hat sich seitdem verfestigt (vgl.  \cite{oecd09}: 3). Der durchschnittliche Anteil der Stellen, die durch befristete Verträge nach dem Arbeitsrecht besetzt werden, liegt bei 20\% (vgl. \cite{eurcom09b}: 9).\par
Eine Anstellung im öffentlichen Dienst scheint für junge Menschen attraktiv zu sein, mit dem Ziel, später besser bezahlte Stellen in der Privatwirtschaft anzunehmen. In den Jahren 2004 und 2005 hatte die Regierung den Versuch gemacht, auch junge Hochschulabsolventen, die ihre Ausbildung im Ausland erhalten hatten, zu einer Bewerbung zu bewegen. (vgl. \cite{oecd09}: 18).\par
Das Gesetz von 2003 zu „Rules of Ethics in Public Administration“ vermischt ethische und juristische Regeln und wird zudem oft ignoriert. Das Gesetz zu „Conflict of Interest“ von 2005 differenziert nicht zwischen Politikern und civil servants, was zu rechtlicher Unklarheit führt, mit unterschiedlicher Anwendung in der Praxis. Auch sind die Rollen der Institutionen, die das Gesetz umsetzen sollen, nicht klar gegeneinander abgegrenzt (vgl. \cite{oecd09}: 3). \par
Die Kapazität des DoPA zur Einhaltung von civil service Verordnungen in den Ministerien wurde geschwächt durch einen häufigen Wechsel der Direktorenposition und den Verlust der Gesamtaufsicht über den civil service. Während das DoPA für das Management des Personalwesens für den civil service zuständig ist, ist als Kontrollmechanismus eine 2002 eingerichtete Civil Service Commission (CSC) vorgesehen, die dem Parlament berichtet. Die CSC hat die Aufgabe, als vorgerichtliche Instanz auf der Grundlage von Beschwerden von civil servants die Legalität der Managemententscheidungen zu überprüfen. Durch Probleme mit der Besetzung eines Mitgliedes der Civil Service Commission über einen längeren Zeitraum, war die Institution allerdings in ihrer Funktion eingeschränkt (vgl. \cite{oecd10b}: 2).\par
Eine umfassende Funktionalreform in hat in den Jahren 2005/2006 in den Ministerien stattgefunden als Folge der Parlamentswahl 2005, die das Machtverhältnis zwischen den beiden großen Parteien umkehrte. In den Jahren 2007/2008 wurden Stellen abgebaut. Dabei ging es auch um Kosteneinsparungen und Effektivitätssteigerung, aber ebenso um politische Entscheidungen. Die freigesetzten Personen wurden auf eine Warteliste für offene Stellen gesetzt und erhielten ihre Bezüge für ein Jahr weiter. Offen werdende Stellen wurden dennoch nicht aus dieser Warteliste nachbesetzt, sondern neue Ausschreibungen fanden statt (vgl.  \cite{oecd09}: 23).\par
Das DoPA hat im Oktober 2010 ein „Concept Paper on a New Civil Service Law” entworfen, das die bisherige Kritik aufnimmt und im Rahmen der PAR-Strategie die Europäischen Prinzipien und Standards besser berücksichtigt. Ein neues Gesetz liegt im Entwurf seit 2011 vor (vgl. \cite{oecd11b}: 7); es ist angesichts der Blockadesituation im Parlament aber erst im Mai 2013 verabschiedet worden.\par
Zusammen mit der Verabschiedung des Gesetzes zum civil service wurde im Jahr 2000 ein Trainingsinstitut, das „Training Institute for Public Administration“ (TIPA), gegründet. TIPA wird von einem beratenden Gremium gelenkt, das sich aus Generalsekretären der Ministerien und Vertretern der Universitäten zusammensetzt, unter direkter Aufsicht des DoPA. Das Angebot umfasst generelle Lehr-Einheiten zur öffentlichen Verwaltung und Verwaltungsrecht sowie speziell zugeschnittene Kurse zu speziellen technischen Fragen, wie EU-Integration und öffentliche Finanzen. Eine Trainingsstrategie wurde verabschiedet, die die EU Integration als wichtiges Element hat. Entsprechende Trainingseinheiten wurden in 2005 begonnen. TIPA führt Kurse durch zu ethischen Standards und der Vermeidung von Conflicts of Interest, zunächst für die zentrale Ebene und seit 2007 auch für die lokalen Strukturen und für unabhängige Institutionen, z.B. das Office for Registration of Immovable Properties, die Agency for Restitution of Properties oder das Ministry of Education and Science (vgl.  \cite{oecd09}: 13).\par
Zusammengefasst zeigt sich, dass die Verwaltungsentwicklung in Albanien seit dem Ende der Diktatur vor besonders großen Aufgaben stand. Dabei lag das Augenmerk in den 1990er Jahren vor allem in der Demokratisierung der zentralen staatlichen Strukturen. Gleichzeitig wurde eine Dezentralisierungsstrategie verfolgt, im Wesentlichen von US-amerikanischen Hilfsorganisationen unterstützt. Diese parallel verlaufenden Entwicklungen waren nicht von ausreichenden finanziellen Zuweisungen an die lokale Ebene verbunden und die Entwicklung ist bis heute nicht konsolidiert. Erschwerend kommt in Albanien der Machtkampf zweier fast gleich starker Parteien hinzu, die in einem erbitterten Streit um die politische Vormachtstellung im Land kämpfen. Viele Reformprojekte werden behindert durch die daraus resultierende Blockadesituation, so zum Beispiel die überfällige Modernisierung der Gesetzgebung zum civil service. Im Zusammenhang mit dem erklärten Ziel des Landes, dem EU-Beitritt, gehen Lippenbekenntnisse zu Reformen fast immer einher mit nicht oder schwer durchsetzbaren Neuerungen.\par
In der Zusammenschau mit der historischen Verwaltungsentwicklung ist Stagnation ein wesentliches und auch aus der Zeit vor der Demokratie bekanntes Element. In dieser Situation arrangiert sich der Einzelne so gut es geht, um sein persönliches Umfeld zu gestalten. Das Gemeinwesen und gesellschaftliche Verantwortung entwickeln sich nicht wesentlich weiter. In diesem Sinne hat sich die Situation nicht substanziell von der Zeit unter der Diktatur oder der davor liegenden osmanischen Herrschaft verändert. Die Sozialisationszusammenhänge der Familie oder des Clans haben Bedeutung, während die Gesellschaft als Bezugspunkt wenig Stellenwert hat. Das Konzept verantwortlicher öffentlicher Verwaltung ist so nicht als allgemeiner Anspruch präsent und damit schwer durchsetzbar.
\section{Fortschrittsberichte der EU}
Die Fortschrittsberichte der EU werden jedes Jahr im Herbst veröffentlicht und die Fortschritte des Landes auf dem Weg der Annäherung an die EU werden beschrieben. Pro Jahr und pro Land werden diese Berichte über die (potenziellen) Beitrittskandidaten von der EU-Kommission verfasst und auf der Website der EU veröffentlich. Sie sind eine der Grundlagen für den weiteren „Beitrittsfahrplan“ der Europäischen Kommission für die einzelnen Länder. Diese Fortschrittsberichte sind aber auch für alle anderen Akteure Grundlage für ihre (Finanz-)Hilfe für die Länder.\par
Im Wesentlichen sind die Berichte analog zu den einzelnen Kapiteln des Acquis communautaire, den es zu übernehmen gilt, strukturiert. Verwaltungsmodernisierung, die zunehmend als wichtiges Thema dieser Fortschrittsberichte anerkannt ist, kommt regelmäßig unter ‚politische Kriterien’ am Anfang der Berichte zur Sprache. Unter diesem Gliederungspunkt enthalten die Berichte jeweils eine detaillierte Beschreibung des Fortschritts im Bereich Verwaltungsentwicklung. Weiterhin findet sich am Ende dieser Darstellung der Fortschritte in diesem Bereich eine Zusammenfassung zur Verwaltungsentwicklung. Dabei fällt auf, dass vor allem diese Zusammenfassung zur Verwaltungsentwicklung in der weiteren Verwertung der Fortschrittsberichte, auch durch andere Publikationen und Institutionen ihren Niederschlag findet. Zur Verdeutlichung werden in der folgenden Tabelle die Zusammenfassungen zur Verwaltungsentwicklung in den Untersuchungsländern für die Jahre 2006 bis 2012 wiedergegeben.

\begin{footnotesize}
\begin{longtable}[H]{|p{1cm}|L{49mm}|L{44mm}|L{43mm}|}
\caption[Fortschrittsberichte der EU zur Verwaltungsentwicklung]{Fortschrittsberichte der EU zur Verwaltungsentwicklung. Überblick 2006-2012}\\\hline
\label{tab:fortschrittsberichte}
Jahr&Montenegro&Mazedonien&Albanien\\\hline
\endfirsthead
\caption[]{(Fortsetzung)}\\\hline
Jahr&Montenegro&Mazedonien&Albanien\\\hline
\endhead 
\hline
\endfoot
\multicolumn{4}{c}{}\\
\multicolumn{4}{C{145mm}}{\normalsize Quelle: European Commission: Progress Report Montenegro 2006, 2007, 2008, 2009, 2010, 2011, 2012; Progress Report Macedonia 2006, 2007, 2008, 2009, 2010, 2011, 2012; Progress Report Albania 2006, 2007, 2008, 2009, 2010, 2011, 2012, jeweils unter Political Criteria, Public Administration, Overall Assessment. Graue Unterlegungen durch die Autorin dieser Arbeit.}
\endlastfoot
2006&Overall, efforts have been made on the side of the Government to upgrade the administrative capacity of Montenegro. But much remains to be done, notably in the areas of transparency and accountability, financial control, public procurement and budget management as well as management of public assets and licensing procedures. Appropriate resources need to be allocated to match the ambitions of Montenegro in this area. For the successful implementation of the SAA, Montenegro needs to upgrade its administrative capacity in the areas covered by the agreement. Particular attention should be paid to enhancing administrative capacity and law enforcement in the area of justice and home affairs, in particular concerning the fight against corruption and organised crime, as well as the protection of personal data (S.9)&
Overall, reforms in the organisation of the public administration are taking place progressively and aim to improve management and increase transparency. However, implementing reforms in the administration and the reform of the police remain serious challenges (S.9).& 
The capacity of the Department of Public Administration to set common management strategies across the public administration remains limited. Career structures, career planning, salaries and performance management in the civil service and other public services remain poor. \hl{Political appointment of higher civil servants remains prevalent, restricting the growth of a professional senior civil service level (S.7).}\\\hline
2007&
Overall, the process of strengthening the administrative and management capacity of the local authorities has been slow. The public administration remains weak and inefficient.\hl{ Further efforts will be needed to ensure the impartiality of public administration and strengthen its capacity.} Future work on decentralisation is expected to continue to strengthen local democracy, upgrade the administrative capacity of the municipalities and clarify sectoral responsibilities in a manner which permits oversight and transparency. The capacity of the municipalities for financial management, including public procurement, needs to be further improved. The process of putting together municipal budgets, including consolidation of revenue and an objective system for establishing and allocating grants, needs to be overhauled. Central government capacity for dealing with local government reform needs to be substantially strengthened (S.10).&
Overall, reforms are gradually being implemented in the area of public administration.\hl{ However, there have been limited results, in particular due to lack of a strong commitment to meet the announced objective of a more transparent, professional and depoliticised public administration and better organised public services.} Public administration remains weak and inefficient. Implementation of the police reform is underway but still at an early stage (S.10).&
Overall, the public administration is stabilising and becoming somewhat more focused.\hl{ Further progress on strengthening the Department of Public Administration and ensuring competent, motivated and impartial staff is now needed (S.8).}\\\hline
2008&
Overall, progress has been made in strengthening the legislative framework for the public administration. Some progress has been made in human resources management and local government reform. However, lack of human and financial resources combined with structural weaknesses and corruption continue to hamper the overall effectiveness of the public administration and, as a whole, administrative capacity remains limited (S.10).&
Overall, some progress has been made in reforming public administration, which is a key priority of the Accession Partnership. \hl{However, greater priority needs to be given to establishing a public administration which is transparent, professional and free of political interference.} In this area the country is at an early stage. Progress was made in implementing the law on police, which is a key priority of the Accession Partnership. Nonetheless, the politicisation of senior police officers is a serious concern. In this area the country partially meets its priorities (S.12).&
\hl{Overall, the public administration is continuing to stabilise, but the lack of transparency and accountability in appointments is endangering its independence. What is now needed is to further strengthen public sector governance by enhancing the impartiality of public administration, a key European Partnership priority. Further progress is needed to establish an independent, merit-based, professional civil service (S.8).}\\\hline
2009&
Overall, some progress has been made on strengthening the legislative framework for the public administration. There was some progress on human resources management but, due to the government’s cost-cutting measures, there has been no overall strengthening of humanvresources. \hl{Significant efforts are required to establish a professional, accountable, transparent and merit-based civil service, free of political interference.} Further reform is required in the fields of financial control, public procurement and licensing procedures. To this effect, internal control mechanisms need to be established throughout the public administration. Efforts need to continue to prepare fully for implementation of the SAA by upgrading administrative capacity in the areas covered by the agreement (S.10).&
Overall, some progress was made on implementing public administration reform, including reform of the civil service, which is a key priority of the Accession Partnership. The amendments to the Law on the civil service strengthened the provisions aiming to ensure merit based recruitment and promotion of civil servants. A functioning training system has been established and some additional staff has been allocated in key areas. \hl{However, further efforts to ensure transparency, professionalism and independence of public administration are required. Respect for the provisions and the spirit of the law needs to be ensured in practice.} Further progress has been made as regards reform of the police, which is a key priority of the Accession Partnership. All the new local and regional commanders are operational, management has improved and the law on internal affairs has introduced a career system into the police service. The reform of the police is well advanced (S.13).&
Overall, the legal framework for public administration reform is in place but the lack of transparency and accountability in appointments remains a key European Partnership priority to be addressed. \hl{Further progress is needed to establish an independent, merit-based and professional civil service, free of political interference.} Full enforcement of the civil service Law and implementation of the Strategy for public administration reform will be key to progress in this regard (S.9).\\\hline
2010&
\hl{Overall, the public administration remains weak and highly politicised.} The general administrative framework, including the Law on general administrative procedure and the Law on civil servants and state employees needs to be reviewed and adapted to European standards and principles. Administrative procedures are cumbersome and time-consuming and must be simplified. \hl{Transparency needs to be improved by facilitating access to public information including on economic governance and allocation of public assets. Significant efforts are still necessary by Montenegro to establish a sound and accountable public administration free of politicisation.} The quality of legislation and of decisions and acts produced by the public administration needs to be considerably improved. This is inextricably linked to improving the quality, capacity and expertise of public servants, with the aid of merit-based recruitment and promotion and continuous training. Further considerable efforts to strengthen administrative capacity to deal with future EU accession obligations are needed (S.16).&
Overall, there was some progress as regards reform of public administration, notably through the adoption of the Law on public servants. \hl{However, significant further efforts are needed to ensure the transparency, professionalism and independence of public administration.} Respect of the legal framework needs to be ensured in practice, in particular as regards staff recruitment. The process of converting a large number of temporary posts into permanent ones in many cases did not provide for competitive and merit-based recruitments. Police reform has made further progress. The new Law on internal affairs entered into force and most necessary implementing legislation has been adopted (S.11).&
Overall, the general administrative law framework and the civil service system are mostly in line with European principles and standards, although some gaps exist. Proper implementation of the legal framework remains a concern as does the lack of transparency and accountability in appointments and the politicisation of the public administration. \hl{Political will and strong efforts are necessary for the full implementation of the civil service law and progress with the public administration reform strategy, which are necessary for the establishment of a civil service that is independent, professional and based on merit.} The pending election of a new Ombudsman as well as the insufficient respect of this institution's recommendations is of concern (S.17).\\\hline
2011&
Overall, Montenegro has taken important steps to address the main challenges posed by the public administration reform. The Government adopted and started to implement a public administration reform strategy. \hl{An improved legal framework in the area of civil service and state administration aiming at efficiency, de-politicisation and merit-based recruitment has been adopted.} Legislation regulating administrative procedures has been amended and a further comprehensive reform has been launched. The HRMA has been strengthened. \hl{Preparations for implementation of the adopted legislation have to be stepped up and focus on enforcing de-politicisation, professionalism and effectiveness and impartiality of the administration, including through merit-based recruitment and promotion.} The capacity of the Ombudsman and of the State Audit Institution needs to be further enhanced. Implementation of the Public Administration Reform Strategy needs to take due account of the need to rationalise administrative structures and strengthen administrative capacity, notably in areas related to European integration, while ensuring the financial sustainability of public administration (S.9).&
Overall, progress was made in the area of public administration reform in terms of policy coordination and legislative developments. A Ministry responsible for public administration reform was created and the Law on General Administrative Procedure was amended. An egovernment interoperability system was launched among several institutions. Progress in implementing the reforms was limited. \hl{Significant additional efforts are needed in order to guarantee transparency, professionalism and independence of the public administration in practice.} Further improvements of the current legal framework are necessary, in particular as regards the Law on general administrative procedures (S.11).&
Overall, despite some reform measures such as the Council of Ministers decision on structure and organisation of public bodies of June 2011, essential steps in public administration reform, which is a key priority of the Opinion, including amendments to the civil service law, have not been completed. Adoption of relevant legislation is pending and contingent on overcoming the persistent political stalemate. Implementation of the existing laws and administrative acts remains weak. In the institutional context, DOPA continues to lack sufficient authority to take up its role fully. \hl{Establishing an independent, merit-based and professional civil service free from political interference has yet to be achieved.} Appointment of the Ombudsman is still pending (S.10).\\\hline
2012&
Overall, Montenegro has taken further steps to address the challenges of public administration reform. The legislative framework and the implementation of the recent legislation need to be improved, in a financially sustainable manner and with adequate verification mechanisms. The capacity of the Ombudsman has been reinforced but needs to be further enhanced (S.9). &
Overall, there was some progress as regards public administration. Services to citizens were improved and e-government has been gradually introduced. Steps on fundamental reforms of the administrative framework and public and civil service have been launched. \hl{Additional efforts are needed to guarantee the transparency, professionalism and independence of the public administration. In particular, respect for the principle of merit-based recruitment together with the principle of equitable representation needs to be ensured (S.10).}&
Overall , there has been progress in public administration reform (a key priority of the opinion) mainly through the adoption of the Laws on Administrative Courts and on the Organisation and Functioning of Public Administration as well as through the appointment of the Ombudsman. It is now essential to adopt the amendments to the civil service Law. Further efforts are needed to implement the adopted legislation and administrative acts. \hl{The legislative and institutional framework for public administration is still marked by deficiencies that need to be addressed with a view to strengthening professionalism, depoliticisation, meritocracy, transparency and accountability (S.10)}.\\\hline
\end{longtable}
\end{footnotesize}
\par
Aus diesem Überblick geht hervor, dass in diesen Zusammenfassungen zu den politischen Kriterien, unter dem Gliederungspunkt Verwaltungsentwicklung der civil service eines der evaluierten Elemente ist. In fast allen Zusammenfassungen der Jahre 2006-2012 wird für den civil service der Untersuchungsländer die Notwendigkeit zu verbesserter Professionalisierung und politischer Neutralität herausgestellt (entsprechende Stellen in Tabelle \ref{tab:fortschrittsberichte} grau unterlegt). Diese Problematik ist sicherlich eine der wesentlichen in den Ländern des Westlichen Balkans, dennoch überrascht die starke Konzentration auf diesen einen Punkt. Die Berichte werden international und national im politischen Zusammenhang von einem breiten Publikum rezipiert. Es besteht die Gefahr, dass dabei nur diese Zusammenfassungen (conclusions) ins Bewusstsein der Rezipienten gelangen. Es sind vor allem die zusammenfassenden Beurteilungen, die in anderen Berichten und Analysen weiterverarbeitet werden.

\section{Zusammenfassende Übersicht zum Stand der aktuellen Entwicklung }

Betrachtet man die Einzelergebnisse für die drei Untersuchungsstaaten im Zusammenhang, so zeigt sich, dass in der Erfüllung öffentlicher Aufgaben für Montenegro und Mazedonien die Praxis des sozialistischen Jugoslawien Fortbestand hat. Eine klare Unterscheidung in der Erfüllung hoheitlicher Aufgaben durch civil servants im Gegensatz zu anderen Staatsbediensteten findet in der Praxis nicht statt. Die Tradition, in der beide Gruppen keiner deutlichen Trennung unterlagen, setzt sich fort, wenngleich die aktuelle Gesetzgebung z. T. eine Unterscheidung in civil servants und andere Staatsbedienstete vorsieht. In Mazedonien und Albanien ist weiterhin zu beobachten, dass mit temporären Stellenbesetzungen, die später in reguläre Stellen umgewandelt werden, die höheren Anforderungen an civil servants umgangen werden. Diese Praxen werden auch genutzt, um politischen Einfluss der jeweiligen Regierungsmehrheit in der öffentlichen Verwaltung geltend zu machen. In der Zusammenschau mit dem historischen Teil der vorliegenden Arbeit wird deutlich, dass politische Einflussnahmen auf die Stellenbesetzungen in der öffentlichen Verwaltung in Zusammenhang gesehen werden können mit der politischen Kontrolle der Partei bis in die Kommunen, wie für Jugoslawien gezeigt werden konnte. Im kommunistischen Albanien war die öffentliche Aufgabenerfüllung stark zentralistisch organisiert und ebenfalls von Loyalität der Partei gegenüber geprägt. \par
Die Europäische Union spricht in ihren regelmäßigen Fortschrittsberichten die aktuell wahrgenommenen Missstände an. Dabei wird vor allem auf die Politisierung der öffentlichen Verwaltung hingewiesen, die einem modernen Staatsverständnis, angelehnt an Werte der EU, im Wege steht. Dieser Befund zieht sich unverändert durch die Fortschrittsberichte.\par
In der Zusammenschau mit den historischen Befunden ist auch interessant, dass in Montenegro mit historisch vergleichsweise starkem österreichisch-ungarischem Einfluss die Verwaltungsgerichtsbarkeit vergleichsweise modern ausgestaltet ist. Dagegen ist das Konzept der Verwaltungsüberprüfung in Albanien, das historisch vor allem osmanischen und kommunistischen Einflüssen ausgesetzt war, auch heute noch sehr gering entwickelt.\par
Aus Gründen der Übersichtlichkeit werden die wesentlichen Ergebnisse stichwortartig in der folgenden Tabelle zusammengefasst:
\begin{footnotesize}
\begin{longtable}[H]{|M{30mm}|C{35mm}|C{35mm}|C{35mm}|}

\caption[Schematische Darstellung zur Verwaltungsentwicklung der Untersuchungsländer ]{Schematische Darstellung zur Verwaltungsentwicklung der Untersuchungsländer }\\\hline
&\textbf{Montenegro}&\textbf{Mazedonien}&\textbf{Albanien}\\\hline
\endfirsthead
\caption[]{(Fortsetzung)}\\\hline
&\textbf{Montenegro}&\textbf{Mazedonien}&\textbf{Albanien}\\\hline
\endhead 
\hline
\endfoot
\multicolumn{4}{c}{}\\
\multicolumn{4}{c}{\normalsize Quelle: Eigene Zusammenstellung.}
\endlastfoot

Größe Quadratkilometer&13.812&25.713&28.748\\\hline
 Einwohner&ca. 600.000&ca. 2 Mio.&ca. 3,2 Mio.\\\hline
BIP 2010 in Millionen \euro{} &4.111,1&9.189,5&11.786,1\\\hline
Basismerkmale des Regierungssystems&
Parlament mit 81 Abgeordneten. 
Parlamentswahlen alle vier Jahre.
Verhältniswahl mit einer landesweiten, geschlossenen Liste.&
Parlament mit 123 Abgeordneten.
Parlamentswahlen alle vier Jahre
Verhältniswahl in 6 Wahlbezirken.&
Parlament mit 140 Abgeordneten.
Parlamentswahlen alle vier Jahre.
Verhältniswahl in 12 Wahlbezirken mit geschlossenen Listen.\\\hline
Staatsaufbau und nationales Verwaltungsprofil&
PAR seit 2003 wesentliche Gesetze verabschiedet.\newline
Neue PAR-Strategie 2011.
Verwaltungsgerichtsbarkeit.
PAR-Koordination bei Innenministerium.
Starke Förderung der Verwaltungsmoderniisierung durch EUProgramme.&
PAR-Strategie 1999 und erneuert 2010.
PAR-Koordination zunächst im Justizministerium, ab 2011 Ministerium für Verwaltung und Information.&
PAR-Strategie von 1997.\newline
Verwaltungsgerichts-barkeit erst 2012 eingeführt.
PAR-Koordination durch Department of Public Administration 
unter Premierminister,
seit 2005 unter Innenministerium.\\\hline
Subnational-dezentrale Verwaltungsebene&
Zwei Ebenen der öffentlichen Verwaltung.\newline
21 Gemeinden.
Dezentralisierung Teil des EU Konzeptes zu PAR.
Dezentralisierung ohne entsprechende fiskalische Ausstattung.&
Zwei Ebenen der öffentlichen Verwaltung. \newline
 84 Gemeinden.
Dezentralisierung ohne entsprechende fiskalische Ausstattung. &
Drei Ebenen der öffentlichen Verwaltung.\newline
12 Regionen und 373 Gemeinden.
Dezentralisierung ohne entsprechende fiskalische Ausstattung. \\\hline
Öffentlicher Dienst&
Keine Trennung zwischen civil servants und state employees.\newline
Human Resources Management Agency 
(ab 2004).\newline
Neues Gesetz zu meritokratischem civil service soll 2013 in Kraft treten.&
 Keine klare Aufgabentrennung von civil servants und state employees in der Praxis.\newline
Civil Servants Agency (ab 2000).
Temporäre Verträge in der öffentlichen Verwaltung unter Umgehung rechtlicher Bestimmungen.&
Trennung in civil servants und public employees. \newline
Praxis der temporären Stellenbesetzungen und damit Umgehung der Gesetzgebung zum civil service.\newline
Neues Gesetz zum civil service 2013 verabschiedet, bislang keine Implementierung.\\\hline
Verwaltungseinflüsse in der Vergangenheit&
Österreich-Ungarn.
K.u.k. Militärverwaltung.
Sozialistische Verwaltung.&
Osmanisches Reich.
Bulgarien, Griechenland. 
K.u.k. Militärverwaltung.
Sozialistische Verwaltung.&
Osmanisches Reich. 
Kommunistische Verwaltung. \\\hline
EU-Beitritt beantragt&
2008&
2004&
2009\\\hline
Einschätzung durch EU&Kandidatenstatus (2010)&
Kandidatenstatus (2005)&
Kandidatenstatus (2014)\\\hline
Hauptprobleme&
Sehr kleines Land.
Starke Auswirkungen der Finanzkrise.
Politische Einflussnahme auf Stellenbesetzungen in der öffentlichen Verwaltung.&
Kleines Land.
Namensstreit mit Griechenland.\newline
Minderheitenproble-matik mit Auswirkungen auf öffentliche Verwaltung.
Politische Einflussnahme auf Stellenbesetzungen in der öffentlichen Verwaltung.&
Kleines Land.
Politische Polarisierung und Gefahr parlamentarischer Blockade.\newline
Politische Einflussnahme auf Stellenbesetzungen in der öffentlichen Verwaltung.
Schwache Implementierung von Gesetzen.\\\hline
\end{longtable}
\end{footnotesize}
\vspace{-0,5cm}
Aus dieser Tabelle wird erkennbar, dass es sich bei den drei Untersuchungsländern um kleine bis sehr kleine Staaten handelt. Es wird deutlich, dass die drei Untersuchungsländer zum Teil unterschiedlichen und zum Teil ähnlichen historischen Einflüssen ausgesetzt waren, die sich auch in der Verwaltungsentwicklung spiegelten. In der Zeit seit der Demokratisierung, also seit Anfang der 1990er Jahre haben externe Akteure, vor allem die EU und die USA auch auf die Modernisierung der Verwaltung Einfluss genommen, bislang mit mäßigen Erfolg, wie aus den regelmäßigen EU-Fortschrittsberichten erkennbar. Während gewisse Fortschritte benannt werden, ergibt sich ein insgesamt stagnierendes Bild in Bezug auf die Modernisierung und und vor allem Professionalisierung der Verwaltung. Es stellt sich also die Frage, wie die öffentliche Verwaltung in den Untersuchungsländern für einen EU-Beitritt besser vorbereitet werden kann.\par
Mit Bezug auf das weitere Verfahren des eventuellen EU-Beitritts dieser Staaten verbleiben Fragen dazu, wie die EU ihre Unterstützung der Verwaltungsentwicklung verbessern kann. Dabei interessieren sowohl die Einschätzung der bisherigen Arbeit der EU zur Verwaltungsentwicklung in den Untersuchungsländern, aber auch Anregungen für die zukünftige Unterstützung der öffentlichen Verwaltung. Diese Fragen sollen in Interviews mit entsprechend tätigen Experten weiter geklärt werden. Aufbauend auf den bisherigen Erkenntnissen der Arbeit wurden Interviewfragen entwickelt.

\chapter{Abschließende Betrachtung}\label{chap:abschlBetrachtung}
Im nun folgenden Abschlusskapitel werden die Ergebnisse der einzelnen Kapitel der Arbeit zum Einfluss der EU-Perspektive auf die Verwaltungsmodernisierung in den Ländern des Westlichen Balkans zusammengeführt im Hinblick auf die Untersuchungsfragen. Im Anschluss werden noch offene Forschungsfragen benannt und mögliche Strategien formuliert. Hier noch einmal die eingangs formulierten Fragen der Untersuchung:
\begin{itemize} \itemsep1pt \parskip0pt \parsep0pt
\item Sind Erfahrungen hinsichtlich der Entwicklung der öffentlichen Verwaltung in den Ländern der letzten Aufnahmewelle übertragbar auf den Westlichen Balkan?
\item Liefert die historische Betrachtung der Verwaltungsentwicklung unter Einschluss früherer Regime der kommunistischen oder sozialistischen Zeit, aber auch der zeitlich davor gelagerten Einflüsse der Imperien verwertbare Erkenntnisse für den aktuellen Modernisierungsprozess?
\item Wie fördert die EU die Verwaltungsmodernisierung in den Beitrittsländern?
\item Wie schätzen Experten der EU und Akteure im Westbalkan die Verwaltungsentwicklung im Kontext der EU-Erweiterung ein?
\item Welche Optionen bestehen für die Verwaltungsentwicklung in den Westbalkanstaaten?
\end{itemize}
Das Thema Verwaltungsmodernisierung im Kontext der EU-Erweiterung fand bislang in der wissenschaftlichen Erforschung überraschenderweise wenig Beachtung. Für die vorliegende Untersuchung wurden neben Sichtung der Literatur auch Berichte der EU und anderer Institutionen ausgewertet. Auf der Basis der Ergebnisse dieser Sichtung wurde eine Expertenbefragung durchgeführt.\par
Um verallgemeinerbare Aussagen zu erhalten wurden drei Ländern des Westlichen Balkan für die Untersuchung herangezogen. Die noch labilen staatlichen Gebilde des Balkan (Kosovo und Bosnien-Herzegowina) wurden von der Untersuchung ausgeschlossen. Ebenso wurde Kroatien nicht betrachtet, da es kurz vor der Aufnahme in die EU stand. Als Untersuchungsländer eigneten sich drei benachbarte kleinere Länder des Westlichen Balkan: Montenegro, Mazedonien und Albanien.\par
Hauptbezugspunkt der Untersuchung ist die Verwaltungsentwicklung in den Untersuchungsländern und der Einfluss der EU-Beitrittsperspektive auf die Verwaltungsmodernisierung. Vor diesem Hintergrund wurden die aktuellen Entwicklungen der Europäisierungsforschung dargestellt, die im Rahmen der Politikwissenschaft aus der Transitionsforschung entwickelt wurden. Für den hier interessierenden EU-Einfluss auf die Verwaltungsentwicklung in den Beitrittsländern wurde die Konditionalitätsforschung als für die vorliegende Untersuchung besonders relevanter Strang der Europäisierungsforschung vorgestellt.\par
Neben diesem theoretischen Gerüst wurden weitere Rahmen in die Untersuchung eingezogen: Die Erfahrungen aus den bisherigen Erweiterungsrunden in Bezug auf die Verwaltungsmodernisierung wurden mittels einer Literaturanalyse herausgearbeitet. Die Verwaltungsentwicklung der Untersuchungsländer seit ihrer Demokratisierung wurde dargestellt. Auf die zeitlich davor bestehenden Systeme und ihre Bedeutung für die Verwaltungsentwicklung wurde ebenfalls eingegangen. Weiterhin wurden die EU-Pro"-gramme vorgestellt in ihrer Bedeutung für die Verwaltungsentwicklung in den Beitrittskandidaten.\par
Zur Überprüfung der Befunde und Klärung noch offener Fragen, wurde eine Expertenbefragung mittels halbstandardisierter Interviews konzipiert und durchgeführt. Befragt wurden mit dem Thema Verwaltungsmodernisierung praktisch und thematisch vertraute Experten aus EU-Institutionen und den drei Untersuchungsländern.\par
Bei der Literaturauswertung zur EU-Osterweiterung fiel auf, dass in den Fortschrittsberichten der EU die Verwaltungsmodernisierung als wichtiges Element begutachtet wurde, nach der vollzogenen Erweiterung aber kein Monitoring zur Verwaltungsentwicklung mehr stattfand.
Aktuelle Untersuchungen zum Status quo der Verwaltungsentwicklung nach der Ost-Erweiterung stellen für viele der Länder der letzten Erweiterungswelle eine Stagnation oder ein Zurückgehen hinter schon Erreichtes in der Verwaltungsentwicklung fest. Einige Länder der letzten Erweiterungswelle haben Reformen angestoßen und unter anderem Gesetze im Bereich Verwaltungsmodernisierung erlassen, jedoch oft nicht umgesetzt, und „some even backtracked once they joined EU“ (\cite{pickering}: 23).\par
Die EU erwartet von potenziellen Mitgliedern die Übernahme des Acquis communautaire. Dabei handelt es sich um das Regelwerk der EU, das Kandidatenländer in nationales Recht übernehmen müssen. Dieses ist in 35 Kapiteln angeordnet und bildet die Grundlage für die Beitrittsverhandlungen. Öffentliche Verwaltung ist keines der Kapitel des Acquis, sondern wird unter politischen Kriterien in den jährlichen Fortschrittsberichten der EU begutachtet. Während sich die Erweiterungspolitik der EU evolutionär entwickelt, hat sich in Bezug auf die Verwaltungsmodernisierung seit der Osterweiterung keine wesentliche Weiterentwicklung ergeben. Nach wie vor ist die Verwaltungsmodernisierung als politisches Kriterium verankert und entfaltet daher im Gegensatz zu den Themen der Kapitel des Acquis nur begrenzten Zugzwang für die Beitrittskandidaten. Die von der EU propagierte Konditionalität gegenüber den Beitrittskandidaten greift beim Thema Verwaltungsentwicklung daher kaum. Dieser Befund wurde schon in der Analyse der EU-Osterweiterung deutlich. Überraschenderweise wird dennoch keine wesentliche Weiterentwicklung der Herangehensweise der EU zum Thema öffentliche Verwaltung und EU-Erweiterung nach Südosteuropa erkennbar. Die Auswertung der Experteninterviews bestätigt diesen Befund.\par
Das eingangs dargestellte Konzept der Konditionalität, das einen wesentlichen Bezugspunkt der EU-Erweiterung darstellt, kommt in Bezug auf das Thema Verwaltungsmodernisierung im Westlichen Balkan im Wesentlichen nicht zum Tragen. Auf der praktischen Ebene wird dies in folgender Aussage deutlich:
\begin{itemize}[label={}]
\item \textit{The EU always states in their progress reports that there should be a stable, professional public administration and career system established and not having political appointees in the institutions. But on the other hand, this message is not clear for the Albanian politicians and nothing happens to them if they change their staff. I think the heads of the institutions need to start feeling responsible for the outcome of their institution” (NGO representative Albania, Frage 6).}
\end{itemize}
In dieser Aussage wird einerseits die propagierte politische Konditionalität der EU gegenüber den Beitrittsländern bestätigt. Gleichzeitig wird deutlich, dass die Nichterfüllung in Bezug auf die öffentliche Verwaltung, hier den öffentlichen Dienst, keine Auswirkungen auf den Erweiterungsprozess hat. Eine Änderung der von der EU kritisierten Praxis der öffentlichen Verwaltung in den Beitrittsländern findet nicht statt.\par
Eine Studie zur EU-Konditionalität im Bereich Demokratie und Rechtsstaatlichkeit kommt zu dem Ergebnis, dass Konditionalität in diesen Bereichen, unter die auch Verwaltungsentwicklung fällt, versagt hat. Gründe werden in den fehlenden Vergleichskriterien und Standards der EU gesehen. „The Commission acted as a prolific myth-maker, asking the candidate countries to embrace non-existent ‚European standards’” (\cite{kochenov}: 300).\par
Dimitrova stellt dazu fest, dass administrative Konditionalität zwar eine “partial conditionality” ist, diese aber ein Land nicht am Beitritt hindern kann, “a candidate country may view itself as to be able to skirt full compliance in that particular policy area without retribution” (\cite{dimit05}: 79). \par
Diese Befunde werden von der vorliegenden Untersuchung gestützt, die auch darauf hinweist, dass Konditionalität im Bereich Verwaltungsentwicklung nicht annähernd gut greift wie in den im Acquis communautaire ausdrücklich geforderten Bereichen. Die Frage ist allerdings, ob man von einem Versagen sprechen kann, wie Kochenov dies vorschlägt (\cite{kochenov}: 300), oder ob es sich dabei nicht vielmehr um eine logische Folge handelt, angesichts der Tatsache, dass kein EU-Acquis zur Verwaltungsmodernisierung definiert ist.\par
Die von der EU propagierte Konditionalität wurde im Wesentlichen auf ökonomischer Ebene angewandt, um das Aufholen der Beitrittskandidaten zu einer entwickelten Marktwirtschaft zu ermöglichen (vgl. \cite{hugsas} : 16). Für die Balkanländer und insbesondere die Untersuchungsländer werden auch für die Bereiche Rechtsstaatlichkeit und öffentliche Verwaltung Mittel seitens der EU bereitgestellt. Eine leistungsfähige öffentliche Verwaltung ist Grundvoraussetzung für die Umsetzung des Acquis communautaire. Dennoch ist die Leistungsfähigkeit der öffentlichen Verwaltungen kaum im Blick der EU. Zwar wird in der EU-Literatur und den Erweiterungsfahrplänen immer wieder auf die zentrale Rolle verwiesen, die die öffentliche Verwaltung im Rahmen der EU-Annäherung eines Landes spielt, dennoch findet keine koordinierte Förderung statt. \par
Bei der historischen Betrachtung fiel auf, dass Elemente moderner öffentlicher Verwaltung vor allem in Montenegro identifiziert wurden. Montenegro war in der historischen Verwaltungsentwicklung von Einflüssen des französischen und des k.u.k. Systems der Verwaltung geprägt, bevor es Teil des sozialistischen Jugoslawien wurde. Aber auch die Verwaltungsentwicklung in Jugoslawien mit seiner besonderen Form des Selbstverwaltungssozialismus führte Elemente moderner Staatlichkeit fort, wie die gerichtliche Überprüfbarkeit von Verwaltungsentscheidungen. Auch war die dezentrale Organisation des Staates im sozialistischen Jugoslawien zumindest in begrenztem Umfang hilfreich nach der Demokratisierung bei der Etablierung von Strukturen unterhalb des Zentralstaates. Hinderlich in diesem Zusammenhang ist allerdings bis heute die während der sozialistischen Zeit etablierte Durchdringung aller Ebenen der öffentlichen Verwaltung mit Parteigängern der Regierungspartei. Diese Tradition setzt sich trotz aller Bemühungen um Reformen in der Gesetzgebung zum civil service fast ungehindert fort. Es stellt sich die Frage, ob dies nur der schwierigen ökonomischen Lage geschuldet ist, oder ob es sich um tiefer liegende Strukturen handelt. \par
An diese Beobachtungen schließt sich die mögliche Bedeutung der historischen Erfahrungen und Traditionen in der öffentlichen Verwaltung an, die eine nicht zu vernachlässigende Wirkung auf die aktuelle Entwicklung in der Demokratie haben. \par
Die parteipolitische Durchdringung der öffentlichen Verwaltung in der Demokratie konnte für alle drei Untersuchungsländer festgestellt werden. Dies zeigt, dass der immense Stellenwert der Partei im sozialistischen Jugoslawien und im kommunistischen Albanien in der Nachwirkung auf die aktuelle Entwicklung nicht zu unterschätzen ist. Zu diesem Zusammenhang wäre weitergehende Forschung wünschenswert. \par
 Während die Modernisierung der öffentlichen Verwaltung eine wesentliche Forderung der EU im Erweiterungsprozess ist, ist das Thema nicht als ein Kapitel des Acquis verankert. Eine moderne leistungsfähige Verwaltung ist eine Grundvoraussetzung der Aufnahme in die EU. Nur eine effektive öffentliche Verwaltung ist in der Lage die Bedingung zu schaffen für eine Annäherung der Verfahren und Gesetzgebung an die EU-Standards. Die öffentliche Verwaltung und der Stand ihrer Entwicklung ist für die EU eine horizontale Aufgabe, d.h. sie ist notwendig für die Umsetzung aller Forderungen aus dem Acquis. In den jährlichen Fortschrittsberichten werden die Fortschritte pro Beitrittskandidat in allen 35 Kapiteln der Acquis ausführlich dargestellt. Die Kapazität und Entwicklung der öffentlichen Verwaltung wird in einem Eingangskapitel unter „politische Kriterien“ begutachtet.\par
In der Analyse der Fortschrittsberichte für die drei Untersuchungsländer von 2006 bis 2012 fällt auf, dass in der Zusammenfassung vor allem die Politisierung des öffentlichen Dienstes als Problem benannt wird. Beim Vergleich über die Länder und die Jahre hinweg ergibt sich ein erstaunlich einheitliches Bild in der Einschätzung durch die EU. Es ist kaum eine Entwicklung festzustellen und die Politisierung des öffentlichen Dienstes in allen drei Untersuchungsländern wird Jahr um Jahr angeprangert. Auffällig und überraschend in der Auswertung des Interviewmaterials ist die starke und fast ausschließliche Konzentration aller Interviewpartner auf das Thema civil service. Im Rahmen der verwaltungswissenschaftlichen Betrachtung würde man das Thema civil service allenfalls als ein Thema unter vielen anderen Themen (z.B. E-Government, Dezentralisierung, regionale Verwaltungszusammenarbeit etc.) der Debatte um Verwaltungsmodernisierung erwarten.\par
Möglicherweise ist diese (einseitige) Konzentration im Zusammenhang mit der Berichterstattung der EU im Rahmen der Fortschrittsberichte zu sehen: Wenn, wie im Fall der Zusammenfassungen zur Verwaltungsentwicklung in den EU-Fortschrittsberichten, (fast) ausschließlich der civil service besprochen wird, überrascht der Befund der vorliegenden Arbeit nicht, dass kaum andere Themen der Verwaltungsmodernisierung von den Gesprächspartnern benannt werden. Mit dieser Schwerpunktsetzung wird der politische Diskurs zwischen der EU und den beitrittswilligen Staaten unter Umständen thematisch vorbestimmt. Diskurse bilden, so Foucault, Wirklichkeit nicht nur ab, sondern stellen sie auch her (vgl. Foucault 1973: 42). Im Sinne Foucaults könnte man sagen, dass die EU mit ihren Fortschrittsberichten die Wahrnehmung aller Akteure beeinflusst und zur Erschaffung der Realität, im Sinne der Konstruktion einer komplexen gesellschaftlichen Situation, beiträgt. Andere Themen der Verwaltungsmodernisierung wie die Erbringung öffentlicher Aufgaben, e-government oder Kundenorientierung werden, wie die Interviewauswertung zeigt, weder von den EU officials noch von den Experten in den Untersuchungsländern benannt. Um diesen Befund weiter auszuleuchten, wäre weitergehende verwaltungswissenschaftliche Forschung wünschenswert. \par
Auf der Ebene der praktischen Unterstützung der Beitrittskandidaten durch die EU findet sich vor allem die Entsendung von Experten und Praktikern aus den EU Ländern in die Beitrittskandidaten im Rahmen der Institutionenhilfe. Diese Entsendungen finden meist statt im Rahmen des Twinning-Ansatzes der EU statt, der aus der Verwaltungshilfe für Entwicklungsländer entstanden ist. Dieser Ansatz wird von den Akteuren in Brüssel und in den Untersuchungsländern generell als angemessen eingeschätzt. Allerdings gibt es Fragezeichen der befragten Experten hinsichtlich der Geeignetheit dieses Ansatzes, insbesondere in Abwesenheit einer verbindlichen Richtschnur der umzusetzenden Standards zur Verwaltungsentwicklung. Als problematisch erlebt wird die Tendenz, Konzepte der jeweiligen Entsendeländer in Bezug auf öffentliche Verwaltung in den Beitrittsländern umzusetzen. Die Nachhaltigkeit und Angemessenheit eines solchen Ansatzes wird seitens der Interviewpartner durchaus in Frage gestellt. Auch zu diesem Zusammenhang böte sich weitere verwaltungswissenschaftliche Forschung an. \par
Verwaltungsentwicklung in den Untersuchungsländern blickt auf eine ca. 20-jährige Zeit seit der Demokratisierung zurück. Zeitlich davor waren die Untersuchungsländer vom jugoslawischen Sozialismus (Montenegro und Mazedonien) und einem kommunistischen System in Albanien geprägt. Die Annahme, dass die vor der Demokratisierung bestehenden Systeme immer noch einen Einfluss entfalten auf die aktuelle Situation der öffentlichen Verwaltung (Legacy-Ansatz), wird von der Auswertung der Interviews gestützt. Die befragten Experten verweisen in einer Reihe von Fällen auf langfristige kulturelle Prägungen bei aktuellen Problemen der Verwaltungsentwicklung. Kulturelle Prägungen sind insbesondere in einem komplexen System wie der öffentlichen Verwaltung nur langsam veränderbar und entfalten sehr langfristigen Einfluss.\par
Die kulturellen Prägungen sind zum großen Teil in der historischen Verwaltungsentwicklung zu suchen. Für Mazedonien und Montenegro ist dieser historische Bezugspunkt Jugoslawien mit seinem spezifischen „Selbstverwaltungssozialismus“, für Albanien dagegen ist es das kommunistische System in seiner isolationistischen Ausprägung. Aber auch die diesen Systemen zeitlich vorgelagerten Einflüsse der Imperien setzen sich in der Verwaltungskultur der Untersuchungsländer immanent fort. Während der osmanischen Zeit entfaltete das Reich vor allem in den Bergregionen kaum administrativen Einfluss und das Gemeinwesen wurde durch die traditionellen Clanbeziehungen geregelt. Diese Besonderheit fand sich in den Bergregionen Albaniens und Montenegros, während in Mazedonien unterschiedliche Nachbarländer immer wieder auf das Land zugriffen.\par
Besonders die in den EU-Fortschrittsberichten oft kritisierte fehlende Implementierung von Gesetzen und Reformen kann zurückverfolgt werden in die Zeit Jugoslawiens. In der spezifischen staatlichen Verfassung in Jugoslawien entwickelte sich die Praxis, Projekte und Initiativen der Föderation zwar aufzunehmen, diese aber in den jeweiligen Republiken nicht umzusetzen und stattdessen regionale Interessen zu verfolgen. Albanien hat aktuell vor allem mit gegenseitiger Blockadepolitik zweier etwa gleich starker Parteien zu kämpfen. Hier kommen Veränderungen nicht voran, da die jeweils andere Gruppierung Gesetze und Reformen blockt. Dies kommt besonders in der Beziehung von Zentrum zu lokaler Ebene zum Tragen, wenn diese unterschiedliche politische Orientierung haben. Im kommunistischen System war die öffentliche Verwaltung straff auf das Zentrum ausgerichtet, ohne Spielraum für die kommunale Ebene. Historisch gesehen gab es in Albanien zu keiner Zeit eine Tradition der Kooperation von lokaler und zentraler Ebene.\footnote{Hier ergibt sich auch ein wesentlicher Unterschied zu den Ländern der Ost-Erweiterung der EU. In diesen Ländern hatte vor ihrer Zugehörigkeit zur Sowjetunion mit entsprechend zentralistischer öffentlicher Verwaltung eine kontinentaleuropäische Verwaltungstradition bestanden, an die angeknüpft werden konnte.
}\par
Zu weiteren historischen Einflüssen, die auf das heutige System der öffentlichen Verwaltung fortwirken, gehört die in der sozialistischen Zeit in Jugoslawien abgeschaffte Unterscheidung in Beamte und andere öffentliche Angestellte. Alle öffentlichen Bediensteten fielen in Jugoslawien unter das allgemeine Arbeitsrecht. Aktuell ist zu beobachten, dass es vor allem in Mazedonien mit dieser Tradition schwierig ist, eine Trennung in Beamte und Angestellte wieder einzuführen. In Albanien, wo es historisch kein Modell der civil servants gegeben hat, steht die Einführung des Konzeptes der civil servants ebenfalls vor großen Schwierigkeiten. Lediglich in Montenegro, das in der Zeit vor seiner Zugehörigkeit zu Jugoslawien über ein funktionierendes Berufsbeamtentum verfügte, sind historisch gesehen Anknüpfungspunkte vorhanden. \par
Eine andere Tradition der kontinentaleuropäischen öffentlichen Verwaltung, die Verwaltungsgerichtsbarkeit und Überprüfbarkeit des Verwaltungshandelns, überdauerte in der Jugoslawischen Zeit fast unbeschadet. Die Verwaltungsgerichtsbarkeit in Montenegro erhält regelmäßig gute Noten in den Fortschrittsberichten der EU. Ganz anders ist die Situation in Albanien, wo es historisch zu keiner Zeit eine Verwaltungskontrolle gab. Eine Verwaltungsgerichtsbarkeit ist in Albanien erst im Entstehen. \par
Deutlich wurde, dass die aktuellen Entwicklungen und Probleme der Modernisierung der Verwaltungen in den Untersuchungsländern eine starke historische Prägung haben. Für die weitere Modernisierung der öffentlichen Verwaltungen in den Beitrittsländern ist es wichtig, diese Prägungen zu berücksichtigen bei der Konzipierung von Unterstützung hin zu einer weiteren Annäherung an die EU.

\section{Konsequenzen für die Förderinstrumente der EU}
In den Untersuchungsländern besteht eine Reihe von hemmenden Faktoren für die Verwaltungsmodernisierung. Zunächst ist die Tradition der zentralistischen Organisation der öffentlichen Verwaltung in den vor-demokratischen Zeiten zu nennen. Die politischen Umwälzungen nach 1989 stellten die Untersuchungsländer vor die Aufgabe eine öffentliche Verwaltung auszubauen, die der geografischen Verortung der Länder in Europa entsprach. Während der politische Umbau hin zu einer parlamentarischen Demokratie in allen Untersuchungsländern schnell stattfand mit der Einführung von demokratischen Wahlen und einem Mehrparteiensystem, ist der Umbau, bzw. Aufbau einer rechtsstaatlichen Verwaltung eine langfristige Aufgabe. Die vorliegende Arbeit macht deutlich, dass dieser Umbau noch nicht vollständig stattgefunden hat. Dabei stehen vor allem Traditionen im Weg, die in der vordemokratischen Zeit geprägt wurden. Fehlender Bezug des Einzelnen zum Gemeinwesen mit stärkerer Prägung auf die Familie und den Clan, der wie gezeigt wurde geschichtlich erklärt werden kann, ist zentral. Ebenfalls geschichtlich herleitbar ist die mangelnde Umsetzung von Gesetzen in den Untersuchungsländern, die eine effektive Verwaltungsmodernisierung behindert. So wurde deutlich, dass einerseits in allen Untersuchungsländern Gesetze zum civil service erlassen wurden und Institutionen eingerichtet wurden zu ihrer Umsetzung. Andererseits werden jedoch Wege gefunden, mit denen die Anforderungen umgangen werden, wie z.B. temporäre Verträge für civil servants.\par
Im Rahmen der Heranführungsstrategie für die Länder, die der EU beitreten wollen, stellt die EU Mittel für die Beitrittskandidaten bereit. Diese Finanzhilfe hat sich evolutionär entwickelt. Für die Staaten des Westbalkans stand EU-Finanzhilfe zunächst vor allem für Wiederaufbau und Infrastrukturprojekte zur Verfügung. Im Zuge der konkreteren EU-Perspektive der Länder seit dem Thessaloniki-Gipfel 2003 ist auch Institutionenaufbau Teil der Unterstützung. In dem aktuellen IPA-Programm der EU für die beitrittswilligen Länder wird auch die administrative Kapazität gefördert. Vor allem durch die Instrumente Twinning und TAIEX wird Verwaltungsexpertise aus den Mitgliedsländern der EU zur Verfügung gestellt.\par
Die Auswertung der Experteninterviews weist auf Verbesserungsmöglichkeiten bei den EU-Pro"-grammen zur Unterstützung der administrativen Kapazität in den Beitrittsländen hin. So gibt es außer den Hinweisen aus den EU-Fortschrittsberichten keine konkrete Orientierung für Projekte zur Verbesserung der öffentlichen Verwaltung. Dies führte und führt oft zur Übertragung von Modellen aus der Verwaltungspraxis der Länder, die die Experten stellen. Auch in diesem Zusammenhang wurde von den Interviewpartnern fehlende Nachhaltigkeit der Hilfe bemängelt. Oft sind die Projekte nicht ausreichend vor Ort abgestimmt und werden in Brüssel konzipiert, womit Akzeptanz vor Ort und institutionelles Lernen erschwert wird.\par
Eine Steuerung der Hilfe, die durch die EU zur Verfügung gestellt wird, ist durch ein komplexes Zuständigkeitsgeflecht in der EU zu bewältigen. Für den Bereich Verwaltungsentwicklung kommen weitere Stellen innerhalb der EU dazu, da es sich um eine horizontale Aufgabe handelt, die nicht in einem Acquis-Kapitel definiert ist. Die Koordinierung wird von den Interviewpartnern sowohl der EU als auch aus den Empfängerländern als schwierig erlebt. \par
Die Hilfspro"-gramme der EU, unter denen auch Verwaltungsmodernisierung gefördert werden kann, haben sich evolutionär entwickelt. Das aktuelle Instrument, IPA, ist in seinem finanziellen Rahmen auf große Projekte ausgelegt mit einem bestimmten Finanzvolumen. In der Erfahrung der Interviewpartner stellt dies vor allem kleinere Länder vor Herausforderungen. Auch werden die Antragsmodalitäten in ihrer Komplexität als problematisch benannt. Die Interviewauswertung weist darauf hin, dass sich aus diesen strukturellen Schwierigkeiten Hemmnisse für die Anwendung des Instrumentariums für die Verwaltungsmodernisierung in den Untersuchungsländern ergeben. \par
In der Zusammenschau der fehlenden Tradition einer rechtsstaatlich orientierten öffentlichen Verwaltung, der mangelnden Umsetzung von Gesetzen und Reformen in den Untersuchungsländern mit den Schwierigkeiten bei der Förderung von Verwaltungsmodernisierung durch EU-Pro"-gramme, besteht die Gefahr von Reform-Attrappen. Für die EU-Erweiterung stellt diese Gefahr ein ernstes Problem dar, da, wie gezeigt wurde, die Konditionalität in Bezug auf die öffentliche Verwaltung keine Kraft entfalten kann und zudem nach dem Beitritt kein Monitoring mehr zur Verwaltungsentwicklung stattfindet. Im Ergebnis würden weitere Länder in die EU aufgenommen, deren öffentliche Verwaltungen den Mindestanforderungen an Professionalität, Transparenz und Effektivität nicht entsprechen. 

\section{Mögliche Strategien }
In der Auswertung der Interviews wird deutlich, dass vor allem in den Untersuchungsländern selbst eine stärkere Orientierung und Anleitung zur Verwaltungsmodernisierung durch die EU gewünscht wird. Es wäre für die weitere Modernisierung der Beitrittsländer zielführend, eine Art von Katalog an die Hand zu bekommen. Auch für die Konzipierung der EU-Projekte zur Unterstützung der Verwaltungsmodernisierung wäre dies sinnvoll. In Abwesenheit verbindlicherer Leitlinien der EU zu Verwaltungsstruktur und Verwaltungsmodernisierung werden in den EU-Projekten meist Modelle aus den jeweiligen Ländern der entsandten Experten übertragen. Eine Praxis, die in Anbetracht der grundsätzlichen Probleme mit der Verwaltungsentwicklung in den Untersuchungsländern aus Sicht von Nachhaltigkeit und „local ownership“ zu hinterfragen ist.\par
Auf der praktischen Ebene wird eine stärkere Outcome- und Impact-Orientierung der EU-Hilfe im Bereich Verwaltungsmodernisierung als förderlich angesehen. Dies würde ein langfristiges Monitoring der Ergebnisse nach Projektabschluss voraussetzen, eine Praxis, die bislang nicht entwickelt ist.\par
Weiterhin sollten gesellschaftliche Gruppen bei der Entwicklung eines Fahrplanes zur Verwaltungsmodernisierung beteiligt werden. Dabei handelt es sich um eine kulturelle Umorientierung, die nur langfristig zu erreichen ist. Möglicherweise ist die Ausgangssituation in Montenegro und Mazedonien etwas günstiger einzuschätzen, da dort in der jugoslawischen Zeit zumindest nominell eine Beteiligung gesellschaftlicher Gruppen an Planungsprozessen propagiert worden war. In Albanien dagegen hat historisch gesehen eine Beteiligung von gesellschaftlichen Gruppen keine Tradition. \par
Zentral wäre eine kontinuierliche Begutachtung der Verwaltungsentwicklung auch nach einem Beitritt. Die Erfahrung aus der EU-Osterweiterung zeigt, dass es nach der Aufnahme in die EU meist zu einer deutlichen Verlangsamung der Reformfortschritte der Länder oder gar zu Rückschritten gekommen ist.\par
Für die Untersuchungsländer wird auch deutlich, dass die Instrumente der EU, vor allem das IPA-Programm, das mit großen Finanzvolumen arbeitet, an die Bedürfnisse von kleineren Ländern angepasst werden sollte. \par
Eine aktuelle Untersuchung zu civil service Reformen und Professionalisierung im Westbalkan kommt zu dem Schluss, dass die Bedingungen für Reformen auf nationaler Ebene in den Beitrittsländern derzeit ungünstig sind. Dies ist zum Teil der aktuellen Finanzlage und damit zusammenhängend Verschärfungen der sozialen Situation geschuldet, aber auch der Abschwächung der europäischen Perspektive der Staaten des Westbalkans. Es wird konstatiert, dass die fehlende Definition administrativer Konzepte seitens der EU zunehmend ein Problem darstellt für die Reformschritte in den Beitrittsländern. So werden dem europäischen Standard entsprechende Konzepte für den civil service zunehmend aufgeweicht mit größerem Einfluss und Entscheidungsfreiheit der Vorgesetzten über die Einstellungspraxis. ”The contestation of the European principles as the most desired concept to guide civil service reform and the lack of specific guidelines for institutional reform have increasingly undermined the capacity of the international community to direct and monitor reform efforts in the Western Balkans. Further reform slippage is likely unless the European principles are reviewed, clarified and re-fashioned in the area of civil service reform” (\cite{meyersah12}: 8). In diesem Zusammenhang schlägt der Autor der Studie vor, die europäischen Prinzipien zum civil service neu zu definieren und in ein umfassenderes Konzept von „better governance“ einzubetten. „Indeed it is worth considering a re-launch of the European principles as a wider initiative for better governance in Europe“ (\cite{meyersah12}: 9). \par
Im Verlauf der vorliegenden Arbeit zeigte sich, dass die EU zwar von administrativer Konditionalität spricht, es aber kein Konzept gibt, an dem sich die Beitrittskandidaten orientieren könnten. Versuche einer PAR checklist sind im Sande verlaufen, nicht zuletzt aufgrund von Widerstand aus den Mitgliedstaaten, die keine Einmischung in die Ausgestaltung ihrer historisch gewachsenen öffentlichen Verwaltung wünschen. Auch das Konzept „European Administrative Space“ hat bislang keine nachhaltige Wirkung entfaltet, nicht zuletzt aufgrund der fehlenden Verbindlichkeit. Somit besteht in der Realität keine wirksame administrative Konditionalität. Notwendig ist eine neue Debatte um eine EU-Vision von Better Governance, die umfassend ist und auch über das Thema civil service hinaus eine moderne und den Herausforderungen angemessene öffentliche Verwaltung in der erweiterten EU entwirft. Dieses Konzept wäre dann nicht nur eine Richtschnur für die Beitrittsländer des Westbalkans, sondern auch anwendbar für die Modernisierung der Verwaltungen in den Ländern, die schon zur EU gehören. Ein Konzept von Better Governance für die EU zu entwickeln ginge auch über die bekannten Elemente von New Public Management hinaus, da es zusätzlich zu Effektivitätskriterien weitere Elemente einbeziehen müsste und wahrscheinlich auch zu einer gewissen Vereinheitlichung administrativer Strukturen führen würde. Dass solch eine Reform insgesamt vonnöten ist, zeigen nicht zuletzt die aktuellen Probleme in der EU, die in Ländern mit intransparenter und nicht effektiver öffentlicher Verwaltung verschärft zum Tragen kommen.


\include{chapter06}
% usw.

\addtocontents{toc}{%
   \protect\setcounter{tocdepth}{1}%
} 
%\renewcommand\section{\@startsection
%   {section}{1}{0mm}%      % name, ebene, einzug
%   {3\baselineskip}%            % vor-abstand
%   {0,1\baselineskip}%            % nach-abstand
%   {\bfseries\rmfamily\large}%           % layout
%   }
% Literaturverzeichnis
\renewcommand*\chapterheadstartvskip{\vspace*{-1cm}}
\nocite{*}

\printbibliography

\vspace{0,5cm}
\textbf{{\Large Archivquellen:}}
\begin{itemize}[label={},leftmargin=*,itemsep=0pt,parsep=10pt]
\item Haus-, Hof- und Saatsarchiv, Wien (HHSTA)
\item HHSTA, PA XVII Montenegro, Kt. 29, Berichte Weisungen 1910-1911, Bericht Freiherr von Giesl an österreichisches Außenministerium 24. Oktober 1911.
\item HHSTA PA XVII Montenegro, Kt. 30 Berichte, Weisungen 1914, Übersetzung Regierungsprogramm, Beilage zu Bericht vom 6. Februar 1914).
\item HHSTA, PA I, Kt. 998, 49f, Bericht des k.u.k. Gesandten an Außenministerium Mitte Dezember 1916.
\item HHSTA, PA I, Kt. 998, 49g, k.u.k. Militärgouverneur von Weber an das k.u.k. Armeekommando, Cetinje, 6. Juni 1916.
\item HHST, PA I, Karton 1001, Geheime Note des k.u.k. Chefs des Generalstabes, 28.Mai 1916, OP.No.25.492.
\item HHSTA, PA I, Nr. 1001, Geheime Note des k.u.k. Chefs des Generalstabes, 28.Mai 1916, OP.No.25.492, Beilage 1.
\item HHSTA, PA I, Karton 1001, Lagebericht Feldpostamt 140, 9.September 1916, Trollmann, k.u.k. XIX Korpskommando, E.V. Nr. 962/IX.
\item HHSTA, PA I, Karton 1006, Z. 56/P.Kral, Shkodra, 5. April 1917. 
\item HHSTA, PA I, Karton 1006, Z. 184
\item HHSTA, PA I, Karton 1006, Z. 184/P.Kral, Shkodra, 16. November 1916.
\item HHSTA, PA I, Karton 1006, Nr. 163/P.Kral, Shkodra, 30. Juni 1918.
\end{itemize}
\vspace{0,5cm}
\textbf{\Large{Internetquellen:}}

\begin{itemize}[label={},leftmargin=*,itemsep=0pt,parsep=10pt]
\item http://www.europarl.europa.eu/brussels/website/media/modul\_01/Zusatzthemen/\\Pdf/Acquis.pdf, (Aufgerufen: 10.9.2012).
\item http://www.consilium.europa.eu/ueDocs/cms\_Data/docs/pressData/de/ec/72924.pdf, 13, (Aufgerufen: 21.9.2012).
\item http://www.stabilitypact.org/about/default.asp, (Aufgerufen: 1.9.2012).
\item http://europa.eu/legislation\_summaries/enlargement/western\_balkans/r18003\_de.htm, (Aufgerufen: 19.10.2012).
\item http://europa.eu/legislation\_summaries/enlargement/2004\_and\_2007\_enlargement/\\e50004\_en.htm, (Aufgerufen: 19.8.2012).
\item http://ec.europa.eu/enlargement/archives/ear/publications/main/documents/
EAR-Article-Montenegro\_Sept2007.pdf, (Aufgerufen: 19.8.2012).
\item http://www.svez.gov.si/nc/en/splosno/cns/news/article/2028/1265/\\EnlargementPackage2006, (Aufgerufen: 15.7.2010).
\item http://assembly.coe.int/ASP/Doc/XrefViewPDF.asp?FileID=18947\&Language=EN, (Aufgerufen: 5.10.2012).
\item http://ec.europa.eu/enlargement/archives/ear/montenegro/montenegro.htm, (Aufgerufen: 5.10.2012).
\item http://ec.europa.eu/enlargement/archives/ear/montenegro/montenegro.htm, (Aufgerufen: 5.10.2012).
\item http://www.pad.gov.al/en/dap.html, (Aufgerufen: 24.10.2012).
\end{itemize}
\providecommand*{\appendixmore}{}% Falls Option headings=appendixprefix nicht verwendet wurde.
\newcommand*{\SavedOriginalchaptertocentry}{}
\appto\appendixmore{%
  \let\SavedOriginaladdchaptertocentry\addchaptertocentry
  \renewcommand*{\addchaptertocentry}[2]{%
    \ifstr{#1}{}{% Eintrag ohne Nummer
      \SavedOriginalchaptertocentry{#1}{#2}%
    }{% Eintrag mit Nummer
      \SavedOriginaladdchaptertocentry{}{%
        \string\expandafter\string\MakeUppercase\string\appendixname
        ~#1:\string\enskip{}#2}%
    }%
  }%
}
% \addcontentsline{toc}{chapter}{Anhang}   % perhaps this line is better placed in the included file
% Inhaltsverzeichnis

\clearpage
% Anhang*

%\marginsize{2cm}{2cm}{2cm}{2cm}
%\newgeometry{a4paper,left=20mm,right=20mm, top=20mm, bottom=20mm}
\appendix
\pagestyle{headings}
\addtocontents{toc}{\protect\value{tocdepth}=0\relax}
%\addtocontents{toc}{%
%    \protect\setcounter{tocdepth}{0}%
%%} 
\renewcommand*\chapterheadstartvskip{\vspace*{-2cm}}
\pagestyle{empty}
\chapter{Questionnaire EU-officials, enlargement experts in the area of Public Administration Reform }
\label{anhang:Questionnaire EU-officials}
1. Which topics/areas are presently dealt with as a priority by the EU regarding Public Administration Reform in Albania/Macedonia and Montenegro? What are the developments you see there?

2. Do you think the EU approach regarding Public Administration Reform in Albania, Macedonia and Montenegro is adequate? Or should other aspects be included from your point of view?

3. Do you perceive differences in the EU approach compared with the experience with PAR during the last wave of enlargement?

4. The literature on enlargement sometimes argues with the legacy theory, in particular regarding the last wave of enlargement. Meaning that structures of previous regime set-ups have an influence on the present development of Public Administration Reform. What is your view on this issue regarding Albania/Macedonia and Montenegro?

5. How do you assess the cooperation within the EU Commission regarding Public Administration reform in Albania/Macedonia and Montenegro with the different Units, DG Enlargement, country desks, special PAR Unit and DG Admin? 

6. Public Administration Reform is not a separate chapter in the Acquis. Should it be a separate chapter? 

7. What is your take on the Treaty of Lisbon regarding Public Administration reform? Does the Lisbon Treaty lead to a different approach of the EU towards Public Administration Reform in the candidate and potential candidate countries? 

8. How do you asses the EU-Insturments to promote Public Administration Reform in Albania/Macedona and Montenegro as regards quantity and effectiveness:
Differentiate per country, if possible
CARDS (phased out)
Twinning
Twinning light
TAIEX
IPA
Did I forget to mention an instrument that is relevant?
9. Are these programmes well designed for the needs of PAR in Albania/Macedonia and Montenegro or do you perceive a need for adjustment in any of them? (Content or technical)


10. In your opinion, are there obstacles to PAR in Albania/Macedonia and Montenegro? And what would be necessary for successful PAR in Albania/Macedonia and Montenegro? 

11. Which other instituions/organizations or bilateral donors are important in regards to PAR in Albania/Macedonia and Montenegro? How do you asses their impact on PAR in the three countries? 

12. Who is responisble for co-ordinating the PAR activities of all the different donors in Albania, Macedonia and Montenegro and what is happening in this respect at the moment? 

13. Should the EU have additional or other priorities in future PAR programming in Albania, Macedonia and Montenegro. 

14. Is there anything else that is important in the context of my research that you would like to comment on?
\clearpage
\addtocontents{lot}{\protect\value{tocdepth}=0\relax}% Ab dem Anhang Einträge im
                                       % Tabellenverzeichnis nicht mehr anzeigen.
\pagestyle{myheadings} 

\chapter[Interview with EU-officials, enlargement experts]{Interview with EU-officials, enlargement experts in the area of Public Administration Reform}
\label{anhang:InterviewEuOfficials}
%---------------------------------------------------------------------------------------------------------------------------
\section{Which topics/areas are presently dealt with as a priority by the EU regarding Public Administration Reform in Albania/Macedonia and Montenegro? What are the developments you see there? }
\label{sec:there}

\markboth{Anhang \thechapter, Frage Nr. \thesection}{Anhang \thechapter, Frage Nr. \thesection}
\textbf{EC official, DG ELARG PAR Coordination team}: PAR, or governance is a key priority in the enlargement process. The updated partnership documents with each country, list the priorities, which might differ from country to country. Mostly priorities related to PAR are found under political criteria and there we have them under Parliament, Government and PA, but also under headings such as civil and political rights and anti-corruption and possibly under chapter 23 Judiciary and Fundamental rights or chapter 32, financial control, some of these functions can also be seen as part of the horizontal PA tasks. The main priority in all the countries is to establish a civil service and a PA that is professional and not influenced by political constellations. There is a tendency, especially after elections to replace many people in the PA. We should distinguish between changes in the legislation related to the public administration and to make them conform with what we call European standards. The other point is implementation and enforcement of those laws. But I must say, in both areas, progress is usually slow. And even when good laws are enacted, you often do not find the administrative capacity in these countries to implement them or the political will. \\
\textbf{OECD/SIGMA team}: There is a tendency to understand PA in terms of civil service and administrative law and to some extent policy making, lately. For PA, we think, this is too narrow. It should be Public Governance. If you are dealing with PA, it should be wider than 3 or 4 main topics. Within the EU's definition of PA in the three countries, there is a strong interest in PAR-Strategies, in Montenegro and Macedonia and perhaps a bit less so in Albania, and in that the main focus tends to be on civil service law and anti-corruption. Lately, there is increasing interest in Admin Procedures and Admin Justice.\\
\textbf{EC official, DG ELRG Evaluation Unit team}: One of the goals is to install democratic stability in these countries with functioning institutions. The institutions we focus on are very much in the sector Justice and Home Affairs and institutions linked to democratic stability. Quite a few of the PAR projects focus on institutional structures and their ability to implement community law. Not in all of the countries we are using all of the instruments. In Montenegro, we are not using Twinning as heavily as in Albania, for example. Twinning is very helpful if you have counterparts in the host country administration. If you do not have that, the results of the projects can be in question. Montenegro is a small country with fewer and smaller institutions and right now, they are very much stretched with their engagement in the pre-accession process. And we know that from the past, even for Slovenia that this is always an extra strain on a small country. So, of course we are careful not to force too many heavy projects on them. \\
\textbf{EC official, DG ELARG Forner Yugoslav Republic of Macedonia team}: The main topic now is related to civil service law and that is the topic of recruitment, the principle of recruitment based on merit and on a transparent process. We also saw overuse of so called temporary employments, which might be a specific case for Macedonia. The state administration for whatever capacity they needed would get staff through private employment agencies for one year on a short term contract to do the job of a civil servant. In summer 2010, the authorities of Macedonia started the process of recruitment and there are indications that not everything was as transparent as it should be. And there are signals that those who were temporarily employed were given an advantage, if they were not directly transferred, which is of course against the principles. There is a Civil Service Agency (CSA) as a body independent from the government and reporting to Parliament. In reality, this arrangement has also created some problems, because for many they are just an agency. So you can imagine that a ministry of finance would hardly listen to someone from an agency telling them how they should do their work. Macedonia is now preparing an updated national PAR strategy, the first update in 10 years. And we understand there is the plan to have a ministry for PA. So the current Ministry for IT will be combined with Ministry for PA. Then the CSA would be included in the organization of this ministry as one of the departments. \\
\textbf{EC official, DG ELARG Albania team}: PAR is an overarching horizontal aspect that goes beyond the political criteria, but that is reflected specifically in the political criteria and PAR is an issue for Albania. We analyze the current situation and we give our view. In our view PAR in Albania is incomplete; there are certain issues we are following up very closely and in detailed discussions and exchanges with SIGMA. We are fully in line with the analysis SIGMA is providing in this regard on the ongoing process of civil service law reform in Albania and strengthening the department that deals with that reform. These are priorities for us in terms of financing.\\
\textbf{EC official, DG ELARG Montenegro team}: Priorities are mainly in the filed of civil service, training issues and the non-political recruitment of civil servants in every ministry. Non political civil servants still needs to improve in Montenegro. A Human Resources Management Agency was created, but unfortunately it is not yet in the lead on reforms. A person in the Deputy Prime Minister's Office has been nominated as the central contact point for PAR. There is a new PAR strategy named AURUM. For 2011, the EU foresees a large IPA project for PAR in Montenegro. In all the WB countries the main IPA projects deal with are in the realm of Rule of Law/Good Governance and PAR. \\
%%\newpage
%---------------------------------------------------------------------------------------------------------------------------
\section{ Do you think the EU approach regarding Public Administration Reform in Albania, Macedonia and Montenegro is adequate? Or should other aspects be included from your point of view? }
\label{sec:view}
\markboth{Anhang \thechapter, Frage Nr. \thesection}{Anhang \thechapter, Frage Nr. \thesection}
\textbf{EC official, DG ELARG PAR Coordination team}: The discussion is what comes first, legislation or culture. You can say that culture is affected by the laws and by enacting new laws, good laws you can influence and bring about change in a country. Another approach is based on trying to draft strategies for change, listing the objectives, and having an action plan with everything carried out according to that plan. But it did not always work like that. Maybe the strategies themselves were not professional. Maybe important elements of a strategy were missing, like a clear definition of the objectives and a realistic, continuously monitored action plan. Sometimes, the scope of PAR and governance were not defined. There is no dedicated Acquis chapter on PAR and thus, no framework for discussions. Fortunately, there seems to be growing awareness, even without Acquis. And in the case of Macedonia, there is a new  high-level working group on PAR. Also checks and balances are very important. Institutions to deal with complaints against the public administration or the government, reform of the ombudsperson institution, but also external audit with Supreme Audit Institutions. According to international standards, a supreme audit should also carry out performance audits of government programmes and activities. While this is just starting for some of these countries, it will contribute to the reform in PA. \\
\textbf{OECD/SIGMA team}: The scope of PAR should be widened to include financial aspects and policy aspects and to focus on results rather than on inputs. What we have been doing in the past and the Commission has been doing in the past is worrying more about who makes a decision than about the decision itself. The other aspect is, not seeing PA as independent from its governance context. Thus, I think that civil service reform is not appropriate to the context. Civil service reform, professionalizing and depoliticizing the civil service, at the moment is a xeno-transplant which will suffer pathological rejection. The second point is that the EU is pushing countries to reform all the time and this is substituting the presence of a reform programme for administrative performance. I think much greater emphasis has to be on the idea of implementing previous policies and previous laws and not pushing people to continuous reforms. This results in diverting resources to perform reform activities away from implementing activities. Lastly, I think that adequacy includes the quality aspect and there is a lot to be said about the quality of support given to the countries for PAR, which is largely driven by the technical assistance with management systems that have been adapted.\\
\textbf{EC official, DG ELARG Evaluation Unit team}: We are at the moment looking at the way we programme accession funds. One of the things we are thinking about is to introduce the sector wide approach, to put the focus on certain priority sectors over a period of three years. The MIPD will be the main document driving this reform, to enable us to focus on priority sectors in the countries. Evaluations of previous enlargement rounds suggest that we maybe covered too much ground at once and the countries found it difficult to prioritize between the different sectors. When MIPDs are drafted, there is a discussion on how to determine the priority sectors with the Delegations and the countries and of course this is linked to the progress reports, accession partnerships and all the top level documents. Some countries came up with a list of three, others with a list of ten priority sectors. These proposals are now being discussed, we have until January (2011). The idea is not only to have sectors, but you also ask countries to have a strategy for each of the sectors. The strategies will be linked to a budget. It is not only us putting money in; it is also the national budget of the countries and the donors that contribute to these sectors. We would end up, let us say, with a set of eight priority sectors and then the countries in cooperation with the Delegations decide which sectors should be covered by which donor.\\
\textbf{EC official, DG ELARG Former Yugoslav Republic of Macedonia team}: I think in our case and also due to the fact that we are the first ones to have this special platform for Macedonia exclusively dedicated to PAR, we took a comprehensive approach. We really want to discuss all aspects of PAR, starting with the basic institutional framework, but looking also into aspects like corruption and transparency as well as donor coordination. It should really be a forum for everything that relates to PA to be discussed. Maybe not everything to the same detail, because we have other fora, such as on corruption and we have a sub-committee on Justice and Home Affairs. But we can talk about prevention and the organizational side more in our special group. This still has to be fine tuned, but we really try to be comprehensive and see PAR as an across the board issue, which is somehow related to many areas of the Acquis.\\
\textbf{EC official, DG ELARG Albania team}: We see PA as overarching and horizontal responsibility because it relates to the question of the foundation of the state, of having good governance, stable institutions and a civil service with the right competences, professionalism and ethics. But we do not have a special programme other than this general wish for good governance. We have PAR as part of the political criteria, which have to be sufficiently met for a country to start negotiations. In that respect, early attention to good governance and setting priorities that have to be met could be seen as approach. We also give it attention in financial assistance; we have the IPA instrument for preaccession assistance. The strong cooperation with OECD/SIGMA is another sign that we want to go into depth in the analysis, in order to find the areas that need to be targeted with advice or assistance. Regarding a definition of PAR, I think we are generally inspired by SIGMA. And we try to adapt it as much as 
possible to our client countries.\\
\textbf{EC official, DG ELARG  Montenegro team}: Montenegro is a special case. Reports mention the poor administrative capacity, but it does have an administration that corresponds to the size of the country. The EC is working on analyzing the specific needs of small countries together with SIGMA. Some EU requests might have to be adjusted to the size of the country, we hope for a new and innovative approach in that respect. For example we had discussion on the advantages of long term TA over  short term assistance in the long term: i.e. the same person coming for one week per month during two years for example. With TA however, sometimes dependency on one foreigner for a long time is the case. When this person leaves, the momentum drops ! Maybe short term consultants will keep the momentum up? SIGMA is more in favour of short term and flexible approaches. TAIEX is a short term assistance focussing on the EU Acquis, it is very efficient.\\
%\newpage
%---------------------------------------------------------------------------------------------------------------------------
\section{Do you perceive differences in the EU approach compared with the experience with PAR during the last wave of enlargement? }
\label{sec:enlargement}
\markboth{Anhang \thechapter, Frage Nr. \thesection}{Anhang \thechapter, Frage Nr. \thesection}
\textbf{EC official, DG ELARG PAR Coordination team}: Yes, as a conclusion from the last wave, we issued a renewed consensus in our enlargement strategy that was issued and adopted in 2006. We focus on a stricter conditionality in all phases of the process, because we realized that in the previous enlargement rounds, we were not as strict as we should have been perhaps, in particular with these two countries that became members in 2007. We also realized that we need to address difficult issues, not only when it comes to judicial reforms, but PAR in general and the fight against corruption much earlier in the enlargement process. This message has been repeated in the following enlargement papers. In the last one for 2009, there was even a special section dedicated to the rule of law. And under a heading 'bringing the citizens and administration closer to the EU', the Commission stated that it will continue to pay close attention to the existence of a professional and functioning PA in line with the focus on basic governance issues. \\
\textbf{OECD/SIGMA team}: There are differences, yes. The last wave of enlargement was driven by a time pressure, which had to do with geopolitical concerns, not with EU-readiness concerns. Time pressure forced people to do things very differently, for example, there was a much greater focus on key risk areas for the internal market. There was a greater focus on sectoral administrative development, and a lesser focus on systemic issues. That is absolutely not a criticism. As for the 8 CEECs, the timing was driven by real valid concerns, which was not the case for Bulgaria and Romania, but unfortunately, the freedom of that not being the case was not used. The Balkans present very different problems; first of all, it is a post-conflict setting. There are large numbers of ethnic and state issues, which are unresolved. Most countries in the Balkans have rather weak states and national (as opposed to ethnic) identities; this was not the case in the CEECs. And the Balkans have weak state institutions with the possible exception of Serbia. So these countries have made very rapid progress, but the institutions of state are still rather weak, and democratic culture and the rule of law culture have not fully been internalized. I do not think the Commissions assistance either in terms of its prioritization or in terms of its delivery mechanisms have been sufficiently adapted to these circumstances. \\
\textbf{EC official, DG ELARG Evaluation Unit team}: Of course, if you look at PAR, the issues are very much the same, but the underlying issues are different. For example, if you have a country like Albania, where the administration is constantly changing when parties in power change, not just in key positions, it is very difficult to operate or start a change process. What I do not see so much in these countries as opposed to the last wave of enlargement are parties that are in opposition of the EI. They do exist, but not in large numbers. So, the main forces in the countries are pro-European. But of course you still have a quite politicized civil service, which is the problem. In the Balkans the general feeling is that the EI-process has to stabilize the region, which in practical terms it is a completely different process than in Eastern Europe. Right now, there is a discussion about looking at these countries not so much already as accession countries, but also as countries in development. And if you read the IPA regulations, it is explicitly stated there that development should be a key part in potential candidates. I don't think that we actually reflected on that enough. We just took instruments like Twinning and Twinning light, TAIEX, etc. and we are just now really adapting them, adapting them fully or revising some of the instruments.\\
\textbf{EC official, DG ELARG Former Yugoslav Republic of Macedonia team}: I think there are a lot of lessons learned. There are similar problems with several countries of the last wave of enlargement. And I think it is due to the lessons learned, why there is this idea of a specialized dialogue on PAR. What we saw before the discussion on PAR was just based on political criteria. Once these political criteria were fulfilled, there was not really a follow up. SIGMA conducted a study on the situation in the countries, which recently joined the EU and their finding is that there was a lot of backsliding in the PA regarding adherence to principles etc. It was quite evident that PA, although it is very important, it is sort of difficult to pinpoint where the boundaries are, so it is not really followed up. With this new approach, what we are trying to do is to have a regular dialogue, where we could really see from one month to another, what really happened. And even when the political criteria are fulfilled and a country received recommendations for opening negotiations, you can still have a place where you can raise issues. In the last wave of enlargement there was no forum to continually look at PA issues after negotiations started. \\
\textbf{EC official, DG ELARG Albania team}: I think in general, yes. This is also a general comment on the political criteria. Of course we have learnt our lessons from the fifth enlargement. We are at an earlier stage addressing certain issues and that includes of course the rule of law, corruption and organized crime. Within this, it is also about governance in these institutions. And also, we do now establish certain targets that need to be reached before we start negotiations. You can see this already in Macedonia, not immediately the opinion, but what was published shortly after. The opinion itself does not give key priorities, but the accession partnership or European partnership do. For Macedonia we have the opinion 2005, and in 2008 we have the updated accession partnership with key priorities that need to be fulfilled before the country can start negotiations. This is a model that could be pursued, which is presently discussed. The philosophy in any case is there. We will want to see the issues addressed at a much earlier stage, even before negotiations start and that could include the priority on public administration.\\
\textbf{EC official, DG ELARG Montenegro team}: There is not really a different approach. But now, PAR is high on the agenda. PAR includes a vast area of topics. It relates not only to the services provided, but also for example to the structures in each ministry. In June 2008, a National Action Plan for Integration (NPI) was designed for implementing the SAA, which is very comprehensive. The NPI will be revised after the opinion on Montenegro will be published. \\
%\newpage
%---------------------------------------------------------------------------------------------------------------------------
\section{The literature on enlargement sometimes argues with the legacy theory, in particular regarding the last wave of enlargement. Meaning that structures of previous regime set-ups have an influence on the present development of Public Administration Reform. What is your view on this issue regarding Albania/Macedonia and Montenegro? }
\label{sec:montenegro1}
\markboth{Anhang \thechapter, Frage Nr. \thesection}{Anhang \thechapter, Frage Nr. \thesection}
\textbf{EC official, DG ELARG PAR Coordination team}: Obviously, there is a common legislative background in all the former Yugoslav countries, when it comes to civil service, which is of course not in line with European standards. And also, unfortunately, when these countries became potential candidate countries, we asked them to reform their legislation, it seems to me that they have been using sometimes experts, who themselves had been brought up under that legislative background. So when it comes to certain legislation of civil service continued to amend those laws in line with those old values, so to say. Obviously those structures or set-ups or values had an influence on the present development. \\
\textbf{OECD/SIGMA team}: The recent Sigma paper No. 44 on civil service reform in the CEECs after accession, does not very much support the legacy theory. But intuitively, the legacy theory must mean something. Probably what the legacy theory does not predict in the CEECs is the differences. But if you take legacy as very basic concept, with these countries starting from a communist system of governance and then switching to a democratic, market oriented, rule of law one, then the legacy theory is an underlying idea, which we have to take into consideration all the time. I think legacy theory in the Balkans is very important, but legacies are different. For example in Serbia and Montenegro, the sanctions regime and the way the states were forced to operate under the sanctions, have an enduring effect. In Albania it is necessary to keep in mind, the harshness of the regime before compared to what was happening in the rest of the Balkans. The rest of the Balkans were relatively open, whereas Albania was totally closed. 16 years on, the legacies are still there in people's mentalities. They are still there in people's understanding of law, both citizens and power elites. And power elites still understand themselves as the architects of law, but not the subjects of law. Now, some of that goes back to pre-communist legacies. That takes you into the area of social, cultural explanations, which is very, very difficult to handle. You have the Austrian/Turkish legacy, we have the communist legacy and we have the post-communist legacy, because after all it is now 20 years after the wall fell. Some of these countries then went through conflict, some of them were under regimes like those of Milosevic and Tudjman, which introduced their own legacies into the system and which stay on in terms of criminal networks and oligarchic arrangements. So, as I said legacies are a very complicated topic. I think you can talk about a sort of substrate, but I think it is very difficult to use legacies for identifying differences. \\
\textbf{EC official, DG ELARG Evaluation Unit team}: There is of course a certain culture in PA and there is the civil service code. But of course what happened after the overthrow of the communist regimes, all of this has been just filed away and it was built up from scratch. A very interesting research question would of course be to compare the old and the new civil service code and see how much of it actually matches. That is a big question how much of the old traditions have carried over into the 'new' institutions. There is a certain mentality. We are mostly dealing with administrations that are stretched to the limit; I am hearing that mostly from our colleagues in the country units. Do not ask them for too many things, because they just do not have this capacity. For example now, we want to organize a training for evaluation and monitoring and just to get the commitment for one training day for maybe 10-12 people, it is almost like shutting down the whole ministry. Especially in Kosovo and Montenegro. Of course we have to take that into account. \\
\textbf{EC official, DG ELARG Former Yugoslav Republic of Macedonia team}:. I think yes. It goes back to the Austro-Hungarian Empire, some of the principles that are embedded in their laws. The law on Administrative Procedures you can trace this back a very long way. The Austro-Hungarian Empire was encompassing countries which were later on transition countries and you could see similar issues or problems in the way to approach things, the heritage, also in the Czech Republic, Slovakia, Hungary and now in the Balkans. While Macedonia never was part of the Austro-Hungarian Empire, it inherited from Yugoslavia, which was heavily based on the older model. So, that is how we can trace the heritage. The country was heavily influenced by the set up of the administration, education and the entire package of Yugoslavia and that is of course why you can see some of the same problems in Croatia, Montenegro and elsewhere in the region. \\
\textbf{EC official, DG ELARG Albania team}: More than a legacy in the structure, there is a legacy in the culture. In Albania probably more so than in any other country. You have a legacy of respect of the highest authority being the only institution that can change things. Although you have that in all ex-communist countries, you have that very strongly still in Albania. So, it is culture more than structures, I would say.\\
\textbf{EC official, DG ELARG Montenegro team}: Until 1989 most of the countries of the last wave of enlargement had central planning. For the Balkan countries the situation is different, as these countries have had the time to start the needed changes. We are much further on in time and also the structure of the state was different than in most of the countries of the last wave of enlargement. In addition, there were wars in the Balkan countries as opposed to the countries of the last wave of enlargement. It is extremely important that these countries talk to each other. Two years ago the DG RTD1 produced a good study based on ethnic research and said among many interesting findings, that we should be careful not to create "ethnocracies".\\
%\newpage
%---------------------------------------------------------------------------------------------------------------------------
\section{How do you assess the cooperation within the EU Commission regarding Public Administration reform in Albania/Macedonia and Montenegro with the different Units, DG Enlargement, country desks, special PAR Unit and DG Admin? }
\label{sec:admin}
\markboth{Anhang \thechapter, Frage Nr. \thesection}{Anhang \thechapter, Frage Nr. \thesection}
\textbf{EC official, DG ELARG PAR Coordination team}: There is very close cooperation within DG Enlargement. There is dedicated country desk for each country and the Delegations in each of the countries. We have the coordination unit both for the political side for producing the annual progress reports, which is unit A1, and unit D1 for the instruments and contracts, which is responsible for correct application of the financial instruments, especially the IPA-Instrument. Take for example financial assistance, there unit D1 regularly organizes meetings, here in Brussels mostly, with the heads of the operational sections in the Delegations. They are constantly kept updated on everything here. There are a number of PAR IPA-projects in each country. IPA projects do not need to be linked to an Acquis chapter; they can also target political criteria. There is a long programming-process, where all the stakeholders, first of all the national authorities themselves, then the Delegations are involved. One important new element in the whole process was when DG Enlargement about three years ago established a so called quality support group (QSG) where drafts of project fiches, which later will be part of the annual national programmes of the countries are discussed quite in detail and are circulated in various units. The aim is to ensure that we plan projects with IPA-support for those areas where we find gaps, where there is a need for reform or a need for institution building. Overall, there are people responsible for financial assistance and others more responsible for the political dialogue. \\
\textbf{OECD/SIGMA team}: We need to make a clear distinction between the political discourse and negotiations on one hand and technical assistance on the other hand, which in my opinion are not always connected, posing something of a problem. The Brussels-based country desks do try to keep the TA part linked to the negotiations. But the TA part tends to be driven by disbursement issues and the Delegations. In DG Enlargement, I think it is fairly tightly connected and to some extent the requirement to produce the regular reports and the multi-annual programming, drives the cooperation process and similarly with the other DGs. With the Delegations, there seem to be two separate issues. One between Delegations and HQ, which will become at least more complex with the arrival of the External Action Service and the other, is the relations in the Delegations between the operational and the political units, where I think coordination could be quite significantly improved. DG Enlargement relies very heavily on external experts and neither DG enlargement nor the Delegations seem to have the technical abilities to steer/control all the technical assistance they are producing. Technical assistance is managed at the administrative contract level and not really at the substance level and the substantive dialogue with countries does not really take place. \\
\textbf{EC official, DG ELARG Evaluation Unit team}: There are several processes and everything we do is cooperation between the units. We want to use the evaluation unit more regarding the question of lessons learned of all the evaluation reports. One question that always is difficult for consultants or our evaluations to answer is on impact and sustainability, based on the five OECD criteria: relevance, efficiency, effectiveness, impact and sustainability. It is quite hard to come to a judgement, if you do not have some sort of a basis. You have to know what was there at the end of a project. Otherwise it is hard to judge what is still there in a year or two years after. With the sector approach there is leverage. You basically ask for a strategy and commitment to certain objectives in the strategy. Things are then formulated out in national programmes and projects. With these projects you can then say, now tell us why you want this project and how does it contribute to your strategy in the Justice sector, let us say. It is all linked up in a logical sequence towards accession. We will most likely have better donor cooperation, more targeted and sequenced funding for assistance and that has of course a large effect for PAR as well.\\
\textbf{EC official, DG ELARG Former Yugoslav Republic of Macedonia}: We have a system of so called chapter desks. For every chapter of the Acquis, like agriculture or fisheries, you have someone in DG Enlargement, who is getting the overview of all countries on that chapter. Somebody will be dealing with Albania, but there is also somebody looking at a chapter in all the countries. This is to make sure there is consistency of approach. We did not have anybody specifically for PAR as it is not a chapter. SIGMA is sub contracted to do work on PA, as we do not have the capacity. For the time being, I am in touch with DG HR, SIGMA, the PAR Coordinator and DG Justice, as DG Justice is the one dealing with aspects of corruption for example. I am in touch with DG Budget on issues of Public Finance and DG Market on public procurement, even with OLAF on issues of anti-fraud. So, because there was and still is no single formal platform on these issues, it has been covered in bits and pieces and other fora. 
EC official, DG ELARG Albania team: Within DG Enlargement there is certainly an increasing attempt to make sense and logic out of this area, which is a horizontal area rather than a specific sector. There is one person in DG ELARG as a sort of horizontal guidance person; also, there has been a working group, so here is some attempt to get the theory right and in this context there has been an increasing cooperation with DG HR. For the upcoming opinion, we have contributions from the line DGs. You have structural funds as well as financial assistance to the "'administrative capacity programme"'. We, together with the Delegations, are interested not only in the Justice sector, but also the reform issues, synergies and good governance.\\
\textbf{EC official, DG ELARG Montenegro team}: There are all sorts of institutional ways to ensure cooperation throughout units, directorates, general directorates, the council, the parliament, as well as with member states (through the IPA committee and other consultations).. If PAR is involved, DG HR is now involved thematically; the chapter desks and the country desks are asked to be active partners in the design of IPA projects (providing comments etc). PAR is not specifically discussed for example in a sub-committee, as it is considered as a horizontal issue.  All assessments, foremost the progress reports are checked for issues that are highlighted as "'in need of progress /efforts or in need of reform"' in order to come up with assistance projects. This is done by my unit, but also the Delegations and the countries themselves. These issues are then discussed by the different stakeholders.\\
%---------------------------------------------------------------------------------------------------------------------------
%\newpage
\section{Public Administration Reform is not a separate chapter in the Acquis. Should it be a separate chapter? }
\label{sec:chapter}
\markboth{Anhang \thechapter, Frage Nr. \thesection}{Anhang \thechapter, Frage Nr. \thesection}
\textbf{EC official, DG ELARG PAR Coordination  team}: When we talk about the Acquis, it is something constantly evolving. Some of these chapters or even the majority are not necessarily based on hard Acquis, EU legislation, EU directives and so on. Some of the chapters appear to be soft in character, meaning that they sometimes refer to international agreements, standards, conventions or treaties issued by other bodies, such as the CoE. The issue of creating a new chapter on PAR is  currently not realistic because there are complex legal and procedural matters and there was also a feeling that it would not be right to add a new chapter as if we would make it more difficult for the new candidate countries compared to the previous ones. Also, there was the argument that perhaps the member states, who would decide on the change in the Acquis might object, because it has at least indirect implications for them. How does it look like if we in a chapter request certain PA reforms, which perhaps are not in place in the MS themselves? But even in the absence of a formal chapter, we can increase the profile of PAR. That means to really discuss it, to conduct a political dialogue with the candidate countries as we do with the chapters. And I think that the fact that we have taken the trouble to list priorities in the Partnership Agreements relating to  PAR and to form indicators, shows that this can be done and it is logical to  enhance the dialogue on PAR.\\
\textbf{OECD/SIGMA team}: The problem is that there is no Acquis regarding PAR, and it is not susceptible to become part of the Acquis in my view. I think PA is far too contextual and social. So one part of the answer is that I do not think you could make it a chapter and my position is to some extent re-enforced by the leading example of what we have been talking about, which is PIFC, which is absolutely not Acquis. It was negotiated into a chapter, now chapter 32. The result in my view was that many of the countries were forced to create systems they could not find models of elsewhere; which were not appropriate or sustainable and which diverted scarce resources into low priority tasks and away from consolidating basic systems. I think it would be far more powerful for the Commission, if it simply relied on the political chapter and ensured that the political chapter was not forgotten about, as soon as negotiations started. At the moment you say, the country meets the political criteria, therefore negotiations can start. Also, I would like to start thinking about outcome measures, for example on administrative reliability. What sort of indicators should we have to measure if administration is acting in a reliable and impartial way? You have for example the analysis of judgements of administrative courts, you have the ombudsman. You could imagine a number of different methods, case based sampling, customer surveys etc. I would like to see a move towards an approach, where we do not say what the inputs are, but what we would like to be the outputs. Also, we should be more concerned about trajectories than about absolutes. And we should be thinking about pathways. However, inputs are easier to objectivise than outputs so both the Commission would be under more pressure to defend judgements rather than "'facts"'. \\
\textbf{EC official, DG ELARG Evaluation Unit team}: Of course it should not be a separate chapter. If there should be a chapter, it should be with the word horizontal in brackets. I think it always comes out again that horizontal PAR is a domestic issue and not something the EC should get too involved in. The Commission should involved in it only as it has large repercussions on the vertical implementation of the Acquis.  \\
\textbf{EC official, DG ELARG Former Yugoslav Republic of Macedonia team}: I think it is difficult to put on a piece of paper or have Acquis to tick off for the countries, so it is difficult to create a chapter. But of course if there was one, it would make things easier to a certain extent it would be less difficult to put your finger on specific things. I think we are going into this direction. We might have a chapter on PAR at some time in the future, but at this point it is difficult to say what would be included into this chapter. \\
\textbf{EC official, DG ELARG Albania team}: No, it is not a separate chapter, but plays a very important role in the political criteria and now the important question is, how much is it an important factor in chapter 23, Justice and Home Affairs. Also other chapters are relevant, like financial control or procurement. But under political criteria we look more at the overarching issues like civil service reform and good governance. Should it be a separate chapter, I don't know. If it would be a separate chapter, it would take out from other chapters and that would not be possible, thus I would say no, but should it be strengthened also in the chapter parts? There, I would say yes. We also have to find guidance and incentives after the opening of negotiations and not stop after we evaluated it. PAR is an overall process and does not stop there. \\
\textbf{EC official, DG ELARG Montenegro team}: There is no Acquis in PAR. But SIGMA is providing each year its assessment on PA in each of the WB countries. These are very helpful to design future assistance projects. There is need for PAR to become a chapter in my point of view.\\
%\newpage
%---------------------------------------------------------------------------------------------------------------------------
\section{What is your take on the Treaty of Lisbon regarding Public Administration reform? Does the Lisbon Treaty lead to a different approach of the EU towards Public Administration Reform in the candidate and potential candidate countries?}
\label{sec:countries1}
\markboth{Anhang \thechapter, Frage Nr. \thesection}{Anhang \thechapter, Frage Nr. \thesection}
OKed by interviewee
\textbf{EC official, DG ELARG PAR Coordination team}: This is an interesting question, because I know that there have been discussions what it means. There were discussions what is the scope of the new paragraphs (art 197 and art 298). Does it really mean that the EU can request candidate countries and new MS to carry out PAR? I do not think so, looking at the text. Art 197 states that the Union may support the efforts of member states to improve their administrative capacity to implement union law but harmonisation of the laws and regulations of the member states is excluded. And art 298 relates more to the Commission itself, and not so much to the administration of MS. In principle, I think it is up to every MS to decide by itself on how to organize its PA. And what paragraph 197 says is more that the Commission can provide support or assistance. To summarize, I do not think this new paragraph opens a door for any radical change when it comes to the approach. \\
\textbf{OECD/SIGMA team}: The interpretation is not very clear. What it means, it seems to me that it limits PA to mean PA to implementing EU policies, which is a lot of the time of course. But I think the EU has become a common law country. And it will be the judgement of the ECJ in the next years to determine what it really means. And principles like "'equal treatment"' will soon force the scope of application to broaden. But there will certainly be an empire built around it and there will be discussions within the Commission on who gets to build the empire. It gives a certain degree of additional legitimacy to the Commission's activities in administrative reform, but not that much more, as it already had been there in the Copenhagen and Madrid councils. So, paragraph 197 as it becomes powerful, will probably have a marginal or higher impact on member states rather than on candidate and potential candidate countries.\\
\textbf{EC official, DG ELARG Evaluation Unit team}: Generally, it is always good to have something in the treaty. There is a larger question at stake. Should we not respect the subsidarity principle regarding candidate countries and I personally think we should. We should not run their countries and we really should not tell the Prime Minister how he has to shape his ministries and how he has to shape his administration. We can only advise them what would be the best way to implement the Acquis, based on the MS experience. Regarding PAR you have no strict limits on how many civil servants you need to implement a certain article of the Acquis. The Acquis never specifies the implementation in detail, while it may specify that you have to have laboratories or border stations. In negotiations, you can ask for certain things, but the negotiator can not sit down and say I want 50 people in this part of the administration. He will probably say, I want you to write this law and I want you to be able to implement it. And then of course our assistance programmes complement the negotiations and Twinning often derives from negotiations. The negotiator might say we provisionally close this chapter, but I want you to have a Twinning on Social Dialogue, for example.
EC official, DG ELARG Former Yugoslav Republic of Macedonia team: I do not think it will have any immediate effect on our work in the enlargement and accession process. There is some kind of work methodology established. But it is good that PAR is noted, that there is a stress on that. This might help with the emphasis we put on the subject. But I do not see any immediate effect. 
EC official, DG ELARG Albania team: I would have to consult the treaty to form an opinion. 
EC official, DG ELARG Montenegro team: Question not answered.\\
%\newpage
%---------------------------------------------------------------------------------------------------------------------------
\section{How do you asses the EU-Instruments to promote Public Administration Reform in
Albania/Macedonia and Montenegro as regards quantity and effectiveness: Differentiate per country, if possible CARDS (phased out) Twinning Twinning light TAIEX IPA Did I forget to mention an instrument that is relevant? }
\label{sec:relevant}
\markboth{Anhang \thechapter, Frage Nr. \thesection}{Anhang \thechapter, Frage Nr. \thesection}
\textbf{EC official, DG ELARG PAR Coordination team}: Twinning and TAIEX were from the start, intended for Acquis-specific issues regarding institution building and not so much for PAR. I would think that the number of activities under Twinning and TAIEX related to PAR are relatively few. I would add two other aspects here: Which are the right instruments to use for certain types of institution building? In which order should they be used? I think that the TAIEX instrument, which is one of the instruments to finance seminars, study trips and expert visits for a few days, is an instrument that could be used so to say to prepare more large scale projects, like assistance, the same with some of SIGMA's technical assistance. So, in principle all the instruments complement each other. Regarding Twinning, what happens very often is that we do not so much transfer common European standards, but that in practice specific MS send Twinners to a candidate country and they are transferring the models in their own countries. But sometimes these models conform to good or even best European standard. For example in external audit, there is support to build up capacity of Supreme Audit Institutions, and there I got the impression that two countries are more involved than others, namely Sweden and the UK.The SAIs in these two countries have a  good reputation.\\
\textbf{OECD/SIGMA team}: I think you are confusing a financing instrument with a delivery instrument. CARDS and IPA are financing instruments. Twinning, Twining light and you should add Technical Assistance and SIGMA, are delivery instruments. EU regulations govern these instruments and determine the efficiency of delivery. For example the time delays and the programming systems, have something of a deleterious effect on the delivery quality of the instruments. In the area of instruments you have Twinning and Twinning light, you now have to add new ones, which is budget support and sector support. So, there is a large spectrum of instruments with varying degrees of effectiveness. I think that Twinning and Twinning light are useful, certainly useful where there is Acquis, and when there is stable political and institutional environment, i.e. if it is a technical issue. Then everybody knows what to do, you only have to put into place the organization of it. When it is non technical and political and or when there is an unstable political context and politically sensitive environment, I do not think they are appropriate and have not been very successful in our areas. Traditional instruments then fall back on trainings, with the exception of some of the financial issues. TAIEX is ok, but it does not have an institutionalized memory, which is important in this area of work.
EC official, DG ELARG Evaluation Unit: There are also Technical Assistance projects and quite often they are combined. Generally you would look at a country and decide what kind of process you want to put in place, what kind of changes you want to see. The next question of course is, how do you achieve this. There the instruments come into place. TAIEX is something that we can organize fairly quickly, these are short missions, up to two weeks, maybe to help draft legislation or bringing the expertise through workshops etc. Twinning is our main instrument to deliver institution building. When Twinning was introduced under CARDS, it was perceived as an accession instrument and there was a lot of interest of the new member states as they had just completed their accession to help others. It is now changing, as Twinning is a quite heavily engaging process and you need a lot of resources for it. This has created something of a step back from Twinning.\\ 
\textbf{EC official, DG ELARG Former Yugoslav Republic of Macedonia team}: They are popular in the country, but it is difficult to compare as they are a bit different in nature. TA are classical projects more expensive and more long term and TAIEX was designed to be user friendly and it is very much used, but it is difficult to asses what the impact directly is. It brings people together from the country to Brussels for example to meet experts or to member states; you can organize short workshops and seminars. So, I think they had their own contribution to the process and they contribute to a better understanding what the Acquis is and it also helps, especially TAIEX; for people to be exposed to the EU way of dealing with things, which is useful. Of course IPA are bigger scale and longer term projects. And for a Twinner, it would mean for example helping to draft a law or streamline the structure of a unit. This Twinner has to respond to the needs that have been identified, in the progress reports. It is not that they can come and do whatever they want to do. The MS is financing it and it is then up to the MS, but as they are so much involved in the assessment of the progress, they would very much follow the same interest in what needs to be done.  \\
\textbf{EC official, DG ELARG Albania team}: IPA finances a number of instruments, including Twinning and Twinning light. It also finances SIGMA. You can differentiate between long term and short term instruments. TAIEX is really a short term instrument, to fill the gaps, conferences, study visits, expert missions. IPA is more for long term projects, also for PAR. In a broader sense, Twinning is the best instrument because it gives you direct experience from the member state administrations to apply to candidate country administration. We all very much like Twinning. Regarding TA, it is a good question, to what degree do you need TA in the area of PA? I would say Twinning is more conducive, but then again SIGMA is more analytical.\\
\textbf{EC official, DG ELARG Montenegro team}: Twinning is not really an instrument for PAR, nor is TAIEX: They are more closely related to the Acquis. IPA is a good programme for TA to implement projects of the national programme (component I). But for small countries large and complex projects pose an absorption problem. Thus, within the national programme for Montenegro 5\% of the IPA budget are kept for small ad hoc projects.\\
%\newpage
%---------------------------------------------------------------------------------------------------------------------------
\section{Are these programmes well designed for the needs of PAR in Albania/Macedonia and Montenegro or do you perceive a need for adjustment in any of them? (Content or technical) }
\label{sec:technical}
\markboth{Anhang \thechapter, Frage Nr. \thesection}{Anhang \thechapter, Frage Nr. \thesection}
\textbf{EC official, DG ELARG PAR Coordination team}: I can imagine that generally speaking in the countries that the procedures are felt to be cumbersome or bureaucratic. But on the other hand, the Commission has a constant dialogue on improving the programmes and trying to reduce red tape or procedures, which result in delays and so on. The IPA-Regulation, when it entered into force in 2007, it was supposed to streamline previous regulations on financial assistance to candidate countries. And there were a lot of discussions on how to do that. Generally speaking, I think that our new rules and programmes are better designed, not only for the needs of PAR, but for all assistance needs in the enlargement process. There is perhaps one area, where things have to be improved and that is the area of donor coordination. Which is also, by the way, mentioned in the IPA regulation. We are very much aware of the latest Developments in that respect, the Paris Declaration on Aid Effectiveness and how to value that in our context of EU-Integration. To give you one example, DG Enlargement has initiated a discussion and process on how to apply a sector approach or programme based approach, and also whether assistance to PAR fits to that approach. The easiest you can apply the sector approach to is to technical sectors like education or transport, where we usually only have one main stakeholder, one ministry in charge. It is a bit different with PAR, which is a kind of cross-cutting sector, making it more difficult. The discussion now is that at least we could select PAR in one or two countries as a sort of pilot sector for this approach. If we do that, we can learn a lot to design assistance to PAR better to the needs of the country and also to avoid a kind of overlap with other donors. \\
\textbf{OECD/SIGMA team}: I think SIGMA needs to have some re-design as well, but it probably is closest to what should be done. Partly, because we have been abstracted from the Commission's rules and regulations, in terms of our operations. The Commission gives us financing for staff and operations as a sort of institutional contract for a period between two and three years. For all operations, for hiring consultants, missions etc., we operate under OECD rules, not EU rules. Maybe it takes two weeks to get consultants contracted, but if it is really urgent, you can do it quicker. Our responsiveness is determined by the production system, not by the approval system. Our staff are themselves experts and they are responsible for work in particular countries in their area and they remain responsible whether or not there is an operational activity and they are ready to alert us if something comes up. Another thing that sets SIGMA apart is to have that continuity. We are geographically and substantively concentrated, meaning specialized. And we try to resist any extension either geographically or substantively. Substantively to the margins where we think it is still relevant to governance. We also use a network of other PA experts, who are practitioners from MS.\\
\textbf{EC official, DG ELARG Evaluation Unit team}: I am not sure if we revised the instruments in the specific needs of these countries, but IPA is by definition quite a flexible instrument.\\
\textbf{EC official, DG ELARG Former Yugoslav Republic of Macedonia team}: Sometimes, the problem with projects is that they are conceived and then it takes a long time before they are implemented and sometimes the situation changes in the meantime. The problem could be from both sides, the national authorities being slow in preparing a project and sometimes it is also from our side. Now we are implementing IPA 2007 projects in Macedonia, which have been prepared even before that. Sometimes what we felt were the priorities then, are no priority any longer. This could be a problem, but not necessarily everywhere. We will also see what happens with the new sectoral approach, which might make it a bit easier to follow.\\
\textbf{EC official, DG ELARG Albania team}: Ideally it should be perfectly merged. Everybody on the political side should know what is going on in regards to financing and the other way around. We try to work that way that we agree on the analysis and then we agree on the priorities. Together with the colleagues who are dealing with financial assistance, we develop the forward looking strategy, the MIPD. This we have now done, also after our workshop with SIGMA.\\
\textbf{EC official, DG ELARG Montenegro team}: It depends on what you want to achieve. CARDS was a different tool, more geared towards reconstruction and infrastructures. It evolved and progressively included PAR in the programmes.  In general, IPA is an adequate tool, which can be adjusted to the needs and much appreciated.\\
%\newpage
%---------------------------------------------------------------------------------------------------------------------------
\section{ In your opinion, are there obstacles to PAR in Albania/Macedonia and Montenegro? And what would be necessary for successful PAR in Albania/Macedonia and Montenegro? }
\label{sec:montenegro2}
\markboth{Anhang \thechapter, Frage Nr. \thesection}{Anhang \thechapter, Frage Nr. \thesection}
\textbf{EC official, DG ELARG PAR Coordination team}: Talking about general obstacles, there is of course a lack of administrative capacity, lack of political will to carry out reforms and often a high corruption in those countries. SIGMA in their reports also talks about a general lack of respect for the law. In all these countries you have of course an outdated PA, which is much politicized. When it comes to more specific reasons, I can imagine that in a country like Macedonia, you have a very specific problem with its Albanian minority. And there you have the Orhid Framework, stipulating that minorities have to be represented in the PA. In Montenegro, although you have a multi-party system, the country has been governed by more or less the same party for many years. And because the country is so small there are very close links between the political and economic elites, which could give rise to what we call state capture, the most serious form of corruption. . Albania, also a small country has a strong historical legacy and everything is very much politicized, , with limited stability in the PA. We have started to discuss how to implement PAR in small countries There has to be domestic support and demand for reform. It means that one way to promote the reform process in these countries is to engage civil society more and the public in general. And we, the Commission and DG Enlargement have to be very clear on our requirements in our political dialogue and in reporting. \\
\textbf{OECD/SIGMA team}: I think the basic obstacle is that the people do not want it in the countries themselves. It is also a question of supply and demand. PAR is basically supply driven. The typical approach is that a project will provide lots of professional civil servants and therefore there will be a demand for them. Regarding service delivery, clients want, it probably. But even there I am not too sure, as they probably do not know what it means. They never had it, so they do not know what it means. Point two is that I do not think that we adapt our notion about service delivery to the basic problems in these countries. For example, one of the typical ideas about service delivery is about turnaround for decision making. Time is money, business need decisions quickly. If you are in a situation, where you have no legal predictability, no reliability about implementation of the law and administrative decision making, maybe getting the wrong decision quicker is actually not what you want; you may be happier to wait to get the "'right"' decision (i.e. the legal one). There may be demand from society, but whether political, administrative and business elites are interested in PAR, I am not so sure. I think you can not reform PA by PAR. We should look at new ways of dealing with it and in certain countries think about consolidating the basic functions of the state, which may require changing the PA. I do not think PAR is treated sufficiently politically and it is the political economy of PAR that is missing. It is treated as a technical issue, which it is not. \\
\textbf{EC official, DG ELARG Evaluation Unit team}: Largely politicised PAs are an obstacle. This will only change when the countries realize that they need a professional civil service, detached to some extent from what is going on politically. Positions are changed after elections, which is a huge obstacle to us and the brain drain related to that actually means, that there is no institutional memory. Also, civil servants are not paid enough and they go to the private sector. Corruption is still a big issue in most of these countries. But it is not only people changing, also procedures change and are in constant flux. And maybe something that has been developed via an EU programme over the years is then pushed off the table. For successful PAR, the milestones in the accession process are quite important. The move from potential candidate to candidate status changes dynamics in a country. Then a country has to get ready to steering other EU funds. You can only get a larger share if you have the right institutions to gear up for it. Also, institutions leaning more towards democracy in the way the three powers interact are needed and this is not fully the case in the three countries you are researching. \\
\textbf{EC official, DG ELARG Former Yugoslav Republic of Macedonia team}: In these countries and also in Macedonia, the concept of independence of PA needs to be accepted; that the public administration is not there to implement the ruling party's ideas and plans. PA is there as a service, which should be working independently. Of course the independence is not as in the judiciary, it is a different kind of independence. You would still have to follow instructions from the ministry, but this concept of a-political and service-oriented PA is really new and that is why we are stumbling with implementation, because even where there are good laws, without an understanding what it means to be non-political and service-oriented, the implementation is not there. We do what we can, we finance projects. It is a very long term process; it is not even finished in some of the member states.\\ 
\textbf{EC official, DG ELARG Albania team}: We are looking into the area of civil service, there is a reform ongoing and we follow SIGMAs assessment in terms of the gaps: Depolitization, merit based and transparent appointments. Stability of the institutions is not the case where position based appointments and politization create instability and staff turnover. In the area of Decentralization, there have been quite a number of measures undertaken, but these have got stuck. For strategic planning, there is a very sophisticated system in place in Albania actually, which needs to be implemented. Anti-corruption is a big issue. A lot of things have been done in Albania also in the context of the visa liberalization. Now it ist important that all systems in place work and are implemented, such as pro-activeness in investigations and prosecutions on all levels. The key issue Shere is impunity. An obstacle certainly is culture. The strong sense of authority of the highest person Sand non-transparency has to change.\\
\textbf{EC official, DG ELARG Montenegro team}: More time is needed, as the change of culture in PA is a long term process. Also, more English speaking personnel in the national PA would help. \\
%\newpage
%---------------------------------------------------------------------------------------------------------------------------
\section{Which other institutions/organizations or bilateral donors are important in regards to PAR in Albania/Macedonia and Montenegro? How do you asses their impact on PAR in the three countries?}
\label{sec:countries2}
\markboth{Anhang \thechapter, Frage Nr. \thesection}{Anhang \thechapter, Frage Nr. \thesection}
\textbf{EC official, DG ELARG PAR Coordination team}: I would need some time to compile details on this issue. \\
\textbf{OECD/SIGMA team}: World Bank, UNDP, some bilateral donors, particularly the US, Austria, less than before the UK. Also, GTZ is very present, but often as an implementer, rather than as a bilateral donor, the Dutch to some extent, in particular regarding some financial issues, and the Norwegians. Problematic is a strong project mentality, which is a whole larger issue, but I will leave you with our phrase, which is that Technical Assistance and PA should be driven by a service model, not by a production model. Most technical assistance is driven as a production system and all the technology of managing technical assistance, especially log frames comes out of engineering and was related to physical projects, which is somewhat distorting. The basic problem is that donor accountability systems are counterproductive to effective delivery of TA. And PA TA design is technocratically conceived and does not correspond to political reality or the constraints of complexity.\\
\textbf{EC official, DG ELARG Evaluation Unit team}: All donors are important and of course most important is that you combine your forces. In many areas presently there is a doubling up of assistance, which in itself is not a problem, but might be an efficiency problem. It is a problem of course, if you pull into different directions. That has happened in some areas and is almost unavoidable. \\
\textbf{EC official, DG ELARG Former Yugoslav Republic of Macedonia team}: This information I am expecting from the government in the special group on PAR. It is sometimes difficult for us to know who is bilaterally dealing with PAR. We know that there is the British government involved in some training projects on PAR. There should be somebody in the government responsible for donor coordination and we have requested an overview. \\
\textbf{EC official, DG ELARG Albania team}: My main source of information is the SIGMA reports on PAR. I remember UNDP and WB being quite active in Turkey, and the same seems to be true for Albania.\\
\textbf{EC official, DG ELARG Montenegro team}: UNDP and multilateral donors are often engaged in sub-sectors, but not often in PAR. UNDP also targets the municipality / local self-government level.\\
%\newpage
%---------------------------------------------------------------------------------------------------------------------------
\section{Who is responisble for co-ordinating the PAR activities of all the different donors in Albania, Macedonia and Montenegro and what is happening in this respect at the moment? }
\label{sec:moment}
\markboth{Anhang \thechapter, Frage Nr. \thesection}{Anhang \thechapter, Frage Nr. \thesection}
\textbf{EC official, DG ELARG PAR Coordination team}: In principle, it should be the country itself, the government, to coordinate the assistance from all the different donors, in line with the ownership principle of the Paris Declaration on Aid Effectiveness. There are different mechanisms in place in all countries for coordinating assistance including also assistance to PAR. Albania for example put in place a new fast tracking mechanism and I think there are similar mechanisms in the two other countries. We in DG Enlargement carried out an evaluation on donor coordination in 2008 and there we have descriptions on the issue in all three countries. Of course since then things have developed and have been improved. \\
\textbf{OECD/SIGMA team}: Who should be responsible, is the country, backed by Europe. The sector approach is supposed to provide greater country ownership over donors. I think the countries rightly suspect that it will imply greater donor ownership over the countries. I think as long as you have accountability arrangements in the donor community, which actually act against effectiveness, you will never have successful donor coordination, or even probably successful projects in PAR, because the requirements are just too hard to fit into the engineering type of contractual framework that the
donor accountability inputs.\\
\textbf{EC official, DG ELARG Evaluation Unit team}: The beneficiary-countries and the Delegations have a good overview. The beneficiary countries all have their coordination units and the Delegation cooperates quite closely or screens this. Putting the sector-wide approach in place will help. While the EU does not focus on horizontal PAR, while other donors like the WB do. There is not really an EU-instrument for PAR. We only concentrate on PAR when there is institutional instability. There is a certain share of responsibility between donors. The WB is moving in certain areas and we do not interfere much with that. The donors are very often attending our internal coordination meetings of the Delegations. And even for the evaluations we are doing now, we invited the donors. Very often the donors also go to the Delegations and rely on information they receive from there. Also, all our documents are on the internet. If donors want to coordinate with us, that is always possible. Twinning really is an instrument for vertical reforms, not horizontal ones. The classical Twinning is for example to train people how to run a border station or in a Ministriy on financial control, which you need to implement regional funds. What we have for horizontal PAR is our SIGMA programme. We do not guide countries towards certain horizontal reforms in detail. There is not lack of knowledge on what all the donors are doing, but it is deliberate to keep out of horizontal issues, as you get involved in politics. With the sector-approach, we are trying to streamline the assistance.\\ 
\textbf{EC official, DG ELARG Former Yugoslav Republic of  Macedonia team}: Reference made to the answer to question 11.\\
\textbf{EC official, DG ELARG Albania team}: GTZ is directly working with the Department of Public Administration (DOPA). GTZ can be seen as bilateral donor, but could also be seen as contractor. Albania is a model regarding donor coordination. They have a whole system in place, which is located in the Council of Ministers with a department that directly reports to the Prime Minister. It is a parallel department to the Strategic Planning Department and they are doing donor coordination. Donor coordination is very important in the context of our assistance and Albania is singled out as good example in this respect.\\
\textbf{EC official, DG ELARG Montenegro team}: The government has recently nominated a person to carry out the function of donor coordination in the Prime Minister's Office. This was long due and hopefully will make a change. It is of utmost importance to better coordinate donors, as they do not always share their views. Some even think that they are competing! An example for good donor coordination at project level in Montenegro is taking place between the EU and the World Bank on agricultural issues. Also with Kreditanstalt für Wiederaufbau (KFW), EU coordination is good.\\
%\newpage
%---------------------------------------------------------------------------------------------------------------------------
\section{Should the EU have additional or other priorities in future PAR programming in Albania, Macedonia and Montenegro?}
\label{sec:montenegro3}
\markboth{Anhang \thechapter, Frage Nr. \thesection}{Anhang \thechapter, Frage Nr. \thesection}
\textbf{EC official, DG ELARG PAR Coordination team}: That depends on  how we define the scope of PAR and  on the possible gaps or problems in these countries. One has first to agree on the priorities and then decide whether these priorities should be supported by assistance. Generally speaking, I would say that our listing  of priorities connected with PAR in the partnership agreements is not uniform, and  not based on a common understanding of PAR or its scope.. It should be added that there is one important organizational tool within DG Elarg, which we call our Matrix system. You have staff in the  country desks, who are horizontally responsible for each chapter. They play an important role in negotiations and in providing input to progress reports and opinions and they also support preparations for sub-committee meetings. The PAR coordinator could play a similar role with regard to PAR  although there is no Acquis chapter for PAR.\\
\textbf{OECD/SIGMA team}: I think the EU should be much more concerned about administrative justice than they are. They are of course concerned about the penal aspects, but they should be much more concerned about administrative justice decision making and financial issues. Also, the EU should focus on some of the governance issues, including on the incentives for individuals and MPs and the capacities of Parliament. And the EU should think about PAR as a support to policy, rather than as PAR as a policy in its own right, because I don't think that approach is very effective.\\
\textbf{EC official, DG ELARG Evaluation Unit team}: Given the amount of funds, we can not tackle all the issues at once. Kosovo is a good example, where we came up with a very narrow list of sectors, three or four. I think it makes sense to have a sequenced approach, as with the sectors. Justice and Home Affairs is a very important sector in that respect. If you tackle brain drain and corruption and all of the main issues and have these out of the way, the other sectors should be easier to reform as well.
\textbf{EC official, DG ELARG Former Yugoslav Republic of Macedonia team}: I think we are fine, we identified what our priorities are, so I do not think there will be any new revealing discoveries, what would be the core of the problem. I think that if we stick to these principles of independent, non-political and service oriented PA, this leads us of course to questions of recruitment and career. I think, if we make progress in this part, we do not have to look anywhere else.\\
\textbf{EC official, DG ELARG,Albania team}: We have established the MIPD. As soon as it is adopted, you will find the priorities. \\
\textbf{EC official, DG ELARG Montenegro team}: The Europe 2020 strategy as well as  the enlargement strategy, are to be taken into account while designing the future programmes. IPA programme should align to these.  This results in topics, such as competitiveness and climate change being high on the agenda in IPA programming. While IPA is not an instrument for the private sector (the private sector is better dealt with by the EBRD), it can certainly participate to climate change for example: for example there is an emphasis on railways and not on roads, which is in line with Europe 2020. There is continuous discussion, development and adjustment of projects with communication lines between the national government, the Delegation and the Commission. The EU knows what the country needs and the country knows this as well. \\
%\newpage
%-------------------------------------------------------------------------------------------------------------------
\section{Is there anything else that is important in the context of my research that you would like to comment on? }
\label{sec:comment on}
\markboth{Anhang \thechapter, Frage Nr. \thesection}{Anhang \thechapter, Frage Nr. \thesection}
\textbf{EC official, DG ELARG PAR Coordination team}: No time was left to ask this question.
\textbf{OECD/SIGMA team}: I think your questions are relatively light on substance and I think it is very difficult to discus. If you took out the word "'PAR"' and replaced it with "'environmental policy"', would your questions still make sense? I suspect yes. But I think PAR has some very specific characteristics. I think that you have to understand the nature of PAR; we are now trying to call it PGR, Public Governance Reform, as being distinct from other sectors. Perhaps what I touched upon is the political economy as important to look at. I think we have discussed some aspects in respect to service delivery, but I do not think this captures the political economy.(Reference to Merilee Grindle, the "'good enough governance"' debate and Sue Unwin the debate on "'drivers of reform"' as well as donor interest in political economy issues). I think PAR could benefit enormously from such thinking, why things work and why things don't work and what drivers you could pursue in order to pursue PAR, for example business interests, although business interests in many countries it turns out that they are not so forceful, because of the oligarchization, but that is the sort of discussion. I think we are implicitly asking countries to go far too quickly and to go too far. Societies are not ready for that sort of adaptation. And I think that one issue that needs to be addressed is the economics, especially in light of the economic crisis. Many of the things that donors are trying to push on them, actually cost too much for them and are not tested. But there is a larger issue, which is (it was the same for the CEECs) that these countries are all poor and have poor state resources. We are asking them to put in place PAs and laws, which are designed for the rich Northern Europeans, but they do not have the tax base to finance that. So we are creating implementation gaps. The second point on that is that our laws, our institutions and our economies developed organically. We are now talking to countries with weak, poor states and asking them to put in place laws for which they do not have the economic or institutional support and so again we are driving implementation gaps. And as a result, what you see is legal formalism. They will produce things according to their perceptions of what we want, with very little sense of ownership or intention to implement. And one last thing that I can think of is that you have three countries that are all small and Montenegro is tiny. Smallness has absolute limits that is to say, even if these countries were rich and did not have the financing gap, you would still find it difficult in Montenegro to develop all the implementation instruments that we require. That is one of the reasons we are doing some analysis on the EI in small states and rationalization of requirements.\\ 
\textbf{EC official, DG ELARG Evaluation Unit team}: I think one issue that is very important in the context of the Balkans, and I often feel like on an island in that respect is the development and reconstruction issue. It never gets mentioned by people, except those who are around longer. When the EAR was dissolved, in many people's heads this was the end of reconstruction in these countries. But having visited the countries and having seen the attitude in these countries, I feel this should not be disregarded. I do not think we have the stability in regards to constitutional reforms and the political party set up. I think there is an underlying element of instability.\\ 
\textbf{EC official, DG ELARG Former Yugoslav Republic of Macedonia team}: What we discussed is actually the core of the problem. I think it would be useful to you to speak to people in the relevant units how the project support is done, to understand the process better.\\
\textbf{EC official, DG ELARG Albania team}: Nothing that comes to my mind immediately.\\
\textbf{EC official, DG ELARG, Montenegro team}: Regarding PAR, the most important issue is the civil service, and some ministries are interested, such as those in charge of Police, Human Resources and Social issues. On other issues like Health, there is not much Acquis and the EU needs to make sure that other donors do complement EU funding, which is more towards Acquis issues. \\

\pagestyle{headings}
\include{appendixC}
\pagestyle{myheadings} 
\chapter{Questionnaire Public Administration Reform experts in Albania, FYROM, Montenegro}

1. Which main topics/areas in the context of Public Administration Reform are presently dealt with as a priority by Albania/Montenegro/FYROM?

2. Is the institution/structure dealing with PAR adequate for the tasks ahead? 

3. In your opinion, are there obstacles to PAR in Albania/Macedonia and Montenegro? And what would be necessary for successful PAR in Albania/Macedonia and Montenegro? 

4. Public Administration Reform is not a separate chapter in the Acquis. Should it be a separate chapter? 

5. How do you assess the cooperation within the EU regarding Public Administration Reform in Albania/Macedonia and Montenegro 

6. Do you think the EU approach regarding Public Administration Reform in Albania, Macedonia and Montenegro is adequate? Or should other aspects be included from your point of view?

7. What is your opinion, how does the new IPA instrument work for PAR in Albania, Montenegro and Macedonia? Examples?

8. Is this programme well designed for the needs of PAR in Albania/Macedonia and Montenegro or do you perceive a need for adjustment ? (Content or technical)

9. Which other institutions/organizations or bilateral donors apart from the EU are important in regards to PAR in Albania/Macedonia and Montenegro? How do you asses their impact on PAR? 

10. Who is responsible for co-ordinating the PAR activities of all the different donors in Albania, Macedonia and Montenegro and what is happening in this respect at the moment? 

11. Should the EU have additional or other priorities in future PAR programming in Albania, Macedonia and Montenegro. 

12. Is there anything else that is important in the context of my research that you would like to comment on?
\include{appendixE}
\pagestyle{plain}
\begin{landscape}

\setlist{nolistsep}
\chapter{Vier zentrale politisch-administrative Traditionen}	
\begin{table}[!hbt]\tiny
\begin{tabular}{|p{4cm}|p{5cm}|p{5cm}|p{5cm}|p{4cm}|}\hline
\vspace{3mm}
&\textbf {\normalsize Deutsch}&	\textbf{\normalsize Französisch}&\textbf {\normalsize Angelsächsisch}&	\textbf{\normalsize Skandinavisch}\\\hline
Verhältnis Staat-Gesellschaft&	organisch&	antagonistisch&	pluralistisch&	organisch\\\hline
Politische Organisation&föderalistisch&zentralistisch	&begrenzt föderalistisch&dezentralisiert, unitaristisch\\\hline
Politikstil&	legalistisch	&korporatistisch, legalistisch&	inkrementell&	konsenuell, technokratisch\\\hline
Dezentrale Elemente&	kooperativer Föderalismus&	regionalisierter Einheitsstaat&	State Power (US), Local Government (UK)&	starke lokale Autonomie\\\hline
Vorherrschende Sichtwiese auf öffentliche Verwaltung&	Öffentliches Recht	&Öffentliches Recht&	Politische Wissenschaft/ Soziologie	&Öffentliches Recht (SWE), Organisationstheorie (NO)\\\hline
Historische Dimension&	Preussische Tradition	&Napoleonische Tradition&	Civic culture Tradition&	Wohlfahrtsstaatsmodell\\\hline
Legale Basis der öffentlichen Verwaltung
&
\begin{itemize}[leftmargin=*]
\item gesonderte Gesetze zu öfentlichem Dienst
\item Verfassungsstatus des öffentlichen Dienstes
\end{itemize}
&
\begin{itemize}[leftmargin=*]
\item gesonderte Gesetze zu öfentlichem Dienst            
\item negative Definition öffentlicher Verwaltung 
\end{itemize}
&
 \vspace{-3mm}
\begin{itemize}[leftmargin=*]
\item gesonderte Gesetze zu öfentlichem Dienst
\item keine Verankerung des öffentlichen Dienstes in der Verfassung
\item Rolle der öffentlichen Verwaltung eher in dienenderer Tradition als in Kontinentaleuropa
 \vspace{-3mm}
 \end{itemize}

&
\begin{itemize}[leftmargin=*]
\item gesonderte Gesetze zu öfentlichem Dienst            
\item Mischung aus Deutschem und Angelsächsischem Modell
\end{itemize}\\\hline
Grad der Zentralisierung&
\begin{itemize}[leftmargin=*]
\item vertikale und horizontale Fragmentierung
\item administrative Dezentralisierung
\item hierarchische Strukturen
\end{itemize}
&
\begin{itemize}[leftmargin=*]
\item unitaristische und stark zentralistische Regierung und öffentliche Verwaltung
\item hierarchische Strukturen
\end{itemize}
 &
\begin{itemize}[leftmargin=*]
\item unitaristische und zentralistische politisch-administrative Strukturen
\item wenig hierarchische Strukturen 	
\end{itemize}
&
Mischung aus Deutschem und Angelsächsischem Modell\\\hline
Koordination innerhalb der öffentlichen Verwaltung	&inter-minsterielle Koordination	&begrenzte inter-ministerielle Koordination&	inter-minsterielle Koordination&	Mischung aus Deutscher und Angelsächsischer inter-ministerieller Koordination\\\hline
Administrativer Rahmen	&
 \vspace{-3mm}
\begin{itemize}[leftmargin=*]
\item einheitlicher administrativer Rahmen auf allen Ebenen
\item föderaler Rahmen mit regional und kommunalverwaltung
\item unterschiedliche Sub-Verwaltungen mit eigenen Kompetenzen
\item vertikale Verteilung von Kompetenzen zwischen den verschiedenen föderalen Ebenen
 \vspace{-3mm}\end{itemize}

&
\begin{itemize}[leftmargin=*]
\item einheitlicher administrativer Rahmen
\item administrative Untereinheiten direkten Weisungen der Zentralregierung unterstellt
\item strikt zentralistische Orientierung
\end{itemize}
&
\begin{itemize}[leftmargin=*]
\item weitgehend autnomome Exekutiv Organe
\item untergeordnete administrative Einheiten mit eingeschränkter finanzieller Autonomie           
\end{itemize}
&
\begin{itemize}[leftmargin=*]
\item zentralistischer Aufbau
\item Mischung aus Deutschem und Angelsächsischem Modell
\end{itemize}\\\hline
Verhältnis von Politik und öffentlicher Verwaltung&	Trennung von öffentlicher Verwaltung und Politik&
\begin{itemize}[leftmargin=*]
\item Trennung von öffentlicher Verwaltung und Politik
\item enge Beziehungen zwischen Politikern und Verwaltern
\end{itemize}
&
 \vspace{-3mm}
\begin{itemize}[leftmargin=*]
\item civic culture und individualistische Tradition
\item Trennung von öffentlicher Verwaltung und Politik
\item Werte des politischen Systems bestimmen auch die öffentliche Verwaltung
 \vspace{-3mm}
 \end{itemize}

&	Mischung aus Deutschem und Angelsächsischem Modell\\\hline
Personalpolitik und Rekrutierung&
\begin{itemize}[leftmargin=*]
\item Primat von Universitätsausbildung im höheren Dienst
\item Hauptsächlich Juristen
\item Beamte sind der personifizierte Staat
\item Ernennung aufgrund von Qualifikation und Leistung, mit begrenzten politischen Ernennungen (höhere Positionen)
\item Lebenszeit und Staatsbediensteten auf Vertragsbasis	
\end{itemize}&
 \vspace{-3mm}
\begin{itemize}[leftmargin=*]
\item vorwiegend administrative Elite
\item hauptsächlich Juristen, aber auch Generalisten
\item homogene mentale und kognitive Übereinstimmung der Bediensteten in der Verwaltung                             \item Rekrutierung vor allem aus spezialisierten Verwaltungsschulen
\item Ernennung aufgrund von Qualifikation und Leistung, mit begrenzten politischen Ernennungen (höhere Positionen)
\item Karriereorientierung
 \vspace{-3mm}
\end{itemize}
&
\begin{itemize}[leftmargin=*]
\item kein Einfluss von Politikern auf Beförderung
\item Universitätsausbildung für höhere Positionen
\item  vorwiegend Generalisten
\item Bestimmte Universitäten bei der Rekrutierung bevorzugt
\item Karrieresystem
\end{itemize}&	Mischung aus Deutschem und Angelsächsischem Modell\\\hline
Länder&	Deutschland, Österreich, Niederlande, Spanien (nach 1978), Belgien (nach 1988)&	Frankreich, Italien, Spanien (bis 1978), Portugal, Griechenland, Belgien (bis 1988) 	&UK, US, Irland&	Schweden, Norwegen, Dänemark\\\hline
\end{tabular}
\end{table}	\footnote{nach Loughlin (1994) aus: Lippert/Umbach (2005), S. 65ff.}
\end{landscape}	



% % Dokumentenende
\end{document}