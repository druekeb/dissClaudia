\chapter{Questionnaire EU-officials, enlargement experts in the area of Public Administration Reform }
1. Which topics/areas are presently dealt with as a priority by the EU regarding Public Administration Reform in Albania/Macedonia and Montenegro? What are the developments you see there?

2. Do you think the EU approach regarding Public Administration Reform in Albania, Macedonia and Montenegro is adequate? Or should other aspects be included from your point of view?

3. Do you perceive differences in the EU approach compared with the experience with PAR during the last wave of enlargement?

4. The literature on enlargement sometimes argues with the legacy theory, in particular regarding the last wave of enlargement. Meaning that structures of previous regime set-ups have an influence on the present development of Public Administration Reform. What is your view on this issue regarding Albania/Macedonia and Montenegro?

5. How do you assess the cooperation within the EU Commission regarding Public Administration reform in Albania/Macedonia and Montenegro with the different Units, DG Enlargement, country desks, special PAR Unit and DG Admin? 

6. Public Administration Reform is not a separate chapter in the Acquis. Should it be a separate chapter? 

7. What is your take on the Treaty of Lisbon regarding Public Administration reform? Does the Lisbon Treaty lead to a different approach of the EU towards Public Administration Reform in the candidate and potential candidate countries? 

8. How do you asses the EU-Insturments to promote Public Administration Reform in Albania/Macedona and Montenegro as regards quantity and effectiveness:
Differentiate per country, if possible
CARDS (phased out)
Twinning
Twinning light
TAIEX
IPA
Did I forget to mention an instrument that is relevant?
9. Are these programmes well designed for the needs of PAR in Albania/Macedonia and Montenegro or do you perceive a need for adjustment in any of them? (Content or technical)


10. In your opinion, are there obstacles to PAR in Albania/Macedonia and Montenegro? And what would be necessary for successful PAR in Albania/Macedonia and Montenegro? 

11. Which other instituions/organizations or bilateral donors are important in regards to PAR in Albania/Macedonia and Montenegro? How do you asses their impact on PAR in the three countries? 

12. Who is responisble for co-ordinating the PAR activities of all the different donors in Albania, Macedonia and Montenegro and what is happening in this respect at the moment? 

13. Should the EU have additional or other priorities in future PAR programming in Albania, Macedonia and Montenegro. 

14. Is there anything else that is important in the context of my research that you would like to comment on?