% FILE: appendixA.tex  Version 2.1
% AUTHOR:
% Universit�t Duisburg-Essen, Standort Duisburg
% AG Prof. Dr. G�nter T�rner
% Verena Gondek, Andy Braune, Henning Kerstan
% Fachbereich Mathematik
% Lotharstr. 65., 47057 Duisburg
% entstanden im Rahmen des DFG-Projektes DissOnlineTutor
% in Zusammenarbeit mit der
% Humboldt-Universitaet zu Berlin
% AG Elektronisches Publizieren
% Joanna Rycko
% und der
% DNB - Deutsche Nationalbibliothek
%\begin{landscape}
\chapter{Interview with Public Administration Reform experts in Albania, Macedonia, Montenegro}
%-----------------------------------------------------------------------------------------------------------------
\markboth{Anhang B , Frage Nr. 1}{Anhang B, Frage Nr. 1}
\textbf{1. Which main topics/areas in the context of Public Administration Reform are presently dealt with as a priority by Albania/Montenegro/Macedonia? }\\
\textbf{Official, Albania}: We are working on two laws at the moment. One is on civil service. The one in place is from 1999 and a 3/5th majority in parliament is needed to change it. There are pitfalls within the existing law and now after 11 years we see that there are a lot of problems with implementation. The other law is the law on organization and functioning of public administration. Presently there is no such law. I think this is really important since we are responsible for the structures of the PA. The respective minister writes a proposal and we write the structure for the Prime Minister to sign. As the position of a minister is more powerful than that of the DOPA, there is not much leverage for DOPA. Thus the law is important and it is good to set standards, not only for the civil service and PA, ministries, line ministries and depending institutions, but also for independent institutions and even for municipalities and cities. This law will require a 3/5th majority and we are working not only with the project within IPA 2008, but also with SIGMA, which is supporting the law on Civil Service. We are also working on laws under justice reform, which will strongly affect PA: the envisaged law on Administrative Courts and the law on Administrative Procedures (the existing law will be changed).\\
\textbf{NGO representative, Albania}: Reforming the public administration has always been a pre condition for other reforms to be implemented and to be pushed forward in the European Integration Process, which is the driving force or at least should be the driving force of reform in Albania. This process is stuck at the moment by the unwillingness of the political class in Albania to deal with this issue. Another problem is more generic and technical. The law on the civil service requires a 3/5th majority in Parliament. This means the reform should be bi-partisan and the opposition parties would have to agree on this reform. These are the two biggest shortcomings of going forward with the reform of the public administration. The law on civil servants is in many parts not properly implemented, e.g., the removal of persons from office is not based on proper argumentation and reasons. Most of the time, replacements were done for political reasons, in particular when a new political force comes into place, but we have seen that it also takes place when there is a change of the head of a Ministry or independent institution. The problem is not the law, but the mentality of dealing with this issue and that is the biggest concern. Also, a lot of judicial decisions withholding the requests of former employees are not implemented by state institutions.\\
\textbf{Official, Macedonia}:We have 10.000 civil servants at central level and 3.000 at local level. This is one tenth of the whole number of approx 100,000 public employees. The government just adopted a new PAR strategy at the end of 2010. The main focus will be on the further professionalization and depolitization of the PA. Every year we receive from Brussels the criticism about the politization of the administration, which exists in reality; we can not deny it especially outside of the civil service, where the rules for employment are basically non-existent. We are using the general labour code, which does not give anything in terms of criteria for selection. The head of a hospital can hire and dismiss at any time. This is not the case in the civil service, where we have since 2000 very precise and detailed regulations. Unfortunately, even there, the CSA was not able to defend the system from political influence. In particular during the past years, this political influence has become enormous and the Civil Servants Agency (CSA) has failed to defend the system from this type of interference. The CSA will continue more as an operational body rather than as a policy making body as before 2010, when the Agency was responsible for the policies on the rights and duties of the civil servants. Having in mind that the new ministry of administration is under the same roof with information society, one of the issues would be to further E-government implementation.\\
\textbf{NGO representative, Macedonia}: For 2011 it is the new PAR strategy, which was developed in the context of an EU project. It will be the main co-ordianator of the Public Administration Reform process in the country. All the duties that used to belong to the Civil Servants Agency are envisaged to be transferred to a new ministry, the ministry of Administration and Information Technology. Basically the existing ministry for IT is being charged with an additional issue, the PAR issue. The Prime Minister has already considered or named the year 2011 of the Public Administration Reforms. There will be a minister held accountable for the reform process on PA. Until now it was the Civil Servants Agency (CSA) directly that was accountable to the parliament and now it is the minister, who is accountable to the government.\\
\textbf{Official, Montenegro}: In the area of Public Administration with the new AURUM strategy adopted by Parliament, rationalization of the PA structure, stabilization of public finance including external and internal financial control, the area concerning the personnel system with implementation of a merit system and a completely new law on public officials. Other developments are one stop shop reform and a new law on Admin Procedure, the issue of the quality of laws, and strategic documents. In this area, especially the coordination of national policies was emphasised, introducing regulatory impact assessments, and regulation guillotine, and the issue of NGO participation in drafting the documents. We are often copying the solutions from the EU, but we do not have a systemic approach to deal with these issues.\\
\textbf{NGO Representative, Montenegro}: From 2003 to 2009 we had a PAR strategy. The drafting and implementation of the strategy was financed by the EU through PARIM I and II projects. Most of that was completed in 2007/8, drafting of new legislation on state administration, state employees, the organization etc. Between 2008 and today, 2011 little was done. The work on the new strategy started at the end of 2009 officially. While we do not have the document yet, a first version was consulted with civil society, SIGMA, CoE and UNDP.In the last quarter of the last year we had preparations for government changes. Preparations for the new president of this gov't, new structures, new ministers, and the new PAR strategy was waiting for the new structure to adopt it.\newpage
%----------------------------------------------------------------------------------------------------------------------
\markboth{Anhang B , Frage Nr. 2}{Anhang B, Frage Nr. 2}
\textbf{2. Is the institution/structure dealing with PAR adequate for the tasks ahead? }\\
\textbf{Official, Albania}: The Department of Public Administration (DOPA) was created 16 years ago under the Prime Minister’s office and since then the capacities have increased from 3 staff working on PAR to 22. Since 2006 DOPA is under the Ministry of Interior. There are three directories, one for organizational structures and salary reform of the whole Public Administration (approx. 90,000 public employees, of which 6,000 are civil servants) including independent agencies. The second directory works on recruitment of the civil service, in particular line ministries and council of ministers, but also on the policies of the civil service. The third directory is the IT department dealing mostly with the register of public employees. We are working together with the Agency for Information Society on constructing a data centre for all institutions dealing with PA and a gov-net structure is presently being developed linking all 12 prefectures to it. In the municipalities and cities there are 50,000 public employees. These do not fall under the government. Of course they are obliged to implement the law on civil service and on procedures, but they are not under the responsibility of the government. These are not controlled in the sense that DOPA is controlling the line ministries. Prefects are also not part of the civil service.
\textbf{NGO representative, Albania}: If we look at the central level, the structure is clear with DOPA they have to draft and facilitate the reform and also to guard the implementation of the law, but in practice this is difficult after the changes in 2005, when the Ministry of Interior included the DOPA as a subordinate agency thus downgrading the role of this institution. Now, every decision and even every official correspondence has to be passed through the Minister. In the previous period, when DOPA was located at the Prime Minister’s Office it had a high degree of autonomy and could enforce decisions with support from the State Minister and line ministries. At the local level, it is even more problematic, as most institutions, municipalities or communes have very few resources. In municipalities there still is the problem that the Head of the municipality can decide on the important positions in that municipality. There is not such a clear division as in other institutions at the central level for the political part and the technical part of the institution. 
\textbf{Official, Former Yugoslav Republic of Macedonia}: This was a major issue, which led us to the establishment of the ministry. Until 2006 there was a structure for co-ordination of all activities, but in the latest review, which we did in 2005/6, the government adopted a very clear structure with very clear responsibilities. The political responsibility was very precisely identified with the Prime Minister. Very soon after this structure was adopted, there were elections and the new administration simply forgot abut this for four years, until last year. And then last year, some new ideas arrived. Some argued that the Ministry for Information Society (MIS) responsible for PAR is not a good choice, as the MIS is in a way part of the economic group of ministries, where we have a lot of money in the budget. Money means possibilities for corruption and also money means that the focus of Heads would be directed to those sectors that have money and the administration as such is not a money generator, but rather spending money. One of the major ideas was establishing this central body, not just responsible for PAR, but in principle for the management and policy making and incorporating most of the functions of the CSA. Having a central body responsible for PAR should be a guarantee for much faster and smoother implementation of any strategic decisions you are going to take. So we will see.
NGO representative, Macedonia: We can not speak about an independent PA, there are many political interferences in many aspects of its functioning and we all the time had a Head of CSA, who was not satisfied, who complained that he can not execute his functions independently, other powers were interfering. I think the plans for the new structure for PAR is a good development, because some of the countries that are already members of the EU do have a Ministry of Public Administration, maybe following them as an example, is a good thing. Still it depends on the capacities of the ministry that was given the task. It is not clear why this ministry in particular was assigned to undertake the reform process of PA. 
\textbf{Official, Montenegro}: Currently, it is all about cutting the expenses and nobody likes establishing new bodies. A sufficient framework for monitoring PAR is in place, although there have been suggestions to strengthen capacities of the Ministry of Internal Affairs and PA Directorate for Personnel and the General Secretary. The Ministry for Internal Affairs and the Personnel Directorate have entered the work in terms of employing additional personnel for these areas. Presently there is the idea is to transfer the responsibilities for PAR to the Ministry of Finance.
\textbf{NGO Representative, Montenegro}: The Ministry of Internal Affairs and Public Administration is responsible for the legislation,  for all state administration and PA. On the other hand there is the Human Resources Management Authority, a government agency, which is implementing policies and legislation regarding HR. At the beginning of the implementation of the previous strategy, the Ministry for Justice was responsible for implementation. In 2006 it was changed and part of the Ministry of Justice that was dealing with PA was moved to the Ministry of Interior and the name of the ministry now incorporated Public Administration. In our comments on the first version, we said it in not possible to draft a strategy without previous serious assessment of the implementation and results of the previous one. Also, we do not have mechanisms as a society and the state to ensure institutional memory. We will always come to this moment, to draft a new strategy, a new law and we are always lacking information on the previous work in the area done and evaluation.\newpage
%------------------------------------------------------------------------------------------------------------------------
\markboth{Anhang B , Frage Nr. 3}{Anhang B, Frage Nr. 3}
\textbf{3. In your opinion, are there obstacles to PAR in Albania/Macedonia and Montenegro? And what would be necessary for successful PAR in Albania/Macedonia and Montenegro? }\\
\textbf{Official, Albania}: The biggest problem for our administration is the lack of consensus between the two main political parties. They might be having this kind of approach thinking that when the opposition comes into place they will need to change people or re-shuffle institutions. On technical terms, everything which is written in the strategy and which is taken care of in the policy paper on civil service has been agreed, even with experts of the opposition parties. There is a new PAR strategy from September 2009. We have support to draft the law and to consult the law, but if there is not the necessary political consensus, it is better not even to send it to Parliament to avoid the same destiny as the Administrative Court Law, which is now for more than two years in pending in Parliament. The opposition is not really interested to have the new legislation, even the municipalities run by the opposition, and most municipalities are run by the opposition. It is a kind of political culture that is blocking a lot of things here, not only this law. Apart from having good laws, we also do not have a strong culture of implementing the laws that are in place.
\textbf{NGO representative, Albania}: The politicians would like to have their hands free as much as possible to appoint people from their staff, wherever they go. I have seen ministers working in one government, going from one ministry to the other and taking their staff with them. Not only political staff, but also technical staff. This makes it impossible for PA to be sustainable in the long term and also to have a proper career system implemented. Such guarantees do not exist, which created this feeling of being appointed ad hoc on a temporary basis although you should have an appointment for life, when you enter the civil service, unless, of course, you breach the requirements. Heads of local institutions should feel that they would be judged for what they produce and the outcome of their work. Then they would feel the need to have proper administration in the institutions established and working professionally. 
\textbf{Official, Former Yugoslav Republic of Macedonia}: Number one is politicization and political interference. The second, again related, is the existence of political will. In terms of professionalizing, we were 2-3 years ago proposing to establish senior civil service, to establish also fast tracking mechanism for juniors for their advancement in their career. It was not accepted, because establishment of senior civil service would mean that the interference of politics would be dramatically decreased. And politicians still are not willing to give up on these tools. Although, formally there were a lot of public hearings regarding the new PAR strategy, these were more of a formal nature rather than substantial. So this could be an obstacle. Another important issue could also be funding as training of the civil service or the PA is expensive.
\textbf{NGO representative,  Macedonia}: The main obstacle is that there is continuous, declaratively announced political will for the de-politization and further reform of the PA, but we have not seen any tangible results from any of the government structures today. For example for the rightsizing process, which has been declaratively initiated since 1999 or 2000, all the political parties that have ruled the country since then have declared, that they would decrease the number of people in the public sector. But there is no concrete strategy until now to solve the problem. There are declarative statements that we are determined to proceed with the process, that we are determined to integrate the country into the EU, but very little has been done in practice and fulfil this vision. Necessary for successful PAR would be strengthening the implementation of laws. While the laws have undergone a lot of changes, still the EU and the experts say the legal framework is good, and an EU approximated one and that it can and should ensure the proper functioning of the PA and the civil service. But as we know we always have this weakness of implementation of legislation in general in the country.
\textbf{Official, Montenegro}: There is a part in the strategy that deals with these possible risks. Political will for reforms is always in the first place. Whenever anything changes in the govt internally, there always is the coalition aspect to be taken int account. The next risk are administrative capacities, a sufficient number of personnel with sufficient knowledge. And the conditions in which this reform is conducted, which are the conditions of crisis, the reform is always costly. There is an ambiguous character of the crisis for PAR. One the one hand, there is a fear of overspending, but on the other hand there is a certain dissatisfaction with the way things work. In time of crisis you are more active in trying to see what to improve with less money and there is a certain readiness for reform. In this context, when the government saw the text of the strategy, they said that is something we need. And two Deputy Prime Ministers said that it is the key reform in the SAA process of Montenegro.
\textbf{NGO Representative, Montenegro}: There are two big issues that are political, economic and social. It is downsizing of the PA in particular in recent times. We are affected by the financial crisis and the global economic crisis is also our economic crisis. In the public sector, not in PA alone, in MNE we have almost 50.000 people employed. The PAR strategy is going to cover state administration and local government. There are some analyses on the side of local governments and Ministry of Finance on the number of employees in the local government and proposals on how to decrease this number, which is very big. So this is one of the issues that needs attention and participation of all interested sides. But what we were most interested in is to how ensure accountability and transparency in the government and in the PA.\newpage
%------------------------------------------------------------------------------------------------------------------------
\markboth{Anhang B , Frage Nr. 4}{Anhang B, Frage Nr. 4}
\textbf{4. Public Administration Reform is not a separate chapter in the Acquis. Should it be a separate chapter?}\\
\textbf{Official, Albania}: It is not a separate chapter, as PA is the responsibility of every country. But since we have these difficulties even with the newcomers in the European Integration like the countries that joined in 2004 and also the Western Balkans and Turkey, maybe it is a good idea to have it as a separate chapter. And this idea, I had even while filling out the questionnaire prior to the EU opinion. It was difficult for the Ministry of Justice as the political criteria were dealt with under Ministry of Justice supervision. If you speak for the whole Public Administration, the DOPA is aware what is happening in the whole PA, but we also have separate areas, for example we have the teachers, the doctors, we have the diplomats, who have different regimes. We had discussions, who is going to answer this and that question. So maybe it is a good idea to have a separate chapter, in order to have even more pressure. Instead of answering all areas together with Justice reforms and Parliamentary reforms, which are also part of this chapter.
\textbf{NGO representative, Albania}: I think it should be a chapter in negotiations (as there is no acquis in this area) and dealt with separately, not only as part of each chapter. Political instability in Albania and high politization of public administration requires special attention by the EU. Part of this attention could be monitoring and keeping a track record of developments in the central public administration (reform, legislation, orders and decisions of heads of Institutions, career system, names of key experts, etc) and especially where EU money and expertise has been invested. So the EU could attach some concrete conditions to state institutions in this area and not only generally addressing this very problematic issue. 
\textbf{Official, Former Yugoslav Republic of Macedonia}: The EC has drafted a checklist recently on PAR. This very fact is driven by the dilemma on whether there should be a chapter or not. I believe the reform should be a chapter although there is no acquis, but the very essence of approximation of the legislation, the very essence of joining the Union is actually the building up of your administration, building the capacities of your administration to be able to cope with the new challenges. It is not just implementation of European legislation itself. It is the bureaucracy which is happening behind the scene. And this of course is pushing these reforms. Thus in the progress reports, as Brussels is always evaluating the administration, there should be a chapter. The first step in that direction is already done. I do not know if you are aware Macedonia is the first candidate country with a special working group on PA. The special working group is at the level of a sub-committee. The PA working group was established last summer. All the problems that are created today are problems you will suffer from as well tomorrow. And this is why you have to be very closely involved and interested what is happening. Brussles will give advice, maybe even a strong one, but that is it. And sometimes advice is not enough and the international community can always find a way to exert more pressure.
\textbf{NGO representative, Macedonia}: It is part of the political criteria and that suffices. It has been the case for all countries joining the EU, why should it be different for Macedonia.
\textbf{Official, Montenegro}: The issue of the organization of the public administration and its reform is specific and it seems that there are no principles or guidelines that can be universally applied. The issue of forming independent regulatory bodies, independent from the state administration, that is the main pre-condition of the EU. Montenegro is requested to form these independent regulatory bodies and over the past 7 or 8 years the number of such bodies doubled. We had 40 or so, but now we have 100. And there is the issue of functionality. And then SIGMA comes asks why have you done this? Montenegro has a population of 650,000. It is not possible for Montenegro to copy anybody else’s experience, because we are such a small country. The EU had a constant request to make state institutions professional and now you have the situation that only the Minister is a political figure in the ministry. All other persons employed, assistants, advisers, secretaries, are chosen via public tender and they have mandates of 5 years or undetermined mandates. That is maximum depolitization on one hand, but on the other hand you have not allowed the minister, when he comes into the position to have his own team. 
\textbf{NGO representative, Montenegro}: I think that most of the issues, like PA, state audit, parliamentary oversight, issues directly related to good governance, democracy control, accountability, transparency, you will never find in the Acquis. And it is really a question for me, is it good not to have them at least at the level of principles defined in the Acquis. I would like to see some of the standards included in the Acquis, it would facilitate the work the candidate and potential candidate countries are conducting in this field.\newpage
%------------------------------------------------------------------------------------------------------------------------
\markboth{Anhang B , Frage Nr. 5}{Anhang B, Frage Nr. 5}
\textbf{5. How do you assess the cooperation within the EU regarding Public Administration Reform in Albania/Macedonia and Montenegro?}\\
\textbf{Official, Albania}: IPA 2008 is supporting my department. The project is strengthening the capacities of the department for the implementation of PAR. Apart from our project, the EU Delegation is dealing with other projects in other fields, which are linked to PAR. But maybe more coordination in this field and more support would be good. Here in the EU Delegation, we mainly deal with the Head of the Operational Department and we have one programme manager, who also deals with PA, but not focussed exclusively on PA. This makes it difficult, as the person is busy with many other issues and PA is only part of the portfolio.
\textbf{NGO representative, Albania}: The problem is the same, there is not a big change from CARDS to IPA in regards to the Public Administration. It is support through technical assistance with CARDS previously and now with IPA. It is the same instrument, but differing technicalities and procedures of reporting. But the most important thing is, how dependent technical assistance is on the will of the institutions to change their priorities or to change in general. We have seen in some ministries, when they provided technical assistance, and EU officials would require other things, they would shift the focus to other areas. An example is the Ministry of European Integration, the EU started the second project of technical assistance to the ministry and although priorities and the work plan were prepared, the minister said, no, I do not want these priorities, my priority now is to answer to the EU questionnaire and I want support only on this issue and in a very short timeframe. Sometimes flexibility is a good thing, but when it hampers the core of assistance, a further look should be taken to see if it is jeopardizing the objectives of the project or not. 
\textbf{Official, Former Yugoslav Republic of Macedonia}:The EU has been very supportive for the CSA. They supported it throughout the accession process through IPA. They were supportive even when the government was excluding the CSA proposals from the programme. The Delegation would always insist to give support to the Agency, which means in the end for the administration reform, so they were very, very supportive. We here with the Delegation staff have very good relations at the institutional and the individual level. So that kind of support was not absent, but was always there.
\textbf{NGO representative, Macedonia}: The EU is serving the country very well on its way to the EU. We have a Delegation of the EU in Macedonia, which publishes regular reports on PAR or regular reports on the overall reforms in the country. But there needs to be an improvement in the domain of national institutions with EU institutions. The general perception of the population is that the government is not that much determined to move the country towards the EU. 
\textbf{Official, Montenegro}: The EU does not help in terms of already giving answers, but in opening up the problematic areas. Sometimes the expectations of the country regarding PAR are unrealistic. By giving assistance in writing the strategy, the EU did not give out answers, but acted more like a supervisor and assisted in methodological aspects. Although, expectations were for more assistance from the EU, maybe this solution is better, as now the country’s capacities are able to draft the strategy. And there was also this element of the EU not wanting to take responsibility for certain solutions, making the country responsible. 
\textbf{NGO representative, Montenegro}: The EU did not support PAR directly for a number of years, although they supported some other inter-connected projects, that indirectly contributed to PA. The EU should every year at least have some component, some project that will be directly supporting PAR. If there is no strategy, you do not have an evaluation, you don’t get the money. But again, I think once the new strategy is adopted, there will be IPA funding to support this. PARIM had a steering committee, the ministry of Justice also had some kind of council. So it had two bodies, but to be honest, the impact of these two bodies was not sufficient, because there was confusion about competences. But it is important to have some structure, where officially, representatives of civil society, interested in PAR or trade unions of PA employees and some other structures would be involved. And this is really lacking in EU funded projects, where they fund the government, they do not require civil society to take part in some kind. It is not a requirement.\newpage
%------------------------------------------------------------------------------------------------------------------------
\markboth{Anhang B , Frage Nr. 6}{Anhang B, Frage Nr. 6}
\textbf{6. Do you think the EU approach regarding Public Administration Reform in Albania, Macedonia and Montenegro is adequate? Or should other aspects be included from your point of view?}\\
\textbf{Official, Albania}: More projects focussing on PA dealing with all the difficulties would have been good. When the new government came to power in 2005, the Department of PA at that time was part of the Prime Ministers office and the new government decided to have a smaller government structure and to distribute some of the agencies or offices into the line ministries. DOPA moved to the ministry of Interior. This has to some extent weakened the institution. In this country, having so many problems with PAR, this was not a good move. The European Delegation at that time did not comment much on the move of DOPA from the government, while now their suggestion is to move DOPA to the Prime Mnister’s office. More capacity building in PA would be good. Staff should go to see how administrations work on exactly the same tasks in European countries, working with them for a whole week or two weeks. The EC Delegation was reluctant as it was seen as travel trips. Instead, experts are sent for coaching or training. Sometimes the trainers are not public administrators, and can not teach practical things, which is what is needed.
\textbf{NGO representative, Albania}: The EU approach is somewhat difficult to understand. The EU provides a lot of help with trainings and technical assistance. A lot of money is invested by the EU in PA, but on the other hand there is a low accountability from the Albanian side on the outcome. The EU always states in their progress reports that there should be a stable, professional public administration and career system established and not having political appointees in the institutions. But on the other hand, this message is not clear for the Albanian politicians and nothing happens to them if they change their staff. Now the Department of Public Administration is under a Ministry and we have seen its role diminishing every year.  I think the heads of the institutions need to start feeling responsible for the outcome of their institution. 
\textbf{Official, Macedonia}:The EU should be more resistant and harder in pursuing their positions. It is very important that this PAR checklist is really made official and put in place here. But also, I think it is an issue where Brussels should maybe start thinking about some kind of soft acquis. If you see what is happening in the new member states, which during the negotiations all established systems of administration which comply with the requirements of Brussels and nowadays they are all more or less going backward, to the old systems, which were much more politically vulnerable, then you see that obviously you need –I do not think that you are loosing your sovereignty if you have a common framework for your administration. I am referring here to SIGMA report No.44 on the progress in new member states on PAR five years after accessioning.
\textbf{NGO representative, Former Yugoslav Republic of Macedonia}: What the EU gives to the country as a task to fulfil is based on the EU model of PA. That is the determination that the country needs, a standard which has the attributes of EU PA.
\textbf{Official, Montenegro}: In the past three years, IPA programmes did not allocate any funds for PA programmes. 2007/2008/2009. But that is our fault, it was not our priority, the initiative is supposed to come from the country. Earlier, when there was the Agency for Reconstruction, there were PARIM I and II programmes, which had a couple of million Euros a year, but that was not the case for IPA. While now, they are working on creating an IPA project for 2011, which would allocate about one Million Euros for PAR. 
\textbf{NGO representative, Montenegro}: One point is that support to PAR should not be left aside in any funding year. The EU should persuade the government to plan certain efforts in this field, in order not to have gaps in the dynamics of  PAR. Also, civil society and other stakeholders should be included. In addition, more money should be allocated to concrete activities and not for consultants.\newpage
%------------------------------------------------------------------------------------------------------------------------
\markboth{Anhang B , Frage Nr. 7}{Anhang B, Frage Nr. 7}
\textbf{7. What is your opinion, how does the new IPA instrument work for PAR in Albania, Montenegro and Macedonia? Examples?}\\
\textbf{Official, Albania}: IPA projects are difficult to write with the project fiche and other documents. We had a training of 20 trainers on the two first IPA components. We try to have training of trainers for IPA 3 and 4 this year and to train not only the central government, but also local governments, because these two components deal more with local administration and we lack capacities there. CARDS was easier to deal with as the EC Delegation implemented it. IPA is, of course, more modern and involves the beneficiaries more. We have new agencies under the Ministry of Finance with auditing of IPA funds, also we have a contracting management unit in order to increase the capacities of our PA, not only to propose and write project fiches, but also to implement everything on our own at a later stage. We had a CARDS 2004 project that dealt with PA and now IPA 2008, nothing in-between. There were no IPA projects in 2009, 2010, and 2011 dealing immediately with PA. We have some problems with performance measurement of staff in PA. It is regulated only for the civil service area, but even there we have 90\% evaluated with the best grade. We would like to have individual based performance evaluations and to link these to institutional based evaluations to get more objectivity. The idea is to pilot these things in our small institution and then to replicate in other line ministries. Another idea is to introduce CAF, first for our institution, but it was perceived by the EU Delegation as being too early for Albania.
\textbf{NGO representative, Albania}: IPA is new and it started to be implemented in 2009. We have to see with the previous projects, how did they go? Sometimes, the EU officials, in order to justify the money spent on a certain area will tend to see things in an optimistic way to justify the work that is done through a project. For example we have seen last year that the EU report was stating that a lot of things were done under a certain project, but the quality of the products was not enough to justify the project and the outcomes. Most of the products were done formally to be in line with the requirements, but if you see how these products were used by the public administration or how relevant these were, you can see that most of them did not work well. One example is the gap analyses on the chapters of the Acquis that were done according to the terms of reference, but in reality it did not exist at all.
\textbf{Official, Former Yugoslav Republic of Macedonia}: Even after 20 years of independence, and 15 years of very strong monitoring and assistance from the EU (we started 1993-94 with PHARE), we still have in many institutions a lack of capacity to drive the wheel. And when you have this lack of capacity then you are driven by the needs of others, of donors. The same goes for IPA. If you do not provide a good project, they will provide you with one. So at the end of the day you might end up with project ToRs that do not fit with your visions, policies and how you would like to develop your institution, organization, government, country. Because it would be drafted by someone else, who was trying to help you but might not see the picture the way you see it. So, the very important aspect is that we have the capacity to do the programming of these plans. Since 2008/2009 we are working with the decentralized system of implementation. We are going to draft and implement our projects and the major challenge is actually planning and programming and not implementing.
NGO representative, Macedonia: I do not have this information, but one of my colleagues has written a special policy report on the usage of IPA funds. 
\textbf{Official Montenegro}: There is a general feeling of disappointment regarding  PAR. There is a certain disappointment by the EU Delegation in Montenegro that certain projects under the previous PARIM programmes were not implemented, were not providing the expected results. There were unpleasant situations when nominating the topic of PAR for IPA 2011. There were members of the EU Delegation saying you have the results of PARIM projects and you have not implemented them. There is a lot of invisible struggle, for example, on the solution of unifying the inspection controls of all the ministries into one. Now there is the situation where a minister has to give up his inspections, or the solution of one stop shops for issuing permits. But that for a minister means that he does not have a say in giving out permits in certain areas. That is why certain govt. officials do not look favourably upon these reforms and the excuse they are using is the failure of similar reforms in other countries.
\textbf{NGO representative, Montenegro}: IPA programming depends on the communication between the national government and the European Commission. There have not been any projects in the area of PAR with IPA funding for a couple of years.\newpage
%------------------------------------------------------------------------------------------------------------------------
\markboth{Anhang B , Frage Nr. 8}{Anhang B, Frage Nr. 8}
\textbf{8. Is this programme well designed for the needs of PAR in Albania/Macedonia and Montenegro or do you perceive a need for adjustment? (Content or technical)}\\
\textbf{Official, Albania}: We have since 2007 been involved with writing the IPA project fiches and on purpose we have left some things broad in the ToR to have more flexibility, for cases when we need to adjust later. CAF is an example for this, we did not think about it in 2007, but now it is 2011 and we have to at least to pilot it to see how does it work with our tradition, our institutions. We will try to write a fiche for IPA 2012, as it is really important, the performance measurement in the PA..
\textbf{NGO representative, Albania}: No answer
\textbf{Official, Macedonia}: Since 2007, the initial year of IPA, almost every year, there was a project directly dealing with Public Administration or supporting institutions in terms of capacity building.
\textbf{NGO representative, Macedonia}: I have heard organizations complain about the institutional setting that is responsible for IPA funds, namely the Ministry of Finance sometimes does not offer proper assistance to the local organizations. On the other hand, there are also deficiencies of civil society organizations here in Macedonia: They do not have enough capacities, resources to lead an IPA application process with regional or international partners. Most of the funds that have been approved for IPA projects are mainly by NGOs that have applied together with other partners, not as leading partners. There is a Think Tank, as co-partner in a project that is building capacities of civil service agents, they are already implementing. The application process for IPA funds is very complex. You need bureaucracy, you need to have a lot of documents at hand, it takes a lot of energy and time and the probability to win it is very low. Maybe they should simplify the procedure for receiving funds.
\textbf{Official, Montenegro}: The logframe that the EU insists upon is a good tool. An issue are the deadlines in which the projects should be conducted. When there is a priority now, there would be money for it in 2 and a half years, but politics do not wait, the new government wants results now. And then it turns out that IPA is used to finance issues that are not a priority anymore. These are usually education, training, capacity building programmes. In regards to the support to the reform of the inspection service controls, there was the idea to form a strong inspection services body. But in order to have it, you have to equip it, you have to invest in it. The EU rather wanted to provide by educating those employed in this body. There is an image of investing large sums of money, but the actual results are missing. For example to have a laptop for each official to do the field work, that costs 250,000 to 300,000 Euros, and now if you ask for that sum or for a Million for consultancy services, they will give you money for consultants. But that is not needed, why should we have that, we know our laws the best.
\textbf{NGO representative, Montenegro}: The government also receives support for example from the Twinning Programme. It is one of the good tools to exchange experience. There was some experience with Slowenien and French PA, and experts were spending some time in Ministries. Again some laws need to be changed and more expertise will be needed. Then, these will have to be implemented, mechanisms within the institutions to monitor and evaluate need to be established. What is positive in the strategy and some parallel efforts is that thinking about policy co-ordination, about capacities to draft policies, to assess draft legislation, regulatory impact have started. Because so far most of our work in policy areas, legislation, drafting was done quite incoherently. Some legislation was just getting the expert to draft the law, this is why I am criticizing the relation with experts. It is important, what will really stay as knowledge here. Do we have public administration employees, who now will know more than before the expert came?  The European Commission is also always talking about co-operation with civil society, local ownership, stakeholders, but in practice this is not much pursued.\newpage
%------------------------------------------------------------------------------------------------------------------------
\markboth{Anhang B , Frage Nr. 9}{Anhang B, Frage Nr. 9}
\textbf{9. Which other institutions/organizations or bilateral donors apart from the EU are important in regards to PAR in Albania/Macedonia and Montenegro? How do you asses their impact on PAR? }\\
\textbf{Official, Albania}: The World Bank is really important as they not only, time after time conduct studies, but also projects on PAR. They have even supported the first law on civil service. It was with support from SIGMA and the World Bank. They also have an IPS trust fund and we had two components from that fund. One was the training plan and the business plan for TIPA 2011-13, the other one was the assessment of the expenditure of this database that we have for all areas of PA and they are still interested in capacity building.
\textbf{NGO representative, Albania}: World Bank, UNDP they both do research and support PAR, also the Dutch Embassy (MATRA project) and they invite people of PA and CSO for two week courses on PA.
Official,Macedonia: OSCE supported PAR very much; they have very interesting small scale projects, focussed on particular issues. The administration reform is also supported by the US government with USAID, in particular the area of education and local administration. Decentralization as a process is very much supported by USAID and also other countries. Norway is very much supportive, mainly through UNDP projects. It is not a problem of donors it is a problem of utilizing donors. 
\textbf{NGO representative, Macedonia}: Mainly the OSCE Mission in Macedonia, they have a rule of law programme, they have a PAR programme, they are publishing annual reports on decentralization and PA. The OECD/SIGMA reports play a great role. And I see the government also uses them. They are referring to SIGMA reports when they are issuing a new document or a new law or strategy. SIGMA reports are of great use for the PAR progress, because they are specialized on PA. They provide very good details. Most of the time, they also provide comparative analysis of different countries of the region, e.g. the report on civil service reform in the Western Balkans. Most of the work of the organizations is consultative with analysis and recommendations.
\textbf{Official Montenegro}: UNDP and apart from UNDP there is an Agency from Luxembourg, which is tasked with functional analysis of sectors in the Ministry of Agriculsture. IFC of the World Bank is financing the regulations guillotine and for the future, support from the World Bank will be sought for certain programmes in PAR.
\textbf{NGO representative, Montenegro}: UNDP has some capacity building fund, which was initiated by the Open Society Institute. They did a lot with the Ministry of Finance and the MFA.\newpage
%------------------------------------------------------------------------------------------------------------------------
\markboth{Anhang B , Frage Nr. 10}{Anhang B, Frage Nr. 10}
\textbf{10. Who is responsible for co-ordinating the PAR activities of all the different donors in Albania, Macedonia and Montenegro and what is happening in this respect at the moment? }\\
\textbf{Official Albania}: We have the Department of Strategy and Donor Coordination under the government. They are dealing with all strategies. We have not only the strategy on PA, but a big umbrella strategy on Integration and Development, under which we have 35 sub-strategies. We have good cooperation with them and we are planning to have a working group with donors dedicated to PAR in Feb 2011. The donor coordination mechanism is quite a good example, even in the region. One of the problems that I have again on my mind is the coordination of capacity building. Every project has capacity building, but they never dealt with TIPA. Even in terms of sustainability, they prepare training manuals, it is good to have them sent to TIPA for possible later replication or even taking into account that people in this ministry or in that institution have received this kind of training, so no need to focus on that in terms of planning the training.
\textbf{NGO representative Albania}: There is a Department of Strategy and Donor Coordination that has regular donor meetings on all different aspects. I think there is some co-ordinator in this structure responsible for PAR and a number of other things, maybe anti-corruption. This Department is under the Prime Ministers office. There is this structure, but still it is difficult to see how they coordinate. The donors also have their own roundtable, where they discuss things between themselves.  
Official Macedonia: Over the past four years we did not have many PAR activities. At the level of training there were some efforts to coordinate donors. At some point in the past there was a body called bi-lateral co-ordination committee, which was responsible for training in the municipalities. The Agency, the Ministry of Local Administration and the Association of Municipalities were co-ordinating all activities, including donors. We certainly had some success in this term. And then the donors continued to co-ordinate themselves alone. But basically, the major donor co-ordination is located in the government in the Secretariat for European Affairs. They have established a central donor assistance database. At the governmental level there was a co-ordination committee for donor assistance, which included the administration, comprised of ministers. We now own the IPA projects and we discussed that for training it would be absolutely important to have a full overview,  either the by Agency at that time or now the ministry with full support of the Delegation to take care to avoid any duplication of efforts regarding training and avoiding gaps. One of the major wastes of money is that every project is doing its own training, which is not a problem, but based on their own curricula. Project management cycle, Human Resource Management were dealt with in all projects and were developing curricula again and again from scratch. This is an enormous waste of European money, donor money.
\textbf{NGO representative Macedonia}: It used to be the CSA, now most probably, the responsibility will be transferred to the Ministry of Administration. I am not sure if there is anything on donor co-ordination in the new PAR strategy.
\textbf{Official Montenegro}: This issue has been dealt with in the strategy and there is a special unit formed in the cabinet of the Prime Minister formed half a year ago, tasked with coordinating donor activity. They have a database of all donors and all projects. There is for example a database on IPA projects by year in Excel. At the same time, the Council for regulatory reform has among its jurisdiction the coordination of donors. This Council comprises the Prime Minister, several ministers and some presidents of municipalities. These are the co-ordination mechanisms envisaged for the future. 
\textbf{NGO representative Montenegro}: In general, there is no donor co-ordination on PAR, but I think that the body that will co-ordinate the process, should have included it either as a sub-group. I think that the government has a responsibility to plan things, to communicate things that there will be no overlap and to use best the available resources of different donors. \newpage
%------------------------------------------------------------------------------------------------------------------------
\markboth{Anhang B , Frage Nr. 11}{Anhang B, Frage Nr. 11}
\textbf{11. Should the EU have additional or other priorities in future PAR programming in Albania, Macedonia and Montenegro?}\\
\textbf{Official, Albania}: The capacity building and introducing some other methods of managing the Public Administration would be good. CAF, as I mentioned, but not only CAF. We still need to work on Job Descriptions and analyses. We are obliged to have Job Descriptions for every civil servant, but they were produced 10 years ago and they need updating. Every time you update the structure you should also change the Job Description, even every time you recruit a new person. An additional problem is, that if you do not have a proper Job Description, you can not evaluate properly.
\textbf{NGO representative, Albania}: There should also be an emphasis on the local level. We have been working with different projects with municipalities and regions and their lack of resources. They can hardly compete for projects, when they have to write and implement using the standard EU required tools. The capacities are weak at the local level, and it is even difficult to find support from the Civil Society or consultancies as they are also weak in these areas. So, I think there should me more attention to the local level staff and structures.
\textbf{Official, Macedonia}: I think the EU support is quite well structured and developed. We do not need to develop anything new, as long as the EU are pursuing a little bit harder these PA subcommittees and working group. Regarding the PAR checklist that should be institutionalized there was a discussion with colleagues from Western European countries, who said they would fail if we tried to implement the checklist in their own countries.
\textbf{NGO representative, Macedonia}: I think the priorities put in front of the country are enough, it is a question how the country will fulfil these. I do not think the country can cope with the existing ones, let alone charging it with new priorities. We should always keep in mind that the country is facing limited resources and limited capacities to cope with the process of PAR. Things should move step by step. We can not expect a massive reform in PA with the current capacities. We should train the personnel and the overall administration in order to carry out the process in the most successful possible way.
\textbf{Official, Montenegro}: One of the seven key conditions presented to Montenegro prior to starting the Association talks, is PAR. And this in addition gave importance to the topic. SIGMA and the EC Delegation in Montenegro have a good cooperation and they were very concrete in proposals and suggestions regarding PAR. That was a surprise for people dealing with this issue, how concrete the suggestions were in the opinion of the EU. SIGMA and the government came up with analyses which brought additional insight into the issue and crystallized things. On the other hand, some of the plans in the EU opinion seem unrealistic. For example, the deadline of six months for adopting a new law on state employees and on the general administrative procedure. This is a process too important and complex to be conducted within the given timeframe.
\textbf{NGO representative, Montenegro}: In our PAR strategy you have a number of areas, which seem potentially too wide, such as external audit, to some extent public internal finances. The danger seems to be of missing the focus of the strategy. Some donors will deal with some aspects and again we will have the problem of policy co-ordination within the strategy. \newpage
%------------------------------------------------------------------------------------------------------------------------
\markboth{Anhang B , Frage Nr. 12}{Anhang B, Frage Nr. 12}
\textbf{12. Is there anything else that is important in the context of my research that you would like to comment on?}\\
\textbf{Official, Albania}: SIGMA started a report on PA in the region. We already have questionnaires that all 1,500 civil servants are expected to fill out. Every country gets the same amount of questionnaires. 
\textbf{NGO representative, Albania}: The local level is important. Also, the independent institutions are not monitored properly. By being not within the civil service law, there is a lot of room to improve the procedures and to be more transparent. There is also this problem with the Civil Service Commission. It is totally blocked now, as they do not reach a decision on the cases brought in front of the Commission. They are fighting internally between the commissioners and although they did a good job previously, now they do not independently asses anymore because of this internal fighting and because of the future establishment of administrative courts. If the law is passed in Parliament, there is discussion of removal of the Commission. I think it is an important factor in the civil service, although now it is not functioning properly anymore. In cases where the DOPA was not supporting staff, which were moved from their jobs, still the Commission was working well until about one year ago and delivering lots of decisions supporting former employees to be re-installed. There was a decision few moths ago in the Parliament to repeal the Commission once the Administrative Courts are operational. 
\textbf{Official, Macedonia}: We discussed a lot of issues.
\textbf{NGO representative, Former Yugoslav Republic of Macedonia}: No time to ask the question
Official, Montenegro: As you are also working on Macedonia and Albania, it might be useful to ask them were they using mutual experiences in the process. You should not forget that all of those countries are from a similar heritage from former Yugoslavia and have similar administrative procedures. We have a similar background with Macedonia, but not with Albania. In a methodological sense, maybe taking into consideration that all three countries have a communist authoritative heritage, even though Albania is something different. It might be the American influence, that led to Albania adopting certain solutions from the USA. And now this is changing. Their law on administrative procedure was done by SIGMA and now they are becoming more oriented towards the EU. And this can be a factor for establishing better cooperation in these reforms. Regional cooperation exists, as does sharing of experiences.
\textbf{NGO representative, Montenegro}: No time to ask the question