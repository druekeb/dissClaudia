
\chapter[Interview with EU-officials, enlargement experts]{Interview with EU-officials, enlargement experts in the area of Public Administration Reform}
\label{anhang:InterviewEuOfficials}
%---------------------------------------------------------------------------------------------------------------------------
\section{Which topics/areas are presently dealt with as a priority by the EU regarding Public Administration Reform in Albania/Macedonia and Montenegro? What are the developments you see there? }
\label{sec:there}

\markboth{Anhang \thechapter, Frage Nr. \thesection}{Anhang \thechapter, Frage Nr. \thesection}
\textbf{EC official, DG ELARG PAR Coordination team}: PAR, or governance is a key priority in the enlargement process. The updated partnership documents with each country, list the priorities, which might differ from country to country. Mostly priorities related to PAR are found under political criteria and there we have them under Parliament, Government and PA, but also under headings such as civil and political rights and anti-corruption and possibly under chapter 23 Judiciary and Fundamental rights or chapter 32, financial control, some of these functions can also be seen as part of the horizontal PA tasks. The main priority in all the countries is to establish a civil service and a PA that is professional and not influenced by political constellations. There is a tendency, especially after elections to replace many people in the PA. We should distinguish between changes in the legislation related to the public administration and to make them conform with what we call European standards. The other point is implementation and enforcement of those laws. But I must say, in both areas, progress is usually slow. And even when good laws are enacted, you often do not find the administrative capacity in these countries to implement them or the political will. \\
\textbf{OECD/SIGMA team}: There is a tendency to understand PA in terms of civil service and administrative law and to some extent policy making, lately. For PA, we think, this is too narrow. It should be Public Governance. If you are dealing with PA, it should be wider than 3 or 4 main topics. Within the EU's definition of PA in the three countries, there is a strong interest in PAR-Strategies, in Montenegro and Macedonia and perhaps a bit less so in Albania, and in that the main focus tends to be on civil service law and anti-corruption. Lately, there is increasing interest in Admin Procedures and Admin Justice.\\
\textbf{EC official, DG ELRG Evaluation Unit team}: One of the goals is to install democratic stability in these countries with functioning institutions. The institutions we focus on are very much in the sector Justice and Home Affairs and institutions linked to democratic stability. Quite a few of the PAR projects focus on institutional structures and their ability to implement community law. Not in all of the countries we are using all of the instruments. In Montenegro, we are not using Twinning as heavily as in Albania, for example. Twinning is very helpful if you have counterparts in the host country administration. If you do not have that, the results of the projects can be in question. Montenegro is a small country with fewer and smaller institutions and right now, they are very much stretched with their engagement in the pre-accession process. And we know that from the past, even for Slovenia that this is always an extra strain on a small country. So, of course we are careful not to force too many heavy projects on them. \\
\textbf{EC official, DG ELARG Forner Yugoslav Republic of Macedonia team}: The main topic now is related to civil service law and that is the topic of recruitment, the principle of recruitment based on merit and on a transparent process. We also saw overuse of so called temporary employments, which might be a specific case for Macedonia. The state administration for whatever capacity they needed would get staff through private employment agencies for one year on a short term contract to do the job of a civil servant. In summer 2010, the authorities of Macedonia started the process of recruitment and there are indications that not everything was as transparent as it should be. And there are signals that those who were temporarily employed were given an advantage, if they were not directly transferred, which is of course against the principles. There is a Civil Service Agency (CSA) as a body independent from the government and reporting to Parliament. In reality, this arrangement has also created some problems, because for many they are just an agency. So you can imagine that a ministry of finance would hardly listen to someone from an agency telling them how they should do their work. Macedonia is now preparing an updated national PAR strategy, the first update in 10 years. And we understand there is the plan to have a ministry for PA. So the current Ministry for IT will be combined with Ministry for PA. Then the CSA would be included in the organization of this ministry as one of the departments. \\
\textbf{EC official, DG ELARG Albania team}: PAR is an overarching horizontal aspect that goes beyond the political criteria, but that is reflected specifically in the political criteria and PAR is an issue for Albania. We analyze the current situation and we give our view. In our view PAR in Albania is incomplete; there are certain issues we are following up very closely and in detailed discussions and exchanges with SIGMA. We are fully in line with the analysis SIGMA is providing in this regard on the ongoing process of civil service law reform in Albania and strengthening the department that deals with that reform. These are priorities for us in terms of financing.\\
\textbf{EC official, DG ELARG Montenegro team}: Priorities are mainly in the filed of civil service, training issues and the non-political recruitment of civil servants in every ministry. Non political civil servants still needs to improve in Montenegro. A Human Resources Management Agency was created, but unfortunately it is not yet in the lead on reforms. A person in the Deputy Prime Minister's Office has been nominated as the central contact point for PAR. There is a new PAR strategy named AURUM. For 2011, the EU foresees a large IPA project for PAR in Montenegro. In all the WB countries the main IPA projects deal with are in the realm of Rule of Law/Good Governance and PAR. \\
%%\newpage
%---------------------------------------------------------------------------------------------------------------------------
\section{ Do you think the EU approach regarding Public Administration Reform in Albania, Macedonia and Montenegro is adequate? Or should other aspects be included from your point of view? }
\label{sec:view}
\markboth{Anhang \thechapter, Frage Nr. \thesection}{Anhang \thechapter, Frage Nr. \thesection}
\textbf{EC official, DG ELARG PAR Coordination team}: The discussion is what comes first, legislation or culture. You can say that culture is affected by the laws and by enacting new laws, good laws you can influence and bring about change in a country. Another approach is based on trying to draft strategies for change, listing the objectives, and having an action plan with everything carried out according to that plan. But it did not always work like that. Maybe the strategies themselves were not professional. Maybe important elements of a strategy were missing, like a clear definition of the objectives and a realistic, continuously monitored action plan. Sometimes, the scope of PAR and governance were not defined. There is no dedicated Acquis chapter on PAR and thus, no framework for discussions. Fortunately, there seems to be growing awareness, even without Acquis. And in the case of Macedonia, there is a new  high-level working group on PAR. Also checks and balances are very important. Institutions to deal with complaints against the public administration or the government, reform of the ombudsperson institution, but also external audit with Supreme Audit Institutions. According to international standards, a supreme audit should also carry out performance audits of government programmes and activities. While this is just starting for some of these countries, it will contribute to the reform in PA. \\
\textbf{OECD/SIGMA team}: The scope of PAR should be widened to include financial aspects and policy aspects and to focus on results rather than on inputs. What we have been doing in the past and the Commission has been doing in the past is worrying more about who makes a decision than about the decision itself. The other aspect is, not seeing PA as independent from its governance context. Thus, I think that civil service reform is not appropriate to the context. Civil service reform, professionalizing and depoliticizing the civil service, at the moment is a xeno-transplant which will suffer pathological rejection. The second point is that the EU is pushing countries to reform all the time and this is substituting the presence of a reform programme for administrative performance. I think much greater emphasis has to be on the idea of implementing previous policies and previous laws and not pushing people to continuous reforms. This results in diverting resources to perform reform activities away from implementing activities. Lastly, I think that adequacy includes the quality aspect and there is a lot to be said about the quality of support given to the countries for PAR, which is largely driven by the technical assistance with management systems that have been adapted.\\
\textbf{EC official, DG ELARG Evaluation Unit team}: We are at the moment looking at the way we programme accession funds. One of the things we are thinking about is to introduce the sector wide approach, to put the focus on certain priority sectors over a period of three years. The MIPD will be the main document driving this reform, to enable us to focus on priority sectors in the countries. Evaluations of previous enlargement rounds suggest that we maybe covered too much ground at once and the countries found it difficult to prioritize between the different sectors. When MIPDs are drafted, there is a discussion on how to determine the priority sectors with the Delegations and the countries and of course this is linked to the progress reports, accession partnerships and all the top level documents. Some countries came up with a list of three, others with a list of ten priority sectors. These proposals are now being discussed, we have until January (2011). The idea is not only to have sectors, but you also ask countries to have a strategy for each of the sectors. The strategies will be linked to a budget. It is not only us putting money in; it is also the national budget of the countries and the donors that contribute to these sectors. We would end up, let us say, with a set of eight priority sectors and then the countries in cooperation with the Delegations decide which sectors should be covered by which donor.\\
\textbf{EC official, DG ELARG Former Yugoslav Republic of Macedonia team}: I think in our case and also due to the fact that we are the first ones to have this special platform for Macedonia exclusively dedicated to PAR, we took a comprehensive approach. We really want to discuss all aspects of PAR, starting with the basic institutional framework, but looking also into aspects like corruption and transparency as well as donor coordination. It should really be a forum for everything that relates to PA to be discussed. Maybe not everything to the same detail, because we have other fora, such as on corruption and we have a sub-committee on Justice and Home Affairs. But we can talk about prevention and the organizational side more in our special group. This still has to be fine tuned, but we really try to be comprehensive and see PAR as an across the board issue, which is somehow related to many areas of the Acquis.\\
\textbf{EC official, DG ELARG Albania team}: We see PA as overarching and horizontal responsibility because it relates to the question of the foundation of the state, of having good governance, stable institutions and a civil service with the right competences, professionalism and ethics. But we do not have a special programme other than this general wish for good governance. We have PAR as part of the political criteria, which have to be sufficiently met for a country to start negotiations. In that respect, early attention to good governance and setting priorities that have to be met could be seen as approach. We also give it attention in financial assistance; we have the IPA instrument for preaccession assistance. The strong cooperation with OECD/SIGMA is another sign that we want to go into depth in the analysis, in order to find the areas that need to be targeted with advice or assistance. Regarding a definition of PAR, I think we are generally inspired by SIGMA. And we try to adapt it as much as 
possible to our client countries.\\
\textbf{EC official, DG ELARG  Montenegro team}: Montenegro is a special case. Reports mention the poor administrative capacity, but it does have an administration that corresponds to the size of the country. The EC is working on analyzing the specific needs of small countries together with SIGMA. Some EU requests might have to be adjusted to the size of the country, we hope for a new and innovative approach in that respect. For example we had discussion on the advantages of long term TA over  short term assistance in the long term: i.e. the same person coming for one week per month during two years for example. With TA however, sometimes dependency on one foreigner for a long time is the case. When this person leaves, the momentum drops ! Maybe short term consultants will keep the momentum up? SIGMA is more in favour of short term and flexible approaches. TAIEX is a short term assistance focussing on the EU Acquis, it is very efficient.\\
%\newpage
%---------------------------------------------------------------------------------------------------------------------------
\section{Do you perceive differences in the EU approach compared with the experience with PAR during the last wave of enlargement? }
\label{sec:enlargement}
\markboth{Anhang \thechapter, Frage Nr. \thesection}{Anhang \thechapter, Frage Nr. \thesection}
\textbf{EC official, DG ELARG PAR Coordination team}: Yes, as a conclusion from the last wave, we issued a renewed consensus in our enlargement strategy that was issued and adopted in 2006. We focus on a stricter conditionality in all phases of the process, because we realized that in the previous enlargement rounds, we were not as strict as we should have been perhaps, in particular with these two countries that became members in 2007. We also realized that we need to address difficult issues, not only when it comes to judicial reforms, but PAR in general and the fight against corruption much earlier in the enlargement process. This message has been repeated in the following enlargement papers. In the last one for 2009, there was even a special section dedicated to the rule of law. And under a heading 'bringing the citizens and administration closer to the EU', the Commission stated that it will continue to pay close attention to the existence of a professional and functioning PA in line with the focus on basic governance issues. \\
\textbf{OECD/SIGMA team}: There are differences, yes. The last wave of enlargement was driven by a time pressure, which had to do with geopolitical concerns, not with EU-readiness concerns. Time pressure forced people to do things very differently, for example, there was a much greater focus on key risk areas for the internal market. There was a greater focus on sectoral administrative development, and a lesser focus on systemic issues. That is absolutely not a criticism. As for the 8 CEECs, the timing was driven by real valid concerns, which was not the case for Bulgaria and Romania, but unfortunately, the freedom of that not being the case was not used. The Balkans present very different problems; first of all, it is a post-conflict setting. There are large numbers of ethnic and state issues, which are unresolved. Most countries in the Balkans have rather weak states and national (as opposed to ethnic) identities; this was not the case in the CEECs. And the Balkans have weak state institutions with the possible exception of Serbia. So these countries have made very rapid progress, but the institutions of state are still rather weak, and democratic culture and the rule of law culture have not fully been internalized. I do not think the Commissions assistance either in terms of its prioritization or in terms of its delivery mechanisms have been sufficiently adapted to these circumstances. \\
\textbf{EC official, DG ELARG Evaluation Unit team}: Of course, if you look at PAR, the issues are very much the same, but the underlying issues are different. For example, if you have a country like Albania, where the administration is constantly changing when parties in power change, not just in key positions, it is very difficult to operate or start a change process. What I do not see so much in these countries as opposed to the last wave of enlargement are parties that are in opposition of the EI. They do exist, but not in large numbers. So, the main forces in the countries are pro-European. But of course you still have a quite politicized civil service, which is the problem. In the Balkans the general feeling is that the EI-process has to stabilize the region, which in practical terms it is a completely different process than in Eastern Europe. Right now, there is a discussion about looking at these countries not so much already as accession countries, but also as countries in development. And if you read the IPA regulations, it is explicitly stated there that development should be a key part in potential candidates. I don't think that we actually reflected on that enough. We just took instruments like Twinning and Twinning light, TAIEX, etc. and we are just now really adapting them, adapting them fully or revising some of the instruments.\\
\textbf{EC official, DG ELARG Former Yugoslav Republic of Macedonia team}: I think there are a lot of lessons learned. There are similar problems with several countries of the last wave of enlargement. And I think it is due to the lessons learned, why there is this idea of a specialized dialogue on PAR. What we saw before the discussion on PAR was just based on political criteria. Once these political criteria were fulfilled, there was not really a follow up. SIGMA conducted a study on the situation in the countries, which recently joined the EU and their finding is that there was a lot of backsliding in the PA regarding adherence to principles etc. It was quite evident that PA, although it is very important, it is sort of difficult to pinpoint where the boundaries are, so it is not really followed up. With this new approach, what we are trying to do is to have a regular dialogue, where we could really see from one month to another, what really happened. And even when the political criteria are fulfilled and a country received recommendations for opening negotiations, you can still have a place where you can raise issues. In the last wave of enlargement there was no forum to continually look at PA issues after negotiations started. \\
\textbf{EC official, DG ELARG Albania team}: I think in general, yes. This is also a general comment on the political criteria. Of course we have learnt our lessons from the fifth enlargement. We are at an earlier stage addressing certain issues and that includes of course the rule of law, corruption and organized crime. Within this, it is also about governance in these institutions. And also, we do now establish certain targets that need to be reached before we start negotiations. You can see this already in Macedonia, not immediately the opinion, but what was published shortly after. The opinion itself does not give key priorities, but the accession partnership or European partnership do. For Macedonia we have the opinion 2005, and in 2008 we have the updated accession partnership with key priorities that need to be fulfilled before the country can start negotiations. This is a model that could be pursued, which is presently discussed. The philosophy in any case is there. We will want to see the issues addressed at a much earlier stage, even before negotiations start and that could include the priority on public administration.\\
\textbf{EC official, DG ELARG Montenegro team}: There is not really a different approach. But now, PAR is high on the agenda. PAR includes a vast area of topics. It relates not only to the services provided, but also for example to the structures in each ministry. In June 2008, a National Action Plan for Integration (NPI) was designed for implementing the SAA, which is very comprehensive. The NPI will be revised after the opinion on Montenegro will be published. \\
%\newpage
%---------------------------------------------------------------------------------------------------------------------------
\section{The literature on enlargement sometimes argues with the legacy theory, in particular regarding the last wave of enlargement. Meaning that structures of previous regime set-ups have an influence on the present development of Public Administration Reform. What is your view on this issue regarding Albania/Macedonia and Montenegro? }
\label{sec:montenegro1}
\markboth{Anhang \thechapter, Frage Nr. \thesection}{Anhang \thechapter, Frage Nr. \thesection}
\textbf{EC official, DG ELARG PAR Coordination team}: Obviously, there is a common legislative background in all the former Yugoslav countries, when it comes to civil service, which is of course not in line with European standards. And also, unfortunately, when these countries became potential candidate countries, we asked them to reform their legislation, it seems to me that they have been using sometimes experts, who themselves had been brought up under that legislative background. So when it comes to certain legislation of civil service continued to amend those laws in line with those old values, so to say. Obviously those structures or set-ups or values had an influence on the present development. \\
\textbf{OECD/SIGMA team}: The recent Sigma paper No. 44 on civil service reform in the CEECs after accession, does not very much support the legacy theory. But intuitively, the legacy theory must mean something. Probably what the legacy theory does not predict in the CEECs is the differences. But if you take legacy as very basic concept, with these countries starting from a communist system of governance and then switching to a democratic, market oriented, rule of law one, then the legacy theory is an underlying idea, which we have to take into consideration all the time. I think legacy theory in the Balkans is very important, but legacies are different. For example in Serbia and Montenegro, the sanctions regime and the way the states were forced to operate under the sanctions, have an enduring effect. In Albania it is necessary to keep in mind, the harshness of the regime before compared to what was happening in the rest of the Balkans. The rest of the Balkans were relatively open, whereas Albania was totally closed. 16 years on, the legacies are still there in people's mentalities. They are still there in people's understanding of law, both citizens and power elites. And power elites still understand themselves as the architects of law, but not the subjects of law. Now, some of that goes back to pre-communist legacies. That takes you into the area of social, cultural explanations, which is very, very difficult to handle. You have the Austrian/Turkish legacy, we have the communist legacy and we have the post-communist legacy, because after all it is now 20 years after the wall fell. Some of these countries then went through conflict, some of them were under regimes like those of Milosevic and Tudjman, which introduced their own legacies into the system and which stay on in terms of criminal networks and oligarchic arrangements. So, as I said legacies are a very complicated topic. I think you can talk about a sort of substrate, but I think it is very difficult to use legacies for identifying differences. \\
\textbf{EC official, DG ELARG Evaluation Unit team}: There is of course a certain culture in PA and there is the civil service code. But of course what happened after the overthrow of the communist regimes, all of this has been just filed away and it was built up from scratch. A very interesting research question would of course be to compare the old and the new civil service code and see how much of it actually matches. That is a big question how much of the old traditions have carried over into the 'new' institutions. There is a certain mentality. We are mostly dealing with administrations that are stretched to the limit; I am hearing that mostly from our colleagues in the country units. Do not ask them for too many things, because they just do not have this capacity. For example now, we want to organize a training for evaluation and monitoring and just to get the commitment for one training day for maybe 10-12 people, it is almost like shutting down the whole ministry. Especially in Kosovo and Montenegro. Of course we have to take that into account. \\
\textbf{EC official, DG ELARG Former Yugoslav Republic of Macedonia team}:. I think yes. It goes back to the Austro-Hungarian Empire, some of the principles that are embedded in their laws. The law on Administrative Procedures you can trace this back a very long way. The Austro-Hungarian Empire was encompassing countries which were later on transition countries and you could see similar issues or problems in the way to approach things, the heritage, also in the Czech Republic, Slovakia, Hungary and now in the Balkans. While Macedonia never was part of the Austro-Hungarian Empire, it inherited from Yugoslavia, which was heavily based on the older model. So, that is how we can trace the heritage. The country was heavily influenced by the set up of the administration, education and the entire package of Yugoslavia and that is of course why you can see some of the same problems in Croatia, Montenegro and elsewhere in the region. \\
\textbf{EC official, DG ELARG Albania team}: More than a legacy in the structure, there is a legacy in the culture. In Albania probably more so than in any other country. You have a legacy of respect of the highest authority being the only institution that can change things. Although you have that in all ex-communist countries, you have that very strongly still in Albania. So, it is culture more than structures, I would say.\\
\textbf{EC official, DG ELARG Montenegro team}: Until 1989 most of the countries of the last wave of enlargement had central planning. For the Balkan countries the situation is different, as these countries have had the time to start the needed changes. We are much further on in time and also the structure of the state was different than in most of the countries of the last wave of enlargement. In addition, there were wars in the Balkan countries as opposed to the countries of the last wave of enlargement. It is extremely important that these countries talk to each other. Two years ago the DG RTD1 produced a good study based on ethnic research and said among many interesting findings, that we should be careful not to create "ethnocracies".\\
%\newpage
%---------------------------------------------------------------------------------------------------------------------------
\section{How do you assess the cooperation within the EU Commission regarding Public Administration reform in Albania/Macedonia and Montenegro with the different Units, DG Enlargement, country desks, special PAR Unit and DG Admin? }
\label{sec:admin}
\markboth{Anhang \thechapter, Frage Nr. \thesection}{Anhang \thechapter, Frage Nr. \thesection}
\textbf{EC official, DG ELARG PAR Coordination team}: There is very close cooperation within DG Enlargement. There is dedicated country desk for each country and the Delegations in each of the countries. We have the coordination unit both for the political side for producing the annual progress reports, which is unit A1, and unit D1 for the instruments and contracts, which is responsible for correct application of the financial instruments, especially the IPA-Instrument. Take for example financial assistance, there unit D1 regularly organizes meetings, here in Brussels mostly, with the heads of the operational sections in the Delegations. They are constantly kept updated on everything here. There are a number of PAR IPA-projects in each country. IPA projects do not need to be linked to an Acquis chapter; they can also target political criteria. There is a long programming-process, where all the stakeholders, first of all the national authorities themselves, then the Delegations are involved. One important new element in the whole process was when DG Enlargement about three years ago established a so called quality support group (QSG) where drafts of project fiches, which later will be part of the annual national programmes of the countries are discussed quite in detail and are circulated in various units. The aim is to ensure that we plan projects with IPA-support for those areas where we find gaps, where there is a need for reform or a need for institution building. Overall, there are people responsible for financial assistance and others more responsible for the political dialogue. \\
\textbf{OECD/SIGMA team}: We need to make a clear distinction between the political discourse and negotiations on one hand and technical assistance on the other hand, which in my opinion are not always connected, posing something of a problem. The Brussels-based country desks do try to keep the TA part linked to the negotiations. But the TA part tends to be driven by disbursement issues and the Delegations. In DG Enlargement, I think it is fairly tightly connected and to some extent the requirement to produce the regular reports and the multi-annual programming, drives the cooperation process and similarly with the other DGs. With the Delegations, there seem to be two separate issues. One between Delegations and HQ, which will become at least more complex with the arrival of the External Action Service and the other, is the relations in the Delegations between the operational and the political units, where I think coordination could be quite significantly improved. DG Enlargement relies very heavily on external experts and neither DG enlargement nor the Delegations seem to have the technical abilities to steer/control all the technical assistance they are producing. Technical assistance is managed at the administrative contract level and not really at the substance level and the substantive dialogue with countries does not really take place. \\
\textbf{EC official, DG ELARG Evaluation Unit team}: There are several processes and everything we do is cooperation between the units. We want to use the evaluation unit more regarding the question of lessons learned of all the evaluation reports. One question that always is difficult for consultants or our evaluations to answer is on impact and sustainability, based on the five OECD criteria: relevance, efficiency, effectiveness, impact and sustainability. It is quite hard to come to a judgement, if you do not have some sort of a basis. You have to know what was there at the end of a project. Otherwise it is hard to judge what is still there in a year or two years after. With the sector approach there is leverage. You basically ask for a strategy and commitment to certain objectives in the strategy. Things are then formulated out in national programmes and projects. With these projects you can then say, now tell us why you want this project and how does it contribute to your strategy in the Justice sector, let us say. It is all linked up in a logical sequence towards accession. We will most likely have better donor cooperation, more targeted and sequenced funding for assistance and that has of course a large effect for PAR as well.\\
\textbf{EC official, DG ELARG Former Yugoslav Republic of Macedonia}: We have a system of so called chapter desks. For every chapter of the Acquis, like agriculture or fisheries, you have someone in DG Enlargement, who is getting the overview of all countries on that chapter. Somebody will be dealing with Albania, but there is also somebody looking at a chapter in all the countries. This is to make sure there is consistency of approach. We did not have anybody specifically for PAR as it is not a chapter. SIGMA is sub contracted to do work on PA, as we do not have the capacity. For the time being, I am in touch with DG HR, SIGMA, the PAR Coordinator and DG Justice, as DG Justice is the one dealing with aspects of corruption for example. I am in touch with DG Budget on issues of Public Finance and DG Market on public procurement, even with OLAF on issues of anti-fraud. So, because there was and still is no single formal platform on these issues, it has been covered in bits and pieces and other fora. 
EC official, DG ELARG Albania team: Within DG Enlargement there is certainly an increasing attempt to make sense and logic out of this area, which is a horizontal area rather than a specific sector. There is one person in DG ELARG as a sort of horizontal guidance person; also, there has been a working group, so here is some attempt to get the theory right and in this context there has been an increasing cooperation with DG HR. For the upcoming opinion, we have contributions from the line DGs. You have structural funds as well as financial assistance to the "'administrative capacity programme"'. We, together with the Delegations, are interested not only in the Justice sector, but also the reform issues, synergies and good governance.\\
\textbf{EC official, DG ELARG Montenegro team}: There are all sorts of institutional ways to ensure cooperation throughout units, directorates, general directorates, the council, the parliament, as well as with member states (through the IPA committee and other consultations).. If PAR is involved, DG HR is now involved thematically; the chapter desks and the country desks are asked to be active partners in the design of IPA projects (providing comments etc). PAR is not specifically discussed for example in a sub-committee, as it is considered as a horizontal issue.  All assessments, foremost the progress reports are checked for issues that are highlighted as "'in need of progress /efforts or in need of reform"' in order to come up with assistance projects. This is done by my unit, but also the Delegations and the countries themselves. These issues are then discussed by the different stakeholders.\\
%---------------------------------------------------------------------------------------------------------------------------
%\newpage
\section{Public Administration Reform is not a separate chapter in the Acquis. Should it be a separate chapter? }
\label{sec:chapter}
\markboth{Anhang \thechapter, Frage Nr. \thesection}{Anhang \thechapter, Frage Nr. \thesection}
\textbf{EC official, DG ELARG PAR Coordination  team}: When we talk about the Acquis, it is something constantly evolving. Some of these chapters or even the majority are not necessarily based on hard Acquis, EU legislation, EU directives and so on. Some of the chapters appear to be soft in character, meaning that they sometimes refer to international agreements, standards, conventions or treaties issued by other bodies, such as the CoE. The issue of creating a new chapter on PAR is  currently not realistic because there are complex legal and procedural matters and there was also a feeling that it would not be right to add a new chapter as if we would make it more difficult for the new candidate countries compared to the previous ones. Also, there was the argument that perhaps the member states, who would decide on the change in the Acquis might object, because it has at least indirect implications for them. How does it look like if we in a chapter request certain PA reforms, which perhaps are not in place in the MS themselves? But even in the absence of a formal chapter, we can increase the profile of PAR. That means to really discuss it, to conduct a political dialogue with the candidate countries as we do with the chapters. And I think that the fact that we have taken the trouble to list priorities in the Partnership Agreements relating to  PAR and to form indicators, shows that this can be done and it is logical to  enhance the dialogue on PAR.\\
\textbf{OECD/SIGMA team}: The problem is that there is no Acquis regarding PAR, and it is not susceptible to become part of the Acquis in my view. I think PA is far too contextual and social. So one part of the answer is that I do not think you could make it a chapter and my position is to some extent re-enforced by the leading example of what we have been talking about, which is PIFC, which is absolutely not Acquis. It was negotiated into a chapter, now chapter 32. The result in my view was that many of the countries were forced to create systems they could not find models of elsewhere; which were not appropriate or sustainable and which diverted scarce resources into low priority tasks and away from consolidating basic systems. I think it would be far more powerful for the Commission, if it simply relied on the political chapter and ensured that the political chapter was not forgotten about, as soon as negotiations started. At the moment you say, the country meets the political criteria, therefore negotiations can start. Also, I would like to start thinking about outcome measures, for example on administrative reliability. What sort of indicators should we have to measure if administration is acting in a reliable and impartial way? You have for example the analysis of judgements of administrative courts, you have the ombudsman. You could imagine a number of different methods, case based sampling, customer surveys etc. I would like to see a move towards an approach, where we do not say what the inputs are, but what we would like to be the outputs. Also, we should be more concerned about trajectories than about absolutes. And we should be thinking about pathways. However, inputs are easier to objectivise than outputs so both the Commission would be under more pressure to defend judgements rather than "'facts"'. \\
\textbf{EC official, DG ELARG Evaluation Unit team}: Of course it should not be a separate chapter. If there should be a chapter, it should be with the word horizontal in brackets. I think it always comes out again that horizontal PAR is a domestic issue and not something the EC should get too involved in. The Commission should involved in it only as it has large repercussions on the vertical implementation of the Acquis.  \\
\textbf{EC official, DG ELARG Former Yugoslav Republic of Macedonia team}: I think it is difficult to put on a piece of paper or have Acquis to tick off for the countries, so it is difficult to create a chapter. But of course if there was one, it would make things easier to a certain extent it would be less difficult to put your finger on specific things. I think we are going into this direction. We might have a chapter on PAR at some time in the future, but at this point it is difficult to say what would be included into this chapter. \\
\textbf{EC official, DG ELARG Albania team}: No, it is not a separate chapter, but plays a very important role in the political criteria and now the important question is, how much is it an important factor in chapter 23, Justice and Home Affairs. Also other chapters are relevant, like financial control or procurement. But under political criteria we look more at the overarching issues like civil service reform and good governance. Should it be a separate chapter, I don't know. If it would be a separate chapter, it would take out from other chapters and that would not be possible, thus I would say no, but should it be strengthened also in the chapter parts? There, I would say yes. We also have to find guidance and incentives after the opening of negotiations and not stop after we evaluated it. PAR is an overall process and does not stop there. \\
\textbf{EC official, DG ELARG Montenegro team}: There is no Acquis in PAR. But SIGMA is providing each year its assessment on PA in each of the WB countries. These are very helpful to design future assistance projects. There is need for PAR to become a chapter in my point of view.\\
%\newpage
%---------------------------------------------------------------------------------------------------------------------------
\section{What is your take on the Treaty of Lisbon regarding Public Administration reform? Does the Lisbon Treaty lead to a different approach of the EU towards Public Administration Reform in the candidate and potential candidate countries?}
\label{sec:countries1}
\markboth{Anhang \thechapter, Frage Nr. \thesection}{Anhang \thechapter, Frage Nr. \thesection}
OKed by interviewee
\textbf{EC official, DG ELARG PAR Coordination team}: This is an interesting question, because I know that there have been discussions what it means. There were discussions what is the scope of the new paragraphs (art 197 and art 298). Does it really mean that the EU can request candidate countries and new MS to carry out PAR? I do not think so, looking at the text. Art 197 states that the Union may support the efforts of member states to improve their administrative capacity to implement union law but harmonisation of the laws and regulations of the member states is excluded. And art 298 relates more to the Commission itself, and not so much to the administration of MS. In principle, I think it is up to every MS to decide by itself on how to organize its PA. And what paragraph 197 says is more that the Commission can provide support or assistance. To summarize, I do not think this new paragraph opens a door for any radical change when it comes to the approach. \\
\textbf{OECD/SIGMA team}: The interpretation is not very clear. What it means, it seems to me that it limits PA to mean PA to implementing EU policies, which is a lot of the time of course. But I think the EU has become a common law country. And it will be the judgement of the ECJ in the next years to determine what it really means. And principles like "'equal treatment"' will soon force the scope of application to broaden. But there will certainly be an empire built around it and there will be discussions within the Commission on who gets to build the empire. It gives a certain degree of additional legitimacy to the Commission's activities in administrative reform, but not that much more, as it already had been there in the Copenhagen and Madrid councils. So, paragraph 197 as it becomes powerful, will probably have a marginal or higher impact on member states rather than on candidate and potential candidate countries.\\
\textbf{EC official, DG ELARG Evaluation Unit team}: Generally, it is always good to have something in the treaty. There is a larger question at stake. Should we not respect the subsidarity principle regarding candidate countries and I personally think we should. We should not run their countries and we really should not tell the Prime Minister how he has to shape his ministries and how he has to shape his administration. We can only advise them what would be the best way to implement the Acquis, based on the MS experience. Regarding PAR you have no strict limits on how many civil servants you need to implement a certain article of the Acquis. The Acquis never specifies the implementation in detail, while it may specify that you have to have laboratories or border stations. In negotiations, you can ask for certain things, but the negotiator can not sit down and say I want 50 people in this part of the administration. He will probably say, I want you to write this law and I want you to be able to implement it. And then of course our assistance programmes complement the negotiations and Twinning often derives from negotiations. The negotiator might say we provisionally close this chapter, but I want you to have a Twinning on Social Dialogue, for example.
EC official, DG ELARG Former Yugoslav Republic of Macedonia team: I do not think it will have any immediate effect on our work in the enlargement and accession process. There is some kind of work methodology established. But it is good that PAR is noted, that there is a stress on that. This might help with the emphasis we put on the subject. But I do not see any immediate effect. 
EC official, DG ELARG Albania team: I would have to consult the treaty to form an opinion. 
EC official, DG ELARG Montenegro team: Question not answered.\\
%\newpage
%---------------------------------------------------------------------------------------------------------------------------
\section{How do you asses the EU-Instruments to promote Public Administration Reform in
Albania/Macedonia and Montenegro as regards quantity and effectiveness: Differentiate per country, if possible CARDS (phased out) Twinning Twinning light TAIEX IPA Did I forget to mention an instrument that is relevant? }
\label{sec:relevant}
\markboth{Anhang \thechapter, Frage Nr. \thesection}{Anhang \thechapter, Frage Nr. \thesection}
\textbf{EC official, DG ELARG PAR Coordination team}: Twinning and TAIEX were from the start, intended for Acquis-specific issues regarding institution building and not so much for PAR. I would think that the number of activities under Twinning and TAIEX related to PAR are relatively few. I would add two other aspects here: Which are the right instruments to use for certain types of institution building? In which order should they be used? I think that the TAIEX instrument, which is one of the instruments to finance seminars, study trips and expert visits for a few days, is an instrument that could be used so to say to prepare more large scale projects, like assistance, the same with some of SIGMA's technical assistance. So, in principle all the instruments complement each other. Regarding Twinning, what happens very often is that we do not so much transfer common European standards, but that in practice specific MS send Twinners to a candidate country and they are transferring the models in their own countries. But sometimes these models conform to good or even best European standard. For example in external audit, there is support to build up capacity of Supreme Audit Institutions, and there I got the impression that two countries are more involved than others, namely Sweden and the UK.The SAIs in these two countries have a  good reputation.\\
\textbf{OECD/SIGMA team}: I think you are confusing a financing instrument with a delivery instrument. CARDS and IPA are financing instruments. Twinning, Twining light and you should add Technical Assistance and SIGMA, are delivery instruments. EU regulations govern these instruments and determine the efficiency of delivery. For example the time delays and the programming systems, have something of a deleterious effect on the delivery quality of the instruments. In the area of instruments you have Twinning and Twinning light, you now have to add new ones, which is budget support and sector support. So, there is a large spectrum of instruments with varying degrees of effectiveness. I think that Twinning and Twinning light are useful, certainly useful where there is Acquis, and when there is stable political and institutional environment, i.e. if it is a technical issue. Then everybody knows what to do, you only have to put into place the organization of it. When it is non technical and political and or when there is an unstable political context and politically sensitive environment, I do not think they are appropriate and have not been very successful in our areas. Traditional instruments then fall back on trainings, with the exception of some of the financial issues. TAIEX is ok, but it does not have an institutionalized memory, which is important in this area of work.
EC official, DG ELARG Evaluation Unit: There are also Technical Assistance projects and quite often they are combined. Generally you would look at a country and decide what kind of process you want to put in place, what kind of changes you want to see. The next question of course is, how do you achieve this. There the instruments come into place. TAIEX is something that we can organize fairly quickly, these are short missions, up to two weeks, maybe to help draft legislation or bringing the expertise through workshops etc. Twinning is our main instrument to deliver institution building. When Twinning was introduced under CARDS, it was perceived as an accession instrument and there was a lot of interest of the new member states as they had just completed their accession to help others. It is now changing, as Twinning is a quite heavily engaging process and you need a lot of resources for it. This has created something of a step back from Twinning.\\ 
\textbf{EC official, DG ELARG Former Yugoslav Republic of Macedonia team}: They are popular in the country, but it is difficult to compare as they are a bit different in nature. TA are classical projects more expensive and more long term and TAIEX was designed to be user friendly and it is very much used, but it is difficult to asses what the impact directly is. It brings people together from the country to Brussels for example to meet experts or to member states; you can organize short workshops and seminars. So, I think they had their own contribution to the process and they contribute to a better understanding what the Acquis is and it also helps, especially TAIEX; for people to be exposed to the EU way of dealing with things, which is useful. Of course IPA are bigger scale and longer term projects. And for a Twinner, it would mean for example helping to draft a law or streamline the structure of a unit. This Twinner has to respond to the needs that have been identified, in the progress reports. It is not that they can come and do whatever they want to do. The MS is financing it and it is then up to the MS, but as they are so much involved in the assessment of the progress, they would very much follow the same interest in what needs to be done.  \\
\textbf{EC official, DG ELARG Albania team}: IPA finances a number of instruments, including Twinning and Twinning light. It also finances SIGMA. You can differentiate between long term and short term instruments. TAIEX is really a short term instrument, to fill the gaps, conferences, study visits, expert missions. IPA is more for long term projects, also for PAR. In a broader sense, Twinning is the best instrument because it gives you direct experience from the member state administrations to apply to candidate country administration. We all very much like Twinning. Regarding TA, it is a good question, to what degree do you need TA in the area of PA? I would say Twinning is more conducive, but then again SIGMA is more analytical.\\
\textbf{EC official, DG ELARG Montenegro team}: Twinning is not really an instrument for PAR, nor is TAIEX: They are more closely related to the Acquis. IPA is a good programme for TA to implement projects of the national programme (component I). But for small countries large and complex projects pose an absorption problem. Thus, within the national programme for Montenegro 5\% of the IPA budget are kept for small ad hoc projects.\\
%\newpage
%---------------------------------------------------------------------------------------------------------------------------
\section{Are these programmes well designed for the needs of PAR in Albania/Macedonia and Montenegro or do you perceive a need for adjustment in any of them? (Content or technical) }
\label{sec:technical}
\markboth{Anhang \thechapter, Frage Nr. \thesection}{Anhang \thechapter, Frage Nr. \thesection}
\textbf{EC official, DG ELARG PAR Coordination team}: I can imagine that generally speaking in the countries that the procedures are felt to be cumbersome or bureaucratic. But on the other hand, the Commission has a constant dialogue on improving the programmes and trying to reduce red tape or procedures, which result in delays and so on. The IPA-Regulation, when it entered into force in 2007, it was supposed to streamline previous regulations on financial assistance to candidate countries. And there were a lot of discussions on how to do that. Generally speaking, I think that our new rules and programmes are better designed, not only for the needs of PAR, but for all assistance needs in the enlargement process. There is perhaps one area, where things have to be improved and that is the area of donor coordination. Which is also, by the way, mentioned in the IPA regulation. We are very much aware of the latest Developments in that respect, the Paris Declaration on Aid Effectiveness and how to value that in our context of EU-Integration. To give you one example, DG Enlargement has initiated a discussion and process on how to apply a sector approach or programme based approach, and also whether assistance to PAR fits to that approach. The easiest you can apply the sector approach to is to technical sectors like education or transport, where we usually only have one main stakeholder, one ministry in charge. It is a bit different with PAR, which is a kind of cross-cutting sector, making it more difficult. The discussion now is that at least we could select PAR in one or two countries as a sort of pilot sector for this approach. If we do that, we can learn a lot to design assistance to PAR better to the needs of the country and also to avoid a kind of overlap with other donors. \\
\textbf{OECD/SIGMA team}: I think SIGMA needs to have some re-design as well, but it probably is closest to what should be done. Partly, because we have been abstracted from the Commission's rules and regulations, in terms of our operations. The Commission gives us financing for staff and operations as a sort of institutional contract for a period between two and three years. For all operations, for hiring consultants, missions etc., we operate under OECD rules, not EU rules. Maybe it takes two weeks to get consultants contracted, but if it is really urgent, you can do it quicker. Our responsiveness is determined by the production system, not by the approval system. Our staff are themselves experts and they are responsible for work in particular countries in their area and they remain responsible whether or not there is an operational activity and they are ready to alert us if something comes up. Another thing that sets SIGMA apart is to have that continuity. We are geographically and substantively concentrated, meaning specialized. And we try to resist any extension either geographically or substantively. Substantively to the margins where we think it is still relevant to governance. We also use a network of other PA experts, who are practitioners from MS.\\
\textbf{EC official, DG ELARG Evaluation Unit team}: I am not sure if we revised the instruments in the specific needs of these countries, but IPA is by definition quite a flexible instrument.\\
\textbf{EC official, DG ELARG Former Yugoslav Republic of Macedonia team}: Sometimes, the problem with projects is that they are conceived and then it takes a long time before they are implemented and sometimes the situation changes in the meantime. The problem could be from both sides, the national authorities being slow in preparing a project and sometimes it is also from our side. Now we are implementing IPA 2007 projects in Macedonia, which have been prepared even before that. Sometimes what we felt were the priorities then, are no priority any longer. This could be a problem, but not necessarily everywhere. We will also see what happens with the new sectoral approach, which might make it a bit easier to follow.\\
\textbf{EC official, DG ELARG Albania team}: Ideally it should be perfectly merged. Everybody on the political side should know what is going on in regards to financing and the other way around. We try to work that way that we agree on the analysis and then we agree on the priorities. Together with the colleagues who are dealing with financial assistance, we develop the forward looking strategy, the MIPD. This we have now done, also after our workshop with SIGMA.\\
\textbf{EC official, DG ELARG Montenegro team}: It depends on what you want to achieve. CARDS was a different tool, more geared towards reconstruction and infrastructures. It evolved and progressively included PAR in the programmes.  In general, IPA is an adequate tool, which can be adjusted to the needs and much appreciated.\\
%\newpage
%---------------------------------------------------------------------------------------------------------------------------
\section{ In your opinion, are there obstacles to PAR in Albania/Macedonia and Montenegro? And what would be necessary for successful PAR in Albania/Macedonia and Montenegro? }
\label{sec:montenegro2}
\markboth{Anhang \thechapter, Frage Nr. \thesection}{Anhang \thechapter, Frage Nr. \thesection}
\textbf{EC official, DG ELARG PAR Coordination team}: Talking about general obstacles, there is of course a lack of administrative capacity, lack of political will to carry out reforms and often a high corruption in those countries. SIGMA in their reports also talks about a general lack of respect for the law. In all these countries you have of course an outdated PA, which is much politicized. When it comes to more specific reasons, I can imagine that in a country like Macedonia, you have a very specific problem with its Albanian minority. And there you have the Orhid Framework, stipulating that minorities have to be represented in the PA. In Montenegro, although you have a multi-party system, the country has been governed by more or less the same party for many years. And because the country is so small there are very close links between the political and economic elites, which could give rise to what we call state capture, the most serious form of corruption. . Albania, also a small country has a strong historical legacy and everything is very much politicized, , with limited stability in the PA. We have started to discuss how to implement PAR in small countries There has to be domestic support and demand for reform. It means that one way to promote the reform process in these countries is to engage civil society more and the public in general. And we, the Commission and DG Enlargement have to be very clear on our requirements in our political dialogue and in reporting. \\
\textbf{OECD/SIGMA team}: I think the basic obstacle is that the people do not want it in the countries themselves. It is also a question of supply and demand. PAR is basically supply driven. The typical approach is that a project will provide lots of professional civil servants and therefore there will be a demand for them. Regarding service delivery, clients want, it probably. But even there I am not too sure, as they probably do not know what it means. They never had it, so they do not know what it means. Point two is that I do not think that we adapt our notion about service delivery to the basic problems in these countries. For example, one of the typical ideas about service delivery is about turnaround for decision making. Time is money, business need decisions quickly. If you are in a situation, where you have no legal predictability, no reliability about implementation of the law and administrative decision making, maybe getting the wrong decision quicker is actually not what you want; you may be happier to wait to get the "'right"' decision (i.e. the legal one). There may be demand from society, but whether political, administrative and business elites are interested in PAR, I am not so sure. I think you can not reform PA by PAR. We should look at new ways of dealing with it and in certain countries think about consolidating the basic functions of the state, which may require changing the PA. I do not think PAR is treated sufficiently politically and it is the political economy of PAR that is missing. It is treated as a technical issue, which it is not. \\
\textbf{EC official, DG ELARG Evaluation Unit team}: Largely politicised PAs are an obstacle. This will only change when the countries realize that they need a professional civil service, detached to some extent from what is going on politically. Positions are changed after elections, which is a huge obstacle to us and the brain drain related to that actually means, that there is no institutional memory. Also, civil servants are not paid enough and they go to the private sector. Corruption is still a big issue in most of these countries. But it is not only people changing, also procedures change and are in constant flux. And maybe something that has been developed via an EU programme over the years is then pushed off the table. For successful PAR, the milestones in the accession process are quite important. The move from potential candidate to candidate status changes dynamics in a country. Then a country has to get ready to steering other EU funds. You can only get a larger share if you have the right institutions to gear up for it. Also, institutions leaning more towards democracy in the way the three powers interact are needed and this is not fully the case in the three countries you are researching. \\
\textbf{EC official, DG ELARG Former Yugoslav Republic of Macedonia team}: In these countries and also in Macedonia, the concept of independence of PA needs to be accepted; that the public administration is not there to implement the ruling party's ideas and plans. PA is there as a service, which should be working independently. Of course the independence is not as in the judiciary, it is a different kind of independence. You would still have to follow instructions from the ministry, but this concept of a-political and service-oriented PA is really new and that is why we are stumbling with implementation, because even where there are good laws, without an understanding what it means to be non-political and service-oriented, the implementation is not there. We do what we can, we finance projects. It is a very long term process; it is not even finished in some of the member states.\\ 
\textbf{EC official, DG ELARG Albania team}: We are looking into the area of civil service, there is a reform ongoing and we follow SIGMAs assessment in terms of the gaps: Depolitization, merit based and transparent appointments. Stability of the institutions is not the case where position based appointments and politization create instability and staff turnover. In the area of Decentralization, there have been quite a number of measures undertaken, but these have got stuck. For strategic planning, there is a very sophisticated system in place in Albania actually, which needs to be implemented. Anti-corruption is a big issue. A lot of things have been done in Albania also in the context of the visa liberalization. Now it ist important that all systems in place work and are implemented, such as pro-activeness in investigations and prosecutions on all levels. The key issue Shere is impunity. An obstacle certainly is culture. The strong sense of authority of the highest person Sand non-transparency has to change.\\
\textbf{EC official, DG ELARG Montenegro team}: More time is needed, as the change of culture in PA is a long term process. Also, more English speaking personnel in the national PA would help. \\
%\newpage
%---------------------------------------------------------------------------------------------------------------------------
\section{Which other institutions/organizations or bilateral donors are important in regards to PAR in Albania/Macedonia and Montenegro? How do you asses their impact on PAR in the three countries?}
\label{sec:countries2}
\markboth{Anhang \thechapter, Frage Nr. \thesection}{Anhang \thechapter, Frage Nr. \thesection}
\textbf{EC official, DG ELARG PAR Coordination team}: I would need some time to compile details on this issue. \\
\textbf{OECD/SIGMA team}: World Bank, UNDP, some bilateral donors, particularly the US, Austria, less than before the UK. Also, GTZ is very present, but often as an implementer, rather than as a bilateral donor, the Dutch to some extent, in particular regarding some financial issues, and the Norwegians. Problematic is a strong project mentality, which is a whole larger issue, but I will leave you with our phrase, which is that Technical Assistance and PA should be driven by a service model, not by a production model. Most technical assistance is driven as a production system and all the technology of managing technical assistance, especially log frames comes out of engineering and was related to physical projects, which is somewhat distorting. The basic problem is that donor accountability systems are counterproductive to effective delivery of TA. And PA TA design is technocratically conceived and does not correspond to political reality or the constraints of complexity.\\
\textbf{EC official, DG ELARG Evaluation Unit team}: All donors are important and of course most important is that you combine your forces. In many areas presently there is a doubling up of assistance, which in itself is not a problem, but might be an efficiency problem. It is a problem of course, if you pull into different directions. That has happened in some areas and is almost unavoidable. \\
\textbf{EC official, DG ELARG Former Yugoslav Republic of Macedonia team}: This information I am expecting from the government in the special group on PAR. It is sometimes difficult for us to know who is bilaterally dealing with PAR. We know that there is the British government involved in some training projects on PAR. There should be somebody in the government responsible for donor coordination and we have requested an overview. \\
\textbf{EC official, DG ELARG Albania team}: My main source of information is the SIGMA reports on PAR. I remember UNDP and WB being quite active in Turkey, and the same seems to be true for Albania.\\
\textbf{EC official, DG ELARG Montenegro team}: UNDP and multilateral donors are often engaged in sub-sectors, but not often in PAR. UNDP also targets the municipality / local self-government level.\\
%\newpage
%---------------------------------------------------------------------------------------------------------------------------
\section{Who is responisble for co-ordinating the PAR activities of all the different donors in Albania, Macedonia and Montenegro and what is happening in this respect at the moment? }
\label{sec:moment}
\markboth{Anhang \thechapter, Frage Nr. \thesection}{Anhang \thechapter, Frage Nr. \thesection}
\textbf{EC official, DG ELARG PAR Coordination team}: In principle, it should be the country itself, the government, to coordinate the assistance from all the different donors, in line with the ownership principle of the Paris Declaration on Aid Effectiveness. There are different mechanisms in place in all countries for coordinating assistance including also assistance to PAR. Albania for example put in place a new fast tracking mechanism and I think there are similar mechanisms in the two other countries. We in DG Enlargement carried out an evaluation on donor coordination in 2008 and there we have descriptions on the issue in all three countries. Of course since then things have developed and have been improved. \\
\textbf{OECD/SIGMA team}: Who should be responsible, is the country, backed by Europe. The sector approach is supposed to provide greater country ownership over donors. I think the countries rightly suspect that it will imply greater donor ownership over the countries. I think as long as you have accountability arrangements in the donor community, which actually act against effectiveness, you will never have successful donor coordination, or even probably successful projects in PAR, because the requirements are just too hard to fit into the engineering type of contractual framework that the
donor accountability inputs.\\
\textbf{EC official, DG ELARG Evaluation Unit team}: The beneficiary-countries and the Delegations have a good overview. The beneficiary countries all have their coordination units and the Delegation cooperates quite closely or screens this. Putting the sector-wide approach in place will help. While the EU does not focus on horizontal PAR, while other donors like the WB do. There is not really an EU-instrument for PAR. We only concentrate on PAR when there is institutional instability. There is a certain share of responsibility between donors. The WB is moving in certain areas and we do not interfere much with that. The donors are very often attending our internal coordination meetings of the Delegations. And even for the evaluations we are doing now, we invited the donors. Very often the donors also go to the Delegations and rely on information they receive from there. Also, all our documents are on the internet. If donors want to coordinate with us, that is always possible. Twinning really is an instrument for vertical reforms, not horizontal ones. The classical Twinning is for example to train people how to run a border station or in a Ministriy on financial control, which you need to implement regional funds. What we have for horizontal PAR is our SIGMA programme. We do not guide countries towards certain horizontal reforms in detail. There is not lack of knowledge on what all the donors are doing, but it is deliberate to keep out of horizontal issues, as you get involved in politics. With the sector-approach, we are trying to streamline the assistance.\\ 
\textbf{EC official, DG ELARG Former Yugoslav Republic of  Macedonia team}: Reference made to the answer to question 11.\\
\textbf{EC official, DG ELARG Albania team}: GTZ is directly working with the Department of Public Administration (DOPA). GTZ can be seen as bilateral donor, but could also be seen as contractor. Albania is a model regarding donor coordination. They have a whole system in place, which is located in the Council of Ministers with a department that directly reports to the Prime Minister. It is a parallel department to the Strategic Planning Department and they are doing donor coordination. Donor coordination is very important in the context of our assistance and Albania is singled out as good example in this respect.\\
\textbf{EC official, DG ELARG Montenegro team}: The government has recently nominated a person to carry out the function of donor coordination in the Prime Minister's Office. This was long due and hopefully will make a change. It is of utmost importance to better coordinate donors, as they do not always share their views. Some even think that they are competing! An example for good donor coordination at project level in Montenegro is taking place between the EU and the World Bank on agricultural issues. Also with Kreditanstalt für Wiederaufbau (KFW), EU coordination is good.\\
%\newpage
%---------------------------------------------------------------------------------------------------------------------------
\section{Should the EU have additional or other priorities in future PAR programming in Albania, Macedonia and Montenegro?}
\label{sec:montenegro3}
\markboth{Anhang \thechapter, Frage Nr. \thesection}{Anhang \thechapter, Frage Nr. \thesection}
\textbf{EC official, DG ELARG PAR Coordination team}: That depends on  how we define the scope of PAR and  on the possible gaps or problems in these countries. One has first to agree on the priorities and then decide whether these priorities should be supported by assistance. Generally speaking, I would say that our listing  of priorities connected with PAR in the partnership agreements is not uniform, and  not based on a common understanding of PAR or its scope.. It should be added that there is one important organizational tool within DG Elarg, which we call our Matrix system. You have staff in the  country desks, who are horizontally responsible for each chapter. They play an important role in negotiations and in providing input to progress reports and opinions and they also support preparations for sub-committee meetings. The PAR coordinator could play a similar role with regard to PAR  although there is no Acquis chapter for PAR.\\
\textbf{OECD/SIGMA team}: I think the EU should be much more concerned about administrative justice than they are. They are of course concerned about the penal aspects, but they should be much more concerned about administrative justice decision making and financial issues. Also, the EU should focus on some of the governance issues, including on the incentives for individuals and MPs and the capacities of Parliament. And the EU should think about PAR as a support to policy, rather than as PAR as a policy in its own right, because I don't think that approach is very effective.\\
\textbf{EC official, DG ELARG Evaluation Unit team}: Given the amount of funds, we can not tackle all the issues at once. Kosovo is a good example, where we came up with a very narrow list of sectors, three or four. I think it makes sense to have a sequenced approach, as with the sectors. Justice and Home Affairs is a very important sector in that respect. If you tackle brain drain and corruption and all of the main issues and have these out of the way, the other sectors should be easier to reform as well.
\textbf{EC official, DG ELARG Former Yugoslav Republic of Macedonia team}: I think we are fine, we identified what our priorities are, so I do not think there will be any new revealing discoveries, what would be the core of the problem. I think that if we stick to these principles of independent, non-political and service oriented PA, this leads us of course to questions of recruitment and career. I think, if we make progress in this part, we do not have to look anywhere else.\\
\textbf{EC official, DG ELARG,Albania team}: We have established the MIPD. As soon as it is adopted, you will find the priorities. \\
\textbf{EC official, DG ELARG Montenegro team}: The Europe 2020 strategy as well as  the enlargement strategy, are to be taken into account while designing the future programmes. IPA programme should align to these.  This results in topics, such as competitiveness and climate change being high on the agenda in IPA programming. While IPA is not an instrument for the private sector (the private sector is better dealt with by the EBRD), it can certainly participate to climate change for example: for example there is an emphasis on railways and not on roads, which is in line with Europe 2020. There is continuous discussion, development and adjustment of projects with communication lines between the national government, the Delegation and the Commission. The EU knows what the country needs and the country knows this as well. \\
%\newpage
%-------------------------------------------------------------------------------------------------------------------
\section{Is there anything else that is important in the context of my research that you would like to comment on? }
\label{sec:comment on}
\markboth{Anhang \thechapter, Frage Nr. \thesection}{Anhang \thechapter, Frage Nr. \thesection}
\textbf{EC official, DG ELARG PAR Coordination team}: No time was left to ask this question.
\textbf{OECD/SIGMA team}: I think your questions are relatively light on substance and I think it is very difficult to discus. If you took out the word "'PAR"' and replaced it with "'environmental policy"', would your questions still make sense? I suspect yes. But I think PAR has some very specific characteristics. I think that you have to understand the nature of PAR; we are now trying to call it PGR, Public Governance Reform, as being distinct from other sectors. Perhaps what I touched upon is the political economy as important to look at. I think we have discussed some aspects in respect to service delivery, but I do not think this captures the political economy.(Reference to Merilee Grindle, the "'good enough governance"' debate and Sue Unwin the debate on "'drivers of reform"' as well as donor interest in political economy issues). I think PAR could benefit enormously from such thinking, why things work and why things don't work and what drivers you could pursue in order to pursue PAR, for example business interests, although business interests in many countries it turns out that they are not so forceful, because of the oligarchization, but that is the sort of discussion. I think we are implicitly asking countries to go far too quickly and to go too far. Societies are not ready for that sort of adaptation. And I think that one issue that needs to be addressed is the economics, especially in light of the economic crisis. Many of the things that donors are trying to push on them, actually cost too much for them and are not tested. But there is a larger issue, which is (it was the same for the CEECs) that these countries are all poor and have poor state resources. We are asking them to put in place PAs and laws, which are designed for the rich Northern Europeans, but they do not have the tax base to finance that. So we are creating implementation gaps. The second point on that is that our laws, our institutions and our economies developed organically. We are now talking to countries with weak, poor states and asking them to put in place laws for which they do not have the economic or institutional support and so again we are driving implementation gaps. And as a result, what you see is legal formalism. They will produce things according to their perceptions of what we want, with very little sense of ownership or intention to implement. And one last thing that I can think of is that you have three countries that are all small and Montenegro is tiny. Smallness has absolute limits that is to say, even if these countries were rich and did not have the financing gap, you would still find it difficult in Montenegro to develop all the implementation instruments that we require. That is one of the reasons we are doing some analysis on the EI in small states and rationalization of requirements.\\ 
\textbf{EC official, DG ELARG Evaluation Unit team}: I think one issue that is very important in the context of the Balkans, and I often feel like on an island in that respect is the development and reconstruction issue. It never gets mentioned by people, except those who are around longer. When the EAR was dissolved, in many people's heads this was the end of reconstruction in these countries. But having visited the countries and having seen the attitude in these countries, I feel this should not be disregarded. I do not think we have the stability in regards to constitutional reforms and the political party set up. I think there is an underlying element of instability.\\ 
\textbf{EC official, DG ELARG Former Yugoslav Republic of Macedonia team}: What we discussed is actually the core of the problem. I think it would be useful to you to speak to people in the relevant units how the project support is done, to understand the process better.\\
\textbf{EC official, DG ELARG Albania team}: Nothing that comes to my mind immediately.\\
\textbf{EC official, DG ELARG, Montenegro team}: Regarding PAR, the most important issue is the civil service, and some ministries are interested, such as those in charge of Police, Human Resources and Social issues. On other issues like Health, there is not much Acquis and the EU needs to make sure that other donors do complement EU funding, which is more towards Acquis issues. \\
