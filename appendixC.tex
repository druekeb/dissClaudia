\chapter{Questionnaire Public Administration Reform experts in Albania, FYROM, Montenegro}
\label{Questionnaire Public Administration Reform experts}
1. Which main topics/areas in the context of Public Administration Reform are presently dealt with as a priority by Albania/Montenegro/FYROM?

2. Is the institution/structure dealing with PAR adequate for the tasks ahead? 

3. In your opinion, are there obstacles to PAR in Albania/Macedonia and Montenegro? And what would be necessary for successful PAR in Albania/Macedonia and Montenegro? 

4. Public Administration Reform is not a separate chapter in the Acquis. Should it be a separate chapter? 

5. How do you assess the cooperation within the EU regarding Public Administration Reform in Albania/Macedonia and Montenegro 

6. Do you think the EU approach regarding Public Administration Reform in Albania, Macedonia and Montenegro is adequate? Or should other aspects be included from your point of view?

7. What is your opinion, how does the new IPA instrument work for PAR in Albania, Montenegro and Macedonia? Examples?

8. Is this programme well designed for the needs of PAR in Albania/Macedonia and Montenegro or do you perceive a need for adjustment ? (Content or technical)

9. Which other institutions/organizations or bilateral donors apart from the EU are important in regards to PAR in Albania/Macedonia and Montenegro? How do you asses their impact on PAR? 

10. Who is responsible for co-ordinating the PAR activities of all the different donors in Albania, Macedonia and Montenegro and what is happening in this respect at the moment? 

11. Should the EU have additional or other priorities in future PAR programming in Albania, Macedonia and Montenegro. 

12. Is there anything else that is important in the context of my research that you would like to comment on?